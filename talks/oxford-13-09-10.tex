\documentclass[10pt]{beamer}

\usepackage[english]{babel}
\usepackage[utf8]{inputenc}
\usepackage{amssymb}
\usepackage{amsthm}
\usepackage[all]{xy}
\usepackage{graphicx}
\usepackage{tikz}
\usetikzlibrary{trees}

%% Stuff
\renewcommand{\le}{\leqslant}
\renewcommand{\ge}{\geqslant}  % comme François le demande...
\newcommand{\blue}[1]{\textcolor{blue}{#1}}  % colouring
%% Algèbre
  \newcommand{\hub}{}
  \newcommand{\rmul}[1]{*_{#1}}
  \newcommand{\dual}[1]{{#1}^\ast}
  
\newcommand{\clot}[1]{\bar{#1}}  % clôture algèbrique
\newcommand{\card}[1]{\# #1}  % cardinalité d'un ensemble
\DeclareMathOperator{\car}{char}  % caractéristique d'un corps
\DeclareMathOperator{\Frac}{Frac}  % corps des fractions
\newcommand{\Z}{\mathbb{Z}}  % les entiers
\newcommand{\K}{\mathbb{K}}  % un corps
\newcommand{\LK}{\mathbb{L}}  % encore un corps
\newcommand{\U}{\mathbb{U}}  % encore un corps
\newcommand{\F}{\mathbb{F}}  % un corps fini
\newcommand{\Q}{\mathbb{Q}}  % les rationnels
\newcommand{\R}{\mathbb{R}}  % les réels
\newcommand{\C}{\mathbb{C}}  % les complexes
\newcommand{\isom}{\cong}  % isomorphisme de corps
\newcommand{\frob}{\varphi}  % fröbenius
\DeclareMathOperator{\Gal}{Gal}  % groupe de Galois
\DeclareMathOperator{\Tr}{Tr}  % trace
\DeclareMathOperator{\PTr}{PTr}  % pseudotrace
\DeclareMathOperator{\Norm}{N} % norme
\newcommand{\euler}{\phi}  % indicatrice d'Euler
\DeclareMathOperator{\ord}{ord}  % l'ordre d'un élément
\newcommand{\AS}[1]{\mathcal{#1}}  % la police des polynômes d'AS
\DeclareMathOperator{\rev}{rev}  % le reverse d'un polynôme
%% Courbes
\DeclareMathOperator{\Jac}{Jac}  % la jacobienne
\newcommand{\Proj}{\mathbb{P}}  % espace projectif
\newcommand{\0}{\mathcal{O}}  % point de base d'une courbe
\newcommand{\ecpoint}[3]{[#1:#2:#3]}  % un point d'une courbe
\newcommand{\isog}[1]{\mathcal{#1}}  % la police des isogénies
\newcommand{\I}{\isog{I}}  % une isogénie I
\newcommand{\Hasse}{H}  % l'invariant de Hasse
\newcommand{\divpol}{f}  % polynôme de division
%% Autre
\newcommand{\tildO}{\tilde{O}}  % la notation O~ qui oublie les log
\newcommand{\Mint}{\mathrm{\sf M}_\text{int}}  % fonction de multiplication
\newcommand{\Mpol}[1][]{\mathrm{\sf M}_\text{pol}^{#1}}  % fonction de multiplication
\newcommand{\Mult}[1][]{\mathrm{\sf M}_{#1}}  % fonction de multiplication
\newcommand{\Push}{\mathrm{\sf P}}  % fonction de push-down
\newcommand{\Lift}{\mathrm{\sf L}}  % fonction de lift-up
\newcommand{\Trace}{\mathrm{\sf T}}  % fonction de trace
\newcommand{\Frob}{\mathrm{\sf F}}  % fonction de frobenius itéré
\newcommand{\Ptr}{\mathrm{\sf PT}}  % fonction de pseudo-trace
\newcommand{\ModComp}{\mathrm{\sf C}}  % fonction de composition modulaire
\newcommand{\alg}[1]{\textsf{#1}}  % la police des algorithmes
\newcommand{\wrt}{\dashv}  % appartenance forte, a\wrt A signifie que a est représenté comme un élément de A
\DeclareMathOperator{\op}{op}  % une opération

\newenvironment{algorithm}[3]{\begin{center}\begin{minipage}{0.85\textwidth}
      \sf
      \rule{\textwidth}{0.2pt}\\
      \makebox[\textwidth][c]{\textbf{#1}}\\
      \rule[0.5\baselineskip]{\textwidth}{0.2pt}\\
      \textbf{Input~~} #2\\
      \textbf{Output} #3
      \smallskip
      \begin{enumerate}
}{\end{enumerate}
      \rule{\textwidth}{0.2pt}
\end{minipage}\end{center}}

% beamer-specific
%\setbeamertemplate{theorem begin}{
%  \begin{\inserttheoremblockenv}{
%      Théorème
%      \ifthenelse{\equal{\inserttheoremaddition}{}}
%		 {}
%		 {(\inserttheoremaddition)}    
		 %  \ifx\inserttheoremaddition\@empty\else\ (\inserttheoremaddition)\fi
%    }
%}

\mode<presentation>{%
  \usetheme[]{Madrid}
  \usefonttheme[onlymath]{serif}
  \usecolortheme{crane}
% \usecolortheme{rose}
}


\title{Research interests}
\author[L.~De~Feo]{L.~De~Feo}
\institute[TANC, LIX]{INRIA Projet TANC \& LIX École Polytechnique}
\date{\today}


% \AtBeginSection[]
% {
%   \begin{frame}<beamer>
%     \frametitle{Plan}
%     \tableofcontents[currentsection]
%   \end{frame}
% }


\begin{document}

\begin{frame}
  \titlepage
\end{frame}

%%

\begin{frame}
  \frametitle{My PhD}
  
  I worked at the interface between Computer Algebra and Algorithmic
  Number Theory. My doctoral dissertation will be about:
  \begin{itemize}
  \item Duality (both in a classic and in a categorical sense) applied
    to algebraic algorithms: the transposition principle;
  \item Fast algorithms for tower of extensions over finite fields:
    the Artin-Schreier case;
  \item Fast algorithms to compute isogenies between elliptic curves.
  \end{itemize}
\end{frame}

%%

\begin{frame}
  \frametitle{The transposition principle}
  \begin{columns}
    \begin{column}{0.6\textwidth}
      \begin{center}
        Arithmetic circuits behave like diagrams: flip the arrows and
        obtain the dual (this can be made precise using
        \emph{categorical semantics}).
      \end{center}

      \centering
      \begin{tikzpicture}
        \tikzstyle{node}=[circle,thick,draw=black,minimum size=4mm]
        \tikzstyle{arg}=[rectangle,thin,draw=black,minimum size=4mm]
        
        \begin{scope} \node[arg](in1){$x_1$}; \node[arg,right
          of=in1](in2){$x_2$}; \node[arg,right of=in2](in3){$x_3$};
          
          \node[node,below of=in1](plus1){$+$}; \node[node,right
          of=plus1](H){$\hub$};
          
          \node[node,below of=plus1](plus2){$+$}; \node[node,right
          of=plus2](times){$\rmul{2}$};
          
          \node[arg,below of=plus2,xshift=6mm](out1){$y_1$};
          \node[arg,right of=out1](out2){$y_2$};
          
          \path[->] (in1) edge (plus1) (in2) edge (H) (in3) edge (out2)
          (H) edge (plus1) (H) edge (times) (plus1) edge (plus2) (times) edge
          (plus2) (plus2) edge (out1);
        \end{scope}
        
        \begin{scope}[xshift=4cm]
          \node[arg](in1){$\dual{x_1}$};
          \node[arg,right of=in1](in2){$\dual{x_2}$};
          \node[arg,right of=in2](in3){$\dual{x_3}$};
          
          \node[node,below of=in1](plus1){$\hub$};
          \node[node,right of=plus1](H){$+$};
          
          \node[node,below of=plus1](plus2){$\hub$};
          \node[node,right of=plus2](times){$\rmul{2}$};
          
          \node[arg,below of=plus2,xshift=6mm](out1){$\dual{y_1}$};
          \node[arg,right of=out1](out2){$\dual{y_2}$};

          \path[<-]
          (in1) edge (plus1)
          (in2) edge (H)
          (in3) edge (out2)
          (H) edge (plus1)
          (H) edge (times) 
          (plus1) edge (plus2)
          (times) edge (plus2)
          (plus2) edge (out1);
        \end{scope}
      \end{tikzpicture}

      \begin{center}
        In particular, in the case of morphisms of free modules this
        corresponds to transposition of matrices.
      \end{center}
      
    \end{column}    
    \begin{column}{0.4\textwidth}
      \begin{align*}
        y_1 &= x_1 + 3x_2\\
        y_2 &= x_3
      \end{align*}
      
      \Large
      \begin{gather*}
        \begin{pmatrix}
          1 & 3 & 0\\
          0 & 0 & 1
        \end{pmatrix}\\
        \updownarrow\\
        \begin{pmatrix}
          1 & 0\\
          3 & 0\\
          0 & 1\\
        \end{pmatrix}
      \end{gather*}
    \end{column}
  \end{columns}

\end{frame}

%%

\begin{frame}
  \frametitle{The transposition principle}

  This carries over to algebraic algorithms:
  \begin{itemize}
  \item \emph{swap} the flow of the algorithm (read from top to bottom),
  \item \emph{dualise} each instruction (swap algebraic input and outputs),
  \item the result is an algorithm for the \emph{dual} problem,
    \textbf{and the algebraic complexity is preserved!}
  \end{itemize}

  Examples:
  \begin{itemize}
  \item transposed multiplication = \emph{middle
      product}~\cite{hanrot+quercia+zimmermann},
  \item transposed Euclidean remainder = extension of linearly
    recurring sequences~\cite{bostan+lecerf+schost:tellegen},
  \item transposed polynomial evaluation = \emph{power
      projection}~\cite{shoup94,shoup95,shoup99,bostan+salvy+schost03};
  \item applications will come next\dots
  \end{itemize}

  \emph{Transposition of algorithms} is a recurring need, but doing it
  by hand is difficult and error prone

  \begin{center}
    \large We applied the theory of functional languages and
    implemented a compiler that performs \emph{automatic
      transposition}~\cite{df+schost10}.
  \end{center}
\end{frame}

%%

\begin{frame}
  \frametitle{Artin-Schreier towers}
  
\end{frame}

%%

\begin{frame}
  \frametitle{Isogenies between elliptic curves}
  
\end{frame}

%%

\begin{frame}
  \frametitle{Perspectives on the transposition principle}
  
\end{frame}

%%

\begin{frame}
  \frametitle{Perspectives on isogenies}
  
\end{frame}

%%

\begin{frame}
  \frametitle{Perspectives on point counting}
  
\end{frame}

%%

\begin{frame}[allowframebreaks]
  \frametitle{References}
  
  \bibliographystyle{plain}
  \bibliography{../defeo}
\end{frame}

\end{document}


% Local Variables:
% mode:flyspell
% ispell-local-dictionary:"british"
% mode:TeX-PDF
% End:
%

% LocalWords:  Isogeny abelian isogenies hyperelliptic supersingular Frobenius
% LocalWords:  isogenous
