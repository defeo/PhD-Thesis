\documentclass[10pt]{beamer}

\usepackage[english]{babel}
\usepackage[utf8]{inputenc}
\usepackage{amssymb}
\usepackage{amsthm}
\usepackage[all]{xy}
\usepackage{graphicx}
\usepackage{tikz}
\usetikzlibrary{trees}

%% Stuff
\renewcommand{\le}{\leqslant}
\renewcommand{\ge}{\geqslant}  % comme François le demande...
\newcommand{\blue}[1]{\textcolor{blue}{#1}}  % colouring
%% Algèbre
  \newcommand{\hub}{}
  \newcommand{\rmul}[1]{*_{#1}}
  \newcommand{\dual}[1]{{#1}^\ast}
  
\newcommand{\clot}[1]{\bar{#1}}  % clôture algèbrique
\newcommand{\card}[1]{\# #1}  % cardinalité d'un ensemble
\DeclareMathOperator{\car}{char}  % caractéristique d'un corps
\DeclareMathOperator{\Frac}{Frac}  % corps des fractions
\newcommand{\Z}{\mathbb{Z}}  % les entiers
\newcommand{\K}{\mathbb{K}}  % un corps
\newcommand{\LK}{\mathbb{L}}  % encore un corps
\newcommand{\U}{\mathbb{U}}  % encore un corps
\newcommand{\F}{\mathbb{F}}  % un corps fini
\newcommand{\Q}{\mathbb{Q}}  % les rationnels
\newcommand{\R}{\mathbb{R}}  % les réels
\newcommand{\C}{\mathbb{C}}  % les complexes
\newcommand{\isom}{\cong}  % isomorphisme de corps
\newcommand{\frob}{\varphi}  % fröbenius
\DeclareMathOperator{\Gal}{Gal}  % groupe de Galois
\DeclareMathOperator{\Tr}{Tr}  % trace
\DeclareMathOperator{\PTr}{PTr}  % pseudotrace
\DeclareMathOperator{\Norm}{N} % norme
\newcommand{\euler}{\phi}  % indicatrice d'Euler
\DeclareMathOperator{\ord}{ord}  % l'ordre d'un élément
\newcommand{\AS}[1]{\mathcal{#1}}  % la police des polynômes d'AS
\DeclareMathOperator{\rev}{rev}  % le reverse d'un polynôme
%% Courbes
\DeclareMathOperator{\Jac}{Jac}  % la jacobienne
\newcommand{\Proj}{\mathbb{P}}  % espace projectif
\newcommand{\0}{\mathcal{O}}  % point de base d'une courbe
\newcommand{\ecpoint}[3]{[#1:#2:#3]}  % un point d'une courbe
\newcommand{\isog}[1]{\mathcal{#1}}  % la police des isogénies
\newcommand{\I}{\isog{I}}  % une isogénie I
\newcommand{\Hasse}{H}  % l'invariant de Hasse
\newcommand{\divpol}{f}  % polynôme de division
%% Autre
\newcommand{\tildO}{\tilde{O}}  % la notation O~ qui oublie les log
\newcommand{\Mint}{\mathrm{\sf M}_\text{int}}  % fonction de multiplication
\newcommand{\Mpol}[1][]{\mathrm{\sf M}_\text{pol}^{#1}}  % fonction de multiplication
\newcommand{\Mult}[1][]{\mathrm{\sf M}_{#1}}  % fonction de multiplication
\newcommand{\Push}{\mathrm{\sf P}}  % fonction de push-down
\newcommand{\Lift}{\mathrm{\sf L}}  % fonction de lift-up
\newcommand{\Trace}{\mathrm{\sf T}}  % fonction de trace
\newcommand{\Frob}{\mathrm{\sf F}}  % fonction de frobenius itéré
\newcommand{\Ptr}{\mathrm{\sf PT}}  % fonction de pseudo-trace
\newcommand{\ModComp}{\mathrm{\sf C}}  % fonction de composition modulaire
\newcommand{\alg}[1]{\textsf{#1}}  % la police des algorithmes
\newcommand{\wrt}{\dashv}  % appartenance forte, a\wrt A signifie que a est représenté comme un élément de A
\DeclareMathOperator{\op}{op}  % une opération

\newenvironment{algorithm}[3]{\begin{center}\begin{minipage}{0.85\textwidth}
      \sf
      \rule{\textwidth}{0.2pt}\\
      \makebox[\textwidth][c]{\textbf{#1}}\\
      \rule[0.5\baselineskip]{\textwidth}{0.2pt}\\
      \textbf{Input~~} #2\\
      \textbf{Output} #3
      \smallskip
      \begin{enumerate}
}{\end{enumerate}
      \rule{\textwidth}{0.2pt}
\end{minipage}\end{center}}

% beamer-specific
%\setbeamertemplate{theorem begin}{
%  \begin{\inserttheoremblockenv}{
%      Théorème
%      \ifthenelse{\equal{\inserttheoremaddition}{}}
%		 {}
%		 {(\inserttheoremaddition)}    
		 %  \ifx\inserttheoremaddition\@empty\else\ (\inserttheoremaddition)\fi
%    }
%}

\mode<presentation>{%
  \usetheme[]{Madrid}
  \usefonttheme[onlymath]{serif}
  \usecolortheme{crane}
% \usecolortheme{rose}
}


\title{Research interests}
\author[L.~De~Feo]{L.~De~Feo}
\institute[TANC, LIX]{INRIA Projet TANC \& LIX, École Polytechnique}
\date{\today}


% \AtBeginSection[]
% {
%   \begin{frame}<beamer>
%     \frametitle{Plan}
%     \tableofcontents[currentsection]
%   \end{frame}
% }


\begin{document}

\begin{frame}
  \titlepage
\end{frame}

%%

\begin{frame}
  \frametitle{My PhD}
  
  I work at the interface between Computer Algebra and Algorithmic
  Number Theory. My doctoral dissertation is about:
  \begin{itemize}
  \item Duality (both in a classic and in a categorical sense) applied
    to algebraic algorithms: the transposition principle;
  \item Fast algorithms for tower of extensions of finite fields: the
    Artin-Schreier case;
  \item Fast algorithms to compute isogenies between elliptic curves.
  \end{itemize}
\end{frame}

%%

\begin{frame}
  \frametitle{The transposition principle}
  \begin{columns}
    \begin{column}{0.6\textwidth}
      \begin{center}
        Arithmetic circuits behave like diagrams: flip the arrows and
        obtain the dual (this can be made precise using
        \emph{categorical semantics}).
      \end{center}

      \centering
      \begin{tikzpicture}
        \tikzstyle{node}=[circle,thick,draw=black,minimum size=4mm]
        \tikzstyle{arg}=[rectangle,thin,draw=black,minimum size=4mm]
        
        \begin{scope} \node[arg](in1){$x_1$}; \node[arg,right
          of=in1](in2){$x_2$}; \node[arg,right of=in2](in3){$x_3$};
          
          \node[node,below of=in1](plus1){$+$}; \node[node,right
          of=plus1](H){$\hub$};
          
          \node[node,below of=plus1](plus2){$+$}; \node[node,right
          of=plus2](times){$\rmul{2}$};
          
          \node[arg,below of=plus2,xshift=6mm](out1){$y_1$};
          \node[arg,right of=out1](out2){$y_2$};
          
          \path[->] (in1) edge (plus1) (in2) edge (H) (in3) edge (out2)
          (H) edge (plus1) (H) edge (times) (plus1) edge (plus2) (times) edge
          (plus2) (plus2) edge (out1);
        \end{scope}
        
        \begin{scope}[xshift=4cm]
          \node[arg](in1){$\dual{x_1}$};
          \node[arg,right of=in1](in2){$\dual{x_2}$};
          \node[arg,right of=in2](in3){$\dual{x_3}$};
          
          \node[node,below of=in1](plus1){$\hub$};
          \node[node,right of=plus1](H){$+$};
          
          \node[node,below of=plus1](plus2){$\hub$};
          \node[node,right of=plus2](times){$\rmul{2}$};
          
          \node[arg,below of=plus2,xshift=6mm](out1){$\dual{y_1}$};
          \node[arg,right of=out1](out2){$\dual{y_2}$};

          \path[<-]
          (in1) edge (plus1)
          (in2) edge (H)
          (in3) edge (out2)
          (H) edge (plus1)
          (H) edge (times) 
          (plus1) edge (plus2)
          (times) edge (plus2)
          (plus2) edge (out1);
        \end{scope}
      \end{tikzpicture}

      \begin{center}
        In particular, in the case of morphisms of free modules this
        corresponds to transposition of matrices.
      \end{center}
      
    \end{column}    
    \begin{column}{0.4\textwidth}
      \begin{align*}
        y_1 &= x_1 + 3x_2\\
        y_2 &= x_3
      \end{align*}
      
      \Large
      \begin{gather*}
        \begin{pmatrix}
          1 & 3 & 0\\
          0 & 0 & 1
        \end{pmatrix}\\
        \updownarrow\\
        \begin{pmatrix}
          1 & 0\\
          3 & 0\\
          0 & 1\\
        \end{pmatrix}
      \end{gather*}
    \end{column}
  \end{columns}

\end{frame}

%%

\begin{frame}
  \frametitle{The transposition principle}

  This carries over to algebraic algorithms:
  \begin{itemize}
  \item \emph{swap} the flow of the algorithm (read from top to bottom),
  \item \emph{dualise} each instruction (swap algebraic input and outputs),
  \item the result is an algorithm for the \emph{dual} problem,
    \textbf{and the algebraic complexity is preserved!}
  \end{itemize}

  Examples:
  \begin{itemize}
  \item transposed multiplication = \emph{middle
      product}~\cite{hanrot+quercia+zimmermann},
  \item transposed Euclidean remainder = extension of linearly
    recurring sequences~\cite{bostan+lecerf+schost:tellegen},
  \item transposed polynomial evaluation = \emph{power
      projection}~\cite{shoup94,shoup95,shoup99,bostan+salvy+schost03};
  \end{itemize}

  \emph{Transposition of algorithms} is a recurring need, but doing it
  by hand is difficult and error prone

  \begin{center}
    \large Schost and I applied the theory of functional languages and
    implemented a compiler that performs \emph{automatic
      transposition}~\cite{df+schost10}.
  \end{center}
\end{frame}

%%

\begin{frame}
  \frametitle{Artin-Schreier towers}
  
  $\K$ of characteristic $p$, Artin-Schreier extension:\hspace{\stretch{1}}
  $\K[X]/X^p-X-\alpha$\hspace{\stretch{1}}

  \begin{itemize}
  \item towers of Artin-Schreier extensions arise naturally when
    computing $p^k$-torsion groups of abelian varieties;
  \item ideally, one would like to do arithmetics in A-S towers of
    height $k$ in $\tildO(p^k)$ time and space;
  \end{itemize}

  Schost and I~\cite{df+schost09} gave nearly optimal algorithms for
  arithmetics\\ (multiplication, inversion, traces, embeddings,
  Frobenius, etc.) in A-S towers of finite fields. We built upon:
  \begin{itemize}
  \item A construction by Cantor~\cite{cantor89} yielding a tower with
    \emph{minimal} coefficients;
  \item Couveignes' algorithm for isomorphisms between A-S
    towers~\cite{couveignes00};
  \item resolution of zero-dimensional ideals using rational
    univariate representation and trace
    formulae~\cite{alonso+becker+roy+wormann,diaz+gonzalez01,rouiller99};
  \item transposed algorithms for \emph{power
      projections}~\cite{shoup94,shoup95,shoup99,bostan+salvy+schost03};
  \item (having transposed hundreds of lines of C code by hand, this
    was the motivation for the work on \emph{automatic
      transposition}).
  \end{itemize}
\end{frame}

%%

\begin{frame}
  \frametitle{Isogenies between elliptic curves} 

  \begin{center}
    Given $E$, $E'$, $\ell$-isogenous over $\F_q$, compute $\I:E\to
    E'$ of degree $\ell$.
  \end{center}

  \begin{itemize}
  \item If $\ker\I$ is known, use Vélu formulae;
  \item if $\car\K\gg\ell$:
    \begin{itemize}
    \item Factoring the modular polynomial $\Phi_\ell(X,j(E))$ gives
      an isogenous curve isomorphic to $E'$, \hfill\alert{costs
        $\tildO(\ell^2)$, using \cite{sutherland10:modpol}}
    \item then, use differential equation satisfied by
      $\I$~\cite{elkies98,bostan+morain+salvy+schost08};
      \hfill\alert{costs $\tildO(\ell)$}
    \end{itemize}
  \item otherwise:
    \begin{itemize}
    \item Couveignes' first algorithm using formal groups~\cite{couveignes94},
      \hfill\alert{$O(\ell^3)$}
    \item Lercier's algorithm for
      $p=2$~\cite{lercier96},\hfill\alert{$O(\ell^3)$?}
    \item Couveignes's second algorithm interpolating over the
      $p^k$-torsion~\cite{couveignes96},\hfill\alert{$\tildO(\ell^3)$}
    \item Lercier-Sirvent algorithm, lifting in the
      $p$-adics~\cite{lercier+sirvent08}.\hfill 
      \alert{$\tildO(\ell^2)$, using \cite{sutherland10:modpol}}
    \end{itemize}
  \end{itemize}

  \begin{center}
    I improved Couveignes second algorithm, applying the
    Artin-Schreier machinery and other improvements~\cite{df10}.
    Complexity is now \alert{$\tildO(\ell^2)$}.
  \end{center}
\end{frame}

%%

\begin{frame}
  \frametitle{Perspectives on the transposition principle}

  \begin{itemize}
  \item Integration of our automatic transposition tool in
    CAS\footnote{Computer Algebra Systems} (Sage?  Mathemagix?).
  \item Categorical semantics give a powerful insight in the structure
    of the transposition principle. Based on this, Boespflug and I
    tried to implement a DSL\footnote{Domain Specific Language} in
    Haskell, but we found obstacles.
  \item We plan to implement a DSL in a language with
    dependent types (Coq? Some extensions of Haskell?).
  \item Using ATP's\footnote{Automated Theorem Provers} to ensure the
    correct semantics of CAS is an emerging trend. We plan to write a
    library to facilitate the use of the transposition principle in
    ATP's (Coq? Isabelle?)
  \item More broadly, I am interested in the interactions between
    ATP's and Computer Algebra. This is the main interest of the
    Calculemus
    network\footnote{\protect\url{http://www.calculemus.net/}} and
    conferences series.
  \end{itemize}
\end{frame}

%%

\begin{frame}
  \frametitle{Perspectives on isogenies}
  
  \begin{itemize}
  \item None of the algorithms to compute isogenies between elliptic
    curves over finite fields reaches the optimal bound of $O(\ell)$
    field operations.  Morain and I have some ideas to reach this
    bound, that we are currently exploring.
  \item Working on Couveignes' second algorithm, I discovered a
    surprising generalisation that permits us to compute isogenies of
    unknown degree at no additional cost. This result is optimal and
    it is the first of its genre in the field. It may have
    applications in
    cryptography~\cite{teske06,rostovtsev+stolbunov06}.
  \item Biasse and I are currently working on an implementation of
    Teske's trapdoor cryptosystem~\cite{teske06} based on the GHS
    attack~\cite{gaudry+hess+smart02,GHS}. This work will give answers
    on the feasibility and the security of the system.
  \end{itemize}
\end{frame}

%%

{
  \setbeamertemplate{navigation symbols}{}

\begin{frame}
  \frametitle{Perspectives on point-counting}
  
  \begin{itemize}
  \item Computing isogenies is the key to efficient point-counting
    using the Schoof-Elkies-Atkin algorithm~\cite{schoof95};
    algorithms for elliptic curves do not generalise well to Jacobians
    of hyperelliptic curves or more general abelian varieties.
  \item Some progress has recently been achieved through the use of
    theta
    functions~\cite{faugere+lubicz+robert10,lubicz+robert10,robert},
    although these results do not permit us to improve Schoof-like
    point-counting methods such as the Schoof-Pila
    algorithm~\cite{pila90}.
  \item I plan to adapt the known methods to compute isogenies between
    elliptic curves to the setting of theta coordinates. This is
    certainly doable, but it is not clear that it could improve the
    complexity of the Schoof-Pila algorithm~\cite{pila90}, even for
    genus $2$ curves. The fundamental part of this research will be on
    improving the complexity of such algorithms, so that they can be
    profitably applied to point counting.
  \item More broadly, I am interested in point counting methods for
    abelian varieties, both $\ell$-adic, like Schoof's algorithm, and
    $p$-adic, like the
    AGM/Satoh~\cite{satoh00,fouquet+gaudry+harley00,mestre00,gaudry02,mestre02,lercier+lubicz06},
    Kedlaya's~\cite{kedlaya01,kedlaya04,denef+vercauteren06,castryck+denef+vercauteren07,harvey07}
    and Lauder's~\cite{lauder04,lauder+wan06} algorithms. Working in
    the number theory group of the Mathematical Institute would be an
    excellent opportunity to study these problems, and I will
    enthusiastically participate to the group's research.
  \end{itemize}

\end{frame}
}
%%

\begin{frame}[allowframebreaks]
  \frametitle{References}
  
  \setbeamertemplate{bibliography item}[text]
  \bibliographystyle{plain}
  \bibliography{../defeo}
\end{frame}

\end{document}


% Local Variables:
% mode:flyspell
% ispell-local-dictionary:"british"
% mode:TeX-PDF
% End:
%

% LocalWords:  Isogeny abelian isogenies hyperelliptic supersingular Frobenius
% LocalWords:  isogenous
