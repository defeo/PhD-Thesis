We let $I$ be a zero-dimensional ideal of $\K[\lst{x}]$ and
$\algeb{A}=\K[\lst{x}]/I$.  To simplify the exposition, from now on we
assume that $I$ is radical, this is equivalent to all the points of
$V(I)=\{a\in\clot{\K}^n|f(a)=0, \forall f\in I\}$ being simple. We
address the reader interested in the case of arbitrary multiplicity to
\cite{mourrain+elkadi}.

To better understand the structure of $\algeb{A}$ it will be important
to study the variety $V(I)$. We denote by $\clot{I}$ the ideal of
$\clot{\K}[\lst{x}]$ generated by $I$ and by $\clot{\algeb{A}}$ the
quotient ring $\clot{\K}[\lst{x}]/\clot{I}$.

\begin{lemma}
  $\clot{\algeb{A}}$ is a finite dimensional $\clot{\K}$-vector
  space. Its dimension is the same as $\algeb{A}$'s dimension as
  $\K$-vector space.
\end{lemma}

In what follows, we suppose that $V(I)$ has cardinality $d$ and we
denote its points by $\zeta_i\in\clot{\K}^n$ for $1\le i\le d$.

\begin{proposition}
  The number of points of $V(I)$ equals the dimension of
  $\clot{\algeb{A}}$ as vector space.
\end{proposition}
\begin{proof}
  Since $\clot{I}$ is radical and zero-dimensional, its primary
  decomposition is
  \begin{equation}
    \label{eq:1}
    \clot{I}=Q_1\cap \cdots \cap Q_d
    \text{,}
  \end{equation}
  where $Q_i$ is the ideal vanishing on $\zeta_i$.
  
  The $Q_i$'s are maximal and pairwise coprime
  (i.e. $Q_i+Q_j=\clot{\K}[\lst{x}]$ whenever $i\ne j$), hence, by the
  Chinese remainder theorem,
  \begin{equation}
    \label{eq:3}
    \clot{\K}[\lst{x}]/Q_1\cap\cdots\cap Q_d \isom \bigoplus_{i=1}^d\clot{\K}[\lst{x}]/Q_i
    \text{.}
  \end{equation}

  But $Q_i$ is maximal, hence $\clot{\K}[\lst{x}]/Q_i$ is an algebraic
  field extension of $\clot{\K}$. Since $\clot{\K}$ is algebraically
  closed, $\clot{\K}[\lst{x}]/Q_i=\clot{\K}$, and $\clot{\algeb{A}}$ has
  dimension $d$ as expected.
\end{proof}


\begin{example}
  Consider the ideal $I=(y-3,x^2-y)$ of $Q[x,y]$. This ideal is prime
  and its variety has no $\Q$-rational points. Since $G=\{y-3,x^2-y\}$
  is a Gröbner basis for $I$ (for grevlex), elements of
  $\algeb{A}=Q[x,y]/I$ are uniquely represented by their normal form
  modulo $G$; for example
  \[x^5y + 3xy + 1 \equiv 36x + 1 \mod I\text{.}\] By analyzing the
  leading monomials of $G$, it is straightforward to realize that all
  normal forms modulo $G$ have degree at most $1$ in $x$ and degree
  $0$ in $y$, thus $\algeb{A}$ has dimension $2$ as vector space.

  Indeed, the variety $V(I)$ consists of two points:
  \[V(I)=\left\{(\sqrt{3},3), (-\sqrt{3},3)\right\}\subset\clot{\Q}^2\text{.}\]
  Hence, $\clot{I}=(x-\sqrt{3},y-3)\cap(x+\sqrt{3},y-3)$ and
  \[\clot{\algeb{A}}\isom \clot{\Q}/(x-\sqrt{3},y-3) \oplus
  \clot{\Q}/(x+\sqrt{3},y-3)\text{.}\] In particular, the element
  $36x+1$ of $\clot{\algeb{A}}$ is mapped to
  \[(1+36\sqrt{3},1-36\sqrt{3})\] by this isomorphism. The reader will
  have noticed that $\algeb{A}$ is isomorphic to $\Q(\sqrt{3})$ as a ring.
\end{example}

We set 
\begin{equation}
  \label{eq:4}
  \clot{\algeb{A}}_i\eqdef \clot{\K}[\lst{x}]/Q_i
  \text{,}
\end{equation}
then by Eq. \eqref{eq:3} 
\begin{equation}
  \label{eq:5}
  \clot{\algeb{A}} = \bigoplus_{i=1}^d\clot{\algeb{A}_i}
  \text{.}
\end{equation}

Now, the $\clot{\algeb{A}}_i$'s are subalgebras of $\clot{\algeb{A}}$
isomorphic to $\clot{\K}$. We denote by $\basis{e}_i$ the unit element
of $\clot{\algeb{A}}_i$, then
\begin{equation}
  \label{eq:6}
  \begin{aligned}
    \basis{e}_i^2 &= \basis{e}_i\text{,}\\
    \basis{e}_i\basis{e}_j &= 0\text{.}
  \end{aligned}
\end{equation}
Hence $(\basis{e}_1,\ldots,\basis{e}_d)$ is a basis of
$\clot{\algeb{A}}$ made of orthogonal idempotents.

\begin{example}
  Continuing the example, 
  \[\clot{\algeb{A}}_1 = \clot{\Q}/(x-\sqrt{3},y-3)
  \quad\text{and}\quad
  \clot{\algeb{A}}_1 = \clot{\Q}/(x+\sqrt{3},y-3)
  \text{.}\]
  The idempotents are given by
  \[\basis{e}_1 = (3+\sqrt{3}x)/6
  \quad\text{and}\quad \basis{e}_1 = (3-\sqrt{3}x)/6 \text{.}\] The
  verification of Eq. \eqref{eq:6} is straightforward. In particular
  \[36x + 1 = (1+36\sqrt{3})\basis{e}_1 + (1-36\sqrt{3})\basis{e}_2
  \text{.}\]
\end{example}


The basis $(\basis{e}_1,\ldots,\basis{e}_d)$ is a very practical one
to represent elements of $\clot{\algeb{A}}$. Unfortunately, in the
general case the idempotents $\basis{e}_i$ may not be elements of
$\algeb{A}$, as the previous example shows; thus, using such basis
comes at the cost of lifting coefficients in $\clot{\K}$. In order to
find a basis better suited to represent elements of $\algeb{A}$, we
shall study the dual of the algebra $\clot{\algeb{A}}$.


\section{Dual}
\label{sec:dual}
We shall denote by $\dual{\algeb{A}}$ the dual space of $\algeb{A}$,
that is the space of $\K$-linear forms on $\algeb{A}$. Similarly, we
shall denote by $\dual{\clot{\algeb{A}}}$ the dual space of
$\clot{\algeb{A}}$.

For any $f\in\clot{\K}[\lst{x}]$, the evaluation $f(\zeta_i)$ of $f$
at $\zeta_i\in V(I)$ only depends on the class of $f$ in
$\algeb{\clot{A}}$, thus there is no ambiguity in writing $a(\zeta_i)$
for $a\in\clot{\algeb{A}}$. The map
\begin{equation}
  \label{eq:8}
  \begin{aligned}
  \basis{1}_{\zeta_i} : \clot{\algeb{A}} &\ra \clot{\K}\\
  a &\mapsto a(\zeta_i)
  \end{aligned}
\end{equation}
is linear; in particular
\begin{equation}
  \label{eq:9}
  \basis{1}_{\zeta_i}(\basis{e}_j) =
  \begin{cases}
    1 &\text{if $i=j$,}\\
    0 &\text{if $i\ne j$.}
  \end{cases}
\end{equation}
Hence $(\basis{1}_{\zeta_1},\ldots,\basis{1}_{\zeta_d})$ is the basis
of $\dual{\clot{\algeb{A}}}$ dual to $(\basis{e}_1,\ldots,\basis{e}_d)$.

The space $\dual{\algeb{A}}$ has a natural $\algeb{A}$-algebra
structure under the law
$\cdot:\algeb{A}\times\dual{\algeb{A}}\ra\dual{\algeb{A}}$ defined by
\begin{equation}
  \label{eq:10}
  \begin{aligned}
    a\cdot\ell : \algeb{A} &\ra \K\\
    b &\mapsto \ell(ab)
    \text{.}
  \end{aligned}
\end{equation}
Similarly $\dual{\clot{\algeb{A}}}$ has an $\clot{\algeb{A}}$-algebra
structure under an analogous law.

\begin{proposition}
  $\dual{\clot{\algeb{A}}}$ and $\clot{\algeb{A}}$ are isomorphic as
  $\clot{\algeb{A}}$-algebras under the mapping
  $\varepsilon:\basis{e}_i\mapsto\basis{1}_{\zeta_i}$ for $1\le i\le d$.
\end{proposition}
\begin{proof}
  The mapping is clearly a vector space isomorphism, we only need to
  prove that it is a morphism of $\clot{\algeb{A}}$-algebras. We want
  to prove that for any $a,b\in\clot{\algeb{A}}$
  \[\varepsilon(ab) = a\cdot\varepsilon(b)\text{;}\]
  it suffices to prove this on the elements of the bases
  $(\basis{e}_1,\ldots,\basis{e}_d)$ and
  $(\basis{1}_{\zeta_1},\ldots,\basis{1}_{\zeta_d})$.

  On one hand
  \begin{equation}
    \label{eq:12}
    \varepsilon(\basis{e}_i\basis{e}_j)=
    \begin{cases}
      \varepsilon(0)=0 &\text{if $i\ne j$,}\\
      \varepsilon(\basis{e}_i)=\basis{1}_{\zeta_i} &\text{if $i=j$.}
    \end{cases}
  \end{equation}
  On the other hand, $\basis{e_i}\cdot\basis{1_{\zeta_j}}$ is the form
  that associates to any $c\in\clot{\algeb{A}}$ the element
  \begin{equation}
    \label{eq:13}
    (\basis{e_i}c)(\zeta_j) = \basis{e}_i(\zeta_j)c(\zeta_j) = 
    \begin{cases}
      0 &\text{if $i\ne j$,}\\
      c(\zeta_j) &\text{if $i=j$,}
    \end{cases}
  \end{equation}
  where the last equality comes from \eqref{eq:9}. Hence
  \begin{equation}
    \label{eq:14}
    \basis{e}_i\cdot\basis{1}_{\zeta_j}=
    \begin{cases}
      0 &\text{if $i\ne j$,}\\
      \basis{1}_{\zeta_i} &\text{if $i=j$.}
    \end{cases}
  \end{equation}
\end{proof}

\begin{note}
  We have thus identified $\dual{\clot{\algeb{A}}}$ to
  $\clot{\algeb{A}}$ as $\clot{\algeb{A}}$-algebras, this implies that
  $\clot{\algeb{A}}$ is a Gorenstein algebra \cite[Chapter
  8]{mourrain+elkadi}. The theory of Gorenstein algebras is much
  deeper than the exposition we give here, and giving a complete
  account of it would be beyond the scope of this
  document. Nevertheless, we will eventually point out the
  relationships between the results proven here and the general
  theory.
\end{note}


% Local Variables:
% mode:flyspell
% ispell-local-dictionary:"american"
% mode:TeX-PDF
% mode:reftex
% TeX-master: "../these"
% End:
