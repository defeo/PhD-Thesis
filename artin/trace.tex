We let $I$ be a zero-dimensional ideal of $\K[\lst{x}]$ and
$\algeb{A}=\K[\lst{x}]/I$.  To simplify the exposition, from now on we
assume that $I$ is radical, this is equivalent to all the points of
$V(I)=\{a\in\clot{\K}^n|f(a)=0, \forall f\in I\}$ being simple. We
address the reader interested in the case of arbitrary multiplicity to
\cite{mourrain+elkadi}.

To better understand the structure of $\algeb{A}$ it will be important
to study the variety $V(I)$. We denote by $\clot{I}$ the ideal of
$\clot{\K}[\lst{x}]$ generated by $I$ and by $\clot{\algeb{A}}$ the
quotient ring $\clot{\K}[\lst{x}]/\clot{I}$.

\begin{lemma}
  $\clot{\algeb{A}}$ is a finite dimensional $\clot{\K}$-vector
  space. Its dimension is the same as $\algeb{A}$'s dimension as
  $\K$-vector space.
\end{lemma}

In what follows, we suppose that $V(I)$ has cardinality $d$ and we
denote its points by $\zeta_i\in\clot{\K}^n$ for $1\le i\le d$.

\begin{proposition}
  The number of points of $V(I)$ equals the dimension of
  $\clot{\algeb{A}}$ as vector space.
\end{proposition}
\begin{proof}
  Since $\clot{I}$ is radical and zero-dimensional, its primary
  decomposition is
  \begin{equation}
    \label{eq:1}
    \clot{I}=Q_1\cap \cdots \cap Q_d
    \text{,}
  \end{equation}
  where $Q_i$ is the ideal vanishing on $\zeta_i$.
  
  The $Q_i$'s are maximal and pairwise coprime
  (i.e. $Q_i+Q_j=\clot{\K}[\lst{x}]$ whenever $i\ne j$), hence, by the
  Chinese remainder theorem,
  \begin{equation}
    \label{eq:3}
    \clot{\K}[\lst{x}]/Q_1\cap\cdots\cap Q_d \isom \bigoplus_{i=1}^d\clot{\K}[\lst{x}]/Q_i
    \text{.}
  \end{equation}

  But $Q_i$ is maximal, hence $\clot{\K}[\lst{x}]/Q_i$ is an algebraic
  field extension of $\clot{\K}$. Since $\clot{\K}$ is algebraically
  closed, $\clot{\K}[\lst{x}]/Q_i=\clot{\K}$, and $\clot{\algeb{A}}$ has
  dimension $d$ as expected.
\end{proof}


\begin{example}
  Consider the ideal $I=(y-3,x^2-y)$ of $Q[x,y]$. This ideal is prime
  and its variety has no $\Q$-rational points. Since $G=\{y-3,x^2-y\}$
  is a Gröbner basis for $I$ (for grevlex), elements of
  $\algeb{A}=Q[x,y]/I$ are uniquely represented by their normal form
  modulo $G$; for example
  \[x^5y + 3xy + 1 \equiv 36x + 1 \mod I\text{.}\] By analyzing the
  leading monomials of $G$, it is straightforward to realize that all
  normal forms modulo $G$ have degree at most $1$ in $x$ and degree
  $0$ in $y$, thus $\algeb{A}$ has dimension $2$ as vector space.

  Indeed, the variety $V(I)$ consists of two points:
  \[V(I)=\left\{(\sqrt{3},3), (-\sqrt{3},3)\right\}\subset\clot{\Q}^2\text{.}\]
  Hence, $\clot{I}=(x-\sqrt{3},y-3)\cap(x+\sqrt{3},y-3)$ and
  \[\clot{\algeb{A}}\isom \clot{\Q}/(x-\sqrt{3},y-3) \oplus
  \clot{\Q}/(x+\sqrt{3},y-3)\text{.}\] In particular, the element
  $36x+1$ of $\clot{\algeb{A}}$ is mapped to
  \[(1+36\sqrt{3},1-36\sqrt{3})\] by this isomorphism. The reader will
  have noticed that $\algeb{A}$ is isomorphic to $\Q(\sqrt{3})$ as a ring.
\end{example}

We set 
\begin{equation}
  \label{eq:4}
  \clot{\algeb{A}}_i\eqdef \clot{\K}[\lst{x}]/Q_i
  \text{,}
\end{equation}
then by Eq. \eqref{eq:3} 
\begin{equation}
  \label{eq:5}
  \clot{\algeb{A}} = \bigoplus_{i=1}^d\clot{\algeb{A}_i}
  \text{.}
\end{equation}

Now, the $\clot{\algeb{A}}_i$'s are subalgebras of $\clot{\algeb{A}}$
isomorphic to $\clot{\K}$. We denote by $\basis{e}_i$ the unit element
of $\clot{\algeb{A}}_i$, then
\begin{equation}
  \label{eq:6}
  \begin{aligned}
    \basis{e}_i^2 &= \basis{e}_i\text{,}\\
    \basis{e}_i\basis{e}_j &= 0\text{.}
  \end{aligned}
\end{equation}
Hence $(\basis{e}_1,\ldots,\basis{e}_d)$ is a basis of
$\clot{\algeb{A}}$ made of orthogonal idempotents.

\begin{example}
  \label{ex:trace}
  Continuing the previous example, 
  \[\clot{\algeb{A}}_1 = \clot{\Q}/(x-\sqrt{3},y-3)
  \quad\text{and}\quad
  \clot{\algeb{A}}_2 = \clot{\Q}/(x+\sqrt{3},y-3)
  \text{.}\]
  The idempotents are given by
  \[\basis{e}_1 = (3+\sqrt{3}x)/6
  \quad\text{and}\quad \basis{e}_2 = (3-\sqrt{3}x)/6 \text{.}\] The
  verification of Eq. \eqref{eq:6} is straightforward. In particular
  \[36x + 1 = (1+36\sqrt{3})\basis{e}_1 + (1-36\sqrt{3})\basis{e}_2
  \text{.}\]
\end{example}

For any $f\in\clot{\K}[\lst{x}]$, we denote by $f(\zeta_i)$ the
evaluation of $f$ at $\zeta_i\in V(I)$. $f(\zeta_i)$ only depends on
the class of $f$ in $\algeb{\clot{A}}$, thus for
$a\in\clot{\algeb{A}}$, we define $a(\zeta_i)$ as the evaluation at
$\zeta_i$ of an arbitrary representative of the class $a$.

For any $a\in\clot{\algeb{A}}$, its class in $\clot{\algeb{A}}_i$ is $a(\zeta_i)$,
by Eq. \eqref{eq:4}. Hence
\begin{equation}
  \label{eq:2}
  a = \sum_{i=1}^da(\zeta_i)\basis{e}_i
  \text{.}
\end{equation}

The basis $(\basis{e}_1,\ldots,\basis{e}_d)$ is a very practical one
to represent elements of $\clot{\algeb{A}}$. Unfortunately, in the
general case the idempotents $\basis{e}_i$ may not be elements of
$\algeb{A}$, as the previous example shows; thus, using such basis
comes at the cost of lifting coefficients in $\clot{\K}$. In order to
find a basis better suited to represent elements of $\algeb{A}$, we
shall study the dual of the algebra $\clot{\algeb{A}}$.


\section{Duality}
\label{sec:dual}
We shall denote by $\dual{\algeb{A}}$ the dual space of $\algeb{A}$,
that is the space of $\K$-linear forms on $\algeb{A}$. Similarly, we
shall denote by $\dual{\clot{\algeb{A}}}$ the dual space of
$\clot{\algeb{A}}$.

The map
\begin{equation}
  \label{eq:8}
  \begin{aligned}
  \basis{1}_{\zeta_i} : \clot{\algeb{A}} &\ra \clot{\K}\\
  a &\mapsto a(\zeta_i)
  \end{aligned}
\end{equation}
is linear; in particular
\begin{equation}
  \label{eq:9}
  \basis{1}_{\zeta_i}(\basis{e}_j) =
  \begin{cases}
    1 &\text{if $i=j$,}\\
    0 &\text{if $i\ne j$.}
  \end{cases}
\end{equation}
Hence $(\basis{1}_{\zeta_1},\ldots,\basis{1}_{\zeta_d})$ is the basis
of $\dual{\clot{\algeb{A}}}$ dual to $(\basis{e}_1,\ldots,\basis{e}_d)$.

The space $\dual{\algeb{A}}$ has a natural $\algeb{A}$-module
structure under the law
$\cdot:\algeb{A}\times\dual{\algeb{A}}\ra\dual{\algeb{A}}$ defined by
\begin{equation}
  \label{eq:10}
  \begin{aligned}
    a\cdot\ell : \algeb{A} &\ra \K\\
    b &\mapsto \ell(ab)
    \text{.}
  \end{aligned}
\end{equation}
Similarly $\dual{\clot{\algeb{A}}}$ has an $\clot{\algeb{A}}$-module
structure under an analogous law.

\begin{proposition}
  \label{th:gorenstein}
  $\dual{\clot{\algeb{A}}}$ and $\clot{\algeb{A}}$ are isomorphic as
  $\clot{\algeb{A}}$-modules under the mapping
  $\rho:\basis{e}_i\mapsto\basis{1}_{\zeta_i}$ for $1\le i\le d$.
\end{proposition}
\begin{proof}
  The mapping is clearly a vector space isomorphism, we only need to
  prove that it is a morphism of $\clot{\algeb{A}}$-modules. We want
  to prove that for any $a,b\in\clot{\algeb{A}}$
  \[\rho(ab) = a\cdot\rho(b)\text{.}\]
  It suffices to prove this on the elements of the bases
  $(\basis{e}_1,\ldots,\basis{e}_d)$ and
  $(\basis{1}_{\zeta_1},\ldots,\basis{1}_{\zeta_d})$.

  On one hand
  \begin{equation}
    \label{eq:12}
    \rho(\basis{e}_i\basis{e}_j)=
    \begin{cases}
      \rho(0)=0 &\text{if $i\ne j$,}\\
      \rho(\basis{e}_i)=\basis{1}_{\zeta_i} &\text{if $i=j$.}
    \end{cases}
  \end{equation}
  On the other hand, $\basis{e_i}\cdot\basis{1_{\zeta_j}}$ is the form
  that associates to any $c\in\clot{\algeb{A}}$ the element
  \begin{equation}
    \label{eq:13}
    (\basis{e_i}c)(\zeta_j) = \basis{e}_i(\zeta_j)c(\zeta_j) = 
    \begin{cases}
      0 &\text{if $i\ne j$,}\\
      c(\zeta_j) &\text{if $i=j$,}
    \end{cases}
  \end{equation}
  where the last equality comes from \eqref{eq:9}. Hence
  \begin{equation}
    \label{eq:14}
    \basis{e}_i\cdot\basis{1}_{\zeta_j}=
    \begin{cases}
      0 &\text{if $i\ne j$,}\\
      \basis{1}_{\zeta_i} &\text{if $i=j$.}
    \end{cases}
  \end{equation}
\end{proof}

\begin{nota}
  We have thus identified $\dual{\clot{\algeb{A}}}$ to
  $\clot{\algeb{A}}$ as $\clot{\algeb{A}}$-modules, this implies that
  $\clot{\algeb{A}}$ is a Gorenstein algebra \cite[Chapter
  8]{mourrain+elkadi}. The theory of Gorenstein algebras is much
  deeper than the exposition we give here, and giving a complete
  account of it would be beyond the scope of this
  document. Nevertheless, we will eventually point out the
  relationships between the results proven here and the general
  theory.
\end{nota}

Since $1$ generates $\clot{\algeb{A}}$ as an $\clot{\algeb{A}}$-module, the form
\begin{equation}
  \label{eq:7}
  \Tr \eqdef \rho(1) = \sum_i\basis{1}_{\zeta_i}
\end{equation}
generates $\dual{\clot{\algeb{A}}}$ as an $\clot{\algeb{A}}$-module.
$\rho(1)$ will play an important role in the sequel; it is called the
\emph{trace form}, the reason for this will be clear in the next
section.

The bilinear form on $\dual{\clot{\algeb{A}}}\times\clot{\algeb{A}}$
defined by
\begin{equation}
  \label{eq:11}
  \braket{\ell}{a} = \ell(a)
\end{equation}
is non-singular by definition (see Section
\ref{sec:linear-algebra:duality}). By means of the isomorphism $\rho$,
we can transport this to a bilinear form on
$\clot{\algeb{A}}\times\clot{\algeb{A}}$: we define
\begin{equation}
  \label{eq:15}
  \braket{a}{b}=\rho(a)(b)
  \text{.}
\end{equation}
By Proposition \ref{th:gorenstein}, by Eq. \eqref{eq:10} and by the
equality
\begin{equation}
  \label{eq:16}
  \rho(a)(b) = \sum_i a(\zeta_i)b(\zeta_i)
  \text{,}
\end{equation}
we deduce that
\begin{equation}
  \label{eq:17}
  \braket{a}{b} = \rho(a)(b) = ab\cdot\Tr(1) = a\cdot\Tr(b) = \Tr(ab) = \braket{b}{a}
  \text{.}
\end{equation}
is a non-singular form on $\clot{\algeb{A}}\times\clot{\algeb{A}}$
that identifies $\clot{\algeb{A}}$ to its dual.

In particular, from Eqs.~\eqref{eq:17} and~\eqref{eq:2} we deduce the
\emph{trace formulas} or \emph{interpolation formulas}:
\begin{equation}
  \label{eq:21}
  a = \sum_{i=1}^d\braket{a}{\basis{e}_i}\basis{e}_i=\sum_{i=1}^da(\zeta_i)\basis{e}_i=\sum_{i=1}^da\basis{e_i}
\end{equation}


\section{Sitckelberger's theorem}
\label{sec:multiplication}

Let $a\in\clot{\algeb{A}}$ and consider the linear map
\begin{equation}
  \label{eq:18}
  M_a:a \mapsto ab
  \text{.}
\end{equation}

\begin{theorem}[Stickelberger]
  \label{th:stickelberger}
  The element $\basis{e_i}$ is an eigenvector of $M_a$ associated to
  the eigenvalue $a(\zeta_i)$. The characteristic polynomial of $M_a$
  is
  \[\prod_{i=1}^d(X-a(\zeta_i))\text{.}\]
\end{theorem}
\begin{proof}
  Using Eqs.~\eqref{eq:21} and~\eqref{eq:6}, we have
  \begin{equation}
    \label{eq:19}
    M_a(\basis{e}_i) = a\basis{e}_i = \braket{a}{\basis{e}_i}\basis{e}_i=a(\zeta_i)\basis{e}_i
    \text{.}
  \end{equation}

  Since the $\basis{e}_i$'s form a basis of $\clot{\algeb{A}}$ as a
  vector space, $M_a$ is diagonalizable and its eigenvalues are the
  $a(\zeta_i)$'s, each counted once.
\end{proof}

\begin{definition}[Trace, norm]
  \label{def:trace}
  We define the \emph{trace} of $a$ as
  \[\Tr(a) = \Tr(M_a)\]
  and its \emph{norm} as
  \[\Norm(a) = \det(M_a)\text{.}\]
\end{definition}
Then, the following corollary is easily derived.

\begin{corollary}
  \label{th:stickelberger-trace-det}
  One has
  \begin{align}
    \label{eq:23}
    \Tr(a) &= \sum_{i=1}^da(\zeta_i)\\
    \label{eq:24}
    \Norm(a) &= \prod_{i=1}^da(\zeta_i)
  \end{align}
\end{corollary}

By Eqs.~\eqref{eq:23} and~\eqref{eq:7}, it is clear that
$\Tr(a)=\rho(1)(a)$, which justifies the notation we employed in the
last section. 

\begin{theorem}
  $\dual{\algeb{A}}$ is isomorphic to $\algeb{A}$ as
  $\algeb{A}$-module under the restriction of $\rho$ to $\algeb{A}$.
\end{theorem}
\begin{proof}
  When $a,b\in\algeb{A}$, the characteristic polynomial of $M_{ab}$
  has coefficients in $\K$. Thus $\braket{a}{b}=\Tr(ab)$ is in $\K$,
  and the restriction of $\rho(a)$ to $\algeb{A}$ is in
  $\dual{\algeb{A}}$.

  Since $a\ne a'$ implies $\rho(a)\ne\rho(a')$, How ???
\end{proof}


\section{Rational Univariate Representation}
\label{sec:rati-univ-repr}
In many circumstances it is useful to switch from a multivariate
representation of the elements of $\algeb{A}$ to an univariate one. A
\emph{rational univariate representation} \cite{rouiller99}, sometimes
also called \emph{geometric resolution} \cite{giusti+lecerf+salvy01},
of $\K[x_1,\ldots,x_n]/I$ consists in expressing the variety $V(I)$ as
the solution of the system
\begin{equation}
  \label{eq:22}
  \begin{aligned}
    f(t) &= 0\text{,}\\
    x_1 &= \frac{g_1(t)}{g(t)}\text{,}\\
    &\vdots\\
    x_n &= \frac{g_n(t)}{g(t)}\text{,}    
  \end{aligned}
\end{equation}
where $t$ is a new variable and $f,g,g_1,\ldots,g_n$ are univariate
polynomials with coefficients in $\K$.

\begin{definition}[Separating element]
  An element $t\in\algeb{A}$ is said to be \emph{separating} if for
  any $\zeta,\zeta'\in V(I)$
  \[\zeta\ne\zeta'\Rightarrow t(\zeta)\ne t(\zeta')\text{.}\]
\end{definition}

Separating elements always exist, provided $\algeb{A}$ is large
enough. We don't give here any proof of this fact because in the
applications we have in mind a separating element is always at hand.

\begin{proposition}
  Let $t$ be a separating element, then $1,t,\ldots,t^{d-1}$ are
  $\clot{\K}$-linearly independent.
\end{proposition}
\begin{proof}
  Let $\sum_{i=0}^{d-1}a_it^i =0$, then the polynomial
  $\sum_{i=0}^{d-1}a_iT^i$ has $d$ roots in $\clot{\K}$, namely
  $t(\zeta_i)$ for $1\le i \le d$, hence it is identically null.
\end{proof}

\begin{lemma}
  \label{th:multi-newton-sums}
  Let $t$ be a separating element of $\algeb{A}$ and let $Q$ be its
  minimal polynomial. Let $T$ be a fresh variable, then
  \begin{equation}
    \label{eq:25}
    \sum_{i\ge0} \frac{\braket{1}{t^{i}}}{T^{i+1}} = \frac{Q'(T)}{Q(T)}
    \text{.}
  \end{equation}
\end{lemma}
\begin{proof}
  By the trace formulas~\eqref{eq:21}
  \begin{equation}
    \label{eq:26}
    t^i = \sum_{j=1}^d\braket{t^i}{\basis{e}_j}\basis{e}_j
    \text{,}
  \end{equation}
  hence
  \begin{equation}
    \label{eq:27}
    \sum_{i\ge0}\frac{\braket{1}{t^i}}{T^{i+1}} =
    \sum_{i\ge0}\sum_{j=1}^d\frac{\braket{1}{\basis{e}_j}\braket{t^i}{\basis{e}_j}}{T^{i+1}} =
    \sum_{i\ge0}\sum_{j=1}^d\frac{t(\zeta_j)^i}{T^{i+1}}
    \text{.}
  \end{equation}
  Swapping the sums, this equals
  \begin{equation}
    \label{eq:28}
    \sum_{j=1}^d\frac{1}{T-t(\zeta_j)} =
    \frac{\sum_{j=1}^d\prod_{j'\ne j}(T-t(\zeta_j))}{\prod_{j=1}^d(T-t(\zeta_j))} =
    \frac{Q'(T)}{Q(T)}
    \text{,}
  \end{equation}
  where the last equality comes from Theorem \ref{th:stickelberger}.
\end{proof}

\begin{remark}
  The polynomial $Q$ can be recovered from its logarithmic derivative
  via the formula
  \begin{equation}
    \label{eq:30}
    Q = \exp\left(\int \frac{Q'}{Q}\right)
    \text{.}
  \end{equation}
  

  When the degree $d$ is known in advance, this suggests an efficient
  algorithm to compute $Q$, provided the characteristic of $\K$ is
  larger than $d$. 

  We know that $\Tr(1)=d$, hence 
  \begin{equation}
    \label{eq:32}
    \frac{Q'}{Q} =
    \frac{d}{T} + \sum_{i\ge 1}\frac{\braket{1}{t^i}}{T^{i+1}} 
    \text{.}
  \end{equation}
  We deduce
  \begin{equation}
    \label{eq:33}
    Q = \exp\left(d\log T + \int\sum_{i\ge1}\frac{\braket{1}{t^i}}{T^{i+1}}\right) =
    T^d\exp\left(-\sum_{i\ge 1}\frac{\braket{1}{t^i}}{iT^i}\right)
    \text{,}
  \end{equation}
  then the power series on the right hand side can be exponentiated
  using a Newton iteration.  But $Q$ is a polynomial of degree $d$,
  hence we can truncate the exponent power series to the order
  $O(T^{-d-1})$.

  In conclusion, it is sufficient to know
  \begin{equation}
    \label{eq:34}
    \Tr(t),\ldots,\Tr(t^d)
  \end{equation}
  in order to compute $Q$.
\end{remark}

\begin{example}
  Continuing Example~\ref{ex:trace}, we want to compute the minimal
  polynomial of $t=36x+1$. We know that
  \[t=36x + 1 = (1+36\sqrt{3})\basis{e}_1 + (1-36\sqrt{3})\basis{e}_2
  \text{,}\] hence $t$ is separating. Its traces are easily computed
  from the previous expression:
  \begin{align*}
    \Tr(t) &= (1 + 36\sqrt{3}) + (1-36\sqrt{3}) = 2\text{,}\\
    \Tr(t^2) &= (1 + 36\sqrt{3})^2 + (1-36\sqrt{3})^2 = 7778\text{.}
  \end{align*}
  We compute the exponential:
  \begin{multline*}
    \exp\left(-\frac{2}{T}-\frac{3889}{T^2} + O(T^{-3})\right)=\\
    \left(1-\frac{2}{T}+\frac{4}{2!T^2}+O(T^{-3})\right)\left(1-\frac{3889}{T^2}+O(T^{-3})\right)=\\
    \left(1 - \frac{2}{T} - \frac{3887}{T^2} + O(T^{-3})\right)
    \text{,}
  \end{multline*}
  hence the minimal polynomial is
  \[T^2-2T-3887\text{.}\]
\end{example}

Thanks to Lemma \ref{th:multi-newton-sums}, we have a way to find the
first line of the representation in Eq.~\eqref{eq:22}, provided that
we know a separating element $t$. We now have to express
$x_1,\ldots,x_n$ as functions of the roots of the minimal polynomial of $t$.

\begin{theorem}
  \label{th:rur}
  Let $t$ be a separating element of $\algeb{A}$ and let $Q$ be its
  minimal polynomial. Let $a\in\algeb{A}$ and set
  \begin{equation}
    \label{eq:38}
    A(T) = Q(T)\sum_{i\ge0}\frac{\braket{a}{t^i}}{T^{i+1}}
    \text{.}
  \end{equation}
  Then $A(T)$ is a polynomial of degree less than $d$, and
  \begin{equation}
    \label{eq:39}
    a = \frac{A(t)}{Q'(t)}
    \text{.}
  \end{equation}
\end{theorem}
\begin{proof}
  We develop the series as in the proof of Lemma
  \ref{th:multi-newton-sums}:
  \begin{equation}
    \label{eq:40}
    \sum_{i\ge0}\frac{\braket{a}{t^i}}{T^{i+1}} =
    \sum_{j=1}^da(\zeta_j)\sum_{i\ge0}\frac{t(\zeta_j)^i}{T^{i+1}}=
    \frac{\sum_{j=1}^da(\zeta_j)\prod_{j'\ne j}(T-t(\zeta_j))}{Q(T)}
    \text{.}
  \end{equation}
  Hence $A(T)$ is a polynomial of degree less than $d$.

  Now we use the trace formulas to decompose 
  $A(t)$ and $Q'(t)$:
  \begin{align}
    \label{eq:41}
    \braket{A(t)}{\basis{e}_i} &=
    \sum_{j=1}^d a(\zeta_j)\prod_{j'\ne j}(\braket{t}{\basis{e}_i}-t(\zeta_j)) =
    a(\zeta_i)\prod_{j\ne i}(t(\zeta_i)-t(\zeta_j))
    \text{,}\\
    \braket{Q'(t)}{\basis{e}_i} &=
    \prod_{j\ne i}(t(\zeta_i)-t(\zeta_j))
    \text{.}
  \end{align}
  Because $t$ is separating, $\braket{Q(t)}{\basis{e}_i}\ne0$ for any
  $i$, hence $Q(t)$ is a unit of $\algeb{A}$. We deduce that
  \begin{equation}
    \label{eq:42}
    \braket{\frac{A(t)}{Q'(t)}}{\basis{e}_i} = a(\zeta_i)
  \end{equation}
  for any $i$. Hence, by the trace formulas
  \begin{equation}
    \label{eq:43}
    \frac{A(t)}{Q'(t)} = \sum_i\braket{\frac{A(t)}{Q'(t)}}{\basis{e}_i}\basis{e}_i =
    \sum_i a(\zeta_i)\basis{e}_i = a
    \text{.}
  \end{equation}
\end{proof}

By taking $a=x_i$, the theorem can be used to find a rational
univariate representation: it suffices to know
\begin{equation}
  \label{eq:44}
  \Tr(x_i),\Tr(x_it),\ldots,\Tr(x_it^{d-1})
\end{equation}
in order to deduce $g_i(T)$ as in the representation~\eqref{eq:22}.

\begin{example}
  We conclude the previous example. We want to find a parameterization
  of $x$ and $y$ with respect to $t=36x+1$. We already know the
  minimal polynomial of $t$:
  \[Q(T) = T^2-2T-3887\text{,}\]
  thus
  \[Q'(T) = 2T-2\text{.}\]
  Now
  \[x=\sqrt{3}\basis{e}_1 -\sqrt{3}\basis{e}_2 \qquad
  y = 3\basis{e}_1 + 3\basis{e}_2\text{,}\]
  hence 
  \begin{align*}
    \Tr(x) &= 0\text{,} & \Tr(xt) &= 216\text{,}\\
    \Tr(y) &= 6\text{,} & \Tr(yt) &= 6\text{.}
  \end{align*}
  We deduce that
  \[x=\frac{216}{2t-2}\text{,}\qquad
  y=\frac{6t-6}{2t-2}\text{.}\]
\end{example}


\section{The univariate case}
\label{sec:univariate-case}



% Local Variables:
% mode:flyspell
% ispell-local-dictionary:"american"
% mode:TeX-PDF
% mode:reftex
% TeX-master: "../these"
% End:
