\section{Level embedding}
\label{sec:level-embedding}

As stated in Section \ref{sec:arithmetics}, an element $v\wrt\U_i$ is
represented as a minimal degree representative of its residue class in
$\F_p[X_i]/Q_i(X_i)$. We also know that $\U_i$ can be seen as the
extension $\U_{i-1}[X_i]/\AS{P}_{i-1}(X_i)$ with $\AS{P}_{i-1}$ as in
Theorem \ref{th:cantor}. We then look for an algorithm, called
\alg{Push-down}, that takes a $v\wrt\U_i$ as input and gives as output
$v_0,\ldots,v_{p-1}\wrt\U_{i-1}$ such that
\begin{equation}
  \tag{\ref{eq:vectorspace}}
  v = v_0 + x_iv_1 + \cdots + x_i^{p-1}v_{p-1} \text{,}
\end{equation}

We also are interested in its inverse algorithm, called
\alg{Lift-up}. We will note respectively $\Push(i)$ and $\Lift(i)$ for
the complexities of the two algorithms in terms of $\F_p$-operations.

We will give the algorithms only for the case where $\AS{P}_{i-1} =
X^p - X - x_{i-1}^{2p-1}$, the other cases being easier and
straightforwardly derivable by this one. To ease the reading, since
our algorithms work only on a bivariate system, we will rename
variables and polynomials and we will set
\begin{gather*}
  Y=X_{i-1},\quad Z=X_i,\quad y=x_{i-1},\quad z=x_i \\
  P=Q_{i-1},\quad Q=Q_i,\quad \biv{R}=\AS{P}_{i-1}
  \text{.}
\end{gather*}
The structure of $\U_i/\U_{i-1}$ is thus described by the two
equivalent systems of equations
\begin{equation}
  \left\{
  \begin{aligned}
    \biv{R}(Z,Y) &= Z^p - Z - Y^{2p-1} = 0\\
    P(Y)         &= 0
  \end{aligned}
  \right.
  %%
  \qquad \leftrightarrow \qquad
  %%
  \left\{
  \begin{aligned}
    Q(Z)       &= 0\\
    Y - \pi(Z) &= 0
  \end{aligned}
  \right.    
  \;\text{.}
\end{equation}
Then one naturally associates an $\F_p$-basis to each system :
\begin{equation}
  \label{eq:bases}
  \begin{aligned}
    D &= (1,y,\ldots,y^{p^{i-1}d-1},z,zy,\ldots,z^{p-1}y^{p^{i-1}d-1})\text{,}\\
    U &= (1,z,\ldots,z^{p^id-1})\text{.}
  \end{aligned}
\end{equation}
So that \alg{Lift-up} just takes an element written over the basis $D$
and returns its representation over the basis $U$, while
\alg{Push-down} does the opposite.

Some aspects of this problem have already been treated in \cite{PS06}
for the more general case of extension rings. Our algorithms extend
and improve the ones in \cite{PS06}.

\subsection{Push-down}
\label{sec:level-embedding:push-down}
The \alg{Push-down} algorithm is really simple : given a polynomial in
the variable $Z$ it simply reduces it modulo $\biv{R}(Z,Y)$ to obtain
a bivariate polynomial $\biv{v}$ of degree at most $p$ in $Z$. In
order to have a good complexity, it uses a divide and conquer
strategy; the recursive algorithm doing this computation is called
\alg{Push-down-$\infty$}.

Then, each of the coefficients of $Z^i$ in $\biv{v}$ is reduced modulo
$P(Y)$ to obtain $p$ elements of $\U_{i-1}$. This is the algorithm
called \alg{Push-down}.

\begin{figure}
  \begin{algorithm}
    {Push-down-$\infty$}
    {$v(Z)\wrt\U_i$}
    {$v_0(T),\ldots,v_{p-1}(T)\wrt\F_p[T]$ s.t. $\sum_{j=0}^{p-1}v_j(y)
      z^j = v(z)$}
  \item If $\deg_Z(v) < p$, let $\biv{v}(Z,T) = v(Z)$.
  \item Else,
    \begin{enumerate}
    \item let $n$ s.t. $2^n < \deg_Z v \le 2^{n+1}$;
    \item let $\;a^{(0)} = v \bmod Z^{2^n}\;$ and $\;a^{(1)} = v \div Z^{2^n}$;
    \item \label{alg:push-down:rec1} compute
      $a^{(0)}_0,\ldots,a^{(0)}_{p-1} = $
      Push-down-$\infty$($a^{(0)}$);
    \item \label{alg:push-down:rec2} compute
      $a^{(1)}_0,\ldots,a^{(1)}_{p-1} = $
      Push-down-$\infty$($a^{(1)}$);
    \item \label{alg:push-down:rec3} compute $z_0,\ldots,z_{p-1} = $
    Push-down-$\infty$($Z^{2^n}$);
    \item let
      \begin{enumerate}
      \item $\biv{a}^{(0)}(Z,T) = \sum_{j=0}^{p-1}a^{(0)}_j(T)Z^j$,
      \item $\biv{a}^{(1)}(Z,T) = \sum_{j=0}^{p-1}a^{(1)}_j(T)Z^j$,
      \item $\biv{z}(Z,T) = \sum_{j=0}^{p-1}z_j(T)Z^j$;
      \end{enumerate}
    \item \label{alg:push-down:bivv} compute \[\biv{v}(Z,T) =
      \biv{a}^{(0)}(Z,T) + (\biv{a}^{(1)}\biv{z})(Z,T) \bmod
      \biv{R}(Z,T)\] by the following algorithm :
      \begin{enumerate}
      \item \label{alg:push-down:prod} compute $\biv{a}^{(1)}\biv{z}$
      by polynomial multiplication ;
      \item let $\;\biv{b}^{(0)} = \biv{a}^{(1)}\biv{z} \bmod Z^p\;$ and
	$\;\biv{b}^{(1)} = \biv{a}^{(1)}\biv{z} \div Z^p$
      \item \label{alg:push-down:sum} compute $\biv{v} =
	\biv{a}^{(0)} + \biv{b}^{(0)} + \biv{b}^{(1)}T^{2p-1} +
	\biv{b}^{(1)}Z$;
      \end{enumerate}
    \end{enumerate}
  \item return $\coeff_Z(\biv{v},j)$ for $j=0,\ldots,p-1$;
  \end{algorithm}
\end{figure}

\begin{figure}
  \begin{algorithm}
    {Push-down} {$v(Z)\wrt\U_i$} {$v_0(Y),\ldots,v_{p-1}(Y)\wrt\U_{i-1}$
      s.t. $\sum_{j=0}^{p-1}v_j(y) z^j = v(z)$}
  \item compute $v_0,\ldots,v_{p-1} = $ Push-down-$\infty$($v$);
  \item \label{alg:push-down:reduction} return $v_j(Y) \bmod P(Y)$ for
    $j = 0,\ldots,p-1$.
  \end{algorithm}
\end{figure}

\begin{theorem}
  The algorithm \alg{Push-down} is correct.
\end{theorem}
\begin{proof}
  To prove the correctness of \alg{Push-down} we just need to show
  that \alg{Push-down-$\infty$} correctly computes some
  representatives of $v_0,\ldots,v_p$ in $\F_p[Y]/P(Y)$, then step
  \algref{alg:push-down:reduction} of \alg{Push-down} insures us that
  the minimal degree representatives of $v_0,\ldots,v_p$ are returned.

  Let then $v\wrt\U_i$, if $\deg_Z(v) < p$ then $v_j = \coeff_Z(v,j)$
  for $j = 0,\ldots,p-1$, thus
  \[\sum_{j=0}^{p-1}v_j(y)z^j = \sum_{j=0}^{p-1}\coeff_Z(v,j)z^j = v(z)
  \text{.}\]
  
  Suppose now that $\deg_Z(v) \ge p$, by induction hypothesis
  \begin{equation}
    \label{eq:push-down:induction}
    \begin{aligned}
      \biv{a}^{(0)}(z, y) &= a^{(0)}(z) \text{,}\\
      \biv{a}^{(1)}(z, y) &= a^{(1)}(z) \text{,}\\
      \biv{z}(z, y)      &= z^{2^n} \text{.}
    \end{aligned}
  \end{equation}
  Let's put 
  \begin{equation}
    \label{eq:push-down:nomod}
    \biv{v}'(Z,T) = \biv{a}^{(0)}(Z,T) + (\biv{a}^{(1)}\biv{z})(Z,T)
  \end{equation}
  so that $\biv{v} = \biv{v}' \bmod \biv{R}$. Then, putting
  \eqref{eq:push-down:induction} and \eqref{eq:push-down:nomod}
  together, we have
  \[\biv{v}'(z,y) = a^{(0)}(z) + z^{2^n}a^{(1)}(z) = v(z) \text{;}\]
  but $\biv{R}(z,y) = 0$, hence 
  \[\sum_{j=0}^{p-1}v_j(y)z^j = \sum_{j=0}^{p-1}\coeff_Z(\biv{v},j)(y)z^j =
  \biv{v}(z,y) = \biv{v}'(z,y) = v(z) \text{.}\]

  To complete the proof, we just need to show that step
  \algref{alg:push-down:bivv} correctly computes $\biv{v}$. This is
  easily seen because $\deg_Z(\biv{a}^{(0)}) < p$ and
  $\deg_Z(\biv{a}^{(1)}\biv{z}) < 2p-1$, then $\deg_Z(\biv{b}^{(0)}) <
  p$ and $\deg_Z(\biv{b}^{(1)}) < p-1$, thus every member of the sum
  at step \algref{alg:push-down:sum} has $\deg_Z$ less than $p$ and
  it's easy to verify that the formula is correct because
  \[\biv{b}^{(1)}(Z,T)Z^P = \biv{b}^{(1)}(Z,T)(Z + T^{2p-1}) \mod \biv{R}(Z,T) 
  \text{.}\]
\end{proof}

Now we want to estimate the complexity class of $\Push(i)$, the cost
of applying \alg{Push-down} to an element $v\wrt\U_i$. For this we'll
need a lemma first.

\begin{lemma}
  \label{th:push-down-degree}
  If $2^{n-1} < \deg_Z(v) \le 2^n$ then each of the output polynomials
  of \alg{Push-down-$\infty$} has $\deg_T$ less than $2^{n+2} <
  8\deg_Z(v)$.\footnote{la vraie borne est plutôt quelque chose du
    type $(2-\frac{1}{p})\deg_Z(V)$, mais j'ai l'impression que cette
    preuve est plus courte et claire.}
\end{lemma}
\begin{proof}
  If $\deg_Z(v) < p$, than each of the outputs has degree zero and the
  thesis holds.

  Else, let $c=0$ if $p=2$ or $c = \lfloor\log_2p\rfloor$ elsewhere
  and let $x=2p+1$. Let $D(n)$ be a bound on $\deg_T$ when $\deg_Z(v)
  \le 2^n$, then obviously $D(c) = 0$.

  When $n > c$, steps \algref{alg:push-down:rec1} through
  \algref{alg:push-down:rec3} give polynomials with $\deg_T \le D(n-1)$,
  step \algref{alg:push-down:prod} combines those polynomials to obtain
  others with $\deg_T \le 2D(n-1)$ and finally step
  \algref{alg:push-down:sum} multiplies part of these last ones by
  $T^x$. The function $D(n)$ is thus seen to obey the following
  recurrence :
  \begin{equation*}
    \begin{cases}
      D(c) = 0\\
      D(n) = 2D(n-1) + x &\text{if $n>c$}
    \end{cases}
  \end{equation*}
  which is easily recognised as the sequence $D(n) = (2^{n-c}-1)x$.

  To conclude we just need to observe that
  \begin{equation*}
    (2^{n-c}-1)(2p-1) < \frac{2^{n+2}p}{2^{c+1}} =
    \frac{2^{n+2}p}{2^{\lceil\log_2p\rceil}} \le 2^{n+2}\text{.}
  \end{equation*}
\end{proof}

\begin{theorem}
  Given an input $v\wrt\U_i$, the algorithm \alg{Push-down} computes
  its output in $O\left(\Mult(p^{i+1}d)\log(p^id)\right)$ amortised
  $\F_p$-operations.
\end{theorem}
\begin{proof}
  If $\deg_Z(v) < p$, then the algorithm simply copies the
  coefficients of $v(Z)$, so its cost is $O(p)$ $\F_p$-operations.

  Let $c=0$ if $p=2$ or $c = \lfloor\log_2p\rfloor$ elsewhere and let
  $C(n)$ be a bound on the number of $\F_p$-operations needed to
  execute \alg{Push-down-$\infty$} on $v\wrt\U_i$ when $\deg_Z(v)\le
  2^n$. Then clearly $C(c) = O(p)$.

  If $n > c$, the three recursive calls \algref{alg:push-down:rec1}
  through \algref{alg:push-down:rec3} contribute $3C(n-1)$ operations
  to $C(n)$. We can reduce this cost by storing the values of
  $z_0,\ldots,z_{p-1}$ once they're are computed; the contribution
  then goes down to $2C(n-1)$ amortised operations.

  By lemma \ref{th:push-down-degree}
  \begin{equation}
    \label{eq:push-down:degT}
    \deg_T(\biv{a}^{(0)}), \deg_T(\biv{a}^{(1)}), \deg_T(\biv{z}_j) < e2^{n-1}
  \end{equation}
  for some constant $e$, and by construction
  \[\deg_Z(\biv{a}^{(0)}), \deg_Z(\biv{a}^{(1)}), \deg_Z(\biv{z}_j) < p\]
  so that step \algref{alg:push-down:prod} costs
  $O\left(\Mult(p2^{n-1})\right)$ operations by Kronecker substitution.
  
  By \eqref{eq:push-down:degT} we deduce
  \[\deg_Z(\biv{b}^{(0)}), \deg_Z(\biv{b}^{(1)}) < e2^n \text{,}\]
  step \algref{alg:push-down:sum} is then done by $p$ sums of
  polynomials in $\F_p[T]$ of degree at most $e2^n+2p-1 < e2^n+2^{c+1}
  \le (e+1)2^n$, for a total cost of $O(p2^n)$ $F_p$-operations.

  Putting things together, we have the following recurrence for
  $C(n)$:
  \begin{equation*}
    \begin{cases}
      C(c) = O(p)\\
      C(n) = 2C(n-1) + O(\Mult(p2^n)) &\text{if $n>c$}
    \end{cases}
  \end{equation*}
  from which we easily deduce
  \[C(n) = O\left(p2^{n-c} +
  \sum_{i=0}^{n-c-1} 2^i\Mult(p2^{n-i})\right)\text{,}\]
  then, neglecting the term $p2^{n-c}$ and using two times definition
  \ref{def:mult}, we conclude
  \[C(n) = O\left(\sum_{i=0}^{n-c-1} \Mult(p2^n)\right) =
  O\left((n-c)\Mult(p2^n)\right) \text{.}\]

  In general, if $v\wrt\U_i$ we know that $\deg_Z(v) < p^id$, so we
  can set $n = \lceil\log_2p^id-1\rceil$ and we get the bound
  $O\left(\Mult(p^{i+1}d)\log (p^id)\right)$ on the number of operations
  in \alg{Push-down-$\infty$}.

  Now, using again lemma \ref{th:push-down-degree}, we know that the
  output of \alg{Push-down-$\infty$} has $\deg_T$ at most $e2^n =
  O(p^id)$, so that step \algref{alg:push-down:reduction} of
  \alg{Push-down} costs $O(p\Mult(p^id))$. We conclude that $P(i) =
  O\left(\Mult(p^{i+1}d)\log (p^id)\right)$.
\end{proof}

\begin{remark}
  The algorithm requires the memoization of the values of
  \alg{Push-down-$\infty$($Z^{2^n}$)} for all the $n$ up to
  $\lceil\log_2p^id - 1\rceil - 1$ for every $\U_i$, this requires to
  store $O(p^id)$ elements of $\F_p$ for each level of the tower.

  A first optimisation is to store the results of
  \alg{Push-down($Z^{2^n}$)} instead, though this optimisation does
  not change the asymptotic number of elements one has to store.

  When $p=2$ all the memoization steps can be avoided thus yielding an
  algorithm whose complexity is not amortised. This is done by
  observing that
  \[Z^{p^n} = Z + \sum_{j=0}^n T^{(2p-1)p^j} \mod \biv{R}(Z,T)\]
  so that multiplication by $\biv{z}$ is done by $n$ polynomial
  shifts, followed by $n-1$ polynomial sums; the complexity of this
  computation stays the same as for the standard algorithm.
\end{remark}

\textbf{NB :} Il existe un algorithme qui, plutôt que couper en deux le
polynôme $v$, le coupe en $p$ parties et procède sans mémoization comme
notre variante pour $p=2$. Sa complexité devrait être plus ou moins la
même, le seul point difficile étant le pas \algref{alg:push-down:bivv}
qui demande de reduire un polynôme de degré $p^2$ modulo $\biv{R}$ et
qui risque de nous faire terminer avec une complexité
$O\left(\Mult(p^{i+2}di\log pd)\right)$ (je n'ai pas vérifié, mais
j'ai vraiment l'impression que ce soit ça). Est-ce que ça vaut le
coup d'en parler ?



\subsection{Lift-up}
\label{sec:level-embedding:lift-up}

Before giving an algorithm for the lift-up, we shall give some
preliminaries first.

\subsubsection{Linear forms}
Given a field $\K$ and a vector space $V$ over it, it is usual to
define the dual space $\dual{V}$ made of all the linear functions (or
linear forms) $V \rightarrow \K$. $\dual{V}$ is again a vector space
over $\K$ and its dual space is $V$. When $V$ is finite dimensional,
$\dual{V}$ has the same dimension, in fact, given any basis
$e_1,\ldots,e_n$ for $V$, the basis $e^1,\ldots,e^n$ defined by
\begin{equation*}
  e^i(e_j) = \begin{cases}
    1 &\text{if $i=j$,}\\
    0 &\text{otherwise}
  \end{cases}
\end{equation*}
is a basis for $\dual{V}$, called the \emph{dual basis} of
$e_1,\ldots,e_n$. Many other properties of interest are verified when
$V$ is a field and we're going to discuss them in the rest of this
Section; the more general case where $V$ is a ring has been treated in
\cite{PS06}.

When a finite field $\K$ is seen as a vector space over $\F_p$, the
\emph{trace} form plays a special role. It has been already introduced
in Section \ref{sec:arithmetics}:
\begin{equation}
  \tag{\ref{eq:tracec0}}
  \Tr_{\K/\F_p} \;:\; x \mapsto \sum_{l=0}^{[\K:\F_p]-1} x^{p^l}
  \text{.}
\end{equation}

Loosely speaking the trace of an element $a\in\K$ is the sum of its
conjugates, eventually repeated a certain number of times if $a$ does
not generate the whole field $\K$.

To make it more precise, we recall the definition of the frobenius map
$\frob : x\mapsto x^p$, the trace can then be written as
\[\Tr_{\K/\F_p}(a) = \sum_{i=0}^{[\K/\F_p]-1} \frob^i(a) \text{.}\]
Let now be $z$ a generator of $\K$, then every element $a\in\K$ has a
representation $a(z)$ as an element of $\F_p[z]$, thus we deduce by
the linearity of $\frob$
\[\frob^i(a) = a\left(\frob^i(z)\right)\text{.}\]
The following lemma is then an easy consequence.

\begin{lemma}
  \label{th:trace}
  Let $\K = \F_p(z)$, let $Q$ be the minimal polynomial of $z$ over
  $\F_p$ and let $B$ be the set
  \[B = \left\{\zeta \in\U_i \;\middle|\; Q(\zeta) = 0 \right\} \]
  of all the conjugates of $z$. Then for every $a\in\K$
  \[\Tr_{\K/\F_p}(a) = \sum_{\zeta\in B} a(\zeta)
  \text{.}\]
\end{lemma}

In the context of finite fields, we can also define the \emph{dual} of
the field multiplication. Given an element $b\in\K$, consider the
application $M_b : \K\rightarrow\K$ given by $M_b(a) = ab$; it is a
$\F_p$-linear application and its dual, noted $\dual{M_b} : \dual{\K}
\rightarrow \dual{\K}$ acts over elements $\ell\in\dual{\K}$ in the
following way
\[\dual{M_b}(\ell)(a) = \ell\left(M_b(a)\right) = \ell(ab) \qquad\text{for every $a\in\K$.}\]

\begin{definition}[Transposed mutliplication]
  \label{def:transmul}
  Let $\K$ be a finite field of characteristic $p$, let $\dual{\K}$ be
  its dual space relatively to $\F_p$, let $b\in\K$ and
  $\ell\in\dual{\K}$. The result of the transposed multiplication of
  $\ell$ by $b$, noted $b\cdot\ell$, is the linear form defined by
  \[(b\cdot\ell)(a) = \ell(ab) \qquad\text{for every $a\in\K$.}\]
\end{definition}


Now, let $b\in\K$, we define $\ell_b$ as the linear form
\begin{equation}
  \ell_b = b\cdot\Tr_{\K/\F_p} \text{,}
\end{equation}
then the following properties are easily verified.
\begin{itemize}
\item $\ell_0 = 0$
\item $\ell_b \ne 0$ if $b\ne0$,
\item $\ell_b + \ell_c = \ell_{b+c}$,
\item $c \cdot \ell_b = \ell_{bc}$.
\end{itemize}
with $b,c\in\K$. Then a counting argument \cite[Theorem 2.24]{LN}
shows that
\begin{theorem}
  \label{th:linearforms}
  Let $\K$ be a finite field. For any linear form $\ell$ over $\K$
  there exists a unique $b\in\K$ such that $\ell =
  b\cdot\Tr_{\K/\F_p}$.
\end{theorem}

Finally, we can now give a lemma that will be at the heart of the
lift-up algorithm.

\begin{lemma}
  \label{th:lift-up}
  Let $b\in\K$ different from $0$ and let $\ell_b =
  b\cdot\Tr_{\K/\F_p}$. Let $z$ be a generator of $\K$ over $\F_p$ and
  $B$ the set of its conjugates. Then the following equality over
  formal series holds:
  \begin{equation*}
    \sum_{j\ge0} \ell_b(z^j)T^j =
    \frac{\sum_{\zeta\in B} b(\zeta)\prod_{\zeta'\ne\zeta}(1 - \zeta' T)}
	 {\prod_{\zeta\in B} (1- \zeta T)}
	 \text{.}
  \end{equation*}
\end{lemma}
\begin{proof}
  By the definition of $\ell_b$
  \[\sum_{j\ge0} \ell_b(z^j)T^j = \sum_{j>0} \Tr_{\K/\F_p}(bz^j)T^j
  \text{,}\]
  which, by lemma \ref{th:trace}, is equal to
  \[\sum_{j\ge0} \sum_{\zeta\in B} b(\zeta) \zeta^j T^j \text{.}\]
  Inverting the sums we get
  \[\sum_{\zeta \in B} \frac{b(\zeta)}{1- \zeta T} \]
  and the thesis follows.
\end{proof}


\subsubsection{Transposition principle}
TODO.

\begin{theorem}[Transposition theorem]
  \label{th:transposition}
  
\end{theorem}


\subsubsection{Power projection}
\label{sec:level-embedding:power-projection}
The power projection problem was first introduced by V. Shoup in
\cite{Sho94} and can be defined as follows.

\begin{definition}[Power projection]
  Let $\K$ be a field of characteristic $p$, and let $\dual{\K}$ be
  its dual space relatively to $\F_p$. Let $a\in\K$ and
  $\ell\in\dual{\K}$.
  
  The power projection problem consists in computing the values
  \[\ell(a^0), \ell(a^1), \ldots, \ell(a^{n-1}) \in\F_p \text{,}\]
  where $n=[\K:\F_p]$.
\end{definition}

Our goal is to apply lemma \ref{th:lift-up}, so we are interested in
the following instance of the power projection problem:
\begin{equation*}
  \K = \U_i,\qquad a=z
  \text{;}
\end{equation*}
hence our problem is to compute 
\begin{equation}
  \label{eq:power-projection}
  \ell(1), \ell(z), \ell(z^2), \ldots, \ell(z^{p^id-1}) \in\F_p
\end{equation}
given an $\ell\in \dual{\K}$. It will be sufficient for our purpose to
solve the power projection problem with $\ell$ represented over
$\dual{D}$, the dual basis of
\begin{equation}
  D = (1,y,\ldots,y^{p^{i-1}d},z,zy,\ldots,z^{p-1}y^{p^{i-1}d}) \text{.}
\end{equation}

In \cite{Sho94} it is observed that the power projection is the
transposed problem of the modular composition of polynomials, so that
the values in \eqref{eq:power-projection} can be computed in
$O(\ModComp(p^id))$ operations in $\F_p$. We can do much better,
though, by observing that our instance is the transposed problem of
the push-down.

In fact, the push-down is an $\F_p$-linear application, more precisely
it is the change of bases from $U$ to $D$ as in equation
\eqref{eq:bases}, so it's matrix is given by
\begin{equation*}
  M = \left(\begin{array}{c|c|c|c}
    %%
    &&&\\
    %%
    \biv{z}_0 & \biv{z}_1 &
    \;\cdots\; & \biv{z}_{p^id-1} \\
    %%
    &&&
  \end{array}\right)
\end{equation*}
where $\biv{z_i}$ is the representation of $z^i$ on the basis $D$.

It is then straightforward to verify the equality 
\begin{equation*}
  \trans{M}\trans{\biv{l}} = 
  \begin{pmatrix}
    \ell(z^0)\\ \ell(z^1)\\ \vdots\\ \ell(z^{p^id-1})
  \end{pmatrix}
\end{equation*}
where $\biv{l}$ is the line vector representing $\ell$ on the basis
$\dual{D}$.

By theorem \ref{th:transposition} we conclude there is an algorithm,
called \alg{Push-down$\trans{}$}, that, given a linear form $\ell$
represented over $\dual{D}$, computes the vector
\[\left(\ell(z^0), \ell(z^1), \ldots, \ell(z^{p^id-1})\right)\]
using $O(P(i))$ operations in $\F_p$.

\subsubsection{Polynomial operations}

TODO : truncated power series ?

\begin{definition}[Reverse]
  \label{def:rev}
  Let $P$ be a polynomial in the variable $T$ of degree at most $d$,
  we define the reverse $P$ around the degree $d$ as
  \[\rev_d(P)(T) = T^dP(1/T) \text{.}\]
\end{definition}
From an algorithmic point of view, a reverse is implemented by adding
enough zeroes to the right of $P$ to make it of formal degree $d$,
then reading its coefficients right to left. The total cost is thus
$O(d)$ operations.

\begin{proposition}
  The following properties hold
  \begin{enumerate}
  \item $\rev_d(\rev_d(P)) = P$,
  \item $\rev_d(P+Q) = \rev_d(P) + \rev_d(Q)$,
  \item $\rev_{d_1+d_2}(PQ) = \rev_{d_1}(P)\rev_{d_2}(Q)$.
  \end{enumerate}
\end{proposition}


\subsubsection{The algorithm}
We are now ready to give the algorithm for the lift-up.

Let $\ell\in\dual{\U_i}$ be any non null linear form represented over
the basis $\dual{D}$ and let $b\in\U_i$ such that
$\ell=b\cdot\Tr_{\U_i/\F_p}$. Suppose we have elements
$v_0,\ldots,v_{p-1}\wrt\U_{i-1}$ that we want to lift-up, that is, we
want to represent the element
\[v = v_0 + v_1z + \ldots + v_{p-1}z^{p-1} \]
over the basis $U$, while we can easily write down its representation
over the basis $D$ simply by concatenating the representations of
$v_0,\ldots,v_{p-1}$.

Set $\ell_v = v\cdot\ell = (bv)\cdot\Tr_{\U_i/\F_p}$, let $B$ be the
set of all the conjugates of $z$, we can apply lemma \ref{th:lift-up}
to obtain the two equalities over elements of $\F_p[[T]]$
\begin{align}
  P_1(T)& = \sum_{j>0}\ell(z^j)T^j =
  \frac{\sum_{\zeta\in B}b(\zeta)\prod_{\zeta'\ne\zeta}(1-\zeta'T)}
       {\prod_{\zeta\in B}(1-\zeta T)} \text{,}\\
  P_v(T)& = \sum_{j>0}\ell_v(z^j)T^j =
  \frac{\sum_{\zeta\in B}b(\zeta)v(\zeta)\prod_{\zeta'\ne\zeta}(1-\zeta'T)}
       {\prod_{\zeta\in B}(1-\zeta T)} \text{.}
\end{align}

Now remark that the reverse of the denominator of $P_1(T)$ and $P_v(T)$ is
the minimal polynomial of $z$, indeed
\[\rev_{p^id}\left(\prod_{\zeta\in B}(1-\zeta T)\right) =
\prod_{\zeta\in B}(T -\zeta) = Q(T)\text{.}\]
We can then easily extract the two numerators
\begin{align}
  N_1(T) &= P_1(T)\rev_{p^id}(Q)(T) =
  \sum_{\zeta\in B}b(\zeta)\prod_{\zeta'\ne\zeta}(1-\zeta'T) \text{,}\\
  N_v(T) &= P_v(T)\rev_{p^id}(Q)(T) =
  \sum_{\zeta\in B}b(\zeta)v(\zeta)\prod_{\zeta'\ne\zeta}(1-\zeta'T)
  \text{,}
\end{align}
and reverse them in order to obtain
\begin{align}
  V_1(T) = \rev_{p^id-1}(N_1)(T) &=
  \sum_{\zeta\in B}b(\zeta)\prod_{\zeta'\ne\zeta}(T-\zeta') \text{,}\\
  V_v(T) = \rev_{p^id-1}(N_v)(T) &=
  \sum_{\zeta\in B}b(\zeta)v(\zeta)\prod_{\zeta'\ne\zeta}(T-\zeta')
  \text{.}
\end{align}

We can now go back to elements in $\F_p[Z]/Q(Z)$ and claim that
\begin{equation}
  v(Z) = \frac{V_v(Z)}{V_1(Z)} \mod Q(Z) \text{,}
\end{equation}
in fact
\begin{equation*}
  \frac{V_v(z)}{V_1(z)} =
  \frac{b(z)v(z)\prod_{\zeta'\ne z}(z-\zeta')}{b(z)\prod_{\zeta'\ne z}(z-\zeta')}
  = v(z) \text{.}
\end{equation*}

This discussion, together with the properties of truncated power
series show the following theorem.

\begin{theorem}
  The algorithm \alg{Lift-up} given below is correct.
\end{theorem}

\begin{figure}
  \begin{algorithm}{Pre-Lift-up}
    {$\ell\in\dual{\U_i}$ written over the basis $\dual{D}$.}
    {$1/V_1(Z) \bmod Q(Z)$.}
  \item \label{alg:pre-lift-up:pow} compute
    $\ell(z^0),\ldots,\ell(z^{p^id-1}) =$ Push-down$\trans{}(\ell)$,
  \item set $P_1(T) = \sum_{j=0}^{p^id-1}\ell(z^j)T^j$,
  \item \label{alg:pre-lift-up:rev1} compute $Q_r = \rev_{p^id}(Q)$,
  \item \label{alg:pre-lift-up:mult} compute $N_1(T) = P_1(T)Q_r(T)
    \bmod T^{p^id}$,
  \item \label{alg:pre-lift-up:rev2} compute $V_1 =
    \rev_{p^id-1}(N_1)$,
  \item \label{alg:pre-lift-up:inv} return $1 / V_1(Z) \bmod Q(Z)$.
  \end{algorithm}
\end{figure}

\begin{figure}
  \begin{algorithm}{Lift-up}
    {$\ell\in\dual{\U_i}$ written over the basis $\dual{D}$,
      $v_0,\ldots,v_{p-1}\wrt\U_{i-1}$.}
    {$v\wrt\U_i$ s.t. $v_0z^0 + \cdots + v_{p-1}z^{p-1} = v$.}
  \item \label{alg:lift-up:transmul} compute $\ell_v = v\cdot\ell$,
  \item \label{alg:lift-up:pow} compute
    $\ell_v(z^0),\ldots,\ell_v(z^{p^id-1}) =$ Push-down$\trans{}(\ell_v)$,
  \item set $P_v(T) = \sum_{j=0}^{p^id-1}\ell_v(z^j)T^j$,
  \item \label{alg:lift-up:rev1} compute $Q_r = \rev_{p^id}(Q)$,
  \item \label{alg:lift-up:mult} compute $N_v(T) = P_v(T)Q_r(T)
    \bmod T^{p^id}$,
  \item \label{alg:lift-up:rev2} compute $V_v = \rev_{p^id-1}(N_v)$,
  \item get $\hat{V}_1 =$ Pre-Lift-up$(\ell)$,
  \item \label{alg:lift-up:mulmod} return $V_v(Z)\hat{V}_1(Z) \bmod Q(Z)$.
  \end{algorithm}
\end{figure}

Now we can analyse the complexity of the algorithm. Step
\ref{alg:pre-lift-up:pow} of \alg{Pre-Lift-up} and step
\ref{alg:lift-up:pow} of \alg{Lift-up} both cost
\[O(\Push(i)) = O\left(\Mult(p^{i+1}d)\log(p^id)\right)\]
as seen in Sections \ref{sec:level-embedding:power-projection} and
\ref{sec:level-embedding:push-down}. Steps \ref{alg:pre-lift-up:rev1}
and \ref{alg:pre-lift-up:rev2} of \alg{Pre-Lift-up} and steps
\ref{alg:lift-up:rev1} and \ref{alg:lift-up:rev2} of \alg{Lift-up} all
cost $O(p^id)$, while step \ref{alg:pre-lift-up:mult} of
\alg{Pre-Lift-up} and steps \ref{alg:lift-up:mult} and
\ref{alg:lift-up:mulmod} of \alg{Lift-up} cost $O(\Mult(p^id))$, so
that the dominant cost is still $O(\Push(i))$.

Step \ref{alg:pre-lift-up:inv} of \alg{Pre-Lift-up} is a modular
inversion and can be done in
\[\Inv(i) = O\left(\Mult(p^id)\log(p^id)\right) \]
as seen in Section \ref{sec:arithmetics}. This is still less than
$O(\Push(i))$ and we remark as well that the algorithm
\alg{Pre-Lift-up} is always called on the same input $\ell$, so that
its output can be stored once for all and its cost amortised over all
the calls of \alg{Lift-up}.

Finally, step \ref{alg:lift-up:transmul} of \alg{Lift-up} is a
transposed multiplication in the basis $\dual{D}$. Theorem
\ref{th:transposition} tells us that its computation can be done in
the same time as a multiplication in the basis $D$ up to a constant
factor, that is $O\left(\Mult(p^id)\right)$ using Kronecker
substitution. A speed up of a constant factor may be achieved by
choosing a suitable $\ell$.\footnote{For example, when $\ell$ is
  chosen to be the residue form, the cost for this drops down to one
  sum in $\U_{i-1}$ and $p$ transposed multiplications in the standard
  basis of $\U_{i-1}$.... Faut-il le justifier ? Est-ce que c'est la
  peine de montrer l'optimalité du résidu (si c'est bien
  optimal... j'en suis pas sûr) ?}

Putting things together, we have proven the theorem
\begin{theorem}
  Let $\ell\in\dual{\U_i}$ be fixed. Given an input
  $v_0,\ldots,v_{p-1}\wrt\U_{i-1}$, the algorithm \alg{Lift-up}
  computes its output in $O(\Push(i))$ amortised $\F_p$-operations.
\end{theorem}



% Local Variables:
% mode:flyspell
% ispell-local-dictionary:"british"
% End:
%
% LocalWords:  Schreier Artin pseudotrace frobenius bivariate memoization
