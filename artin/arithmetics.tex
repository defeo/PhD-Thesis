\section{Preliminaries}
\label{sec:arithmetics}

As a general rule, variables and polynomials are in upper
case; elements algebraic over $\F_p$ (or some other field, that will
be clear from the context) are in lower case.
 
%%%%%%%%%%%%%%%%%%%%%%%%%%%%%%%%%%%%%%%%%%%%%%%%%%%%%%%%%%%%

\subsection{Element representation}\label{ssec:rep}

Let $Q_0$ be in $\F_p[X_0]$ and let $(G_i)_{0 \le i < k}$ be
a sequence of polynomials over $\F_p$, with $G_i$ in
$\F_p[X_0,\dots,X_i]$. We say that the sequence $(G_i)_{0\le i <k}$
{\em defines the tower} $(\U_0,\dots,\U_k)$ if for $i \ge 0$, 
$\U_i=\F_p[X_0,\dots,X_i]/K_i$, where $K_i$ is
the
ideal generated by
$$\left | \begin{array}{l}
P_i=X_i^p-X_i -G_{i-1}(X_0,\dots,X_{i-1})\\
~~~\,~\vdots\\
P_1=X_1^p-X_1-G_0(X_0)\\
Q_0(X_0)
\end{array}\right .$$
in $\F_p[X_0,\dots,X_i]$, and if $\U_i$ is a field. The residue class of
$X_i$ (resp. $G_i$) in $\U_i$, and thus in $\U_{i+1},\dots$, is
written $x_i$ (resp. $\gamma_i$), so that we have
$x_i^p-x_i=\gamma_{i-1}$.

Finding a suitable $\F_p$-basis to represent elements of a tower
$(\U_0,\dots,\U_k)$ is a crucial question. If $d=\deg(Q_0)$, a natural
basis of $\U_i$ is the multivariate basis $\bB_i=\{x_0^{e_0} \cdots
x_i^{e_i}\}$ with $0 \le e_0 < d$ and $0\le e_j < p$ for $1 \le j \le
i$. However, in this basis, we do not have very efficient arithmetic
operations, starting from multiplication. Indeed, the natural
approach to multiplication in $\bB_i$ consists in a polynomial
multiplication, followed by reduction modulo $(Q_0,P_1,\dots,P_i)$;
however, the initial product gives a polynomial of partial degrees
$(2d-2,2p-2,\dots,2p-2)$, so the number of monomials appearing is not
linear in $[\U_i:\F_p]=p^id$.  See~\cite{LiMoSc07} for details.

As a workaround, we introduce the notion of a {\em primitive tower},
where for all $i$, $x_i$ generates $\U_i$ over $\F_p$. In this case,
we let $Q_i\in \F_p[X]$ be its minimal polynomial, of degree
$p^id$. In a primitive tower, unless otherwise stated, we represent
the elements of $\U_i$ on the $\F_p$-basis
$\bC_i=(1,x_i,\dots,x_i^{p^id-1})$.

To stress the fact that $v\in\U_i$ is represented on the basis
$\bC_i$, we write $v\wrt\U_i$. In this basis, assuming $Q_i$ is known,
additions and subtractions are done in time $p^id$, multiplications in
time $O(\Mult(p^id))$~\cite[Ch.~9]{vzGG} and inversions in time
$O(\Mult(p^id)\log(p^id))$~\cite[Ch.~11]{vzGG}.

Remark that having fast arithmetic operations in $\U_i$ enable us to
write fast algorithms for polynomial arithmetic in $\U_i[Y]$, where
$Y$ is a new variable. Extending the previous notation, let us write
$A \wrt\U_i[Y]$ to indicate that a polynomial $A \in \U_i[Y]$ is
written on the basis $(x_i^\alpha Y^\beta)_{0 \le \alpha < p^id, 0 \le
  \beta}$ of $\U_i[Y]$.  Then, given $A,B \wrt \U_i[Y]$, both of
degrees less than $n$, one can compute $AB \wrt \U_i[Y]$ in time
$O(\Mult(p^id n))$ using Kronecker's
substitution~\cite[Lemma~2.2]{GaSh92}.

One can extend the fast Euclidean division algorithm to this context,
as Newton iteration reduces Euclidean division to polynomial
multiplication. The analysis of~\cite[Ch.~9]{vzGG} implies that
Euclidean division of a degree $n$ polynomial $A \wrt \U_i[Y]$ by a
monic degree $m$ polynomial $B \wrt \U_i[Y]$, with $m \le n$, can be
done in time $O(\Mult(p^id n))$.

Finally, fast GCD techniques carry over as well, as they are based on
multiplication and division. Using the analysis
of~\cite[Ch.~11]{vzGG}, we see that the extended GCD of two monic
polynomials $A,B \wrt \U_i[Y]$ of degree at most $n$ can be computed
in time $O(\Mult(p^id n \log(n)))$.

%%%%%%%%%%%%%%%%%%%%%%%%%%%%%%%%%%%%%%%%%%%%%%%%%%%%%%%%%%%%

\subsection{Trace and pseudotrace}\label{ssec:tpt}


We continue with a few useful facts on traces. Let $\U$ be a field and
let $\U'=\U[X]/Q$ be a separable field extension of $\U$, with
$\deg(Q)=n$. For $a \in \U'$, the {\em trace} $\Tr(a)$ is the trace of
the $\U$-linear map $M_a$ of multiplication by $a$ in $\U'$.

The trace is a $\U$-linear form; in other words, $\Tr$ is in the dual
space $\dual{\U'}$ of the $\U$-vector space $\U'$; we write it
$\Tr_{\U'/\U}$ when the context requires it. In finite fields, we
also have the following well-known properties:
\begin{align}
  \tag{$\bP_1$} &\begin{array}{c}  
  \Tr_{\F_{q^n}/\F_q}: a \mapsto \sum_{\ell=0}^{n -
    1}a^{q^\ell} \text{,}
  \end{array}\\
  \tag{$\bP_2$}\label{eq:trcomp}
  &\Tr_{\F_{q^{mn}}/\F_q} = \Tr_{\F_{q^m}/\F_q} \circ
  \Tr_{\F_{q^{mn}}/\F_{q^m}}\text{.}
\end{align}

Besides, if $\U'/\U$ is an Artin-Schreier extension generated by a
polynomial $Q$ and $x$ is a root of $Q$ in $\U'$, then
\begin{equation}
  \tag{$\bP_3$}\label{eq:pd} \Tr_{\U'/\U}(x^j) = 0~ \text{for}~j
  <p-1; \quad \Tr_{\U'/\U}(x^{p-1}) = -1\text{.}
\end{equation}
Following~\cite{Couveignes00}, we also use a generalization of the
trace. The $n$th {\em pseudotrace} of order $m$ is the
$\F_{p^m}$-linear operator
\begin{equation*}
\begin{array}{c}  \PTr_{(n,m)}: a \mapsto \sum_{\ell=0}^{n-1}a^{p^{m\ell}};\end{array}
\end{equation*}
for $m=1$, we call it the $n$th pseudotrace and write $\PTr_n$.

In our context, for $n=[\U_i:\U_j]=p^{i-j}$ and $m=[\U_j:\F_p]=p^jd$,
$\PTr_{(n,m)}(v)$ coincides with $\Tr_{\U_{i}/\U_j}(v)$ for $v$ in
$\U_i$; however $\PTr_{(n,m)}(v)$ remains defined for $v$ not in
$\U_i$, whereas $\Tr_{\U_{i}/\U_j}(v)$ is not.

%%%%%%%%%%%%%%%%%%%%%%%%%%%%%%%%%%%%%%%%%%%%%%%%%%%%%%%%%%%%

\subsection{Duality}\label{ssec:duality}

Finally, we discuss two useful topics related to duality,
starting with the transposition of algorithms.

Introduced by Kaltofen and Shoup, the \emph{transposition principle}
relates the cost of computing an $\F_p$-linear map $f:\ V \to W$ to
that of computing the transposed map $\dual{f}:\ \dual{W} \to
\dual{V}$.  Explicitly, from an algorithm that performs an $r \times
s$ matrix-vector product $b \mapsto M b$, one can deduce the existence
of an algorithm with the same complexity, up to $O(r+s)$, that
performs the transposed product $c \mapsto M^t c$;
see~\cite{BuClSh97,Kaltofen00,BoLeSc03}. However, making the
transposed algorithm explicit is not always straightforward; we will
devote part of Section~\ref{sec:level-embedding} to this issue.

We give here first consequences of this principle,
after~\cite{Sho94,Shoup99,BoLeSc03}. Consider a degree $n$ field
extension $\U \to \U'$, where $\U'$ is seen as an $\U$-vector
space. For $w$ in $\U'$, recall that $M_w: \U'\rightarrow\U'$ is the
multiplication map $M_w(v) = vw$.  Its dual $\dual{M_w}: \dual{\U'}
\rightarrow \dual{\U'}$ acts on $\ell\in\dual{\U'}$ by
$\dual{M_w}(\ell)(v) = \ell\left(M_w(v)\right) = \ell(vw)$ for $v$ in
$\U'$. We prefer to denote the linear form $\dual{M_w}(\ell)$ by
$w\cdot\ell$, keeping in mind that $(w\cdot\ell)(v) = \ell(vw)$.

Suppose then that $\bD$ is a $\U$-basis of $\U'$, in which we can
perform multiplication in time $T$. Then by the transposition
principle, given $w$ on $\bD$ and $\ell$ on the dual basis
$\dual{\bD}$, we can compute $w\cdot \ell$ on the dual basis
$\dual{\bD}$ in time $T+O(n)$.  This was discussed already
in~\cite{Shoup99,BoLeSc03}, and we will get back to this in
Section~\ref{sec:level-embedding}.

Suppose finally that $\U'$ is separable over $\U$ and that $b\in \U'$
generates $\U'$ over $\U$; we will denote by $Q \in \U[X]$ the minimal
polynomial of $b$. Given $w$ in $\U'$, we want to find an expression
$w=A(b)$, for some $A \in \U[X]$. Hereafter, for $P \in \U[X]$ of
degree at most $e$, we write $\rev_e(P)=X^eP(1/X) \in \U[X]$. Then,
recalling that $n=[\U':\U]$, we define $\ell=w\cdot\Tr_{\U'/\U} \in
\dual{\U'}$ and
\begin{equation}
  \label{eq:MN}
  M = \sum_{j < n}\ell(b^j)X^j,\quad N = M\rev_{n}(Q) \bmod X^n.
\end{equation}
This construction solves our problem: Theorem~3.1
in~\cite{Rouillier99} shows that $w=A(b)$, with $A=\rev_{n-1}(N)
{Q'}^{-1} \bmod Q$. We will hereafter denote by
$\alg{FindParameterization}(b,w)$ a subroutine that computes this
polynomial $A$; it follows closely a similar algorithm given
in~\cite{Sho94}. Since this is the case we will need later on, we give
details for the case where $Q$ is Artin-Schreier (so $n=p$): then,
$Q'=-1$, so no work is needed to invert it modulo $Q$.

In the following algorithm, we suppose that $\U'$ is presented as
$\U'=\U[X]/P$, where $P$ is Artin-Schreier. We let $x$ be the residue
class of $X$ in $\U'$.
\begin{algorithm}{FindParameterization}
  {$w \in \U'$ written as $w_0 + \cdots + w_{p-1} x^{p-1}$,  
   $b \in \U'$ written as $b_0 + \cdots + b_{p-1} x^{p-1}$}
  {A polynomial $A$ of degree less than $p$ such that $w=A(b)$}
\item\label{alg:para:trmul} let $\ell = w \cdot\Tr_{\U'/\U}$
\item\label{alg:para:trmodcomp} let $M= \sum_{j < p}\ell(b^j)X^j$
\item\label{alg:para:multrunc} let $N = M \rev_{p}(Q) {\sf ~mod~} X^{p}$
\item\label{alg:para:mulmod} return $-\rev_{p-1}(N)$
\end{algorithm}
\begin{proposition}
  \label{th:findparameterization}
  If $Q$ is Artin-Schreier, the cost of $\alg{FindParameterization}$ is
  $O(p^2)$ operations $(+,\times)$ in $\U$.
\end{proposition}
\begin{proof} By~\ref{eq:pd}, the representation of $\Tr_{\U'/\U}$ in
$\U'^\ast$ is simply $(0,\ldots,0,-1)$. Then by the discussion above,
if $T$ is the cost of multiplying two elements of $\U'$ in the basis
$(1,\ldots,x^{p-1})$, step~\ref{alg:para:trmul} costs $T + O(p)$; this
stays in $O(p^2)$ by taking a naive
multiplication. Step~\ref{alg:para:trmodcomp} fits into the same
bound, by the proof of~\cite[Th.~4]{Sho94}. Taking the $\rev$'s in
steps~\ref{alg:para:multrunc} and~\ref{alg:para:mulmod} is just
reading the polynomials from right to left, thus this costs no
arithmetic operation. Finally, step~\ref{alg:para:multrunc} features a
polynomial multiplication truncated to the order $p$, this costs
$O(p^2)$ operations by a naive algorithm.\end{proof}

Note that this cost can be improved with respect to $p$, by using fast
modular composition as in~\cite{Sho94}; we do not give details, as this
would not improve the overall complexity of the algorithms of the next
sections.


% Local Variables:
% mode:flyspell
% ispell-local-dictionary:"american"
% End:
%
% LocalWords:  Schreier Artin pseudotrace frobenius bivariate memoization
