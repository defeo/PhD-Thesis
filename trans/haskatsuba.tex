\chapter{Linearity inference of karatsuba multiplication}
\label{cha:line-infer-karats}

\lstset{language=haskell}

We show here an example of inference of linearity in Haskell, using
the technique described in Section~\ref{sec:inference}. We define type
classes \lstinline{Ring} and \lstinline{Module} to represent
left-linear operations on rings and free modules. We instantiate them
with integers as base ring, and lists of integers as free module
(representing polynomials over $\Z[X]$).

We implement Karatsuba multiplication over $\Z[X]$, using only the
methods defined in \lstinline{Ring} and \lstinline{Module}. This
allows the type checker to deduce that Karatsuba multiplication is
linear in its first argument, once the second argument and the degree
of the polynomials are fixed. The code makes use of functional
dependencies, it must be run with the switch \verb|-fglasgow-exts| on.

\begin{xcomment}{lstlisting}
\chapter{Linearity inference of karatsuba multiplication}
\label{cha:line-infer-karats}
\lstset{language=haskell}

\begin{lstlisting}
data L = L Integer
data S = S Integer

class Ring r where
  zero :: r
  (<+>) :: r -> r -> r
  (<*>) :: r -> S -> r
  neg :: r -> r
  
class Ring r => Module m r | m -> r where
  zeroM :: m
  (<<*) :: m -> S -> m
  (>>>) :: m -> Integer -> r
  (<<<) :: r -> Integer -> m
  (<++>) :: m -> m -> m
  add :: m -> m -> Integer -> m
  add a b n = foldl (<++>) zeroM 
              [((a>>>i) <+> (b>>>i))<<<i | i <- [1..n]]
  
instance Ring L where
  zero = L 0
  (L x) <+> (L y) = L (x+y)
  (L x) <*> (S y) = L (x*y)
  neg (L x) = L (-x)
  
instance Ring S where
  zero = S 0
  (S x) <+> (S y) = S (x+y)  
  (S x) <*> (S y) = S (x*y)
  neg (S x) = S (-x)

one = S 1

instance Ring r => Module [r] r where
  zeroM = [zero]
  [] <<* x = []
  (x:xs) <<* y = (x <*> y):(xs <<* y)
  [] >>> i = zero
  (x:xs) >>> i =
    if i < 1 then zero else if i == 1 then x else xs >>> (i-1)
  x <<< i = if i <= 1 then [x] else zero:(x <<< (i-1))
  [] <++> [] = []
  (x:xs) <++> [] = x:(xs <++> [])
  [] <++> (y:ys) = y:([] <++> ys)
  (x:xs) <++> (y:ys) = (x <+> y):(xs <++> ys)
  add [] [] n = []
  add [] (y:ys) n =
    if n > 0 then y:(add [] ys (n-1)) else []
  add (x:xs) [] n =
    if n > 0 then x:(add xs [] (n-1)) else []
  add (x:xs) (y:ys) n =
    if n > 0 then (x<+>y):(add xs ys (n-1)) else []


-- Karatsuba multiplication : the system will infer
-- shift :: Ring r => [r] -> Integer -> [r]
-- split :: Ring r => [r] -> Integer -> ([r], [r])
-- kara :: Ring r => [r] -> [S] -> Integer -> [r]

shift x n = if n <= 0 then x else shift (zero:x) (n-1)

split [] n = ([], [])
split (x:xs) n =
  if n <= 0
  then ([], x:xs)
  else let (a, b) = split xs (n-1) in (x:a, b)

kara [] y n = []
kara x [] n = []
kara x y n =
  if n <= 0
  then []
  else if n == 1 
       then [(x!!0) <*> (y!!0)]
       else 
         let h = n `div` 2 in
         let (a0, a1) = split x h in
         let (b0, b1) = split y h in
         let x0 = kara a0 b0 h in
         let x2 = kara a1 b1 (n-h) in
         let xx1 = kara (a1 <++> a0) (b1 <++> b0) (n-h) in
         let x1 = xx1 <++> ((x0 <++> x2) <<* (neg one)) in
         (shift x2 n) <++> (shift x1 h) <++> x0
\end{lstlisting}                    
                  

% Local Variables:
% mode:flyspell
% ispell-local-dictionary:"american"
% mode:TeX-PDF
% mode:reftex
% TeX-master: "../these"
% End:
%

\end{xcomment}


                  

% Local Variables:
% mode:flyspell
% ispell-local-dictionary:"american"
% mode:TeX-PDF
% mode:reftex
% TeX-master: "../these"
% End:
%
