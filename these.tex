\documentclass[a4paper]{report}

\usepackage[french,american]{babel}
\usepackage[utf8]{inputenc}
\usepackage{bbm}
\usepackage{amsmath}
\let\equation\gather
\let\endequation\endgather
\usepackage{extarrows}
\usepackage{amsthm}
\usepackage{amssymb}
\usepackage{mathrsfs}
\usepackage{mdwlist}
\usepackage{graphicx}
\usepackage{float}
\usepackage[pdftex]{hyperref}
\hypersetup{pdfborderstyle={/S/U/W 0.5},citebordercolor={0 0.4 1}}
\usepackage[author={Luca De Feo}]{pdfcomment}
%\renewcommand{\pdfmargincomment}[1]{}
\usepackage{url}
\usepackage{color}
\usepackage[all]{xypic}
\usepackage{tikz}
\usetikzlibrary{arrows,shapes,automata}
\usepackage{eurosym}
\usepackage[chapter]{algorithm}
\usepackage[noend]{algorithmic}
\usepackage[final]{listings}
\usepackage{textcomp}
\usepackage{makeidx}
\usepackage[refpage,intoc]{nomencl}
\usepackage{xcomment}

%%%%%%%%%%%%%% Code listings
\lstset{
  upquote=true,
  basicstyle=\ttfamily,          % print whole listing in typewriter
  keywordstyle=\color{blue}\bfseries, % bold blue keywords
  %identifierstyle=,           % nothing happens
  commentstyle=\color{green}, % green comments
  stringstyle=\color{red},      % typewriter type for strings
  showstringspaces=false     % no special string spaces
}


%%%%%%%%%%%%%% Aliases

%% Typographie
%\renewcommand{\le}{\leqslant}
%\renewcommand{\ge}{\geqslant}  % comme François le demande...
\newcommand{\todo}{\dots}  % un marqueur pour un trou

%% Bases
\newcommand{\card}[1]{\# #1}  % cardinalité 
\newcommand{\diffset}{\,\backslash\,}  % difference ensembliste
\newcommand{\ndiv}{\nmid}  % ne divise pas
\newcommand{\wrt}{\dashv}  % appartenance forte, a\wrt A signifie que a est représenté comme un élément de A
\newcommand{\isom}{\cong}
\newcommand{\eqdef}{\overset{\mathrm{def}}{\equiv}}

%% Nombres, corps
\newcommand{\N}{\mathbb{N}}  % les naturels
\newcommand{\Z}{\mathbb{Z}}  % les entiers
\newcommand{\K}{\mathbb{K}}  % un corps
\newcommand{\F}{\mathbb{F}}  % un corps fini
\newcommand{\Q}{\mathbb{Q}}  % les rationnels
\newcommand{\R}{\mathbb{R}}  % les réels
\newcommand{\C}{\mathbb{C}}  % les complexes
\newcommand{\LK}{\mathbb{L}}  % encore un corps
\newcommand{\U}{\mathbb{U}}  % encore un corps

%% Algèbre
\newcommand{\trans}[1]{#1^\top}  % transposé
\newcommand{\dual}[1]{#1^\ast}  % dual
\newcommand{\clot}[1]{\bar{#1}}  % clôture algèbrique
\newcommand{\frob}{\varphi}  % l'isomorphisme de frobenius
\newcommand{\res}{\rho}  % the residue form
\newcommand{\euler}{\varphi}  % indicatrice d'Euler
\newcommand{\Cyclo}{\Phi}  % le polynome cyclotomique
\newcommand{\Proj}{\mathbb{P}}  % espace projectif
\newcommand{\op}{\mathrm{op}}  % opposite
\newcommand{\bra}[1]{\left\langle #1 \right\rvert}
\newcommand{\ket}[1]{\left\lvert #1 \right\rangle}
\newcommand{\braket}[2]{\left\langle #1 \middle| #2 \right\rangle}
\newcommand{\braketop}[3]{\left\langle #1 \middle| #2 \middle| #3 \right\rangle}  % notation braket
\newcommand{\poinc}{\mathcal{H}}  % the Poincaré half plane

%% Courbes
\newcommand{\0}{\mathcal{O}}  % point de base d'une courbe
\newcommand{\ecpoint}[3]{[#1:#2:#3]}  % un point d'une courbe
\newcommand{\isog}[1]{\mathcal{#1}}  % la police des isogénies
\newcommand{\I}{\isog{I}}  % une isogénie I
\newcommand{\frobisog}{\phi}  % l'isogénie de frobenius
\newcommand{\Hasse}{H}  % l'invariant de Hasse
\newcommand{\divpol}{f}  % polynôme de division
\newcommand{\Modpol}{\Phi}  % modular polynomial

%% Diagrammes
\newcommand{\la}{\leftarrow}
\newcommand{\ra}{\rightarrow}
\newcommand{\La}{\Leftarrow}
\newcommand{\Ra}{\Rightarrow}

%% Langages
\newcommand{\bom}{\perp}  % bottom _|_
\newcommand{\al}{\prec}  % Haskell's -<
\newcommand{\tAL}{\textsf{transAL}}
\newcommand{\tALpy}{\texttt{transalpyne}}
\newcommand{\Sbasis}{\mathcal{S}}  % standard basis
\newcommand{\Tbasis}{\mathcal{L}}  % linear basis
\newcommand{\lmul}[1]{{}_{#1}*}  % left multiplication node
\newcommand{\rmul}[1]{*_{#1}}  % right multiplication node
%\newcommand{\RMod}[1]{#1\text{\sf-Mod}}  % The category of left R-modules
%\newcommand{\ModR}[1]{\text{\sf Mod-}#1}  % The category of right R-modules
\newcommand{\pspace}{\mathcal{P}}  % the parameter space
\newcommand{\lap}{\overset{+}{\leftarrow}}
\newcommand{\lat}{\overset{*}{\leftarrow}}  % self-increments


%% Complexité
\newcommand{\tildO}{\tilde{O}}  % la notation O~ qui oublie les log
\newcommand{\Mint}{\mathsf{M}_\text{int}}  % fonction de multiplication
\newcommand{\Mpol}[1][]{\mathsf{M}_\text{pol}^{#1}}  % fonction de multiplication
\newcommand{\Mult}{\mathsf{M}}  % fonction de multiplication
\newcommand{\MulM}{\mathsf{Mul}}  % multiplication
\newcommand{\Inv}{\mathsf{Inv}}  % inversion
\newcommand{\Push}{\mathsf{P}}  % fonction de push-down
\newcommand{\Lift}{\mathsf{L}}  % fonction de lift-up
\newcommand{\Trace}{\mathsf{T}}  % fonction de trace
\newcommand{\Frob}{\mathsf{F}}  % fonction de frobenius itéré
\newcommand{\Ptr}{\mathsf{PT}}  % fonction de pseudo-trace
\newcommand{\ModComp}{\mathsf{C}}  % fonction de composition modulaire
\newcommand{\sC}{\mathsf{K}}  % fonction de composition polynomiale

%% Polices
\newcommand{\basis}{\mathbf}  % la police des bases
\newcommand{\lst}{\mathbf}  % lists of coordinates
\newcommand{\algeb}{\mathcal}  % la police des algèbres
\newcommand{\alg}[1]{{\sf #1}}  % la police des algorithmes
\newcommand{\algref}[1]{\alg{\ref{#1}}}  % la police des algorithmes

%% Opérateurs
\DeclareMathOperator{\im}{im}  % image
\DeclareMathOperator{\End}{End}  % Endomorphism ring
\DeclareMathOperator{\Aut}{Aut}  % Automorphism group
\DeclareMathOperator{\Mat}{\mathcal{M}}  % Matrices
\DeclareMathOperator{\GL}{GL}  % General linear group
\DeclareMathOperator{\SL}{SL}  % Special linear group
\DeclareMathOperator{\inp}{in}
\DeclareMathOperator{\outp}{out}  % input and output ports/nodes in circuits
\DeclareMathOperator{\eval}{eval}
\DeclareMathOperator{\lave}{coeval}  % (co)evaluation of an arithmetic circuit
\DeclareMathOperator{\size}{size}
\DeclareMathOperator{\depth}{depth}  
\DeclareMathOperator{\hub}{\&}  % a hub node
\DeclareMathOperator{\id}{id}  % identity
\DeclareMathOperator{\diff}{d}  % differential operator
\DeclareMathOperator{\car}{char}  % caractéristique d'un corps
\DeclareMathOperator{\Frac}{Frac}  % corps des fractions
\DeclareMathOperator{\Gal}{Gal}  % groupe de Galois
\DeclareMathOperator{\Tr}{Tr}  % trace
\DeclareMathOperator{\PTr}{T}  % pseudotrace
\DeclareMathOperator{\Norm}{N} % norme
\DeclareMathOperator{\ord}{ord}  % l'ordre d'un élément
\DeclareMathOperator{\ev}{ev}  % evaluation
\DeclareMathOperator{\proj}{proj}  % evaluation
\DeclareMathOperator{\flip}{flip}  % flipper


%% Théorèmes
\theoremstyle{plain}
\newtheorem{theorem}{Theorem}[chapter]
\newtheorem{lemma}[theorem]{Lemma}
\newtheorem{corollary}[theorem]{Corollary}
\newtheorem{proposition}[theorem]{Proposition}
\newtheorem{principle}[theorem]{Principle}
\theoremstyle{definition}
\newtheorem{definition}[theorem]{Definition}
\newtheorem{example}[theorem]{Example}
\theoremstyle{remark}
\newtheorem{remark}[theorem]{Remark}
\newtheorem*{nota}{Note}

%% Algorithmes
\renewcommand{\algorithmicrequire}{\textbf{Input~~~:}}
\renewcommand{\algorithmicensure}{\textbf{Output~:}}


\title{Fast Algorithms for Towers of Finite Fields and Isogenies}
\author{Luca De Feo}
\makeindex
\makenomenclature

%% PDF metadata
\hypersetup{pdftitle={Fast Algorithms for Towers of Finite Fields and Isogenies},pdfauthor={Luca De Feo}}

\begin{document}
%\frontmatter

\maketitle
%% these.tex
%% Copyright 2010 Luca De Feo
%% All rights reserved


\selectlanguage{french}
\chapter[Intoduction (Français)][Introduction]{Introduction}

Les corps finis sont au cœur de la technologie moderne, à tel point
que la dernière génération de processeurs Intel Core possède une
instruction matérielle (CLMUL) pour la multiplications dans
$\F_{2^m}$~\cite{intel-carryless}. Cela tient au fait que les corps
finis apparaissent partout dans le génie des télécommunications, en
particulier en Codes Correcteurs d'Erreurs et Cryptographie. Cette
thèse applique des techniques algorithmiques et algébriques avancées
aux calculs dans les tours d'extensions sur les corps finis, avec pour
but des applications à la cryptographie à base de courbes elliptiques.

\paragraph*{Courbes elliptiques}
En cryptographie à base de courbes elliptiques, afin de construire un
système de chiffrement sûr, il faut sélectionner une courbe au nombre
de points divisible par un grand nombre premier. La méthode préférée
consiste à sélectionner une courbe au hasard et à appliquer un
algorithme de comptage de points pour déterminer sa cardinalité. Le
premier algorithme de comptage de points de courbes elliptiques de
complexité polynomiale fut donné par Schoof~\cite{schoof85}, puis
amélioré par Atkin et Elkies~\cite{atkin88,elkies98,schoof95}, et par
la suite nommé SEA.

L'algorithme SEA suscita de l'intérêt pour l'utilisation effective des
isogénies: ce sont des morphismes de groupes algébriques entre courbes
elliptiques. Lorsqu'on calcule des isogénies sur des corps finis, il
faut distinguer entre la caractéristique grande et la caractéristique
quelconque. Dans le premier cas, on peut utiliser des algorithmes
conçus pour la caractéristique $0$, et ensuite réduire le résultat;
les méthodes de Elkies\cite{elkies98,morain95}, Atkin~\cite{schoof95}
et Bostan, Morain, Salvy et Schost~\cite{bostan+morain+salvy+schost08}
appartiennent à cette famille. Quand la réduction modulo la
caractéristique introduit des divisions par $0$, ces algorithmes ne
s'appliquent plus.

Les deux premiers algorithmes pour calculer des isogénies en
caractéristique quelconque furent donnés par
Couveignes~\cite{couveignes94,couveignes96}; les deux ont complexité
polynomiale en la caractéristique, ce qui les rend peu pratiques pour
des valeurs supérieures à $2$ ou $3$. Un algorithme spécifique pour la
caractéristique $2$ fut donné par Lercier~\cite{lercier96}: en
pratique il est plus rapide que l'algorithme de Couveignes, mais sa
complexité n'est pas bien comprise.

Après la découverte de méthodes $p$-adiques alternatives à
SEA~\cite{satoh00,fouquet+gaudry+harley00}, l'intérêt pour le calcul
d'isogénies s'est estompé. Pourtant, deux algorithmes $p$-adiques pour
le calcul d'isogénies en caractéristique quelconque ont récemment été
proposés par Joux et Lercier~\cite{joux+lercier06} et Lercier et
Sirvent~\cite{lercier+sirvent08}; ils montrent qu'il est possible
d'éviter les divisions par $0$ en liftant les courbes dans les
$p$-adiques. Le second algorithme est actuellement celui qui a la
meilleure complexité dans le cas de la caractéristique quelconque, sa
dépendance en la caractéristique est seulement logarithmique.

Il est tout de même intéressant de remarquer qu'aucun algorithme pour
le calcul d'isogénies n'a une complexité optimale ou quasi-optimale,
avec la seule exception de~\cite{bostan+morain+salvy+schost08} dans un
cas très spécifique.

Le point de départ de ce travail a été le deuxième algorithme de
Couveignes~\cite{couveignes96}. Il calcule une isogénie par
interpolation sur les points de $p^k$-torsion de la courbe, pour $k$
assez grand; quand ces points ne sont pas définis sur le corps de
base, il faut travailler dans des extensions de corps pour les
trouver. Les extensions qui apparaissent naturellement dans ce calcul sont des corps de rupture de polynômes de la forme
\[X^p-X-\alpha\text{;}\] de telles extensions sont appelées
d'Artin-Schreier.

\paragraph*{Tours de corps finis}
\label{sec:tours-de-corps}
Mis à part l'addition, la multiplication et l'inversion, les
opérations arithmétiques importantes dans une tour d'extensions finies
sont sans aucun doute les traces relatives, les polynômes minimaux et
les inclusions de corps. Pour les corps finis, il est possible
d'ajouter des groupes de Galois effectifs à la liste, puisqu'il est
relativement facile de calculer avec ces objets.

L'arithmétique des tours de corps finis est une question de première
importance pour tout système de calcul formel, pourtant elle a reçu
peu ou pas d'attention. On sait que Magma permet de gérer des
diagrammes quelconques de corps finis depuis
longtemps~\cite{bosma+cannon+steel97}, mais il est difficile de dire
quels algorithmes y sont implantés de nos jours et avec quelles
complexités. Tous les autres résultats qui peuvent éventuellement
s'appliquer aux tours de corps finis ont été obtenus dans le contexte
plus général de la résolution de systèmes polynomiaux et de la
géométrie algébrique effective, en particulier pour la résolution des
ensembles
triangulaires~\cite{diaz+gonzalez01,giusti+lecerf+salvy01,bostan+salvy+schost03,pascal+schost06,li+moreno+schost07,dahan+jin+moreno+schost08,boulier+lemaire+moreno01,FGLM,rouiller99,alonso+becker+roy+wormann}.

Dans le cas spécifique des tours d'Artin-Schreier, il n'y a pas
énormément de littérature non plus. En s'appuyant sur des idées
contenues dans~\cite{Conway:ONAG2000}, Cantor~\cite{cantor89}
construisit une tour d'Artin-Schreier avec des propriétés spécifiques,
qu'il appliqua à la multiplication par FFT dans
$\F_2[X]$. Dans~\cite{couveignes00}, Couveignes donna un algorithme pour
le calcul d'isomorphismes entre tours d'Artin-Schreier; néanmoins, son
algorithme nécessite une multiplication rapide dans une tour, appelée
\og{}tour de Cantor\fg{} dans~\cite{couveignes00}, ayant la même forme que
celle de~\cite{cantor89}. Un tel algorithme n'est malheureusement pas
dans la littérature, ce qui rend les résultats de~\cite{couveignes00}
difficiles à exploiter en pratique.

\paragraph*{Le principe de transposition}
Un des outils algorithmiques que nous allons étudier en détail et
appliquer tout le long du document est le \emph{principe de
  transposition}, qui est à la théorie des langages ce que la dualité
est à l'algèbre.

Le principe de transposition fut découvert dans la théorie des
circuits électri\-ques par Bordewijk~\cite{bordewijk57}, puis prouvé
dans sa forme générale par Fiduccia~\cite{fiduccia:phd}; mais ce n'est
que bien plus tard, à travers les travaux de Kaltofen, Yagati, Shoup,
von zur Gathen et
autres~\cite{kaltofen+lakshman89,vzgathen+shoup92,shoup94,shoup95,shoup99,hanrot+quercia+zimmermann},
qu'il est devenu populaire en calcul formel. L'un des énoncés
possibles est le suivant:
\begin{quote}
  Soit $\pspace$ un ensemble quelconque. À tout algorithme
  $R$-algébrique, qui calcule une famille de fonctions linéaires
  $(f_p:M\ra N)_{p\in\pspace}$, correspond un algorithme
  $R$-algébrique $\dual{A}$ qui calcule la \emph{famille duale}
  $(\dual{f}_p:\dual{N}\ra\dual{M})_{p\in\pspace}$. Les complexités
  algébriques en temps et espace de $\dual{A}$ sont bornées par la
  complexité en temps de $A$.
\end{quote}

Le principe de transposition est important en calcul formel car il
permet d'obtenir des algorithmes asymptotiquement bons qui n'auraient
pas paru évi\-dents autrement. Un grand pas en avant dans sa
compréhension fut fait par Bostan, Lecerf et
Schost~\cite{bostan+lecerf+schost:tellegen} qui, en généralisant un
travail de Shoup~\cite{shoup95}, remarquèrent que la transposition
peut être appliquée de façon systématique à un langage de
programmation restreint. Il est aussi intéressant de remarquer que le
principe de transposition a des liens importants avec la
différentiation
automatique~\cite{baur+strassen83,kaltofen+lakshman89,Ka2K,gashkov+gashkov05,sergeev08}.

Dans ce document nous enquêtons plus en détail sur les rapports entre
la transposition et les langages de programmation. Nous travaillons
dans le cadre de la théorie des langages purement fonctionnels
typés~\cite{pierce}, car sa structure mathématique élégante nous
permet de raisonner sur les programmes à un niveau algébrique.

\paragraph*{Organisation du document, résultats}
Ce document est divisé en quatre parties. Dans la
partie~\ref{part:prerequisites} nous revenons sur les notions
fondamentales d'algèbre et calcul formel dont nous allons nous servir
par la suite.

La partie~\ref{part:transp-princ} a pour objet le principe de
transposition. Au Chapitre~\ref{cha:algebr-compl-dual} nous rappelons
le modèle des circuits arithmétiques et le modèle des programmes sans
branchements, puis prouvons le théorème de transposition pour
chacun. Ensuite nous évoquons les liens avec la différentiation
automatique. En complément, dans
l'Annexe~\ref{cha:basic-categ-theory}, nous donnons une nouvelle
preuve du théorème de transposition, à base de sémantique catégorique,
et étudions ses implications pour l'implantation d'un DSL en Haskell;
il s'agit un travail commun avec Mathieu Boespflug.

Le Chapitre~\ref{cha:autom-transp-code} est une collaboration avec
Éric Schost. Nous étudions les liens entre les circuits arithmétiques
et les langages fonctionnels, puis montrons que la transposition peut
être appliquée algorithmiquement à un langage fonctionnel générique.

La Partie~\ref{part:fast-arithm-using} est dédiée à l'arithmétique
dans les tours d'extensions. Nous commençons par rappeler la théorie
des idéaux zéro-dimensionnels et la représen\-tation univariée
rationnelle au Chapitre~\ref{cha:trace-computations}. Ici, les
résultats de la Partie~\ref{part:transp-princ} sont la clef pour
obtenir des algorithmes asymptotiquement rapides. Les algorithmes de
ce chapitre sont ensuite appliqués au
Chapitre~\ref{cha:artin-schr-towers}, où nous fournissons des
algorithmes asymptotiquement bons pour les tours d'Artin-Schreier
(fruit d'une autre collaboration avec Éric Schost).

Enfin, la Partie~\ref{part:appl-isog-comp} applique les résultats des
chapitres précédents au calcul d'isogénies. Après quelques rappels sur
les courbes elliptiques au Chapitre~\ref{cha:ellipt-curv-isog}, nous
passons en revue les algorithmes asymptotiquement meilleurs pour le
calcul d'isogénies sur les corps finis. Nous commençons par rappeler
l'algorithme BMSS pour le cas de la grande
caractéristique~\cite{bostan+morain+salvy+schost08} et sa
généralisation à la caractéristique quelconque de Lercier et
Sirvent~\cite{lercier+sirvent08}; puis nous rappelons l'algorithme
original de Couveignes~\cite{couveignes96} et présentons des variantes
améliorées avec un meilleur comportement asymptotique: les clefs pour
ces résultats sont le Chapitre~\ref{cha:artin-schr-towers} et de
nouvelles idées algorithmiques pour l'interpolation dans les tours
d'extensions. Nous présentons aussi en Section~\ref{sec:bounded} une
généralisation surprenante de l'algorithme de Couveignes, qui permet
le calcul d'isogénies de degré inconnu au même prix que le calcul
d'isogénies de degré prescrit. Cette découverte éclaire davantage la
(sous)optimalité de l'algorithme de Couveignes et pourrait avoir des
applications en
cryptologie~\cite{gaudry+hess+smart02,GHS,hess03,teske06}.

\pdfmctwo{Rachel trouvait que "ce manuscrit n'aurait pas d'intérêt"
  était une figure de style qui n'a pas sa place dans l'introduction
  d'une thèse, et que "nos paquets logiciels" faisait trop
  commercial.}  La théorie ne suffirait pas sans pratique. De la même
façon, ce manuscrit ne serait pas complet s'il n'était accompagné par
les paquets logiciels que nous avons développés. La grande majorité
des algorithmes présentés ici a été implantée, paquetée et distribuée
avec des licences \emph{open source}. Ainsi, tous les algorithmes du
Chapitre~\ref{cha:artin-schr-towers} sont disponibles dans la
bibliothèque \texttt{FAAST}, écrite en \texttt{C++} et disponible à
l'adresse \url{http://www.lix.polytechnique.fr/~defeo/FAAST/}.  Au
moment où nous écrivons, le compilateur pour le langage
\texttt{transalpyne} du Chapitre~\ref{cha:autom-transp-code} n'est pas
encore distribué; nous travaillons en ce moment à la première
\emph{stable release} et espérons commencer la distribution au début
de 2011. Il sera disponible à l'adresse
\url{ http://transalpyne.gforge.inria.fr/}.



\selectlanguage{american}

%%% Local Variables: 
%%% mode:flyspell
%%% ispell-local-dictionary:"francais"
%%% mode: TeX-PDF
%%% mode: reftex
%%% TeX-master: "../these"
%%% End: 

\chapter[Intoduction (English)][Introduction]{Introduction}

\pdfmcone{Removed my beautiful incipit.}
Finite field arithmetics are at the hearth of modern technology; this
is so true, that the last generation of Intel Core processors supports
a hardware instruction (CLMUL) for multiplication in
$\F_{2^m}$~\cite{intel-carryless}. The reason is that finite fields
appear everywhere in telecommunications engineering, in particular in
Error Correcting Codes and Cryptography. This thesis applies advanced
algorithmic and algebraic techniques to computations in towers of
extensions of finite fields, in view of applications to elliptic curve
cryptography.


\paragraph*{Elliptic curves}
In elliptic curve cryptography, in order to build a secure
cryptosystem, one must select a curve whose number of points contains
a large enough prime factor. The preferred method for doing this is to
randomly select a curve and then use a point-counting algorithm to
determine its cardinality. The first polynomial time point counting
algorithm for elliptic curves was due to Schoof~\cite{schoof85}, then
improved by Atkin and Elkies~\cite{atkin88,elkies98,schoof95},
henceforth named SEA.

The SEA algorithm raised interest in explicit computations with
isogenies, i.e.\ algebraic group morphisms of elliptic curves. When
computing isogenies over finite fields one must distinguish between
the large and arbitrary characteristic. In the first case, one can use
algorithms that work for characteristic $0$, and then reduce the
result; the methods of Elkies~\cite{elkies98,morain95},
Atkin~\cite{schoof95} and Bostan, Morain, Salvy and
Schost~\cite{bostan+morain+salvy+schost08} belong to this family. When
the reduction modulo the characteristic introduces division by $0$,
these algorithms are not of help.

The first two algorithms to compute isogenies in arbitrary
characteristic are due to Couveignes~\cite{couveignes94,couveignes96}:
both have a polynomial dependency in the characteristic, which makes
them unpractical for values higher than $2$ or $3$. An algorithm
specific to characteristic $2$ was given by Lercier~\cite{lercier96};
in practice it performs faster than Couveignes' algorithms, but its
complexity is not well understood. 

\pdfmcone{Rapidly recall the existence of p-adic methods.}
After the discovery of $p$-adic alternatives to the SEA
algorithm~\cite{satoh00,fouquet+gaudry+harley00} interest in computing
isogenies in small characteristic was lost.  Nevertheless, two
$p$-adic algorithms were recently proposed by Joux and
Lercier~\cite{joux+lercier06} and Lercier and
Sirvent~\cite{lercier+sirvent08} to solve the isogeny problem in
arbitrary characteristic. They show that it is possible to avoid
divisions by $0$ by lifting the curves in the $p$-adics. The last
algorithm is currently the one having the best asymptotic complexity
for the arbitrary characteristic case; its complexity in the
characteristic is only logarithmic.

It is interesting to remark, however, that no algorithm to compute
isogenies has optimal or quasi-optimal complexity, with the only
exception of~\cite{bostan+morain+salvy+schost08} on a very special
case. 

The starting point of this work was Couveignes' second
algorithm~\cite{couveignes96}. It computes an isogeny by interpolating
it over the $p^k$-torsion points of the elliptic curves for a large
enough $k$; when those points are not defined on the base field, one
has to take towers of field extensions to find them. The field
extensions that naturally arise when doing this computation are
splitting fields of polynomials of the form
\[X^p - X -\alpha\text{;}\] such extension are called Artin-Schreier
extensions. 


\paragraph*{Towers of finite fields}
Besides addition, multiplication and inversion, the arithmetic
operations of interest in a tower of finite extensions arguably are
relative traces, minimal polynomials and embeddings. For finite fields
one could add explicit Galois groups to the list as these are
relatively easy to compute with.

The arithmetic of towers of finite fields is a central question for
any computer algebra system, however it has received few attention, if
any. Magma is known for having had support for lattices of finite
fields for a long time~\cite{bosma+cannon+steel97}, but it is hard to
tell which algorithms it implements nowadays and what their
complexities are. All other results that can possibly apply to towers
of finite fields were derived in the more general context of
polynomial system solving and effective algebraic geometry, in
particular in the resolution of triangular
sets~\cite{diaz+gonzalez01,giusti+lecerf+salvy01,bostan+salvy+schost03,pascal+schost06,li+moreno+schost07,dahan+jin+moreno+schost08,boulier+lemaire+moreno01,FGLM,rouiller99,alonso+becker+roy+wormann}.

In the specific case of Artin-Schreier towers, the literature is not
extensive either.  Using ideas from~\cite{Conway:ONAG2000},
Cantor~\cite{cantor89} constructs a particular Artin-Schreier tower
that he applies to FFT multiplication in $\F_2[X]$.
In~\cite{couveignes00}, Couveignes gives an algorithm to compute
isomorphisms between Artin-Schreier towers; however, his algorithm
needs as a prerequisite a fast multiplication algorithm in a tower,
called a ``Cantor tower'' in~\cite{couveignes00}, having the same
shape as the one in~\cite{cantor89}. Such an algorithm is
unfortunately not in the literature, making the results
of~\cite{couveignes00} non practical.


\paragraph*{Transposition principle}
One algorithmic tool that we shall study in depth and apply throughout
the whole document is the \emph{transposition principle}, which is the
language-theoretic counterpart to algebraic duality.

The transposition principle was discovered in electrical network
theory by Bordewijk~\cite{bordewijk57}, then proved in its general
form by Fiduccia~\cite{fiduccia:phd}; but it only became popular in
computer algebra much later through the works of Kaltofen, Yagati,
Shoup, von zur Gathen and
others~\cite{kaltofen+lakshman89,vzgathen+shoup92,shoup94,shoup95,shoup99,hanrot+quercia+zimmermann}. One
possible statement is:
\begin{quote}
  Let $\pspace$ be an arbitrary set. To any $R$-algebraic algorithm
  $A$ computing a family of linear functions $(f_p:M\ra
  N)_{p\in\pspace}$ corresponds an $R$-algebraic algorithm $\dual{A}$
  computing the \emph{dual family}
  $(\dual{f}_p:\dual{N}\ra\dual{M})_{p\in\pspace}$. The algebraic time
  and space complexities of $\dual{A}$ are bounded by the time
  complexity of $A$.
\end{quote}

The transposition principle is important in computer algebra because
it permits to derive asymptotically good algorithms that were not
otherwise evident. One big step forward in the understanding of it was
done by Bostan, Lecerf and Schost~\cite{bostan+lecerf+schost:tellegen}
who, extending work of Shoup~\cite{shoup95}, remarked that
transposition can be systematically applied to a restricted
programming language. It is also remarkable that the transposition
principle has a strong connection with automatic
differentiation~\cite{baur+strassen83,kaltofen+lakshman89,Ka2K,gashkov+gashkov05,sergeev08}.

In this document we investigate more in depth the relationships
between the transposition principle and programming languages. We use
the theory of typed purely functional languages~\cite{pierce} as
framework, because its elegant mathematical structure permits to
reason at an algebraic level on programs.


\paragraph*{Outline of our contributions}
This document is divided in four parts. Part~\ref{part:prerequisites}
recalls the basic notions from algebra and computer algebra that we
will use later.

Part~\ref{part:transp-princ} studies the transposition principle. In
Chapter~\ref{cha:algebr-compl-dual} we review the arithmetic circuit
model and the straight line program model, and prove the transposition
theorem in them. Then we discuss the relationships with automatic
differentiation. As a complement, in
Appendix~\ref{cha:basic-categ-theory} we also give a new proof of the
transposition theorem, using categorical semantics, and discuss its
consequences on the implementation of a DSL in Haskell; this is joint
work with Boespflug.

Chapter~\ref{cha:autom-transp-code} is a collaboration with Schost. We
study the relationships between the arithmetic circuit model and
functional programming languages, then we show that transposition can
be applied algorithmically to a generic functional language. 

Part~\ref{part:fast-arithm-using} is devoted to arithmetics in towers
of extensions. We start by reviewing the general theory of
zero-dimensional ideals and rational univariate representations in
Chapter~\ref{cha:trace-computations}. Here, the results of
Part~\ref{part:transp-princ} are the key to obtain asymptotically fast
algorithms. The algorithms of this chapter are then applied in
Chapter~\ref{cha:artin-schr-towers}, where we provide asymptotically
good algorithms for Artin-Schreier towers (fruit of another
collaboration with Schost).

\pdfmcone{More emphasis on C2-UD and its cryptographic interest.}
Finally Part~\ref{part:appl-isog-comp} applies the results of the
previous chapters to isogeny computation. After some general
references on elliptic curves in Chapter~\ref{cha:ellipt-curv-isog},
we review in Chapter~\ref{cha:algor-small-char} the asymptotically
fastest algorithms to compute isogenies over finite fields.  We start
by reviewing the BMSS algorithm for large
characteristic~\cite{bostan+morain+salvy+schost08} and its
generalization for arbitrary characteristic by Lercier and
Sirvent~\cite{lercier+sirvent08}; then we review Couveignes' original
algorithm~\cite{couveignes96}, and present some improved variants with
better asymptotic behavior: the key to this results are
Chapter~\ref{cha:artin-schr-towers} and new ideas on interpolation in
towers of extensions.  We also present in Section~\ref{sec:bounded} a
surprising generalization of Couveignes' algorithm that allows to
compute isogenies of unknown degree at the same cost of computing an
isogeny of a given degree.  This discovery sheds new light on the
(sub)optimality of Couveignes' algorithm and can possibly find
applications in
cryptology~\cite{gaudry+hess+smart02,GHS,hess03,teske06}.

Theory would be meaningless without practice. Similarly, this
manuscript would make no sense if it was not accompanied by our
software packages. The great majority of the algorithms we present
here have been implemented, packaged and distributed under open source
licences. So, all the algorithms of
Chapter~\ref{cha:artin-schr-towers} can be found in the \texttt{C++}
library \texttt{FAAST}, available from
\url{http://www.lix.polytechnique.fr/~defeo/FAAST/}. At the
moment we write, the compiler for the language \texttt{transalpyne} of
Chapter~\ref{cha:autom-transp-code} is not distributed yet; we are
currently working on the first stable release and hope to start
distributing it by the beginning of 2011. It will be available from
\url{ http://transalpyne.gforge.inria.fr/}.




%%% Local Variables: 
%%% mode:flyspell
%%% ispell-local-dictionary:"american"
%%% mode: TeX-PDF
%%% mode: reftex
%%% TeX-master: "../these"
%%% End: 

\tableofcontents
%\listoffigures
%\listoftables
\listofalgorithms

%\mainmatter
\chapter[Intoduction (English)][Introduction]{Introduction}

\pdfmcone{Removed my beautiful incipit.}
Finite field arithmetics are at the hearth of modern technology; this
is so true, that the last generation of Intel Core processors supports
a hardware instruction (CLMUL) for multiplication in
$\F_{2^m}$~\cite{intel-carryless}. The reason is that finite fields
appear everywhere in telecommunications engineering, in particular in
Error Correcting Codes and Cryptography. This thesis applies advanced
algorithmic and algebraic techniques to computations in towers of
extensions of finite fields, in view of applications to elliptic curve
cryptography.


\paragraph*{Elliptic curves}
In elliptic curve cryptography, in order to build a secure
cryptosystem, one must select a curve whose number of points contains
a large enough prime factor. The preferred method for doing this is to
randomly select a curve and then use a point-counting algorithm to
determine its cardinality. The first polynomial time point counting
algorithm for elliptic curves was due to Schoof~\cite{schoof85}, then
improved by Atkin and Elkies~\cite{atkin88,elkies98,schoof95},
henceforth named SEA.

The SEA algorithm raised interest in explicit computations with
isogenies, i.e.\ algebraic group morphisms of elliptic curves. When
computing isogenies over finite fields one must distinguish between
the large and arbitrary characteristic. In the first case, one can use
algorithms that work for characteristic $0$, and then reduce the
result; the methods of Elkies~\cite{elkies98,morain95},
Atkin~\cite{schoof95} and Bostan, Morain, Salvy and
Schost~\cite{bostan+morain+salvy+schost08} belong to this family. When
the reduction modulo the characteristic introduces division by $0$,
these algorithms are not of help.

The first two algorithms to compute isogenies in arbitrary
characteristic are due to Couveignes~\cite{couveignes94,couveignes96}:
both have a polynomial dependency in the characteristic, which makes
them unpractical for values higher than $2$ or $3$. An algorithm
specific to characteristic $2$ was given by Lercier~\cite{lercier96};
in practice it performs faster than Couveignes' algorithms, but its
complexity is not well understood. 

\pdfmcone{Rapidly recall the existence of p-adic methods.}
After the discovery of $p$-adic alternatives to the SEA
algorithm~\cite{satoh00,fouquet+gaudry+harley00} interest in computing
isogenies in small characteristic was lost.  Nevertheless, two
$p$-adic algorithms were recently proposed by Joux and
Lercier~\cite{joux+lercier06} and Lercier and
Sirvent~\cite{lercier+sirvent08} to solve the isogeny problem in
arbitrary characteristic. They show that it is possible to avoid
divisions by $0$ by lifting the curves in the $p$-adics. The last
algorithm is currently the one having the best asymptotic complexity
for the arbitrary characteristic case; its complexity in the
characteristic is only logarithmic.

It is interesting to remark, however, that no algorithm to compute
isogenies has optimal or quasi-optimal complexity, with the only
exception of~\cite{bostan+morain+salvy+schost08} on a very special
case. 

The starting point of this work was Couveignes' second
algorithm~\cite{couveignes96}. It computes an isogeny by interpolating
it over the $p^k$-torsion points of the elliptic curves for a large
enough $k$; when those points are not defined on the base field, one
has to take towers of field extensions to find them. The field
extensions that naturally arise when doing this computation are
splitting fields of polynomials of the form
\[X^p - X -\alpha\text{;}\] such extension are called Artin-Schreier
extensions. 


\paragraph*{Towers of finite fields}
Besides addition, multiplication and inversion, the arithmetic
operations of interest in a tower of finite extensions arguably are
relative traces, minimal polynomials and embeddings. For finite fields
one could add explicit Galois groups to the list as these are
relatively easy to compute with.

The arithmetic of towers of finite fields is a central question for
any computer algebra system, however it has received few attention, if
any. Magma is known for having had support for lattices of finite
fields for a long time~\cite{bosma+cannon+steel97}, but it is hard to
tell which algorithms it implements nowadays and what their
complexities are. All other results that can possibly apply to towers
of finite fields were derived in the more general context of
polynomial system solving and effective algebraic geometry, in
particular in the resolution of triangular
sets~\cite{diaz+gonzalez01,giusti+lecerf+salvy01,bostan+salvy+schost03,pascal+schost06,li+moreno+schost07,dahan+jin+moreno+schost08,boulier+lemaire+moreno01,FGLM,rouiller99,alonso+becker+roy+wormann}.

In the specific case of Artin-Schreier towers, the literature is not
extensive either.  Using ideas from~\cite{Conway:ONAG2000},
Cantor~\cite{cantor89} constructs a particular Artin-Schreier tower
that he applies to FFT multiplication in $\F_2[X]$.
In~\cite{couveignes00}, Couveignes gives an algorithm to compute
isomorphisms between Artin-Schreier towers; however, his algorithm
needs as a prerequisite a fast multiplication algorithm in a tower,
called a ``Cantor tower'' in~\cite{couveignes00}, having the same
shape as the one in~\cite{cantor89}. Such an algorithm is
unfortunately not in the literature, making the results
of~\cite{couveignes00} non practical.


\paragraph*{Transposition principle}
One algorithmic tool that we shall study in depth and apply throughout
the whole document is the \emph{transposition principle}, which is the
language-theoretic counterpart to algebraic duality.

The transposition principle was discovered in electrical network
theory by Bordewijk~\cite{bordewijk57}, then proved in its general
form by Fiduccia~\cite{fiduccia:phd}; but it only became popular in
computer algebra much later through the works of Kaltofen, Yagati,
Shoup, von zur Gathen and
others~\cite{kaltofen+lakshman89,vzgathen+shoup92,shoup94,shoup95,shoup99,hanrot+quercia+zimmermann}. One
possible statement is:
\begin{quote}
  Let $\pspace$ be an arbitrary set. To any $R$-algebraic algorithm
  $A$ computing a family of linear functions $(f_p:M\ra
  N)_{p\in\pspace}$ corresponds an $R$-algebraic algorithm $\dual{A}$
  computing the \emph{dual family}
  $(\dual{f}_p:\dual{N}\ra\dual{M})_{p\in\pspace}$. The algebraic time
  and space complexities of $\dual{A}$ are bounded by the time
  complexity of $A$.
\end{quote}

The transposition principle is important in computer algebra because
it permits to derive asymptotically good algorithms that were not
otherwise evident. One big step forward in the understanding of it was
done by Bostan, Lecerf and Schost~\cite{bostan+lecerf+schost:tellegen}
who, extending work of Shoup~\cite{shoup95}, remarked that
transposition can be systematically applied to a restricted
programming language. It is also remarkable that the transposition
principle has a strong connection with automatic
differentiation~\cite{baur+strassen83,kaltofen+lakshman89,Ka2K,gashkov+gashkov05,sergeev08}.

In this document we investigate more in depth the relationships
between the transposition principle and programming languages. We use
the theory of typed purely functional languages~\cite{pierce} as
framework, because its elegant mathematical structure permits to
reason at an algebraic level on programs.


\paragraph*{Outline of our contributions}
This document is divided in four parts. Part~\ref{part:prerequisites}
recalls the basic notions from algebra and computer algebra that we
will use later.

Part~\ref{part:transp-princ} studies the transposition principle. In
Chapter~\ref{cha:algebr-compl-dual} we review the arithmetic circuit
model and the straight line program model, and prove the transposition
theorem in them. Then we discuss the relationships with automatic
differentiation. As a complement, in
Appendix~\ref{cha:basic-categ-theory} we also give a new proof of the
transposition theorem, using categorical semantics, and discuss its
consequences on the implementation of a DSL in Haskell; this is joint
work with Boespflug.

Chapter~\ref{cha:autom-transp-code} is a collaboration with Schost. We
study the relationships between the arithmetic circuit model and
functional programming languages, then we show that transposition can
be applied algorithmically to a generic functional language. 

Part~\ref{part:fast-arithm-using} is devoted to arithmetics in towers
of extensions. We start by reviewing the general theory of
zero-dimensional ideals and rational univariate representations in
Chapter~\ref{cha:trace-computations}. Here, the results of
Part~\ref{part:transp-princ} are the key to obtain asymptotically fast
algorithms. The algorithms of this chapter are then applied in
Chapter~\ref{cha:artin-schr-towers}, where we provide asymptotically
good algorithms for Artin-Schreier towers (fruit of another
collaboration with Schost).

\pdfmcone{More emphasis on C2-UD and its cryptographic interest.}
Finally Part~\ref{part:appl-isog-comp} applies the results of the
previous chapters to isogeny computation. After some general
references on elliptic curves in Chapter~\ref{cha:ellipt-curv-isog},
we review in Chapter~\ref{cha:algor-small-char} the asymptotically
fastest algorithms to compute isogenies over finite fields.  We start
by reviewing the BMSS algorithm for large
characteristic~\cite{bostan+morain+salvy+schost08} and its
generalization for arbitrary characteristic by Lercier and
Sirvent~\cite{lercier+sirvent08}; then we review Couveignes' original
algorithm~\cite{couveignes96}, and present some improved variants with
better asymptotic behavior: the key to this results are
Chapter~\ref{cha:artin-schr-towers} and new ideas on interpolation in
towers of extensions.  We also present in Section~\ref{sec:bounded} a
surprising generalization of Couveignes' algorithm that allows to
compute isogenies of unknown degree at the same cost of computing an
isogeny of a given degree.  This discovery sheds new light on the
(sub)optimality of Couveignes' algorithm and can possibly find
applications in
cryptology~\cite{gaudry+hess+smart02,GHS,hess03,teske06}.

Theory would be meaningless without practice. Similarly, this
manuscript would make no sense if it was not accompanied by our
software packages. The great majority of the algorithms we present
here have been implemented, packaged and distributed under open source
licences. So, all the algorithms of
Chapter~\ref{cha:artin-schr-towers} can be found in the \texttt{C++}
library \texttt{FAAST}, available from
\url{http://www.lix.polytechnique.fr/~defeo/FAAST/}. At the
moment we write, the compiler for the language \texttt{transalpyne} of
Chapter~\ref{cha:autom-transp-code} is not distributed yet; we are
currently working on the first stable release and hope to start
distributing it by the beginning of 2011. It will be available from
\url{ http://transalpyne.gforge.inria.fr/}.




%%% Local Variables: 
%%% mode:flyspell
%%% ispell-local-dictionary:"american"
%%% mode: TeX-PDF
%%% mode: reftex
%%% TeX-master: "../these"
%%% End: 


\part{Prerequisites}

\chapter{Algebra}
\label{cha:algebra}
%% these.tex
%% Copyright 2010 Luca De Feo
%% All rights reserved


Here we recall the concepts from abstract algebra that will constitute
the background for all the chapters that follow. One chapter is
certainly not enough to present such a vast subject, hence we just
recall the few definitions and properties that will help the reader
understand the results presented in this document. The material of
this chapter is mainly drawn from
\cite{lang,lidl+niederreiter:2,silverman:elliptic}.

\section{Linear algebra}
\label{sec:linear-algebra}
In Part~\ref{part:transp-princ} we shall apply some classical linear
algebraic tools to free modules over non-commutative ring. We recall
here the fundamental concepts.


\subsection{Bra-ket notation}
\label{sec:linear-algebra:bra-ket}

It will be convenient to (ab)use Dirac's
\index{bra-ket~notation}bra-ket notation to represent elements of
modules. If $(M,+,\cdot)$ is a left $R$-module and $x\in (M,+)$ is an
element of its underlying group, by $\ket{x}_R$ we mean the element
obtained by lifting $x$ in $(M,+,\cdot)$. We call
$\ket{x}$\symb[braket-1]{$\ket{x}$}{Ket, element of a left module} a
\index{ket}\emph{ket} and read it as ``ket x''.

The external multiplication by an element $a\in R$ will be written
$a\ket{x}_R$; if $f:M\ra N$ is a left module morphism, we write
$f\ket{x}_R$ for $f(\ket{x}_R)$. By a slight abuse of notation we may
write $\ket{a x}_R$ and $\ket{f(x)}_R$ for $a\ket{x}_R$ and
$f\ket{x}_R$ respectively. When $R$ is clear from the context, a ket
can be simply written as $\ket{x}$.

\symb[braket-2]{$\bra{x}$}{Bra, element of a right module} In a
symmetric way, elements of right $R$-modules will be written
${}_R\!\bra{x}$, which we call a \index{bra}\emph{bra} and read as
``bra x''. External multiplication will be written as ${}_R\!\bra{x}a$
and application of a right module morphism as ${}_R\!\bra{x}f$.

Let $M$ be a right module and $N$ a left module. A
\index{bilinear~form}\emph{bilinear
  form}\symb[braket-3]{$\braket{x}{y}_f$}{Bilinear form} on
$M\times N$ is a map $f:M\times N\ra R$ such that for any $x\in M$,
the map
\[\ket{y}\mapsto f(x, y)\]
is a left module morphism, and for any $y\in N$, the map
\[\bra{x}\mapsto f(x, y)\]
is a right module morphism. If $f$ is a bilinear form, we write
$\braket{x}{y}_f$ for $f(x,y)$, or simply $\braket{x}{y}$ when $f$ is
clear from the context. Note that textbooks usually define bilinear
forms only when $R$ is commutative, in our more general setting some
common properties of bilinear forms fail to hold, for example
\pdfmcthree{Changed inequality in "not necessarily".}
$\braket{xa}{y}$ is not necessarily equal to $\braket{x}{ay}$.

If $M$ is a left (right) module, we denote by $\dual{M}=\hom(M,R)$ the
\index{dual~module}\emph{dual
  module}\symb[dual]{$\dual{M}$}{Dual of a module or vector
  space: $\dual{M}=\hom(M,R)$} of $M$, it is a right (left) module.
Any bilinear form $f$ gives rise to a morphism $ \phi_f : M \ra
\dual{N}$ of right modules where $\bra{x}\phi_f$ is the linear form $y
\mapsto \braket{x}{y}$. Similarly, $f$ gives rise to a morphism
$\phi^f:N\ra\dual{M}$ of left modules. The maps $f\mapsto\phi_f$,
$f\mapsto\phi^f$ and their obvious inverses induce group isomorphisms
between $\hom(M,\dual{N})$, $\hom(N,\dual{M})$ and the group of
bilinear forms on $M\times N$.  A bilinear form $f$ is said to be
\index{bilinear~form!degenerate}\emph{non-degenerate} if $\phi_f$
and $\phi^f$ are module isomorphisms.
\pdfmcone{The distinction
  non-singular/non-degenerate comes from Lang, but it is not very
  standard, indeed. I removed the distinction: now I call
  non-degenerate what I used to call non-singular before.}


\subsection{Matrices and morphisms}
\label{sec:linear-algebra:matrices}
$M=M_1\oplus\cdots\oplus M_n$ be a left module and
$N=N_1\oplus\cdots\oplus N_m$ be a right module.  Let $\iota_i$ be the
injections $M_i\ra M$ and let $\pi_j$ be the projections $N\ra N_i$,
then a linear map $f:M\ra N$ is uniquely determined by the maps
$\pi_j\circ f\circ\iota_i$. If we consider $m\times n$ matrices whose
$(j,i)$-th coefficient is in $\hom(M_i,N_j)$, then we verify that
there is a group isomorphism between $\hom(M,N)$ and this group of
matrices. Furthermore, let $f:M\ra N$ and $g:N\ra O$ and let $M_f$ and
$M_g$ be the matrices that are associated respectively, then the
matrix associated to $g\circ f$ is $M_gM_f$, where the product of two
entries is defined as composition of morphisms. This induces a ring
isomorphism between $\End(M)$ and the ring of square matrices with
entries in $\hom(M_i,M_j)$.

Consider $R$ as an $R$-module over itself, a linear map from $R$ to
itself is uniquely determined by the image of $1$, hence $\End(R)\isom
R^\op$. As a consequence, there is a group isomorphism
$\hom(R^n,R^m)\isom\Mat_{m\times n}(R^\op)$, and matrix multiplication
is equivalent to composition as above.  Hence, if $M$ is a free
module, for any fixed basis $\basis{B}$ of cardinality $n$ we have an
isomorphism of rings $\End_R(M)\isom\Mat_n(R^\op)$; in particular
$\Aut(M)\isom\GL_n(R^\op)$ as groups.

Let $R$ be commutative, then $R^\op = R$. We denote by
$M_{\basis{B}}(f)$ the matrix associated to $f\in\End_R(M)$ with
respect to the basis $\basis{B}$.  If $\basis{B'}$ is another basis,
it has the same cardinality as $\basis{B}$. Then, there is an
invertible matrix $B$ such that $A\mapsto B^{-1}AB$ is the
automorphism of $\Mat_n(R)$ that sends $M_{\basis{B}}(f)$ over
$M_{\basis{B'}}(f)$. Hence, any property of matrices that is invariant
by similarity, can be defined for linear operators. We define the
\index{trace!of~an~operator} \index{trace}\emph{trace} of a linear
operator as $\Tr f = \Tr M(f)$\symb[Tr]{$\Tr$}{Trace of a
  matrix, of a linear operator}, and its
\index{determinant}\emph{determinant} as $\det f = \det
M(f)$\symb[det]{$\det$}{Determinant of a matrix, of a linear
  operator}.


\subsection{Duality}
\label{sec:linear-algebra:duality}
We fix a non-degenerate bilinear form $f$ on $M\times N$. Let
$g\in\End(M)$, then the map
\[(x,y) \mapsto \braket{g(x)}{y}_f\] is a bilinear form. On the other
hand, let $h$ be a bilinear form on $M\times N$, for any $x\in M$ the
map $h_x : \ket{y} \mapsto \braket{x}{y}_h$ is a linear form on $N$,
thus $h_x\in\dual{N}$. From the non-degeneracy of $f$ we deduce that
there is an unique element $x'\in M$ such that
$\braket{x'}{y}_f=\braket{x}{y}_h$ and it is clear that the map
$\bra{x}\mapsto \bra{x'}$ is an endomorphism of $M$. It is evident
that the two maps are each other's inverse, thus we have a group
isomorphism between $\End(M)$ and the group of bilinear forms. An
analogous argument shows that $\End(N)$ is isomorphic to the group of
bilinear forms and ultimately $\End(M)\isom\End(N)$.

A consequence of this is that for any linear operator $g\in\End(M)$
there is an operator $\dual{g}\in\End(N)$ such that
\[\braket{g(x)}{y}_f = \braket{x}{\dual{g}(y)}_f\]
for any $x\in M$ and $y\in N$. We define similarly $\dual{h}$ when
$h\in\End(N)$, obviously $\dual{(\dual{g})} = g$. The operator
$\dual{g}$ is called the
\index{dual~operator}\emph{dual}\symb[dual]{$\dual{g}$}{Dual
  operator: $\braket{g(x)}{y}=\braket{x}{\dual{g}(y)}$} of $g$ with
respect to $f$. In general, whenever it is clear from the context that
$g$ belongs to $\End(M)$ (or to $\End(N)$), we simply
write\symb[braket-4]{$\braketop{x}{g}{y}$}{Bilinear form with linear operator}
\[\braketop{x}{g}{y} \eqdef \braket{g(x)}{y} = \braket{x}{\dual{g}(y)}\text{.}\]

More generally, Let $f:M\times M'\ra R$ and $g:N'\times N\ra R$ be two
non-degenerate bilinear forms, by the same technique as above we can
show that there is a group isomorphism between $\hom_R(N,M')$,
$\hom(M,N')$ and the bilinear forms on $M\times N$. Then, for any
$h:N\ra M'$ there is an unique $\dual{h}:M\ra N'$ such that
\[\braketop{x}{h}{y}\eqdef\braket{x}{h(y)}_f = \braket{\dual{h}(x)}{y}_g 
\text{.}\]
We also call $\dual{h}$ the \emph{dual} of $h$.

The canonical example of non-degenerate bilinear forms is obtained by
considering the family of forms on $\dual{M}\times M$ defined by
\[\braket{\ell}{x} = \ell(x) \text{.}\]
For any $f:M\ra N$, we define the dual map
$\dual{f}:\dual{N}\ra\dual{M}$ as the map that sends a form
$\ell\in\dual{N}$ over the form $\ell\circ f$ in $\dual{M}$; it is
easy to verify that
\[\braketop{\ell}{f}{x} =  \braket{\ell}{f(x)} = \braket{\dual{f}(\ell)}{x}
= \ell(f(x))\text{.}\]


\pdfmcone{Introduced definition of dual basis, clarified messy
  comments about columns and vectors.}  If $M$ is a free module and
$\basis{B}=\{\basis{e}_1,\ldots,\basis{e}_n\}$ a basis, the
\index{dual~basis}\emph{dual
  basis}\symb[dual]{$\dual{\basis{B}}$}{Dual basis}
$\dual{\basis{B}}$ is the unique basis
$\{\dual{\basis{e}_1},\ldots,\dual{\basis{e}_1}\}$ of $\dual{M}$ such
that
\begin{equation*}
  \braket{\dual{\basis{e}_i}}{\basis{e}_j} =
  \begin{cases}
    1 &\text{if $i=j$,}\\
    0 &\text{if $i\ne j$.}
  \end{cases}
\end{equation*}
If elements of $M$ and $\dual{M}$ are represented, respectively, as
vectors over $\basis{B}$ and $\dual{\basis{B}}$, then the bilinear
form $\braket{\ell}{x}=\ell(x)$ is given by the inner product
\begin{equation*}
  \left\langle\begin{matrix}
      \ell_1 &\cdots & \ell_n
    \end{matrix}\right\rvert
  \left\lvert\begin{matrix}
    x_1\\
    \vdots\\
    x_n
  \end{matrix}\right\rangle
  =
  \sum_i x_i\ell_i
\end{equation*}
(notice how the product is swapped, this is because $\End(R)\isom
R^\op$).  Now, if $M$ and $N$ are free modules with a fixed basis, a
linear map $f:M\ra N$ is isomorphic to a matrix with entries in
$R^\op$. Then the application $f\ket{x}$ is just matrix-vector
multiplication, while $\bra{\ell}\dual{f}$ is vector-matrix
multiplication by the same matrix. This justifies the notation
$\braketop{\ell}{A}{x}$ where $A$ is the matrix associated to
$f$.


\section{Basic Galois theory}
\label{sec:basic-galois-theory}
In Parts~\ref{part:fast-arithm-using} and~\ref{part:appl-isog-comp} we
shall need some basic Galois theory of finite fields. We recall here
the general concepts.

\subsection{Galois extensions}
\label{sec:basic-galois-theory:galois-extensions}
\pdfmcone{Removed pastiche about splitting fields} Let $\K$ be a
field. The \index{splitting~field}\emph{splitting field} of a family
of polynomials $(Q_i)_{i\in I}$ in $\K[X]$ is defined as an extension
$\LK$ of $\K$ where all the $Q_i$'s factor completely into linear
factors, and such that $\LK$ is generated over $\K$ by the roots of
the $Q_i$; the splitting field is unique up to isomorphism. An
algebraic field extension $\LK/\K$ such that $\LK$ is the splitting
field of a family of polynomials in $\K[X]$ is called a
\index{normal~field~extension}\emph{normal extension}.

Let $\LK/\K$ be an algebraic field extension, an element $x\in\LK$ is
said to be \index{separable!element}\emph{separable} over $\K$ if
its minimal polynomial over $\K$ has no multiple roots in $\LK$.
$\LK/\K$ is said to be \index{separable!field~extension}
\emph{separable} if every $x\in\LK$ is separable over $\K$. An
algebraic field extension is said to be a
\index{Galois~field~extension}\emph{Galois extension} if it is both
separable and normal.

\begin{theorem}
  Let $\LK/\K$ be a finite Galois extension, then there exists an
  element $x\in\LK$, called a
  \index{primitive~element}\emph{primitive element}, such that
  $\LK\isom\K[x]$.
\end{theorem}

Let $\LK/\K$ be a Galois extension, the group of automorphisms of
$\LK$ that fix $\K$ is called the \index{Galois~group}\emph{Galois
  group} of $\LK/\K$ and is denoted by
$\Gal(\LK/\K)$\symb[Gal]{$\Gal(\LK/\K)$}{Galois group}.  Let
$G$ be a group of automorphisms of a field $\K$, by
$\K^G$\symb[KG]{$\K^G$}{Fixed field, the subfield of $\K$
  fixed by the action of $G$} we denote the subfield of $\K$
consisting in the elements such that $\sigma(x)=x$ for any $\sigma\in
G$. Obviously, $\K=\LK^{\Gal(\LK/\K)}$.

\begin{theorem}
  Let $\LK/\K$ be a finite Galois extension. Let $H$ be a subgroup of
  $G=\Gal(\LK/\K)$, the map $H\mapsto \LK^H$ is a bijection between
  the subgroups of $G$ and the subfields of $\LK$ containing $\K$. The
  extension $\LK^H/\K$ is Galois if and only if $H$ is a normal
  subgroup of $G$; in this case its Galois group is isomorphic to
  $G/H$.
\end{theorem}

Let $\LK/\K$ be a Galois extension and let $x\in\LK$. The elements
$\sigma(x)$ for $\sigma\in\Gal(\LK/\K)$ are called the
\index{conjugate~element}\emph{conjugates} of $x$ under the action
of $\Gal(\LK/\K)$; they are the roots of the minimal polynomial of $x$
over $\K$.

Let $\K$ be a field, an element $x\in \K$ such that $x^n=1$ is called
an $n$-th \index{root~of~unity}\emph{root of unity}. If the
characteristic of $\K$ does not divide $n$, the polynomial $X^n-1$ has
$n$ distinct roots in $\clot{\K}$ and they form a multiplicative
group, denoted by $\mu_n$\symb[mu]{$\mu_n$}{Group of the $n$-th roots
  of unity}; it is a cyclic group, its generators are called the
\index{root~of~unity!primitive} \emph{primitive} roots of unity.  If
$\K$ has characteristic $p>0$, then $X^{p^m}-1$ has only one root,
namely $1$, thus $\mu_{p^m}$ is the trivial group.

The \index{Euler~function}\emph{Euler function}
$\euler:\N\ra\N$\symb[f]{$\euler$}{Euler totient function} is
defined as
\begin{align*}
  \euler(1) &= 1\text{,}\\
  \euler(p^r) &= p^{r-1}(p-1) &\text{for $p$ prime, $r\ge1$,}\\
  \euler(nm) &= \euler(n)\euler(nm) &\text{when $\gcd(n,m)=1$.}
\end{align*}
The Euler function counts the number of generators of the cyclic group
with $n$ elements, thus, when the characteristic of the field does not
divide $n$, the number of primitive roots of unity is equal to $\euler(n)$.

\begin{theorem}
  Let $x$ be a primitive $n$-th root of unity in an algebraic closure
  of $\Q$, then
  \[[\Q(x):\Q]=\euler(n)\text{.}\]
\end{theorem}

If $x$ is an $n$-th root of unity, its minimal polynomial over $\Q$ is
called the $n$-th \index{cyclotomic~polynomial}\emph{cyclotomic
  polynomial} and is denoted by
$\Cyclo_n$\symb[f]{$\Cyclo_n$}{$n$-th Cyclotomic polynomial};
it is a monic polynomial with coefficients in $\Z$. $\Cyclo_n$ is an
irreducible factor of $X^n-1$ over $\Q$, its roots are all the
primitive $n$-th roots of unity, hence $\deg\Cyclo_n=\euler(n)$.

\pdfmcone{Removed the Galois requirement.}  Let $\LK/\K$ be a finite
extension and let $x\in\LK$, the map $M_x:a\mapsto xa$ is an
automorphism of the $\K$-vector space $\LK$. The minimal polynomial of
its matrix with respect to any basis is equal to the minimal
polynomial of $x$ over $\K$.  The trace of $M_x$ is called the
\index{trace!of~a~field~extension}\emph{trace} of $x$ and is denoted
by $\Tr_{\LK/\K}(x)$\symb[Tr]{$\Tr_{\LK/\K}$}{Trace of a field
  extension}; its determinant is called the \index{norm}\emph{norm} of
$x$ and is denoted by
$\Norm_{\LK/\K}(x)$\symb[Norm]{$\Norm_{\LK/\K}$}{Norm of a field
  extension}.

\begin{proposition}
  \label{th:basic-galois-theory:trace}
  \pdfmcthree{"finite extension" is more precise than "field extension".}
  Let $\LK/\K$ and $\K/k$ be finite extensions and let
  $G=\Gal(\LK/\K)$. We have the following identities
  \begin{align*}
    \Tr_{\LK/\K}(x) &= \sum_{\sigma\in G}\sigma(x) \text{,}&
    \Tr_{\LK/k} &= \Tr_{\K/k}\circ\Tr_{\LK/\K}\text{,}\\
    \Norm_{\LK/\K}(x) &= \prod_{\sigma\in G}\sigma(x) \text{,}&
    \Norm_{\LK/k} &= \Norm_{\K/k}\circ\Norm_{\LK/\K}\text{.}
  \end{align*}
  The trace is a morphism of $\K$-vector spaces from $\LK$ to $\K$,
  the norm is a multiplicative morphism of groups from $\LK^\ast$ to
  $\K^\ast$.
\end{proposition}


\subsection{Finite fields}
\label{sec:basic-galois-theory:finite-fields}

Let $\K$ be a \index{finite~field} \emph{finite field}. It has
necessarily characteristic $p>0$, thus it must contain $\Z/p\Z$ as a
subfield. $\Z/p\Z$ is called the \index{prime~field}\emph{prime
  field} of $\K$ and is denoted by $\F_p$. Since $\K$ is a vector
space over $\F_p$, it must have cardinality $q=p^n$ for some $n$,
hence its multiplicative group has order $q-1$.

As a consequence, the elements of $\K^\ast$ must be roots of the
polynomial $X^{q-1}-1$. The fact that $p$ does not divide $q-1$
implies that $\K$ is isomorphic to $\F_p[\zeta]$, where $\zeta$ is a
primitive $(q-1)$-th root of unity in $\clot{\F}_p$. This implies
that, up to isomorphism, there is an unique finite field containing
$q$ elements, we denote by $\F_q$\symb[FiniteField]{$\F_q$}{Finite
  field of cardinality $q$} this field.

\pdfmcthree{Slightly reformulated.}
Using the same arguments, it is easy to show that for any $m\ge 1$,
$\F_{q^m}$ contains a subfield isomorphic to $\F_q$.  The map
$\frob_q:\F_{q^m}\ra\F_{q^m}$ sending $x\mapsto
x^q$\symb[f]{$\frob_q$, $\frob$}{Frobenius automorphism} is a morphism
of fields that fixes $\F_q$, it is called the
\index{Frobenius~automorphism}\emph{Frobenius automorphism} of
$\F_{q^m}/\F_q$. We now give the main result about the Galois theory
of finite fields.

\begin{proposition}
  The Galois group of $\F_{q^m}/\F_q$ is a cyclic group of order $m$;
  it is generated by the Frobenius automorphism $\frob_q$.
\end{proposition}


\section{Basic algebraic geometry}
\label{sec:basic-algebr-geom}

\subsection{Noetherian rings}
\label{sec:noetherian-rings}
A ring $R$ is called \index{Noetherian~ring}\emph{Noetherian} if any
ascending chain of ideals eventually terminates. Being Noetherian is a
very stable condition: fields and principal ideal domains, quotients
of Noetherian rings, rings of polynomials in finitely many variables
over a Noetherian ring, are all Noetherian. In particular, all the
rings we will work with in this document are Noetherian.

\pdfmcone{Changed definition of primary ideal.}
A proper ideal $I$ is \index{ideal!maximal}\emph{maximal} if it is
not strictly contained in any proper ideal, this is equivalent to
$R/I$ being a field. A proper ideal is
\index{ideal!prime}\emph{prime} if $R/I$ is an integral domain;
\index{ideal!primary}\emph{primary} if $ab\in I$ implies that $a\in
I$ or $b^n\in I$ for some $n$. The
\index{ideal!radical}\emph{radical} of an ideal $I$ is the
ideal\symb[I]{$\sqrt{I}$}{Radical of an ideal}
\begin{equation}
  \label{eq:212}
  \sqrt{I} = \{f \,|\, f^r\in I \text{ for some $r\ge0$.}\}
  \text{.}
\end{equation}
An ideal is said to be radical if $\sqrt{I}=I$. The radical of a
primary ideal is prime.

\pdfmcone{Forgot strict inclusion for reducibility.}
An ideal $I$ is said to be \index{ideal!reducible}\emph{reducible}
if it is strictly contained in two ideals $I_1,I_2$ such that
$I=I_1\cap I_2$, \index{ideal!irreducible}\emph{irreducible}
otherwise. Any primary ideal is irreducible; we have the following two
fundamental results about reducibility.

\begin{proposition}
  Let $R$ be Noetherian. Any radical ideal $I$ admits an unique
  decomposition
  \begin{equation}
    \label{eq:213}
    I = P_1\cap\cdots\cap P_n
  \end{equation}
  with $P_i$ prime and $P_i\not\subset P_j$ for $i\ne j$.
\end{proposition}

\begin{theorem}[Primary decomposition]
  Let $R$ be Noetherian. \index{primary~decomposition}Any ideal $I$
  admits a decomposition
  \begin{equation}
    \label{eq:214}
    I = Q_1\cap\cdots\cap Q_n
  \end{equation}
  into primary ideals. Furthermore, $\sqrt{Q_i}$ is uniquely
  determined.
\end{theorem}

Now we state a fundamental lemma that we will repeatedly use in the
next chapters.

\begin{lemma}[Chinese remainder theorem]
  \label{th:chinese-remainder}
  \index{Chinese~remainder~theorem}
  Let $I_1,\ldots,I_n$ be pairwise coprime ideals (i.e.\ $I_i+I_j=R$ if
  $i\ne j$).  Then the canonical morphism $A\ra\prod_j A/I_j$ gives an
  isomorphism of rings
  \begin{equation}
    \label{eq:211}
    A/I_1\cap\cdots\cap I_n \isom \prod_j A/I_j
    \text{;}
  \end{equation}
  and the intersection $I_1\cap\cdots\cap I_n$ equals the product
  $I_1\cdots I_n$.
\end{lemma}


\subsection{Algebraic varieties}
\label{sec:algebraic-varieties}
\pdfmcone{Substituted the ambiguous use of "variety" with "set of
  zeros".}  We now consider the polynomial ring $\K[x_1,\ldots,x_n]$,
where $\K$ is a perfect field with algebraic closure $\clot{\K}$. To
any ideal $I$, we associate its
\index{set~of~zeros}\symb[ideal]{$V(I)$}{Set of zeros of the ideal
  $I$}\emph{set of zeros}
\begin{equation}
  \label{eq:215}
  V(I) = \{x\in\clot{\K}^n \,|\, f(x) = 0 \text{ for any } f\in I\}
  \text{.}
\end{equation}
\pdfmcone{Detailed definition of I(V), so that it is compatible with
  the version of the Nullstellensatz given later.} Reciprocally, to
any $V\subset\clot{\K}^n$ we associate the ideal
\index{ideal!vanishing}\symb[ideal]{$I(V)$}{Ideal vanishing at the
  algebraic set $V$}\emph{vanishing at $V$}
\begin{equation}
  \label{eq:216}
  I(V) = \{f\in\clot{\K}[x_1,\ldots,x_n] \,|\, f(x) = 0 \text{ for any } x\in V\}
  \text{.}
\end{equation}

\pdfmcone{Defined the affine space (and the projective space three
  paragraphs later).} A subset of $\clot{\K}^n$ is called an
\index{algebraic~set!affine}\emph{affine algebraic set} if it is the
set of zeros of an ideal of $\clot{\K}[x_1,\cdots,x_n]$.  The
\index{affine~space}\emph{affine space} of dimension $n$, denoted by
\symb[affspace]{$\mathbb{A}^n$}{$n$-dimensional affine
  space}$\mathbb{A}^n$, is the affine algebraic set associated to the
zero ideal.

An algebraic set $V$ is
\index{algebraic~set!defined~over~a~field}\emph{defined over} $\K$ if
$I(V)$ has a set of generators in $\K[x_1,\ldots,x_n]$; in this case
we denote by \symb[ideal]{$V(\K)$}{$\K$-rational points of an
  algebraic set}$V(\K)$ the subset $V\cap\K^n$.

An algebraic set is
\index{algebraic~set!irreducible}\emph{irreducible} if it cannot be
written as the union of two proper algebraic sets; equivalently, it is
irreducible if $I(V)$ is prime. An irreducible affine algebraic set is
called an \index{variety!affine}\emph{affine variety}.

There is also an equivalent notion of
\index{algebraic~set!projective}\index{variety!projective}\emph{projective
  variety} for homogeneous ideals. The projective variety associated to
the zero ideal is called the \index{projective~space}\emph{projective
  space} of dimension $n$, and is denoted by
\symb[affspace]{$\Proj^n$}{$n$-dimensional projective space}$\Proj^n$.

In the sequel we shall drop the qualificatives ``affine'' or
``projective'', and simply speak of
\index{variety!algebraic}\emph{algebraic varieties} whenever
definitions/theorems are identical.


\begin{theorem}[Nullstellensatz]
  Let $I$ be an ideal and $V$ an algebraic set. We have the following
  identities
  \begin{equation}
    \label{eq:217}
    I(V(I)) = \sqrt{I}
    \text{,}\qquad
    V(I(V)) = V
    \text{.}
  \end{equation}
\end{theorem}

\pdfmcone{Made more explicit that V must be a variety.}
If $V$ is a variety defined over $\K$, its
\index{coordinate~ring}\emph{coordinate
  ring}\symb[KV]{$\K[V]$}{Coordinate ring of an algebraic
  variety} is
\begin{equation}
  \label{eq:218}
  \K[V] \eqdef \K[x_1,\ldots,x_n]/I(V)
  \text{;}
\end{equation}
the \index{function~field}\emph{function field}
$\K(V)$\symb[KV]{$\K(V)$}{Function field of an algebraic
  variety} is its field of fractions. 

The \index{dimension~of~a~variety}\emph{dimension} of a variety $V$
is the length $d$ of the longest chain of distinct non-empty
subvarieties of $V$
\begin{equation}
  \label{eq:219}
  V_d \subset \cdots \subset V_1 \subset V
  \text{.}
\end{equation}
Equivalently, it is the length of the longest strictly decreasing
chain of prime ideals in $\K[V]$. Yet another way of defining it, is
the degree of transcendence of $\K(V)$ over $\K$.

If $V_1$ and $V_2$ are two varieties, an
\index{rational~map}\emph{affine rational map} is a map
\begin{equation}
  \label{eq:220}
  \begin{aligned}
    \phi : V_1&\ra V_2\text{,}\\
    x &\mapsto (f_1(x),\ldots,f_n(x))\text{,}
  \end{aligned}
\end{equation}
with $f_1,\ldots,f_n\in\clot{\K}(V_1)$ and such for any point $P$ at
which $f_1,\ldots,f_n$ are defined, $\phi(P)\in V_2$. An equivalent
definition exists for \emph{projective rational maps}.

A rational map that is defined at any point of $V_1$ is called a
\index{morphism~of~varieties}\emph{morphism}. A rational map (a
morphism) is
\index{rational~map!defined~over~a~field}\index{morphism~of~varieties!defined~over~a~field}\emph{defined
  over} $\K$ if $f_1,\ldots,f_n\in\K(V)$.



%%% Local Variables: 
%%% mode:flyspell
%%% ispell-local-dictionary:"american"
%%% mode: TeX-PDF
%%% mode: reftex
%%% TeX-master: "../these"
%%% End: 




\chapter{Algorithms and complexity}
\label{cha:algor-compl}
\section{Asymptotic complexity}
\label{sec:asympt-compl}
% Many algorithms below rely on fast multiplication; thus, we let $\Mult
% : \N \rightarrow \N$ be a {\em multiplication function}, such that
% polynomials in $\F_p[X]$ of degree less than $n$ can be multiplied in
% $\Mult(n)$ operations, under the conditions of~\cite[Ch.~8.3]{vzGG}.
% Typical orders of magnitude for $\Mult(n)$ are $O(n^{\log_2(3)})$ for
% Karatsuba multiplication or $O(n\log (n) \log\log (n))$ for FFT
% multiplication. Using fast multiplication, fast algorithms are
% available for Euclidean division or extended GCD~\cite[Chapter~9 \&
% 11]{vzGG}.

% The cost of {\em modular composition}, that is, of computing $F(G)
% \bmod H$, for $F,G,H\in\F_p[X]$ of degrees at most $n$, will be
% written $\ModComp(n)$. We refer to~\cite[Chapter~12]{vzGG} for a
% presentation of known results in an algebraic computational model: the
% best known algorithms have subquadratic (but superlinear) cost in
% $n$. Note that in a boolean RAM model, the algorithm of~\cite{KeUm08}
% takes quasi-linear time.

\section{Fundamental algorithms}
\label{sec:fund-algor}
In this section we review some fundamental algorithms that we will
repeatedly use in the rest of the document. Most of the algorithms we
present are taken from~\cite{vzGG}; another source of inspiration
is~\cite{poly-formel}.

\subsection{Polynomial multiplication}
\label{sec:polyn-mult}
Multiplication of polynomials with coefficients in a ring is a
fundamental brick to which most of the algorithms in computer algebra
reduce.

In the previous section we introduced the notation $\Mult(n)$ to
denote the number of operations in $R$ required to multiply two
polynomials of degree at most $n$ in $R[X]$.  Using the school-book
algorithm, we have $\Mult(n) \in O(n^2)$. The first major step forward
in the complexity of multiplication was done by \index{Karatsuba
  multiplication}Karatsuba~\cite{karatsuba}. He observed that using
the formula
\begin{gather*}
  f = f_1X^n + f_2\text{,}\qquad g = g_1X^n + g2\text{,}\\
  fg = f_1g_1X^{2n} + \bigl((f_1+f_2)(g_1+g_2)-f_1g_1-f_2g_2\bigr)X^n + f_2g_2
  \text{,}
\end{gather*}
multiplication can be computed recursively using only $3$ recursive
calls.  It follows that $\Mult(n)\in O(n^{\log_23})$.

When the base ring $R$ is a field containing a primitive $n$-th root
of unit $\omega$, polynomials can be multiplied by evaluating at the
powers of $\omega$, multiplying each evaluation, and interpolating
back. The map that sends a polynomial of degree $n$ over its
evaluations at the $n$-th roots of unit is called
\index{discrete~Fourier~transform}\textbf{discrete Fourier transform},
there are many algorithms of complexity $O(n\log n)$ to compute it,
they all go under the generic name of
\index{FFT}\index{fast~Fourier~transform}\textbf{fast Fourier transform}
(FFT).

Thus, multiplication in certain fields can be carried out in time
$O(n\log n)$. In the famous paper~\cite{schonage+strassen}, Schönage
and Strassen showed how performing an FFT in an extension ring of $R$
containing the required roots of unit yields an algorithm of
complexity $O(n\log n\log\log n)$ to multiply polynomials in $R[X]$.

\subsection{Formal power series}
\label{sec:formal-power-series}
We denote by $R[[X]]$ the ring of
\index{formal~power~series}\textbf{formal power series} on $R$. Its
elements are the sequences $(f_i)_{i>0}$ of elements of $R$, they are
denoted by
\begin{equation}
  \label{eq:197}
  f(X) = \sum_{i>0}f_iX^i
  \text{.}
\end{equation}
Multiplication and evaluation are defined in the obvious way. An
element $f\in R[[X]]$ is invertible if and only if $f(0)$ is an unit
of $R$.

Since formal power series are infinite objects, to be used in a
discrete algorithm they must be approximated. We denote by $f\bmod
X^n$ the polynomial
\begin{equation}
  \label{eq:198}
  f\bmod X^n = \sum_{0\le i < n}f_iX^i
  \text{.}
\end{equation}
We write $f = g + O(X^n)$, where $g$ is a polynomial or a power
series, whenever
\[f\bmod X^n=g\bmod X^n\text{,}\] and we say that $g$ approximates $f$
to the precision $n$.

Using polynomial multiplication, the product of two series known up to
precision $n$ can be computed in $O(\Mult(n))$ operations.

\paragraph{Derivative, integral}
\label{sec:derivative-integral}
If $R$ contains $\Q$, we define the
\index{formal~power~series!derivative}derivative and the
\index{formal~power~series!integral}integral of a power series as
\begin{align}
  \label{eq:200}
  f'(X) &= \sum_{i\ge0}(i+1)f_{i+1}X^i\text{,}\\
  \int f(X) &= \sum_{i\ge0}\frac{f_i}{i+1}X^{i+1}\text{.}
\end{align}
Derivatives and integrals up to precision $n$ can be computed in
$O(n)$ operations by their definition.

\paragraph{Logarithm, exponential}
\label{sec:logarithm}
The \index{formal~power~series!logarithm}logarithm of a power series
$f$ such that $f(0)=1$ is defined as
\begin{equation}
  \label{eq:194}
  \log f = \int\frac{f'}{f}
  \text{.}
\end{equation}
The \index{formal~power~series!exponential}exponential of a power
series $f$ such that $f(0)=0$, is defined as
\begin{equation}
  \label{eq:195}
  \exp(f) = 1 + f/1! + f^2/2! + \cdots
\end{equation}

\paragraph{First order linear differential equations}
\label{sec:first-order-linear}
All the usual identities involving multiplication, derivatives,
integrals, logarithms and exponentials are verified on power
series. An immediate consequence of this is a formula to solve first
order linear differential equations to Brent and
Kung~\cite{brent+kung}.

Let $f,g\in
R[[X]]$, the equation
\begin{equation}
  \label{eq:208}
  y' = f(X)y + g(X)
\end{equation}
with initial condition $y(0)=a$ has solution
\begin{equation}
  \label{eq:209}
  y(X) =  \frac{1}{j(X)}\left( a + \int g(X)j(X)\right)
  \text{,}
\end{equation}
where $j = \exp(-\int f)$; the verification is immediate.

In the next subsection we shall see that multiplicative inverses,
logarithms, exponentials and powers up to precision $n$ can all be
computed in time $O(\Mult(n))$, thus formula~\eqref{eq:209} can also
be applied at the same cost.


\begin{nota}
  If $R$ does not contain $\Q$, but has characteristic $0$, it is easy
  to use the previous definitions by working in
  $R[2^{-1},3^{-1},\ldots]$ and taking the result back in $R$ when
  needed. In characteristic different from $0$, these definition do
  not make sense anymore because Eq.~\eqref{eq:200} introduces a
  division by $0$. However, when $2,3,\ldots,n$ are invertible in $R$,
  we can still do computations on power series truncated to the order
  $n$.
\end{nota}


\subsection{Newton's iteration}
\label{sec:newtons-iteration}
Let $\Phi:\R\ra\R$ be a $C^1$ function, the
\index{Newton's~iteration}Newton's iteration is a classical method to
approximate a root $x$ of $\Phi$. Start from an approximation $x_0$, and
\emph{linearize} $\Phi$ to compute
\begin{equation}
  \label{eq:192}
  x_1 = x_0 - \frac{\Phi(x_0)}{\Phi'(x_0)}
  \text{,}
\end{equation}
then iterate this step until the desired precision is obtained. When
$x_0$ is taken close enough to a root, and when the derivative at this
root is non-zero, Newton's iteration converges \emph{quadratically} to
the solution, meaning that at each iteration the distance to the
solution is squared.

In computer algebra, Newton's iteration is applied to operators
$\Phi:R[[X]]\ra R[[X]]$ on formal power series; in this context,
\emph{quadratic} convergence means that the number of correct terms is
doubled at each iteration. Many fast algorithms for some fundamental
operations on power series and polynomials are obtained by this
method, here we summarize the most important ones.

\paragraph{Inversion}
If $f\in R[[X]]$ is invertible, the operator $\Phi(y) = 1/y - f$
applied to $y_0=1$ converges quadratically to the inverse of
$f$. Since the iteration associated to $\Phi$ is
\begin{equation}
  \label{eq:193}
  y_{i+1} = y_i(2 - y_if)
  \text{,}
\end{equation}
the cost of inverting a power series is $O(\Mult(n))$. From
Eq.~\eqref{eq:194} we deduce that computing the logarithm of a power
series has the same cost.

Another important consequence of this algorithm is that the Euclidean
division of polynomials of degree at most $n$ can also be performed in
$O(\Mult(n))$ operations.

\paragraph{Exponential}
If $f$ is such that $f(0)=0$, we compute its exponential using the
operator $\Phi(y)=f-\log y$, which gives the iteration
\begin{equation}
  \label{eq:196}
  y_{i+1} = y_i(1 + f - \log y)
  \text{.}
\end{equation}
Thus, the cost of computing an exponential is $O(\Mult(n))$ too. Using
the formula
\begin{equation}
  \label{eq:201}
  f^\alpha = \exp(\alpha\log f)
  \text{,}
\end{equation}
we deduce that, in characteristic $0$, computing arbitrary rational
powers of power series costs $O(\Mult(n))$ too.


\subsection{Modular composition}
\label{sec:modular-composition}
Given polynomials $f,g,h\in R[X]$ of degree at most $n$, the
\index{modular~composition}\emph{modular composition} requires to
compute
\begin{equation}
  \label{eq:190}
  f(g(X)) \mod h(X)
  \text{;}
\end{equation}
the special case where $h=X^n$ permits to compute the
\index{formal~power~series!composition}\emph{composition of power
  series} truncated to the order $n$.

Modular composition is a fundamental algorithm with lots of
applications, the most relevant being polynomial
factorization\cite{vzgathen+shoup92,kaltofen+shoup98} and computation
of minimal polynomials (see Remark~\ref{rk:shoups-algorithm-1}). In
the next subsection we shall also see an application of composition of
power series to solve differential equations.

Since many algorithms in this document make use of modular
composition, we introduced the notation $\ModComp(n)$ for its
complexity. A naive algorithm implies $\ModComp(n)\in
O(n\Mult(n))$. The first improvement to this bound was given by Brent
and Kung in~\cite{brent+kung}: they devise a baby step-giant step
algorithm of complexity $O\left(\sqrt{n}\Mult(n) +
  n^{\frac{\omega+1}{2}}\right)$; in the same paper they also gave an
algorithm of complexity $O\left(\sqrt{n\log n}\Mult(n)\right)$ for
composition of power series. Bernstein~\cite{bernstein98} found the
bound $O(\Mult(n)\log n)$ for the composition of power series in case
the characteristic of the base ring is small, however for a long time
Brent and Kung's algorithm and its
variants\cite{huang+pan98,kaltofen+shoup98} have stood as the only
generic algorithm for modular composition. A major breakthrough has
been recently achieved by Kedlaya and
Umans~\cite{umans:08,kedlaya+umans08}, who give an algorithm for
modular composition over a finite field $\F_q$ of \emph{binary}
complexity $n^{1+O(1)}\log^{1+O(1)}q$, using a reduction to
multivariate multipoint evaluation.


\paragraph{Computing iterated Frobenius and pseudotrace}
\label{sec:comp-frob-trace}
Fast modular composition can be used to compute Frobenius
automorphisms and \emph{pseudotraces} in finite fields. This algorithm
is due to von zur Gathen and Shoup~\cite{vzgathen+shoup92}, who
applied it to polynomial factorization. We will repeatedly use it in
Chapters~\ref{cha:artin-schr-towers} and~\ref{cha:algor-small-char}.

Consider the field extension $\F_{q^d}/\F_q$, its Galois group is
generated by the Frobenius automorphism
\begin{equation}
  \label{eq:199}
  \begin{aligned}
  \frob_q : \F_{q^d}&\ra\F_{q^d}\text{,}\\
  x &\mapsto x^{q}\text{.}
  \end{aligned}
\end{equation}
For any $n<d$, we also define the $n$-th
\index{pseudotrace}\emph{pseudotrace}\footnote{In~\cite{vzgathen+shoup92},
  this map goes under the name of \index{trace~map}\emph{trace map}.}
as
\begin{equation}
  \label{eq:202}
  \begin{aligned}
    \PTr_n : \F_{q^d}&\ra\F_{q^d}\text{,}\\
    x&\mapsto\sum_{i=0}^{n-1} x^{q^i}\text{.}
  \end{aligned}
\end{equation}
Notice that, when $n=d$, the pseudotrace coincides with the trace
$\Tr_{\F_{q^d}/\F_q}$; in this case one can use much faster
algorithms.

We suppose that elements of $\F_{q^d}$ are represented as residue
classes in $\F_q[X]/f(X)$ for some irreducible polynomial $f$, then
the Frobenius morphism can be computed with $O(\log q)$
multiplications in $\F_{q^d}$ plus one modular composition as
\begin{align}
  \label{eq:203}
  \Phi_1(X) &= X^q \bmod f(X)\text{,}\\
  \frob_q(a) &= X^q\circ a\bmod f = a\circ X^q\bmod f = a\circ \Phi_1\bmod f
  \text{.}
\end{align}


\begin{algorithm}
  \label{alg:itfrob}
  \caption{Iterated Frobenius}
  \begin{algorithmic}[1]
    \REQUIRE $0<i<d$, $a\in\F_q[X]/f(x)$, $\Phi_1(X) = X^q\bmod f(x)$.
    \ENSURE $\frob_q^i(a)$.
    \STATE let $i=\sum b_j2^j$ be the binary expansion of $i$;
    \STATE $k \la 1$;
    \FOR {$j=\lfloor\log_2 i\rfloor -1$ \TO $0$}
    \IF{$b_j=0$}
    \STATE $\Phi_{2k} \la \Phi_k\circ\Phi_k \bmod f$;
    \STATE $k \la 2k$;
    \ELSE
    \STATE $\Phi_{2k} \la \Phi_k\circ\Phi_k \bmod f$;
    \STATE $\Phi_{2k+1} \la \Phi_{2k}\circ\Phi_1 \bmod f$;
    \STATE $k \la 2k +1$;
    \ENDIF
    \ENDFOR
    \STATE return $a\circ\Phi_i \bmod f$.
  \end{algorithmic}
\end{algorithm}

Iterating $i$ times the $\frob_q$ can be done with only $O(\log i)$
modular compositions via square-and-multiply as shown in
Algorithm~\ref{alg:itfrob}.

Thus the cost of computing the $i$-th iterated Frobenius is
\begin{equation}
  \label{eq:204}
  O(\ModComp(d)\log i)
\end{equation}
operations in $\F_q$ plus a precomputation costing $O(\Mult(d)\log
q)$.

\begin{algorithm}
  \caption{Pseudotrace}
  \begin{algorithmic}[1]
    \REQUIRE $0<i<d$, $a\in\F_q[X]/f(x)$, $\Phi_1(X) = X^q\bmod f(x)$.
    \ENSURE $\PTr_n(a)$.
    \STATE let $n=\sum b_j2^j$ be the binary expansion of $n$;
    \STATE $\Theta_o \la 0$, $\Theta_1 \la a\circ\Phi_{1}$;
    \STATE $k=b_0$;
    \FOR{$j=1$ \TO $\lfloor\log_2n\rfloor$}
    \STATE $\Phi_{2^j} \la \Phi_{2^{j-1}}\circ\Phi_{2^{j-1}} \bmod f$;
    \STATE $\Theta_{2^j} \la \Theta_{2^{j-1}} + \Theta_{2^{j-1}}\circ\Phi_{2^{j-1}} \bmod f$;
    \IF{$b_j=1$}
    \STATE $\Theta_{2^j+k} \la \Theta_{2^j} + \Theta_{k}\circ\Phi_{2^j} \bmod f$;
    \STATE $k \la 2^j + k$;
    \ENDIF
    \ENDFOR
    \STATE return $\Theta_n$.
  \end{algorithmic}
\end{algorithm}

We apply the same idea to compute the $n$-th pseudotrace in time
$O(\ModComp(d)\log n)$; note that we use a dynamic programming
technique to keep the complexity into this bound. The key equation is
\begin{equation}
  \label{eq:205}
  \PTr_{n+m}(a) = \PTr_{n}(a)+\frob_q^n(\PTr_{m}(a))
  \text{.}
\end{equation}


\subsection{Chinese remainder algorithm and interpolation}
\label{sec:chin-rema-algor}

\cite[$\S$10]{vzGG}




\subsection[XGCD, Cauchy interpolation and RFR]{Euclidean algorithm,
  Cauchy interpolation and rational fraction reconstruction}
\label{sec:eucl-algor-rati}
Let $\K$ be a field, given two polynomials $f,g\in\K[X]$ of degrees
$m,n$, the \index{Euclidean~algorithm}Euclidean algorithm permits to
compute their \index{GCD}GCD using $O(mn)$ operations in $\K$. Let $r$
be the GCD of $f$ and $g$, a \index{Bézout~relation}\emph{Bézout
  relation} is an equation of the form
\begin{equation}
  \label{eq:154}
  fu + gv = r
  \text{,}
\end{equation}
with $u,v\in\K[X]$. If we ask $\deg(ur)<\deg(g)$ and
$\deg(vr)<\deg(a)$, the Bézout relation is unique; computing it is
called the extended GCD problem (\index{XGCD}XGCD) and can be computed
by the \index{extended Euclidean algorithm}\emph{extended Euclidean
  algorithm}.

\begin{algorithm}
  \caption{Extended Euclidean algorithm}
  \begin{algorithmic}[1]
    \REQUIRE $f,g\in\K[X]$.
    \ENSURE $u,v,r\in\K[X]$ such that $fu+gv=r$.
    \STATE let $r_0\la f$, $u_0\la1$, $v_0\la0$;
    \STATE let $r_1\la f$, $u_1\la0$, $v_1\la1$;
    \STATE $i\la1$;
    \WHILE{$r_i\ne0$}
    \STATE compute $r_{i-1} = q_ir_i + r_{i+1}$ by Euclidean division;
    \STATE compute $u_{i+1} \la u_{i-1} - q_iu_i$, $v_{i+1} \la v_{i-1} - q_iv_i$;
    \STATE $i\la i+1$;
    \ENDWHILE
    \STATE return $u_i,v_i,r_i$.
  \end{algorithmic}
\end{algorithm}

One important application of XGCD's is computing modular inverses. Let
$f,g\in\K[X]$ with $\deg(g)<\deg(f)$ and $f$ prime to $g$, then $r$ is
a unit in $\K$, and a Bézout relation implies
\begin{equation}
  \label{eq:206}
  g\frac{v}{r} \equiv 1 \mod f
  \text{.}
\end{equation}

More generally, the polynomials computed at each iteration by the
extended Euclidean algorithm satisfy
\begin{equation}
  \label{eq:156}
  fu_i + gv_i =  r_i
  \qquad\text{for any $i$;}
\end{equation}
each of these is also called a Bézout relation. These relations have
two major applications: \emph{Cauchy interpolation} and \emph{rational
  fraction reconstruction}.

Given $n$ pairs $(x,e)\in\K\times\K$ with all $x$ distinct and an
integer $\ell<n$, \index{Cauchy~interpolation}Cauchy interpolation
computes, if it exists, a rational fraction $\frac{r}{v}\in\K(X)$ with
$\deg r<\ell$ and $\deg v \le n-\ell$, such that $\frac{r(x)}{v(x)}=e$
for any $(x,e)$. Let $f=\prod (X-x)$, by interpolation one obtains the
unique polynomial $g\in\K[X]/f$ such that $g(x)=e$ for any
$(x,e)$. Then a Bézout relation for $f$ and $g$ gives
\begin{equation}
  \label{eq:207}
  \frac{r_i}{v_i} \equiv g \mod f
  \qquad\text{for any $i$,}
\end{equation}
thus in particular $\frac{r_i(x)}{v_i(x)}=e$ for any $(x,e)$. It can
be proven that a solution to the Cauchy interpolation problem exists
if and only if one of the intermediate results of the extended
Euclidean algorithm is such that $\deg(r_i)<\ell$ and $\deg(v_i)\le
n-\ell$.

\index{rational~fraction~reconstruction}Rational fraction
reconstruction (\index{RFR}RFR) is very similar to Cauchy
interpolation, and it can be viewed as a generalization of it using
multiplicities. Let $g\in\K[[X]]$ be a power series, we want to
compute a rational fraction $\frac{r}{v}\in\K(X)$ with $\deg(r)<\ell$
and $\deg(v)\le n-\ell$, such that $\frac{r}{v}=g+O(X^n)$ in
$\K[[x]]$. Such a rational fraction is called a
\index{Padé~approximant}\emph{Padé approximant} of type
$(\ell-1,n-\ell)$ of $g$. Again, it can be shown that a Padé
approximant of type $(\ell-1,n-\ell)$ exists if and only if
\begin{equation}
  \label{eq:210}
  \frac{r_i}{v_i}\equiv g \mod X^{n+m+1}
\end{equation}
is one of the intermediate results computed by the extended Euclidean
algorithm.

The extend Euclidean algorithm is not optimal. We address the reader
to~\cite[$\S$11.1]{vzGG} for the description of an algorithm that
takes $f,g\in\K[X]$ of degree at most $n$ and $\ell\le n$, and
computes, using $O(\Mult(n)\log n)$ operations, the rows $u_i,v_i,r_i$
and $u_{i+1},v_{i+1},r_{i+1}$ of the Extended Euclidean algorithm such
that $\deg(r_i)\ge n-\ell$ and $\deg(r_{i+1})<n-\ell$. A consequence of
this algorithm is that both Cauchy interpolation and rational fraction
reconstruction can be computed in time $O(\Mult(n)\log n)$.



\subsection{Multivariate polynomials}
\label{sec:mult-polyn}
Kronecker substitution (vzGS92)\\
maybe bivariate operations (some Pascal-Schost and some Li-Moreno-Schost) ?

\subsection{Transposed algorithms}
\label{sec:transp-algor}
transposed mul, transposed mod,


%%% Local Variables: 
%%% mode:flyspell
%%% ispell-local-dictionary:"american"
%%% mode: TeX-PDF
%%% mode: reftex
%%% TeX-master: "../these"
%%% End: 


%%% Local Variables: 
%%% mode:flyspell
%%% ispell-local-dictionary:"american"
%%% mode: TeX-PDF
%%% mode: reftex
%%% TeX-master: "../these"
%%% End: 

\newcommand{\bom}{\perp}  % bottom _|_
\newcommand{\al}{\prec}
\newcommand{\tAL}{\textsf{transAL}}
\newcommand{\Sbasis}{\mathcal{S}}  % standard basis
\newcommand{\Tbasis}{\mathcal{L}}  % linear basis
\newcommand{\lmul}[1]{{}_{#1}*}  % left multiplication node
\newcommand{\rmul}[1]{*_{#1}}  % right multiplication node
\newcommand{\op}{\mathrm{op}}  % opposite
\newcommand{\RMod}[1]{#1\text{\sf-Mod}}  % The category of left R-modules
\newcommand{\ModR}[1]{\text{\sf Mod-}#1}  % The category of right R-modules
\newcommand{\pspace}{\mathcal{P}}  % the parameter space
\newcommand{\s}{\underline}
\renewcommand{\l}{\overline}


\lstset{
  upquote=true,
  basicstyle=\ttfamily\small,          % print whole listing in typewriter
  keywordstyle=\color{blue}\bfseries, % bold blue keywords
  %identifierstyle=,           % nothing happens
  commentstyle=\color{green}, % green comments
  stringstyle=\color{red},      % typewriter type for strings
  showstringspaces=false     % no special string spaces
}


\part{The transposition principle}

The complexity of algebraic algorithms is often more easily described
in a non-Turing model where one assumes that any algebraic operation
can be done in a unit of time and any other operation is
free. \emph{Algebraic complexity} studies precisely the computational
models that behave this way.

For algorithms over finite rings, the algebraic complexity gives a
precise estimate for the complexity in the Turing model (also called
the \emph{binary complexity}). For other rings, the algebraic estimate
may be way off target, but it can nevertheless give useful information.

In this chapter we study models that allow one to study the algebraic
complexity of linear operators. We first present the \emph{arithmetic
  circuit}, then the \emph{straight line program}. Because of their
algebraic structure, these models support some algebraic
manipulations.  Our principal interest will be the \emph{transposition
  theorem}, stating that it is possible to apply classical duality (in
the sense of Section~\ref{sec:linear-algebra:duality}) to programs,
while preserving some complexity invariants. The interest for the
transposition theorem comes from the applications we have seen in
Section~\ref{sec:transp-algor} and other more advanced applications
that we will see in the next chapters.

Finally, in Section~\ref{sec:word-about-automatic}, we study the
relationship between the transposition theorem and the classical
theory of \emph{automatic differentiation}.



% Local Variables:
% mode:flyspell
% ispell-local-dictionary:"american"
% mode:reftex
% mode:TeX-PDF
% TeX-master: "../these"
% End:
%

\section{\index{arithmetic circuit}Arithmetic circuits}
\label{sec:circuits}

In this section we present the arithmetic circuit model and give some
classical results. Since we have in mind applications to the theory of
transposition, our presentation is slightly different from classical
presentations of the subject \cite{BuClSh, Vollmer}.

\begin{definition}[Arithmetic operator, arity]
  Let $R$ be a (non necessarily commutative) ring with unit. An
  arithmetic operator over $R$ is a function $f:R^i\ra R^o$ for some
  $i,o\in\N$.  Here $i$ is called the \emph{in-arity} of $f$ or simply
  \emph{arity}, $o$ is called the \emph{out-arity} of $f$.
\end{definition}

The exact meaning of the product $R^n$ varies depending on what is
meant by ``function'': a category must be specified in order for this
notions to be well defined. In this paper we will use two different
definitions:
\begin{itemize}
\item If we want ``function'' to mean any function in a set-theoretic
  sense, we work in the category $\mathsf{Set}$. Then $R^n$ is the
  product $\prod^nR$ of the category (the Cartesian product) and $R^0$
  is the terminal object of the category, i.e. any singleton. For
  convenience we note $\{\bom\}$ for $R^0$, with $\bom$ being its
  unique element.
\item In some cases it will be convenient to restrict ``function'' to
  mean ``morphism of left $R$-modules'', then we will work in the
  category $\RMod{R}$ of left $R$-modules with morphisms. Then $R^n$
  is the product $\prod^nR$ of the category and $R^0$ is the zero
  module.  We note $0$ for the zero module and $\bom$ for its unique
  element in order to avoid confusion with the $0$ of $R$ (and to
  stress the analogy with $\mathsf{Set}$). In what follows by
  ``module'' we will always mean ``left module''.
\end{itemize}

\begin{definition}[Arithmetic basis]
  Let $R$ be a ring.  An arithmetic $R$-basis is a (not necessarily
  finite) set of arithmetic operators over $R$.
\end{definition}

Two arithmetic bases will be important to us. The first one is the
\emph{standard basis}, noted $\Sbasis$. It is composed of the
following operators
\begin{equation}
  \label{eq:sbasis}
  \tag{$\Sbasis$}
  \begin{aligned}
    + : R\times R &\ra R    & * : R\times R &\ra R &  \hub : R &\ra R\times R\\
        a, b &\mapsto a+b   &     a, b &\mapsto ab &         a &\mapsto a,a\\ \\
    \eta_a : \{\bom\} &\ra R     &&& \omega : R &\ra \{\bom\} \\
          \bom &\mapsto a &&&          a &\mapsto \bom
  \end{aligned}
\end{equation}
Remark that $\eta_a$ defines a possibly infinite family of operators,
one for every $a\in R$.

The other one is the \emph{linear basis}, noted $\Tbasis$.  It
is composed of
\begin{equation*}
  \label{eq:tbasis}
  \tag{$\Tbasis$}
  \begin{aligned}
    + : R\times R &\ra R    &\quad
    \rmul{a} : R &\ra R      &\quad
    0 : 0 &\ra R  \\
    a, b &\mapsto a+b   &
    b &\mapsto ba &  
    \bom &\mapsto 0  \\
    \\
    \hub : R &\ra R\times R  &\quad
    &&
    \omega : R &\ra 0 \\
    a &\mapsto a,a    &
    &&
    a &\mapsto \bom
  \end{aligned}
\end{equation*}
The linear basis is reproduced in figure \ref{fig:nodes}.

Finally, the \emph{opposite linear basis}, noted $\Tbasis^\op$, is
obtained from $\Tbasis$ by substituting $\rmul{a}$ with
$\lmul{a}:b\mapsto ab$. Notice that this is just a shorthand notation
for the arithmetic $R^\op$-basis $\Tbasis$.

In terms of category theory, $\Sbasis$ only makes sense when working
with $\mathsf{Set}$, while $\Tbasis$ can be defined even when working
with $\RMod{R}$ (equivalently, $\Tbasis^\op$ can be defined when
working with $\ModR{R}$, the category of right $R$-modules).
\begin{figure}[!ht]
  \label{fig:nodes}
  \centering
  \begin{tikzpicture}
    \tikzstyle{node}=[circle,thick,draw=black,minimum size=4mm]
    \tikzstyle{arg}=[rectangle,thin,draw=black,minimum size=4mm]
    
    \begin{scope}
      \node(in1){};
      \node(nop)[right of=in1]{};
      \node(in2)[right of=nop]{};

      \node[node](plus)[below of=nop]{$+$};
      \node(out)[below of=plus]{};

      \path[->]
      (in1) edge (plus)
      (in2) edge (plus)
      (plus) edge (out);
    \end{scope}
    
    \begin{scope}[xshift=30mm]
      \node(in){};
      \node[node](hub)[below of=in]{$\hub$};
      \node(nop)[below of=hub]{};
      \node(out1)[left of=nop]{};
      \node(out2)[right of=nop]{};

      \path[->]
      (in) edge (hub)
      (hub) edge (out1)
      (hub) edge (out2);
    \end{scope}
    
    \begin{scope}[xshift=45mm]
      \node(in){};
      \node[node](times)[below of=in]{$\rmul{a}$};
      \node(out)[below of=times]{};

      \path[->]
      (in) edge (times)
      (times) edge (out);
    \end{scope}

    \begin{scope}[xshift=55mm]
      \node(nop){};
      \node[node][below of=nop](create){$0$};
      \node(out)[below of=create]{};
      \path[->] (create) edge (out);
    \end{scope}

    \begin{scope}[xshift=65mm]
      \node(in){};
      \node[node][below of=in](destroy){$\omega$};
      \node(nop)[below of=destroy]{};
      \path[->] (in) edge (destroy);
    \end{scope}

    \begin{scope}[xshift=75mm]
      \node[arg](in){$x$};
      \node[below of=in](out){};
      \path[->] (in) edge (out);
    \end{scope}

    \begin{scope}[xshift=85mm]
      \node(nop){};
      \node[below of=nop](in){};
      \node[arg](out)[below of=in]{$y$};
      \path[->] (in) edge (out);
    \end{scope}
  \end{tikzpicture}
  \caption{Nodes over the linear basis: round ones are evaluation
    nodes, square ones are input and output nodes.}
\end{figure}


Arithmetic circuits are directed acyclic multigraphs carrying
information from an arithmetic basis; the formal definition follows.

\begin{definition}[Arithmetic node]
  Let $R$ be a ring and $\mathcal{B}$ be an $R$-basis. A node over
  $(R,\mathcal{B})$ is a tuple $v=(I, O, f)$ such that
  \begin{itemize}
  \item $I$ and $O$ are finite ordered sets, 
  \item $f$ is either an element of $\mathcal{B}$ or the special value
    $\emptyset$.
  \item If $f=\emptyset$, one of the two following conditions must hold:
    \begin{itemize}
    \item $I$ is a singleton and $O$ is empty, in this case we say
      that $v$ is an \emph{input node};
    \item $I$ is empty and $O$ is a singleton, in this case we say
      that $v$ is an \emph{output node}.
    \end{itemize}
  \item If $f\ne\emptyset$, the cardinality of $I$ matches the
    in-arity of $f$ and the cardinality of $O$ matches the out-arity
    of $f$; in this case we say that $v$ is an \emph{evaluation node}.
  \end{itemize}
\end{definition}

We call \emph{input ports} the elements of $I$ and \emph{output ports}
the elements of $O$, we note respectively $\inp(v)$ and
$\outp(v)$. The cardinalities of $I$ and $O$ are called, respectively,
the \emph{in-degree} and \emph{out-degree} of $v$.  We call $f$ the
value of $v$ and note $\beta(v)$.

\begin{definition}[Arithmetic circuit]
  Let $R$ be a ring and $\mathcal{B}$ be an $R$-basis. An arithmetic
  circuit over $(R,\mathcal{B})$ is a tuple $C=(V,E)$ such that
  \begin{enumerate}
  \item $V$ is a finite ordered set of nodes over $(R,\mathcal{B})$,
  \item let $I=\biguplus_{v\in V}\inp(v)$ and $O=\biguplus_{v\in
      V}\outp(v)$, then $E$ is a bijection from $O$ to $I$.
  \end{enumerate}
\end{definition}

It is useful to see $E$ as a set of pairs $(o,i)$ with $i\in I$ and
$o\in O$. Then the elements of $E$ are called the \emph{edges} of the
circuit. The edges \emph{incident} to $v\in V$ are all the $(o,i)\in
E$ such that $i\in\inp(v)$; the edges \emph{stemming} from $v\in V$
are all the $(i,o)\in E$ such that $o\in\outp(v)$. An edge stemming
from $v$ and incident to $v'$ is said to \emph{connect} $v$ to $v'$.
We call \emph{inputs} and \emph{outputs} of a circuit, respectively,
the input and output nodes in $V$; we note $\inp(C)$ and $\outp(C)$.

By forgetting the ordering of $V$, a circuit induces a multiDAG with
labelled nodes, we call it the \emph{graph} of $C$. In what follows we
will often refer to graph-theoretic properties of the graph without
explicitly making the difference between a circuit and its graph.

Figure \ref{fig:circuit} shows an example of arithmetic circuit, the
analogy with multiDAGs is evident. We draw input and output nodes in
square boxes and evaluation nodes in round boxes. The orders on input
and output ports are implicitly represented by ordering edges
clockwise starting from 9 on the clock. The order on $V$ is actually
only important for input and output nodes: we implicitly represent
this information by arranging input and output nodes increasingly from
left to right, but we omit it for evaluation nodes. We will always use
these conventions when drawing circuits.

\begin{figure}[!ht] \centering
  \begin{tikzpicture}
\tikzstyle{node}=[circle,thick,draw=black,minimum size=4mm]
\tikzstyle{arg}=[rectangle,thin,draw=black,minimum size=4mm]
    
    \begin{scope} \node[arg](in1){$x_1$}; \node[arg,right
of=in1](in2){$x_2$}; \node[arg,right of=in2](in3){$x_3$};
      
      \node[node,below of=in1](plus1){$+$}; \node[node,right
of=plus1](H){$\hub$};

      \node[node,below of=plus1](plus2){$+$}; \node[node,right
of=plus2](times){$*_2$};

      \node[arg,below of=plus2,xshift=6mm](out1){$y_1$};
\node[arg,right of=out1](out2){$y_2$};

      \path[->] (in1) edge (plus1) (in2) edge (H) (in3) edge (out2)
(H) edge (plus1) (H) edge (times) (plus1) edge (plus2) (times) edge
(plus2) (plus2) edge (out1);
    \end{scope}

    \begin{scope}[xshift=4cm]
      \node[arg](in1){$\dual{x_1}$};
      \node[arg,right of=in1](in2){$\dual{x_2}$};
      \node[arg,right of=in2](in3){$\dual{x_3}$};
      
      \node[node,below of=in1](plus1){$\hub$};
      \node[node,right of=plus1](H){$+$};

      \node[node,below of=plus1](plus2){$\hub$};
      \node[node,right of=plus2](times){$*_2$};

      \node[arg,below of=plus2,xshift=6mm](out1){$\dual{y_1}$};
      \node[arg,right of=out1](out2){$\dual{y_2}$};

      \path[<-]
      (in1) edge (plus1)
      (in2) edge (H)
      (in3) edge (out2)
      (H) edge (plus1)
      (H) edge (times) 
      (plus1) edge (plus2)
      (times) edge (plus2)
      (plus2) edge (out1);
    \end{scope}
  \end{tikzpicture}
  \caption{Two arithmetic circuits over $\Tbasis$. The linear map
    $y_1=x_1+3x_2, y_2=x_3$ is computed by the circuit on the left and
    its transpose is computed by the circuit on the right.}
  \label{fig:circuit}
\end{figure}

Circuits would be meaningless if they hadn't a semantic attached to
them. Intuitively the semantic corresponds to evaluation of the nodes
along the flow of the multiDAG.

\begin{definition}[Evaluation of an arithmetic circuit]
  \label{def:eval}
  Let $C$ be an arithmetic circuit with $i$ inputs and $o$ outputs,
  then its evaluation is a function $\eval_C:R^i\ra R^o$.

  In order to define it, we simultaneously define the evaluation
  $\eval_v$ of each $v\in V$ and the evaluation $\eval_e$ of each
  $e\in E$. We will note by $<_v$ the orders on the input and the
  output ports of $v$; we will also note by $<_V$ the order on $V$.
  \begin{itemize}
  \item Let $v\in V$ have out-degree $n$, let its evaluation be
    $\eval_v:R^i\ra R^n$ and let $\pi_1,\ldots,\pi_n$ be the canonical
    projections from $R^n$ to $R$. Let $o_1<_v\cdots<_vo_n$ be the
    output ports of $v$ and let $e_j=\bigl(o_j,E(o_j)\bigr)$ be the
    corresponding edges stemming from $v$, then $\eval_{e_j} =
    \pi_j\circ\eval_v$ for any $j$.
  \item Let $x_1<_V\cdots<_Vx_n$ be the input nodes and let
    $\pi_1,\ldots,\pi_i$ be the canonical projections from $R^i$ to
    $R$, then $\eval_{x_j}=\pi_j$ for any $j$.
  \item For every evaluation node $v$ with in-degree $m$, let
    $i_1<_v\cdots<_vi_m$ be the input ports of $v$ and let
    $e_j=\bigl(E^{-1}(i_j),i_j\bigr)$ be the corresponding edges
    incident to $v$, then
    \begin{equation}
      \label{eq:eval_v}
      \eval_v = \beta(v) \circ (\eval_{e_1} \times \cdots \times \eval_{e_m})
      \text{.}
    \end{equation}
  \item For every output node $y$, let $e\in E$ be the only edge
    incident to $y$, then $\eval_y=\eval_e$.
  \end{itemize}

  We can finally define $\eval_C:R^i\ra R^o$. Let $y_1<_V\cdots<_Vy_o$
  be the output nodes, then
  \begin{equation}
    \label{eq:eval}
    \eval_C = \eval_{y_1} \times \cdots \times \eval_{y_o}
    \text{.}
  \end{equation}
  We also say that $C$ \emph{computes} $\eval_C$.
\end{definition}

Observe that the products of functions in equation \eqref{eq:eval_v}
and \eqref{eq:eval} are formally defined via the universal property of
the product: for any collection of functions $f_j:X\ra R$, the product
$f=\prod_jf_j$ is the unique function $f:X\ra \prod_jR$ such that
$f_j=\pi_j\circ f$ for any $j$. This allows us to easily show the
following property.

\begin{proposition}
  The evaluation of an arithmetic circuit is an arrow (a function) in
  the category of interest. In particular, the evaluation of a circuit
  over $(R,\Tbasis)$ is a morphism of $R$-modules.
\end{proposition}

The converse is not necessarily true : not any arrow of the category
can be computed by a circuit. However for the basis $\Tbasis$ it is
possible to give a partial converse.

\begin{proposition}
  Any morphism of free finite-dimensional $R$-modules can be computed
  by an arithmetic circuit over $(R,\Tbasis)$.
\end{proposition}
\begin{proof}
  Take the matrix associated to such morphism and create a circuit
  that performs the matrix-vector product.
\end{proof}

Circuits can be composed in a natural way by connecting their inputs
and outputs.

\begin{definition}[Composition of circuits]
  Let $C=(V,E)$ and $C'=(V',E')$ be two circuits over
  $(R,\mathcal{B})$ and let $\mathcal{R}:\outp(C)\ra\inp(C')$ be an injective
  partial function. The composition of $C$ and $C'$ through
  $\mathcal{R}$, noted $C'\overset{\mathcal{R}}{\circ}C$, is the
  circuit $(V'', E'')$ such that
  \begin{itemize}
  \item $V'' = \bigr(V - \mathcal{R}^{-1}(\inp(C'))\bigr) \uplus
    \bigl(V' - \mathcal{R}(\outp(C))\bigr)$ with the orders inherited
    from $V$ and $V'$ and the supplementary condition that $v<v'$
    whenever $v\in V$ and $v'\in V'$.
  \item $E''(o) = \begin{cases}
      E'(o') & \text{if $E(o)\in\inp(v')$ and $\mathcal{R}(v') = v''$ and $\outp(v'')=\{o'\}$,}\\
      (E\uplus E')(o) & \text{otherwise.}
    \end{cases}$
  \end{itemize}

  When the function $\mathcal{R}$ is total, bijective and monotone
  (w.r.t. the orders on $V$ and $V'$), we say the the composition is
  \emph{natural} and simply note $C'\circ C$.
\end{definition}


\begin{figure}[!ht] \centering
  \begin{tikzpicture}
    \tikzstyle{node}=[circle,thick,draw=black,minimum size=4mm]
    \tikzstyle{arg}=[rectangle,thin,draw=black,minimum size=4mm]
    
    \begin{scope}
      \node[arg](in1){$x_1$};
      \node[arg,right of=in1](in2){$x_2$};
      
      \node[node,below of=in1](H){$\hub$};

      \node[arg,below of=H, xshift=-6mm](out1){$y_1$};
      \node[arg,right of=out1](out2){$y_2$};
      \node[arg,right of=out2](out3){$y_3$};

      \path[->]
      (in1) edge (H)
      (in2) edge (out3)
      (H) edge (out1)
      (H) edge (out2);

      \node[arg,below of=out1,yshift=-6mm](in3){$x'_1$};
      \node[arg,right of=in3](in4){$x'_2$};
      \node[arg,right of=in4](in5){$x'_3$};

      \node[node,below of=in3,xshift=6mm](plus1){$+$};
      
      \node[node,below of=plus1,xshift=6mm](plus2){$*$};

      \node[arg,below of=plus2](out4){$y'_1$};

      \path[->]
      (in3) edge (plus1)
      (in4) edge (plus1)
      (plus1) edge (plus2)
      (in5) edge (plus2)
      (plus2) edge (out4);

      \path[->,dashed]
      (out2) edge (in5)
      (out3) edge (in3);
    \end{scope}

    \begin{scope}[xshift=3cm,yshift=-3cm]
      \Large
      \node(=>){$\Ra$};
    \end{scope}

    \begin{scope}[xshift=4cm,yshift=-2cm]
      \node[arg](in1){$x_1$};
      \node[arg,right of=in1](in2){$x_2$};
      \node[arg,right of=in2](in4){$x'_2$};
      
      \node[node,below of=in1,xshift=6mm](H){$\hub$};

      \node[node,below of=in2,xshift=6mm](plus1){$+$};
      
      \node[node,below of=plus1, xshift=-4mm](plus2){$*$};

      \node[arg,below of=plus2,xshift=4mm](out4){$y'_1$};
      \node[arg,left of=out4](out1){$y_1$};

      \path[->]
      (in1) edge (H)
      (in2) edge[bend left=40] (plus1)
      (H) edge[bend left=40] (plus2)
      (H) edge (out1);

      \path[->]
      (in4) edge[bend right=40] (plus1)
      (plus1) edge[bend right=40] (plus2)
      (plus2) edge (out4);
    \end{scope}
  \end{tikzpicture}
  \caption{Composition of circuits through $\mathcal{R} = \{y_2\mapsto
    x'_3, y_3\mapsto x'_1\}$.}
  \label{fig:composition}
\end{figure}
An example of composition is given in figure \ref{fig:composition}.
It is evident that the concept of natural composition coincides with
composition of functions, the proof of the following proposition is
trivial.

\begin{proposition}
  Let $C$ and $C'$ be circuits over $(R,\mathcal{B})$ and let $C$
  have as many outputs as $C'$ has inputs, then
  \[\eval_{C'\circ C} = \eval_{C'}\circ\eval_{C} \text{.}\]
\end{proposition}

Another evidence that we state without proof is the fact that any
circuit over $(R,\mathcal{B})$ can be obtained by composition of small
elementary circuits.

\begin{proposition}
  We call \emph{elementary} a circuit that has one unique evaluation
  node. Any circuit can be obtained by composition of elementary
  circuits.
\end{proposition}

We finally define a way of substituting nodes, first syntactically,
then semantically.

\begin{definition}[Syntactic substitution]
  Let $C=(V,E)$ be a circuit over $(R,\mathcal{B})$ and let
  $C'=(V',E')$ be a circuit over $(R,\mathcal{B}')$. Let $C'$ have $i$
  inputs and $o$ outputs and let $v\in V$ have in-degree $i$ and
  out-degree $o$. 

  Let $\mathcal{I}$ and $\mathcal{O}$ be monotone bijections
  respectively from $\inp(v)$ to $\inp(C')$ and from $\outp(C')$ to
  $\outp(v)$. We note by $C[C'/v]$ the circuit $(V'',E'')$ over
  $(R,\mathcal{B}\cup\mathcal{B}')$ defined as follows:
  \begin{itemize}
  \item $V'' = V\uplus (V' - \inp(C') - \outp(C'))$,
  \item $E''(o) = \begin{cases}
      E'(o') & \text{if $E(o)\in\inp(v)$ and $\mathcal{I}(E(o))=v'$ and $\outp(v')=\{o'\}$,}\\
      E(o') & \text{if $E'(o)\in\outp(C')$ and $\mathcal{O}(E'(o))=o'$,}\\
      (E\uplus E')(o) & \text{otherwise.}
    \end{cases}$
  \end{itemize}
\end{definition}

\begin{definition}[Semantic substitution]
  Let $C$ be a circuit over $(R,\mathcal{B}\cup\{f\})$ and let $F$ be
  a circuit over $(R,\mathcal{B})$ such that $\eval_F=f$.

  We note by $C[F/f]$ the circuit over $(R,\mathcal{B})$ where any
  node $v$ of $C$ such that $\beta(v)=f$ has been syntactically
  substituted by $F$.

  It is obvious that
  \[\eval_{C[F/f]} = \eval_C \text{.}\]
\end{definition}

As a shorthand notation, we will draw octogones to signify that a node
has been syntactically substituted by a circuit, without giving the
actual shape of the substituting circuit. Figure
\ref{fig:substitution} shows an example.

\begin{figure}[!ht]
  \label{fig:substitution}
  \centering
  \begin{tikzpicture}
    \tikzstyle{node}=[circle,thick,draw=black,minimum size=4mm]
    \tikzstyle{arg}=[rectangle,thin,draw=black,minimum size=4mm]
    \tikzstyle{subst}=[regular polygon,regular polygon sides=8,thick,draw=black,minimum size=4mm]
    
    \begin{scope}
      \node[arg](in1){$x_1$};
      \node[arg,right of=in1](in2){$x_2$};
      
      \node[node](H)[below of=in2]{$\hub$};
      \node[subst](F)[left of=H]{$F$};

      \node[arg,below of=F,xshift=-6mm](out1){$y_1$};
      \node[arg,right of=out1](out2){$y_2$};
      \node[arg,right of=out2](out3){$y_3$};

      \path[->]
      (in1) edge (F)
      (in2) edge (H)
      (H) edge[bend left=10] (F)
      (H) edge[bend right=10] (F)
      (F) edge (out1)
      (F) edge (out2)
      (F) edge (out3);
    \end{scope}

  \end{tikzpicture}
  \caption{Arithmetic circuit where a node has been syntactically
    substituted by a circuit $F$ with $3$ inputs and $3$ outputs.}
\end{figure}

This shows that circuits are a good formalism to describe
functions. However, they carry more information than simply their
evaluation; the next step is to classify them in a
complexity-theoretic way.

\begin{definition}[Size, depth]
  \label{def:size}
  Let $C$ be a circuit over $(R,\mathcal{B})$. The size of $C$, noted
  $\size(C)$ is the number of evaluation nodes in $V$; the depth of
  $C$, noted $\depth(C)$ is the length of the longest directed path
  --in a graph-theoretic sense-- in $(V,E)$.

  Sometimes it is useful to only count certain nodes. Let
  $X\subset\mathcal{B}$, the $X$-weighted size of $C$, noted
  $\size_X(C)$ is the number of nodes $v\in V$ such that $\beta(v)\in
  X$.
\end{definition}


\subsection{Coevaluation}
When dealing with a construction in category theory, it is natural to
simultaneously study its dual, that is the construction obtained by
\emph{reversing all the arrows}. If in definition \ref{def:eval} we
substitute the product $\prod^nR$ by its dual $\coprod^nR$, called
\emph{coproduct}, we obtain a new way of evaluating an arithmetic
circuit that we will call \emph{coevaluation}. We study here the
properties of coevaluation, its interest will be clear in the next
sections.

In this context, we will make an abuse by using the same notation
$R^n$ we used for the product to signify the coproduct $\coprod^nR$ in
the category. Whether $R^n$ is product or coproduct will always be
clear from the context.

\begin{definition}[Arithmetic co-operator, cobasis]
  Let $R$ be a ring. An arithmetic co-operator over $R$ is a function
  $f:R^i\ra R^o$ for some $i,o\in\N$; here $R^n$ is coproduct.

  An arithmetic $R$-cobasis is a set of arithmetic co-operators over
  $R$.
\end{definition}

In particular when the category is $\mathsf{Set}$ the coproduct is the
disjoint union of sets, thus the bases $\Sbasis$ and $\Tbasis$ make no
sense in this context.

The definitions of node and circuit naturally extend to cobases, but
we need to define a new evaluation for arithmetic circuits defined
over them.

\begin{definition}[coevaluation of an arithmetic circuit]
  \label{def:coeval}
  Let $C$ be an arithmetic circuit with $i$ inputs and $o$ outputs
  over a cobasis $\mathcal{B}$. Its coevaluation is a function
  $\lave_C:R^i\ra R^o$.

  We use the same notation as in definition \ref{def:eval}. As we did
  there, we simultaneously define $\lave_v$ for each $v\in V$ and
  $\lave_e$ for each $e\in E$.
  \begin{itemize}
  \item Let $v\in V$ have in-degree $m$, let its coevaluation be
    $\lave_v:R^m\ra R^o$ and let $\iota_1,\ldots,\iota_n$ be the
    canonical injections from $R$ to $R^m$. Let $i_1<_v\cdots<_vi_m$
    be the input ports of $v$ and let
    $e_j=\bigl(i_j,E^{-1}(i_j)\bigr)$ be the corresponding edges
    incident to $v$, then $\lave_{e_j} = \lave_v\circ\iota_j$ for any
    $j$.
  \item Let $y_1<_V\cdots<_Vy_n$ be the output nodes and let
    $\iota_1,\ldots,\iota_o$ be the canonical injections from $R$ to
    $R^o$, then $\lave_{y_j}=\iota_j$ for any $j$.
  \item For every evaluation node $v$ with out-degree $n$, let
    $o_1<_v\cdots<_vo_n$ be the output ports of $v$ and let
    $e_j=\bigl(E(o_j),o_j\bigr)$ be the corresponding edges
    stemming from $v$, then
    \begin{equation}
      \label{eq:lave_v}
      \lave_v = (\lave_{e_1} \oplus \cdots \oplus \lave_{e_n}) \circ \beta(v) 
      \text{.}
    \end{equation}
  \item For every input node $x$, let $e\in E$ be the only edge
    stemming from $x$, then $\lave_x=\lave_e$.
  \end{itemize}

  We can finally define $\lave_C:R^i\ra R^o$. Let $x_1<_V\cdots<_Vx_i$
  be the input nodes, then
  \begin{equation}
    \label{eq:lave}
    \lave_C = \lave_{x_1} \oplus \cdots \oplus \lave_{x_i}
    \text{.}
  \end{equation}
\end{definition}

As before, the sums of equations \eqref{eq:lave_v} and \eqref{eq:lave}
are formally defined via the universal property of the coproduct.

The coevaluation in general does not attach the same semantics to a
circuit as the evaluation. For example in the case of $\mathsf{Set}$
the coevaluation is a function from the disjoint union of $i$ copies
of $R$ to the disjoint union of $o$ copies of $R$. We can regard
circuits over cobases in $\mathsf{Set}$ as objects that are fed one
single element of $R$ on one out of their $n$ inputs and then take
decisions depending on which input was fed. An example is given in
figure \ref{fig:coffee}.

\begin{figure}[!ht]
  \centering
  
  \begin{tikzpicture}
    \tikzstyle{node}=[circle,thick,draw=black,minimum size=4mm]
    \tikzstyle{arg}=[rectangle,thin,draw=black,minimum size=4mm]

    \begin{scope}
      \node[arg](in){$x$};

      \node[node,below of=in](s10){$\ge_{10}$};

      \node[node,right of=s10,xshift=2mm](s20){$\ge_{20}$};
      \node[arg,below of=s10](o10){$10$c};

      \node[node,right of=s20,xshift=2mm](s50){$\ge_{50}$};
      \node[arg,below of=s20](o20){$20$c};

      \node[node,right of=s50,xshift=2mm](s1){$\ge_{100}$};
      \node[arg,below of=s50](o50){$50$c};

      \node[arg,below of=s1](o1){$1$\euro};
      \node[arg,right of=o1](o2){$2$\euro};

      \path[->]
      (in) edge (s10)
      (s10) edge (o10)
      (s10) edge (s20)
      (s20) edge (o20)
      (s20) edge (s50)
      (s50) edge (o50)
      (s50) edge (s1)
      (s1) edge (o1)
      (s1) edge (o2);
    \end{scope}
  \end{tikzpicture}  
  
  \caption{The coffee machine circuit. On input $r\in R$, the operator
    $\ge_x:R\ra R\uplus R$ gives $r$ on its first output if $r\ge x$, on its
    second output otherwise. The circuit is an euro coin separator.}
  \label{fig:coffee}
\end{figure}

In some cases, howevever, evaluation and coevaluation coincide. The
following lemma shows one important case when this happens.

\begin{lemma}
  \label{th:coeval}
  Let $C$ be a circuit over $(R,\Tbasis)$. In the category $\RMod{R}$
  $\eval_C\simeq\lave_C$.
\end{lemma}
\begin{proof}
  For finite dimensional modules, the product and the direct sum are
  the same object. More formally, when working in $\RMod{R}$ there is
  a natural isomorphism $\prod^n R\simeq\coprod^n R$ for any $n$ (this
  is true for any additive category, see \cite[VIII.2]{McLane}). Thus
  $\Tbasis$ is both a basis and a cobasis, up to isomorphism, and both
  evaluation and coevaluation of circuits over it are meaningful.

  The rest of the proof is just induction on the size of the
  circuit. First, it is obvious that for elementary circuits with an
  unique evaluation node $v$ we have $\eval_C \simeq \beta(v) \simeq
  \lave_C$. Then it suffices to show that the property is maintained
  upon composition of circuits.
\end{proof}

The equivalence of evaluations and coevalutions suggests that there is
some unexploited symmetry in circuits over $\Tbasis$. The next
section explores it.


\subsection{The transposition theorem}
\label{sec:tellegen}

Since we are in a non-commutative setting, we have to precisely define
what we mean by ``transposition''. We start by recalling some well
known facts.


The transposition theorem says that from a circuit that computes the
matrix-vector product $x\mapsto \trans{x}M$ for a fixed matrix $M$, one can
deduce another circuit that computes $y\mapsto My$. We now
give such construction.

\begin{definition}[Dual circuit]
  \label{def:dual}
  Let $C=(V,E)$ be a circuit over $(R,\Tbasis)$, the dual circuit of
  $C$, noted $\dual{C}$, is the arithmetic circuit $\dual{C} = (V',
  E)$ over $(R^{\op},\Tbasis)$ where for any node $v=(I,O,f)$ in $V$
  there is a node $\dual{v}=(O,I,f')$ in $V'$ where
  \begin{equation}
    \label{eq:dual-circuit}
    f' = \begin{cases}
      \rmul{a} & \text{if $f = \rmul{a^\op}$}\\
      + & \text{if $f = \hub$}\\
      \hub & \text{if $f = +$}\\
      \omega & \text{if $f = 0$}\\
      0 & \text{if $f = \omega$}
    \end{cases}
  \end{equation}
  The ordering of $V'$ is the same as the one of $V$.
\end{definition}

In particular, this makes $(V',E)$ the reverse graph of $(V,E)$ in a
graph-theoretic sense. Figure \ref{fig:circuit} shows two circuits
that are each other's dual.

\begin{theorem}[Transposition theorem, Fiduccia '72]
  \label{th:tellegen}
  Let $C$ be a circuit over $(R,\Tbasis)$ that computes a module
  homomorphism $f$, then $\dual{C}$ computes the dual homomorphism
  $\dual{f}$.
\end{theorem}
\begin{proof}
  We apply the functor $\dual{()}$ to the construction of $\eval_C$ as
  defined in \ref{def:eval}. It is routine to verify that
  \begin{itemize}
  \item canonical projections in $\RMod{R}$ are taken to canonical
    injections in $\RMod{R^\op}$;
  \item products are taken to coproducts, in particular
    $\dual{\left(\prod^n R\right)} \simeq \coprod^nR^\op$ and the isomorphism is canonical;
  \item any morphism $t\in\Tbasis$ is taken it to its dual $\dual{t}$
    in $\Tbasis^\op$, the same correspondences as in equation
    \eqref{eq:dual-circuit} hold.
  \end{itemize}
  Furthermore, since $\dual{()}$ is contravariant, all the arrows are
  reversed. By comparing this with definitions \ref{def:coeval} and
  \ref{def:dual}, we clearly see that up to a natural isomorphism
  $\dual{\eval_C}\simeq\lave_{\dual{C}}$.

  The claim follows from lemma \ref{th:coeval}.
\end{proof}

\begin{corollary}
  A linear function $f:R^n\ra R^m$ and its dual can be computed by
  arithmetic circuits of same sizes and depths. In particular if $C$
  computes $f$ and $\dual{C}$ computes $\dual{f}$,
  \begin{gather*}
    \size_{\{+\}}(C) = \size_{\{\hub\}}(\dual{C}), \qquad  
    \size_{\{\hub\}}(C) = \size_{\{+\}}(\dual{C}), \\
    \size_{\{\rmul{a}\}}(C) = \size_{\{\rmul{a}\}}(\dual{C})\quad\text{for any $a\in R$},\\
    \size_{\{0\}}(C) = \size_{\{\omega\}}(\dual{C}), \qquad 
    \size_{\{\omega\}}(C) = \size_{\{0\}}(\dual{C}).
  \end{gather*}
\end{corollary}

In the commutative world, if $M$ is the matrix associated to the
homomorphism $f$, $\trans{M}$ is the matrix associated to $\dual{f}$
(up to isomorphism), but when $R$ is not commutative
$\trans{(AB)}\ne\trans{B}\trans{A}$ and this correspondence fails. We
can nevertheless define another transformation on circuits that acts
in a way similar to transposition in some special cases.

\begin{definition}[Opposite circuit]
  Let $C=(V,E)$ be a circuit over $(R,\Tbasis)$, the \emph{opposite
    circuit} of $C$, noted $C^\op$, is the arithmetic circuit over
  $(R^{\op},\Tbasis)$ where any $\beta(v)=\rmul{a}$ has been changed to $\rmul{a^\op}$.
\end{definition}

Now the opposite of a circuit certainly computes some $R^\op$-module
homomorphism, but there is no unique relationship between $\eval_C$
and $\eval_{C^\op}$ due to the lack of commutativity. A special case
of interest is stated in the next corollary.

\begin{definition}[Bilinear chain]
  Let $R$ be non-commutative and let $S\subset R$ be a subring of its
  center. A circuit $C$ over $(R,\Tbasis)$ such that no directed path
  in $C$ contains two nodes $v\ne v'$ with $\beta(v)=\rmul{a}$ and
  $\beta(v')=\rmul{a'}$ where $a,a'\not\in S$ is called an
  $S$-\emph{bilinear chain}.
\end{definition}

\begin{corollary}
  Let $C$ be a bilinear chain and let $\eval_C(x)=\trans{x}M$ for some
  matrix $M$, then $\eval_{C^\op}(x)=\trans{M}x$.
\end{corollary}
\begin{proof}
  Let $x_1,\ldots,x_n$ be the inputs of $C$ and $y_1,\ldots,y_m$ its
  outputs and let $M$ be the $n\times m$ matrix associated to
  $\eval_C$. Each path connecting $x_i$ to $y_j$ contributes linearly
  to the entry $(M)_{i,j}$; more precisely, let $p$ be a path from
  $x_i$ to $y_j$ and let $\rmul{a_1},\ldots,\rmul{a_h}$ be the
  multiplication nodes appearing (in that order) on it, we associate
  to $p$ the element $C_p=a_1\cdots a_h\in R$, then
  \begin{equation}
    (M)_{i,j}=\sum_{p\in\mathcal{P}(x_i,x_j)}C_p
  \end{equation}
  where $p$ ranges over the paths from $x_i$ to $x_j$. Obviously, to
  any path in $C$ corresponds an unique path in $C^\op$, by nothing
  them with the same letter $p$ we have
  \begin{equation}
    C_p^\op=a_1^\op\cdots a_n^\op=(a_n\cdots a_1)^\op=(C_p)^\op
    \text{ ,}
  \end{equation}
  where the last equality comes from the fact that all the $a_i$
  except at most one are in the center of $R$. We conclude that if
  $M'$ is the matrix associated to $C^\op$ then
  \begin{equation}
    (M')_{i,j}=\sum_{p\in\mathcal{P}(x_i,x_j)}C_p^\op=(M)_{i,j}^\op
    \text{ ;}
  \end{equation}
  and thus $\trans{x}M'=\trans{\left(\trans{M}x\right)}$.
\end{proof}



\subsection{Uniformity}
\label{sec:uniformity}

A circuit is limited to compute one specific function with inputs and
outputs of fixed size (in term of elements of $R$). However complexity
theory is interested in algorithms that compute on inputs of variable
size. This will lead us to study families of circuits.

\begin{definition}[Circuit family]
  Let $R$ be a ring, $\mathcal{B}$ a basis over $R$ and $\pspace$ a
  set. A \emph{circuit family} over $(R,\mathcal{B},\pspace)$ is a
  family of circuits over $(R,\mathcal{B})$ indexed by $\pspace$.
  $\pspace$ is called the \emph{parameter space} of the family.
\end{definition}

Algebraic complexity textbooks usually take $\pspace=\N$ and force
$C_n$ to have $n$ inputs. Our construction is more general and its
interest will be clear in the next sections.

\begin{definition}[Size and depth functions]
  Let $\mathcal{C} = (C_j)_{j\in\pspace}$ be a circuit family, we
  define the size and depth function as
  \begin{align*}
    \size^{\mathcal{C}}:\pspace&\ra\N  & \depth^{\mathcal{C}}:\pspace&\ra\N\\
                   j&\mapsto\size(C_j) &    j&\mapsto\depth(C_j)
  \end{align*}
  respectively.

  As in definition \ref{def:size}, for $X\subset\mathcal{B}$ we also
  define
  \begin{align*}
    \size^{\mathcal{C}}_X:\pspace&\ra\N\\
                     j&\mapsto\size_X(C_j)
                     \text{ .}
  \end{align*}
\end{definition}

Circuit families are interesting in theory and theorem
\ref{th:tellegen} easily generalizes to them. However this model of
computation is strictly more powerful than the BSS one (see \cite[Obs.
2.3]{Vollmer}). It is sometimes desirable to restrict oneself to a
more constrained model.

\begin{definition}[Uniform circuit family]
  Let $\mathcal{C} = (C_j)_{j\in\pspace}$ be a circuit family. Fix a
  binary representation of arithmetic circuits over $(R,\mathcal{B})$
  and a binary representation of $\pspace$.

  $\mathcal{C}$ is said to be uniform if there is a multi-band Turing
  machine $\mathcal{M}$ that on input $x$ writes the binary
  representation of $C_x$ on its output tape if $x$ is the binary
  representation of an element of $\pspace$, or an error token otherwise.

  The complexity of $\mathcal{M}$ is called the \emph{uniform
    complexity} of $\mathcal{C}$.
\end{definition}

We are mainly interested in uniform circuit families since they are
equivalent to computable functions, theorem \ref{th:tellegen} easily
generalizes to them. We won't study uniform circuit families more in
depth. Instead, we directly work on computer programs that implicitly
represent circuit families, and automatically deduce the transposed
family without actually using the circuit model. More details on
uniform circuit families can be found in \cite{Vollmer}.



% Local Variables:
% mode:flyspell
% ispell-local-dictionary:"american"
% mode:TeX-PDF
% mode:reftex
% TeX-master: "../these"
% End:
%

\section{Multilinearity}
\label{sec:multi}

In this section we develop an extension of the transposition theorem
to the multilinear case.  The subject has already been treated by
Hopcroft and Musinski \cite{hopcroft+musinski73} and Fiduccia
\cite{fiduccia:phd}, this section is a mild generalization of their
methods.

\subsection{Multilinear circuits}
\label{sec:multilinear-circuits}
We shall consider \index{multilinear~circuit}\emph{multilinear
  circuits}, i.e.\ circuits that, besides the operators
of~\ref{eq:tbasis}, also contain binary multiplication nodes. The
definitions of the previous section must be adapted to deal with
operators that are not module morphisms, but this generalization is
straightforward (see also Appendix~\ref{cha:basic-categ-theory} for a
clean way of defining arithmetic circuits that supports both linear
and arbitrary circuits).

Multilinear circuits are constructed using the
\index{standard~multilinear~basis}\emph{standard multilinear basis}
$\Sbasis$
\begin{equation}
  \label{eq:sbasis}
  \tag{$\Sbasis$}
  \begin{aligned}
    + : R\times R &\ra R\text{,}    & * : R\times R &\ra R\text{,} &  \hub : R &\ra R\times R\text{,}\\
        a, b &\mapsto a+b\text{,}   &     a, b &\mapsto ab\text{,} &         a &\mapsto a,a\text{,}\\ \\
    \eta_a : \{\bom\} &\ra R\text{,}     &&& \omega : R &\ra \{\bom\}\text{,} \\
          \bom &\mapsto a\text{,} &&&          a &\mapsto \bom\text{.}
  \end{aligned}
\end{equation}


We are going to define a transformation process that transforms a
circuit over $(R,\Sbasis)$ in a (uniform) circuit family over
$(R,\Tbasis)$; the idea is to make any bilinear multiplication node
linear by \emph{fixing} one of its inputs (see
Figure~\ref{fig:linearization}). We call this process
\emph{linearization}, a special case of it has been used in
\cite{gashkov+gashkov05,sergeev08} to transpose circuits that compute
differentials.

\begin{definition}[Zero edge, null edge]
  Let $C$ be a circuit over $(R,\Sbasis)$, a \emph{zero edge} is any
  edge $e$ in $C$ such that one of the following conditions holds:
  \begin{itemize}
  \item $e$ stems from a node $v$ with $\beta(v)=\eta_0$, such an edge
    is also called a \emph{normal} zero edge;
  \item $e$ stems from a node $v$ with $\beta(v)=+$ and whose incident
    edges are both zero;
  \item $e$ stems from a node $v$ with $\beta(v)=*$ and such that at
    least one of the incident edges of $v$ is zero;
  \item $e$ stems from a node $v$ with $\beta(v)=\hub$ and whose input
    edge is zero.
  \end{itemize}
  A \emph{null edge} is any edge $e$ such that one of the following
  conditions holds:
  \begin{itemize}
  \item $e$ is incident to a node $v$ with $\beta(v)=\omega$, such an
    edge is also called a \emph{normal} null edge;
  \item $e$ is incident to a node $v$ with $\beta(v)=\hub$ and whose
    stemming edges are both null;
  \item $e$ is incident to a node $v$ with $\beta(v)\in\{+,*\}$ such
    that its stemming edge is null.
  \end{itemize}
  An output node whose incident edge is zero is called a \emph{zero
    output}, an input node whose stemming edge is null is called a
  \emph{null input}.  A \emph{normal} circuit is a circuit whose zero
  and null edges are all normal.
\end{definition}

Notice that the evaluation of a zero edge or output is the zero
function, the converse is not true.  There is an obvious normalization
technique that takes a generic circuit and transforms it in a normal
circuit having the same evaluation; clearly, the normalization does
not increase the size and the depth of the circuit (it generally
increases $\size_{\{\eta_0,\omega\}}$, though). When necessary, we
will restrict ourselves to normal circuits.

\begin{definition}[Linearization]
  \label{def:linearization}
  \index{linearization}
  Let $C=(V,E)$ be a circuit over $(R, \Sbasis)$. Let $0=\{v\in
  V|\beta(v)=\eta_0\}$, a linearization of $C$ is a subset
  $\ell\subset\inp(C)\cup 0$ such that:
  \begin{itemize}
  \item for every $v\in V$ with $\beta(v)=+$ either none of its
    incident edges is reachable from $\ell$, or both are;
  \item for every $v\in V$ with $\beta(v)=*$ at most one of the edges
    incident to $v$ is reachable from $\ell$; if $R$ is
    non-commutative, such edge is always the right (left) edge and the
    linearization is called a
    \index{linearization!left}\index{linearization!right}\emph{left
      (right) linearization}.
  \end{itemize}
  If $\ell=\emptyset$, the linearization is called
  \index{linearization!trivial}\emph{trivial}.

  The elements of $\ell\cap\inp(C)$ and $s = \inp(C) - \ell$ are
  respectively called the \index{linear~input}\emph{linear} and
  \index{scalar~input}\emph{scalar} inputs. An edge reachable from
  $\ell$ is called \index{linear~edge}\emph{linear},
  \index{scalar~edge}\emph{scalar} otherwise.
\end{definition}


\begin{definition}[Linearized circuit, scalar part]
  Let $C=(V,E,\le,\le_i,\le_o)$ be a normal circuit over $(R,\Sbasis)$
  and let $\ell$ be a left (resp. right) linearization; without loss
  of generality, we suppose that all the scalar inputs precede the
  linear inputs in the order $\le_i$ (one can always permute inputs).
  
  Let $n$ be the number of scalars for the linearization $\ell$ and
  let $x_1,\ldots,x_n$ be distinct indeterminates over $R$. The
  \index{linearized~circuit}\emph{linearized circuit}
  \begin{equation*}
    C_\ell=(V_\ell,E_\ell,\le_\ell,\le_{i,\ell},\le_{o,\ell})
  \end{equation*}
  is the circuit over $(R[x_1,\ldots,x_n], \Tbasis)$
  (resp. $(R^\op[x_1,\ldots,x_n],\dual{\Tbasis})$) obtained from $C$
  as follows:
  \begin{itemize}
  \item $E_\ell$ is the subset of $E$ containing the linear edges,
  \item $V_\ell$ are the nodes of $V$ adjacent to $E_\ell$, where
    $\beta(v)$ has incurred the following substitutions:
    \begin{itemize}
    \item $\eta_0$ becomes $0$; $+, \hub, \omega$ are preserved;
    \item if $\beta(v)=*$, let $e$ be the only non-linear edge
      incident to $v$ and let
      $a=\eval_e(x_1,\ldots,x_n,\bullet,\ldots,\bullet)$, then $*$
      becomes $\rmul{a}$;
    \item The orders $\le_\ell,\le_{i,\ell},\le_{o,\ell}$ are the
      restriction to $V$ of the original ones.
    \end{itemize}
  \end{itemize}

  The sets $V-V_\ell$ and $E-E_\ell$ are called the \emph{scalar part}
  of $C$.
\end{definition}

Observe that non-linear edges do not depend on linear inputs, thus the
substitution for $*$ is well defined. The trivial linearization gives
the trivial linear circuit with no nodes, hence, its evaluation is the
trivial map $0\ra0$. Figure \ref{fig:linearization} shows an example
of linearized circuit (in the case $R$ is commutative), we gray out
the scalar part of the circuit.

\begin{figure}[!ht]
  \centering
  \begin{tikzpicture}
    \tikzstyle{node}=[circle,thick,draw=black,minimum size=4mm]
    \tikzstyle{arg}=[rectangle,thin,draw=black,minimum size=4mm]
    \tikzstyle{nodeg}=[circle,thick,draw=gray,minimum size=4mm]
    \tikzstyle{argg}=[rectangle,thin,draw=gray,minimum size=4mm]
    
    \begin{scope}
      \node[arg](in1){$x_1$};
      \node[arg,right of=in1](in2){$x_2$};
      \node[arg,right of=in2](in3){$x_3$};
      \node[node,below of=in1,xshift=5mm](times1){$*$};
      \node[node,below of=times1,xshift=5mm](times2){$*$};
      \node[arg,below of=times2](out){$y_1$};

      \path[->]
      (in1) edge (times1)
      (in2) edge (times1)
      (times1) edge (times2)
      (in3) edge (times2)
      (times2) edge (out);
    \end{scope}

    \begin{scope}[xshift=4cm]
      \node[argg](in1){\color{gray}{$x_1$}};
      \node[arg,right of=in1](in2){$x_2$};
      \node[argg,right of=in2](in3){\color{gray}{$x_3$}};
      \node[node,below of=in1,xshift=5mm](times1){$\rmul{x_1}$};
      \node[node,below of=times1,xshift=5mm](times2){$\rmul{x_3}$};
      \node[arg,below of=times2](out){$y_1$};

      \path[->]
      (in2) edge (times1)
      (times1) edge (times2)
      (times2) edge (out);
      
      \path[->,draw=gray,dashed]
      (in1) edge (times1)
      (in3) edge (times2);
    \end{scope}

    \begin{scope}[xshift=8cm]
      \node[arg](in1){$x_1$};
      \node[arg,right of=in1](in2){$\dual{y_1}$};
      \node[arg,right of=in2](in3){$x_3$};
      \node[node,below of=in3,xshift=-5mm](times1){$*$};
      \node[node,below of=times1,xshift=-5mm](times2){$*$};
      \node[arg,below of=times2](out){$\dual{x_2}$};

      \path[->]
      (in2) edge (times1)
      (times1) edge (times2)
      (times2) edge (out)
      (in1) edge (times2)
      (in3) edge (times1);
    \end{scope}
  \end{tikzpicture}
  \caption{A circuit over a commutative ring, its linearization for
    $\ell=\{x_2\}$ (scalar edges are grayed out) and its
    $\ell$-dual.}
  \label{fig:linearization}
\end{figure}

Any linearized circuit obviously defines an uniform circuit family
over $(R,\Tbasis)$ (resp. $(R^\op,\dual{\Tbasis})$), thus we can apply
the transposition theorem to the family. But there's more: from a
linearized circuit we can deduce a new circuit over $(R,\Sbasis)$ that
computes the same function as the transposed family.

\begin{definition}[$\ell$-dual]
  \label{def:ell-dual}\index{arithmetic~circuit!l-dual@$\ell$-dual}
  Let $C$ be a normalized circuit over $(R,\Sbasis)$ and let $\ell$ be
  a linearization.  The $\ell$-dual of $C$ is the circuit over
  $(R,\Sbasis)$ obtained by dualizing the linearized circuit $C_\ell$,
  then connecting back the edges of the scalar part to the
  corresponding nodes in $C_\ell$: in doing this nodes with
  $\beta(v)=\rmul{a}$ are changed back to $\beta(v)=*$. The order on
  the nodes of the $\ell$-dual is arbitrary.
\end{definition}

By abuse of notation, the $\ell$-dual will also be noted
$\dual{C_\ell}$. Figure \ref{fig:linearization} shows an example of
$\ell$-dual; notice that $\dual{C_\ell}$ is only defined up to
reordering of the nodes, we will adopt the convention of preserving
the ordering of the linearized circuit, while we take the freedom to
permute the scalar part as it will be more convenient.

\begin{proposition}
  The size of the $\ell$-dual is the same as that of $C$, more
  precisely
  \begin{align*}
    \size_{\{+,\hub\}}(C) &= \size_{\{+,\hub\}}(\dual{C_\ell}), 
    &\size_{\{*\}}(C) &= \size_{\{*\}}(\dual{C_\ell}),\\
    \size_{\{\eta_0,\omega\}}(C) &= \size_{\{\eta_0,\omega\}}(\dual{C_\ell}), 
    &\size_{\{\eta_a|a\ne0\}}(C) &= \size_{\{\eta_a|a\ne0\}}(\dual{C_\ell}).
  \end{align*}
  Its depth is at most twice that of $C$.
\end{proposition}


\subsection{Bilinear chains}
\label{sec:bilinear-chains}

The case of bilinear circuits has received particular interest because
it permits to give lower bounds on the complexity of matrix
multiplication~\cite{fiduccia:phd}. In this section we just point out
how the results of Hopcroft and Musinski~\cite{hopcroft+musinski73}
and Fiduccia~\cite{fiduccia:phd} reduce to ours.

\begin{definition}[Linear chain]
  Let $R$ be non-commutative and let $S\subset R$ be a subring of its
  center. A circuit $C$ over $(R,\Tbasis)$ such that no directed path
  in $C$ contains two nodes $v\ne v'$ with $\beta(v)=\rmul{a}$ and
  $\beta(v')=\rmul{a'}$ where $a,a'\not\in S$ is called an
  \index{linear~chain}\emph{$S$-linear chain}.
\end{definition}

We have seen in Remark~\ref{rk:tellegen} that in the non commutative
case the transposition principle does not \emph{transpose
  matrices}. It is however possible to transpose linear chains.

\begin{definition}[Opposite circuit]
  Let $C=(V,E)$ be a circuit over $(R,\Tbasis)$, the
  \index{arithmetic~circuit!opposite}\emph{opposite circuit} of $C$,
  noted $C^\op$, is the arithmetic circuit over $(R^{\op},\Tbasis)$
  where any $\beta(v)=\rmul{a}$ has been changed to $\rmul{a^\op}$.
\end{definition}

\begin{proposition}
  Let $C$ be a linear chain and let $\eval_C(x)=\trans{x}M$ for some
  matrix $M$, then $\eval_{C^\op}(x)=\trans{M}x$.
\end{proposition}
\begin{proof}
  This is a consequence of the
  \hyperref[th:electrical-network]{electrical network lemma}.  The
  matrix $M$ associated to $\eval_C$ is given by
  \begin{equation}
    \label{eq:243}
    m_{ij} = \pi_j\circ\eval_C\circ\iota_i(1)
    = \sum_{p\in x_i\leadsto y_j}\eval_p(1) =
    \sum_{p\in x_i\leadsto y_j} p_1p_2\cdots p_{n_p}
    \text{,}
  \end{equation}
  where $\rmul{p_1},\ldots,\rmul{p_{n_p}}$ are the scalar
  multiplication nodes on the path $p$.

  Now, by the definition of linear chain, on any path there is at most
  one element not in the center of $R$, thus
  \begin{equation}
    \label{eq:245}
    (p_1p_2\cdots p_{n_p})^\op=p_{1}^\op p_2^\op\cdots p_{n_p}^\op
    \text{,}
  \end{equation}
  and the claim follows.
\end{proof}

Thus, to any linear chain one can associate the four circuits
$C,\dual{C},C^\op,\dual{{C^\op}}$. In~\cite{hopcroft+musinski73,fiduccia:phd},
bilinear chains are considered, i.e.\ bilinear circuits whose only two
non-trivial linearizations are linear chains. The opposite circuit of
a \index{bilinear~chain}bilinear chain is defined by swapping every
multiplication node. Thus, if $C$ is a bilinear chain and
$\ell_1,\ell_2$ its linearizations, one obtains the six circuits
\[C,C^\op,\dual{C_{\ell_1}},\dual{{C_{\ell_1}^\op}},\dual{C_{\ell_2}},\dual{{C_{\ell_2}^\op}}\text{.}\]

Finally, the complexity bounds of~\cite{hopcroft+musinski73}, are
obtained by considering the sets $D=\{\rmul{a}|a\in S\}$ and
$M=\{\rmul{a}|a\not\in S\}$ and realizing that both $\size_D$ and
$\size_M$ are preserved taking the $\ell$-dual and/or the opposite.



% Local Variables:
% mode:flyspell
% ispell-local-dictionary:"american"
% mode:TeX-PDF
% mode:reftex
% TeX-master: "../these"
% End:
%

\section{From circuits to function-level programming}
\label{sec:fp}
\lstset{language=haskell}

In Section \ref{sec:circuits} we saw that the transposition theorem
holds for uniform circuit families.  Informally, an uniform circuit
family can be simulated in a BSS-like model \cite{BSS} such as a
Turing machine with an additional input/output tape where each cell
contains an element of $R$. Then the transposition theorem can be
extended to this context and it is easy to prove that some fundamental
measures such as space and time complexity are preserved by it. This
kind of approach has been used in \cite{BoLeSc03}.

We won't go further in defining a model and proving a transposition
theorem as this is just matter of posing the right definitions and
deriving the (boring) consequences. Here, instead, we are interested
in the real world scenario of transposing a computer program written
in a general purpose programming language. This section studies how an
uniform circuit family can be efficiently simulated in the Haskell
programming language and how its transposition can automatically be
obtained. Nevertheless, this approach has some limitations that we
will address in the next sections.

As we already pointed out in Section \ref{sec:circuits}, evaluation
(and co-evaluation) of arithmetic circuits can be formally defined by
means of a category with finite products (and coproducts). Arithmetic
circuits over $\Tbasis$, in particular, are defined by means of an
\emph{additive category}, that is a category such that $\hom$ sets are
abelian groups, it has \emph{zero morphisms} and it has all finite
\emph{biproducts}.

It can be proven \cite[VIII.2]{McLane} that in an additive category
finite products and coproducts are naturally isomorphic to biproducts.
This is the key argument for the proof of our lemma \ref{th:coeval}
and ultimately leads to the transposition theorem in the special case
of $\RMod{R}$.

Modern functional languages too are constructed around the concept of
category: roughly, the types of the language are viewed as objects and
programs as arrows. In this setting, categories are required to be
\emph{Cartesian closed}. This allows to perform \emph{partial
  evaluation} of a function, an operation known as
\emph{currying}. Examples of Cartesian closed categories are
$\mathsf{Set}$ and $\mathsf{Hask}$ --the category of Haskell types--.

Cartesian closed categories cannot be additive, thus it may seem that
our attempts to natively express circuits in functional languages are
doomed. However the Haskell type system is powerful enough to
represent some categories inside it, in particular subcategories of
$\mathsf{Hask}$. It provides a \emph{type class} \lstinline+Category+,
that we reproduce here --following \cite{Yor09}, we use an infix
operator \lstinline+(~>)+ instead of a prefix one as in the standard
Haskell library--.

\begin{lstlisting}
  class Category (~>) => where
    id :: (a ~> a)
    (.) :: (b ~> c) -> (a ~> b) -> (a ~> c)
\end{lstlisting}

In order to behave as a category, an instance of this class shall form
a monoid for the operation \lstinline+(.)+, with \lstinline+id+ being
the identity element. Now this class can be extended to model additive
categories: we first define a class that mimics \emph{Ab-categories},
or \emph{preadditive} categories, that is categories whose $\hom$ sets
are abelian groups.

\begin{lstlisting}
  class Category (~>) => AbCategory (~>) where
    zeroArrow :: (a ~> b)
    (<+>) :: (a ~> b) -> (a ~> b) -> (a ~> b)
\end{lstlisting}
To behave as an Ab-category, an instance shall form an abelian group
for the operation \lstinline|<+>|, with \lstinline+zero+ being the
identity element; it shall also obey a \emph{bilinear law}:
\begin{lstlisting}
  (f <+> g).(h <+> i) = f.h <+> f.i <+> g.h <+> g.i
\end{lstlisting}

Now we give one possible definition of a class that mimics additive
categories.
\begin{lstlisting}
  class AbCategory (~>) => AdditiveCategory (~>) where
    (&&&) :: (a ~> b) -> (a ~> c) -> (a ~> (b, c))
    (|||) :: (a ~> c) -> (b ~> c) -> ((a, b) ~> c)

  first f = f <**> zeroArrow
  second f = zeroArrow <**> f
  left f = f <++> zeroArrow
  right f = zeroArrow <++> f
  f *** g = (first f) ||| (second g)
\end{lstlisting}
For an instance to behave as an additive category it shall satisfy
\begin{lstlisting}
  id *** id = id
  zeroArrow &&& zeroArrow = zeroArrow ||| zeroArrow = zeroArrow
  (f &&& g) <+> (f' &&& g') = (f <+> f') &&& (g <+> g')
  (f ||| g) <+> (f' ||| g') = (f <+> f') ||| (g <+> g')
  (f ||| g) &&& (f' ||| g') = (f &&& f') ||| (g &&& g')
  (f ||| g)  .  (f' &&& g') = f.f' <+> g.g'
\end{lstlisting}

Haskell programmers may have recognized a familiar pattern:
\emph{arrows}. Arrows were introduced in \cite{Hug00} as a
generalization of \emph{monads}, they have been successfully applied
to many different settings such as, for example, solving ordinary
differential equations \cite{LH10}. They are usually understood as
circuits \cite{Pat01}: the standard library class \lstinline+Arrow+ is
roughly equivalent to evaluation of an arithmetic circuit in
$\mathsf{Hask}$ (or $\mathsf{Set}$), while \lstinline+ArrowChoice+ is
roughly equivalent to co-evaluation.

Our \lstinline+AdditiveCategory+ shares similarities with the standard
classes \lstinline+Arrow+, \lstinline+ArrowChoice+,
\lstinline+ArrowZero+ and \lstinline+ArrowPlus+. In particular, it is
equivalent to evaluation of an arithmetic circuit in an additive
category and, by lemma \ref{th:coeval}, to co-evaluation.

An arrow expression is an expression formed uniquely from elementary
arrows and the combinators of \lstinline+AdditiveCategory+. Given an
arrow expression, it is trivial to form its dual: one substitutes
\lstinline|first| with \lstinline|left|, \lstinline|second| with
\lstinline|right|, \lstinline|&&&| with \lstinline&|||& and changes
any \lstinline|f.g| into \lstinline|g.f|. This corresponds to forming
the dual circuit as in definition \ref{def:dual}, thus if
\lstinline+AdditiveCategory+ is instantiated with arrows that
correspond to the elements of the basis\footnote{It is actually enough
  to create the elementary arrows that correspond to $*_a$ for any
  $a\in R$. The combinators \lstinline+&&&+ and \lstinline+|||+
  already imply $+$ and $\hub$.} $\Tbasis$, one obtains the transposed
circuit as in theorem \ref{th:tellegen}. Also notice that Haskell
defines a \lstinline+do+-notation \cite{Pat01} to write down arrow
expressions more conveniently. Dualizing this notation boils down to
applying the transposition algorithm given in Section \ref{todo}.

By writing Haskell functions that return arrows, or, more generally,
functions that have arbitrary types involving arrows, one can express
circuit families and apply the transposition theorem to them by
dualizing each arrow expression appearing in the computation. Figure
\ref{fig:karahask} shows an example of Karatsuba multiplication
written using this technique.

\begin{figure}
  \centering
  
  \caption{The left-linear Karatsuba algorithm.}
  \label{fig:karahask}
\end{figure}


One limitation to this approach is that it cannot be implemented
inside Haskell. As we have defined them, arrows do not remember their
history: the arrow \lstinline+f.g+ is a new arrow that knows nothing
about the pieces that compose it, thus no Haskell function can
transform it in \lstinline+g.f+. There are three solutions to this:
\begin{enumerate}
\item Define \emph{biarrows} instead of arrows \cite{ASWEP05}. This
  way the dual is computed as a side effect of the computation of the
  arrow. The main advantage of this approach is that it is easy to
  implement, its main disadvantage is that it is not compatible with
  the \lstinline+do+ notation, as observed in \cite{ASWEP05}. Another
  minor disadvantage is that the dualization is not transparent as it
  does not happen at the level of the arrow expression, but inside its
  computation.
\item Write a precompiler for Haskell that dualizes arrow
  expressions. The great advantage of this approach is that it permits
  to treat \lstinline+do+-expressions.
\item Implement arrows at the type-level, instead of the function
  level, by defining a type that describes circuits. However this
  approach is not realistic because of its poor efficiency.
\end{enumerate}

A second, fundamental, limitation is that this approach requires the
user to explicitly identify the linear computations by wrapping them
in arrows. This process may be almost as hard as transposing a program
by hand as done by \cite{BoLeSc03} and the resulting code may be
difficult to read (see figure \ref{fig:karahask}). The next Section
studies an alternative approach where the user if freed from the
burden of \emph{linearizing} the program.



% Local Variables:
% mode:flyspell
% ispell-local-dictionary:"american"
% mode:TeX-PDF
% TeX-master: "transAL"
% End:
%

In this chapter we present a joint work with
Schost~\cite{df+schost10}. We study the
\index{automatic~transposition}\emph{automatic transposition} of
generic code (i.e.\ not limited to straight line programs).
Section~\ref{sec:autom-diff} has shown that this has applications in
automatic differentiation, and we will see other applications in the
next chapters.

By looking at a specific subproblem of automatic differentiation, our
goal is to be more efficient and more general. In particular, compared
to the existing implementations of AD tools, we want to:
\begin{itemize}
\item avoid unnecessary space overhead;
\item handle algebraic, rather than just numerical code;
\item handle advanced programming constructs, including recursion and
  algebraic data types;
\item transpose code parameterized by arbitrary algebraic variables.
\end{itemize}

In this chapter we shall abandon the algebraic RAM model we used in
Section~\ref{sec:stra-line-progr} and work on source code
transformation. Implementation details such as knowing what the cost
of copying variables is, shall be ignored: one can assume that a good
compiler will optimize most of these details. Hence, we shall assume
that Theorem~\ref{th:tellegen-R-algeb} really reflects the behavior of
the code we generate.


\section{Inferring linearity}
\label{sec:inference}
\lstset{language=haskell}

By looking at Section~\ref{sec:transp-algor} one sees that often we
want to transpose families of $R$-algebraic algorithms parameterized
by algebraic elements (e.g., we want to transpose the code that for
any $a\in R$ evaluates the map $b\mapsto ab$). This is also necessary
in automatic differentiation, when the code for $\diff_x f$ not only
depends on $\diff x_1,\ldots,\diff x_n$, but also on $x$.

The next section will address the question of how to transpose such
code. This section, instead, asks the question: can a compiler guess
by itself which inputs to a function are parameters, and which are
linear arguments?

The answer is yes. We show how the type system of common statically
typed functional languages can be extended to automatically infer all
the possible \index{linearization}\emph{linearizations} of a computer
program. We first present the non-commutative case, which can be fully
expressed inside the Haskell type system, then we discuss how to
extend to the commutative case.

\paragraph{Linears, scalars}
\label{sec:linears-scalars}
Suppose we have defined some data type \lstinline{R} representing
elements of a ring $R$ together with the usual constants (say
\lstinline{zeroR}, \lstinline{oneR}, etc.), arithmetic operations (say
\lstinline{plus}, \lstinline{times}, etc.), tests and so on. To
simplify, we assume --as usual in algebraic complexity theory-- that
the type \lstinline{R} is isomorphic to $R$, i.e.\ the elements of $R$
can be represented exactly, the operations do not introduce any
rounding error, etc.

For any term involving elements of type \lstinline{R} we would like
the type system to tell us whether its outputs are linear in its
inputs. For example the term
\begin{lstlisting}
  \x y -> plus x y
\end{lstlisting}
has type \lstinline{R -> R -> R}, but we would like the type checker
to also output something like $\ell\ra\ell\ra\ell$ ($\ell$ for
\index{linear~input}\emph{linear}) telling that the term is a
(curryfied) left module homomorphism from $R^2$ to $R$. For
consistency, we want to view constants as mappings from $R^0$ to $R$,
thus for the term \lstinline{zeroR} we want the type checker to
compute something like $0\ra\ell$, that we simply write as $\ell$.

Now, what do we expect about \lstinline{oneR} or \lstinline{times}?
The former is the mapping $\bom\mapsto1$, which is not a module
homomorphism; then, by analogy with
Definition~\ref{def:linearization}, we want the type checker to output
something like $0\ra s$, or simply $s$ ($s$ for
\index{scalar~input}\emph{scalar}). The second can be made into a
linear mapping by \emph{fixing} its second argument (remember that for
the moment we are restricting to left modules) as we did in
Section~\ref{sec:multi}; thus we expect the type checker to output
$\ell\ra s\ra\ell$, meaning that
\begin{lstlisting}
  \x -> times x y
\end{lstlisting}
is a left module homomorphism $R\ra R$ for any \lstinline{y::R}.

Finally consider the following term
\begin{lstlisting}
  z x n = if n <= 0 then zeroR else plus x (z x (n-1))
\end{lstlisting}
as before we expect something like $\ell\ra\N\ra\ell$, meaning that
\begin{lstlisting}
  \x -> z x n
\end{lstlisting}
is a homomorphism $R\ra R$ for any integer \lstinline{n}.

Observe that in order to make a correct inference about a term such as
\begin{lstlisting}
  \x y -> times x (plus y y)
\end{lstlisting}
we must also admit for any of the previous cases the possibility where
everything is a scalar, so that from the hypothesis that
\lstinline{plus} has type $s\ra s\ra s$ we can deduce the correct type
$\ell\ra s\ra\ell$ for the term above. Summarizing, we would like to
have two types \lstinline{L} and \lstinline{S} such that the following
equations hold
\begin{lstlisting}
  plus :: L -> L -> L
  plus :: S -> S -> S
  times :: L -> S -> L
  times :: S -> S -> S
  zeroR :: L
  zeroR :: S
  oneR :: S
\end{lstlisting}

\pdfmcone{Better with newtypes, thanks to Mathieu.}
If we define \lstinline{L} and \lstinline{S} as wrappers around
\lstinline{R}
\begin{lstlisting}
  newtype L = Lin R
  newtype S = Sca R
\end{lstlisting}
then, using Haskell type classes
\cite{Walder+Blott-ad-hoc-polymorphism}, we can conveniently express
all the equations above as
\begin{lstlisting}
  class Ring r where
    zero :: r
    (<+>) :: r -> r -> r
    neg :: r -> r
    (<*>) :: r -> S -> r
\end{lstlisting}
together with the obvious \lstinline{instance} definitions (see the
example in Appendix~\ref{cha:line-infer-karats}). 

\pdfmcone{Better terminology, thanks to Mathieu.}  For our
inference to work, it is important that \lstinline{L} be an abstract
data type with only the above functions in its interface. On the other
hand, any other function acting on \lstinline{R} can be wrapped inside
a function acting on \lstinline{S} as, for example,
\begin{lstlisting}
  one = Sca oneR
  (Sca a) == (Sca b) = a == b
\end{lstlisting}
or, simply, using a \lstinline{deriving} clause in the declaration of
\lstinline{S}
\begin{lstlisting}
  newtype S = Sca R deriving (Eq)
\end{lstlisting}

If we restrict to terms that do not use the type constructor
\lstinline{Lin}, then we can show that the semantic of a term with
type
\begin{lstlisting}
  L->...->L->L
\end{lstlisting}
is a left module homomorphism.  \pdfmcone{More details on the
  simply typed lambda calculus, as suggested by Mathieu.}  The proof
for the full language would be too long, thus we restrict to a simply
typed $\lambda$-calculus with constants. Its terms are defined by the
following grammar
\begin{equation}
  \label{eq:lambda}
  t ::= c \;|\; x \;|\; t_0 t_1 \;|\; \lambda x . t  \text{ ,}
\end{equation}
where $x$ are identifiers and $c$ are constants; its types are
defined by the grammar
\begin{equation}
  \label{eq:77}
  \tau ::= \ell \;|\; s \;|\; \beta \;|\; \tau \ra \tau
  \text{ ,}
\end{equation}
where $\beta$ are the usual base types (integers, booleans, etc.). If
$\Gamma$ is a type environment, by $\Gamma\vdash t::\tau$ we mean that
the term $t$ has type $\tau$ in $\Gamma$.  The semantic of our
calculus is the usual one, based on $\beta\eta$-reduction.

\pdfmcone{Careful about lists and tuples, thanks Mathieu!}  We
suppose all the constants above are defined, plus the usual constants
for the other base types; observe that our grammar forbids type
constructors altogether (including \lstinline{Lin} and
\lstinline{Sca}). In this context we use the type names $\ell$ and $s$
in place of the Haskell types \lstinline{L} and \lstinline{S} defined
above. For simplicity, we shall also assume that lists and tuples are
not part of the types of our language; see the end of this section for
a discussion about them.

\begin{definition}[Flipper]
  \pdfmcone{Defined the flipper and changed the proof of the
    lemma, thanks to Léo.}  Let $\tau$ be the type
  \begin{equation}
    \label{eq:58}
    \tau = \alpha_0\ra\alpha_1\ra\cdots\ra\alpha_n
  \end{equation}
  with $\alpha_n$ not a function type. Let $I\subset[0,\ldots,n-1]$
  such that $\alpha_i\ne\ell$ if and only if $i\in I$, and let
  $m=\card{I}$.  The \emph{flipper} for $\tau$, denoted by $\flip_\tau$,
  is the term
  \begin{equation}
    \label{eq:60}
    \flip_\tau = \lambda t.\lambda x_{i_1}.\ldots.\lambda x_{i_m}.
    \lambda x_{j_1}.\ldots.\lambda x_{j_{n-m}}.tx_0\cdots x_{n-1}
    \text{,}
  \end{equation}
  with $i_1,\ldots,i_m\in I$ and $j_1,\ldots,j_{n-m}\in\bar{I}$.
\end{definition}

\begin{lemma}
  \label{th:lininference}
  Let $\Gamma\vdash t::\tau$ be a term, let $m,n\ge0$, and let
  $\Gamma\vdash \flip_{\tau}t::\sigma$ with
  \begin{equation}
    \label{eq:264}
    \sigma=\alpha_1\ra\alpha_2\ra\cdots\ra\alpha_m\ra\underbrace{\ell\ra\cdots\ra\ell}_{\text{$n$ times}}\ra\beta
    \text{,}    
  \end{equation}
  with $\alpha_i\ne\ell$ and $\beta$ not a function type. Let
  $\Delta_i\vdash s_i::\alpha_i$ for $1\le i\le m$. The semantic of
  \begin{equation}
    \label{eq:270}
    \Gamma,\Delta_1,\ldots,\Delta_m\vdash \flip_\tau ts_1\cdots s_m
  \end{equation}
  is
  \begin{enumerate}
  \item\label{item:1} a constant function if $\beta\ne\ell$,
  \item\label{item:2} a module homomorphism $R^n\ra R$ if $\beta=\ell$,
  \end{enumerate}
  assuming the free variables in $t,s_1,\ldots,s_m$
  satisfy~\ref{item:1} or~\ref{item:2}.
\end{lemma}
\begin{proof}
  We distinguish the following cases.
  \begin{itemize}
  \item $\Gamma\vdash c$. All the constants satisfy
    either~\ref{item:1} or~\ref{item:2}. We just work out $0$ and $+$
    defined above and leave the others to the reader; in both cases
    $\flip_\tau c=c$ up to $\beta\eta$-conversion.
    \begin{itemize}
    \item $\Gamma\vdash0::\ell$ is the map $\bom\mapsto 0$, thus a
      (constant) morphism.
    \item $\Gamma\vdash0::s$ is the map $\bom\mapsto 0$, thus a constant
      (morphism).
    \item $\Gamma\vdash+::\ell\ra\ell\ra\ell$ is the map $a,b\mapsto
      a+b$. A morphism.
    \item $\Gamma\vdash+::s\ra s\ra s$. Take any $a::s$ and $b::s$, then
      $\Gamma\vdash a+b::s$ is a constant.
    \end{itemize}
  \item $\Gamma,x::\alpha\vdash x::\alpha$. The claim follows because
    $x$ is free.
  \item $\Gamma\vdash t_0t_1::\tau$. This is the only real case
    to prove. We distinguish two cases:
    \begin{itemize}
    \item $\Gamma\vdash t_1::\ell$, then, by induction its semantic is
      a morphism $0\ra R$ (because it is $\beta\eta$-equivalent to
      $\flip_\ell t_1$). 

      Let $\Gamma\vdash \flip_\tau t_0t_1::\sigma$, with $\sigma$ as in
      Eq.~\eqref{eq:264}, and let $\Delta_i\vdash s_i$ for $1\le i\le
      m$ be as in the hypothesis.  Let $\Gamma\vdash t_0::\tau_0$ and
      $\Gamma\vdash\flip_{\tau_0}t_0::\sigma_0$, then by induction
      \begin{equation}
        \label{eq:271}
        \Gamma,\Delta_1,\ldots,\Delta_m\vdash t_0'\eqdef\flip_{\tau_0}t_0s_1,\ldots,s_m
      \end{equation}
      is either a morphism $R^{n+1}\ra R$ or a constant function. In
      the first case $t_0't_1$ is a morphism $R^{n'}\ra R$, in the second
      case it is a constant function; in both cases
      \begin{equation}
        \label{eq:74}
        \flip_\tau(t_0t_1)s_1\cdots s_m\xleftrightarrow{\beta\eta}t_0't_1        
      \end{equation}
      and the claim follows.
    \item $\Gamma\vdash t_1::\alpha$ with $\alpha\ne\ell$. Then the
      claim follows directly by induction on $t_0$ and
      $\beta\eta$-conversion, by choosing $s_1=t_1$.
    \end{itemize}
  \item $\Gamma\vdash \lambda x.t::\alpha_1\ra\alpha_2$. By induction
    $\Gamma,x::\alpha_1\vdash t::\alpha_2$ satisfies~\ref{item:1}
    or~\ref{item:2} (assuming $x$ does). We distinguish two cases
    \begin{itemize}
    \item $\alpha_1\ne\ell$, then
      \begin{equation}
        \label{eq:73}
        \lambda x.\flip_{\alpha_2}t\xleftrightarrow{\beta\eta}\flip_{\alpha_1\ra\alpha_2}(\lambda x.t)
        \text{;}
      \end{equation}
    \item $\alpha_1=\ell$, then
      \begin{equation}
        \label{eq:76}
        \lambda x.\flip_{\alpha_2}ts_1\cdots s_m\xleftrightarrow{\beta\eta}
        \flip_{\alpha_1\ra\alpha_2}(\lambda x.t)s_1\cdots s_m
        \text{.}
      \end{equation}
    \end{itemize}
    In both cases, $\lambda x.t$ satisfies~\ref{item:1}
    or~\ref{item:2} accordingly.
  \end{itemize}
\end{proof}

\begin{proposition}
  Let $t:\tau$ be a closed term, let $n\ge0$ and let
  \begin{equation}
    \tau=\underbrace{\ell\ra\cdots\ra\ell}_{\text{$n$ times}}\ra\beta
    \text{,}    
  \end{equation}
  with $\beta$ not a function type. Then, the semantic of $t$ is
  \begin{enumerate}
  \item a constant function if $\beta\ne\ell$,
  \item a module homomorphism $R^n\ra R$ if $\beta=\ell$.
  \end{enumerate}
\end{proposition}

By the proof, it should be now clear why we forbid the type
constructor \lstinline{Lin}. In fact, introducing a term as
\lstinline{Lin oneR :: L} tricks the proof (the type checker) by
making it believe that the function $\bom\mapsto 1$ is a morphism.


\paragraph{The commutative case}
\label{sec:commutative-case}
In the commutative case we shall add a second multiplication operator
allowing multiplication on the left by a scalar
\begin{lstlisting}
  class Ring r => CommRing r where
    (>*<) :: S -> r -> r 
\end{lstlisting}
but this would force the user to chose between the two operators any
time he multiplies two elements of $R$. To avoid this we need to
overload the operator \lstinline{(<*>)} with both type signatures, a
technique sometimes called \emph{ad-hoc} polymorphism
\cite{strachey00}, but this is not possible in the Haskell type system
since the two types are contradictory.  To make it possible we need to
extend the type inference algorithm: our idea is not new, but it has
been rarely implemented because it is not practical for solving
generic \emph{ad-hoc} polymorphism; it perfectly fits the needs of our
special case, though.

First observe that type classes can be translated to ordinary types of
the Hindley-Milner type system as explained in \cite[$\S
4$]{Walder+Blott-ad-hoc-polymorphism}, thus it suffices to modify the
classic type inference algorithm
\cite{Damas+Milner,Cardelli:Typechecking}. Second, observe that there
is some redundancy between the two signatures of \lstinline{(<*>)} and
that a more concise version is
\begin{lstlisting}
  (<*>) :: Ring r => r -> S -> r
  (<*>) :: S -> L -> L
\end{lstlisting}

A review of the Hindley-Milner algorithm and its implementation can be
found in \cite{Cardelli:Typechecking}. The idea is to first assign
type variables to terms, then solve type equations by unifying them.
In our generalization, instead of handling a single unification, we
keep a list of possible unifications: when a type equation implies
that a certain unification is not acceptable, the unification is
discarded from the list; if the list gets empty the term cannot by
typed and an error is returned, otherwise any unification in the list
is valid and is returned.

In practice, the only term that makes the list of unification grow is
\lstinline{(<*>)}: any time an equation involving it has to be solved,
the list of unifications potentially doubles. This exponential
increase is the reason why this solution is not practical to solve
generic \emph{ad-hoc} polymorphism; but in our case we really are
interested in knowing all the possible types of a term because each of
them gives rise to a different linearization and, hence, to a
different transposition.

\paragraph{Modules}
\label{sec:modules}
Finally we remark that by allowing tuples and lists,
Lemma~\ref{th:lininference} can be generalized to morphisms $R^m\ra
R^n$ and even to infinite dimensional modules using lazy
lists. Elements of type \lstinline{L}, \lstinline{[L]},
\lstinline{(L,L)}, etc. share a common pattern: they can be viewed as
$R$-modules. It is convenient to summarize their properties in an
unique interface\footnote{We make use of some experimental modules of
  Haskell: this codes needs the flags
  \lstinline{-XMultiParamTypeClasses},
  \lstinline{-XFunctionalDependencies} and
  \lstinline{-XFlexibleInstances} in order to work.}
\begin{lstlisting}
  class Ring r => Module m r | m -> r where
    zeroM :: m
    (<<*) :: m -> S -> m
    (>>>) :: m -> Integer -> r
    (<<<) :: r -> Integer -> m
    (<++>) :: m -> m -> m
    add :: m -> m -> Integer -> m
    add a b n = foldl (<++>) zeroM
                [((a>>>i) <+> (b>>>i))<<<i | i <- [1..n]]
\end{lstlisting}

Instances of this class represent free $R$-modules: \lstinline{zeroM}
is the zero element, \lstinline{(<<*)} is scalar multiplication,
\lstinline{(<++>)} is addition, \lstinline{(<<<)} and
\lstinline{(>>>)} are canonical injections and projections.

This interface adds nothing to the linearity inference system, but we
will need it in Section~\ref{sec:texttttransalpyne}.  Also notice the
presence of the operator \lstinline{add} that performs addition up to
a truncation order, it is of no great importance in this section, but
for efficiency reasons we will eventually prefer it to plain addition.

A fully worked Haskell example of the ideas presented in this section
(without the extension to the commutative case) is given in
Appendix~\ref{cha:line-infer-karats} where we implement Karatsuba
multiplication of polynomials in $\Z[X]$.


% Local Variables:
% mode:flyspell
% ispell-local-dictionary:"american"
% mode:TeX-PDF
% mode: reftex
% TeX-master: "../these"
% End:
%

\section{\tAL}
\label{sec:transAL}

We saw in Section~\ref{sec:stra-line-progr} that the key to
transposition of $R$-algebraic algorithms is partial evaluation. In
this section we discuss how to implement partial evaluation on a
\index{domain~specific~language}\emph{domain specific language}
(\index{DSL@see{domain~specific~language}}DSL) similar to straight
line programs.

If we want the linearity inference of the previous section to work, we
cannot use standard straight line programs. We shall instead use a
functional language, nicknamed
\index{transAL@see{transposable~Algebraic~Language}}\tAL{} (the
\index{transposable~Algebraic~Language}transposable Algebraic
Language).

\paragraph{\tAL{}}
\label{sec:ta}
We informally describe \tAL{}: giving formal grammar and semantics
would be useless. We shall write algebraic variables in upper case,
and non-algebraic variables and constants in lower case.

For simplicity, the only algebraic instructions in \tAL{} are
\begin{align}
  A &\la B + C
  \text{,}\\
  A &\la B * a
  \text{,}\\
  A &\la B * C
  \text{.}
\end{align}
It is straightforward to generalize to arbitrary algebraic
instructions, including division (in a field). \tAL{} also contains
arbitrary expressions involving non-algebraic variables.

Function calls are written
\begin{equation}
  \label{eq:58}
  A,\ldots,B;x,\ldots,y \la f(C, \ldots, D; t, \ldots, z)
  \text{;}
\end{equation}
they can be recursive. Functions are defined using the $\proc$ and
$\return$ keywords
\begin{equation}
  \label{eq:60}
  \begin{aligned}
    &\proc f (A,\ldots,B;x,\ldots,y)\\
    &\quad\ldots\\
    &\return C,\ldots,D;t,\ldots,z\text{.}
  \end{aligned}
\end{equation}

If statements are the only real difference with the usual imperative
style of writing algebraic algorithms. They are typed and are written
using the $\talif$, $\talelse$ and $\return$ keywords
\begin{equation}
  \label{eq:73}
  \begin{aligned}
    &A, \ldots, B;x,\ldots, y \la \talif (z)\\
    &\quad\cdots\\
    &\return C, \ldots, D;t,\ldots,u\talelse\\
    &\quad\cdots\\
    &\return E, \ldots, F;v,\ldots,w\text{.}
  \end{aligned}
\end{equation}
The scope of variables is local to the if-else-return block.

For simplicity, we do not allow arbitrary variable names to be used
twice on the left of an assignment.

Here is an example \tAL{} program that multiplies an element $A$ by a
scalar $n\in\Z$:
\begin{equation}
  \label{eq:tALprog}
  \begin{aligned}
    &\proc f(A; n)\\[-1ex]
    &\qquad B \la \talif (n == 0)\\[-1ex]
    &\qquad\qquad C \la \text{zero}()\\[-1ex]
    &\qquad \return C \talelse\\[-1ex]
    &\qquad\qquad n' \la n - 1\\[-1ex]
    &\qquad\qquad D \la f(A; n').\\[-1ex]
    &\qquad\qquad E \la A + D\\[-1ex]
    &\qquad \return E\\[-1ex]
    &\return B\text{.}
  \end{aligned}
\end{equation}

A type inference as in the previous section gives types to \tAL{}
variables, in particular we are interested in the types $\ell$ and
$s$. When we need to make clear what the type of a variable is, we use
the notation $A::\ell$.

\paragraph{Partial evaluation}
\label{sec:partial-evaluation}
The partial evaluation at a point $p\in\pspace$ is done in two steps:
first we evaluate all the statements depending from $p$, then we strip
those statements off the partially evaluated program. Let us explain
this through an example. Consider the program
\begin{equation}
  \label{eq:77}
  \begin{aligned}
    &\proc f(A,B;m)\\[-1ex]
    &\qquad C,n \la \talif (m\le 0)\\[-1ex]
    &\qquad \return A,n \talelse\\[-1ex]
    &\qquad\qquad o\la m-1\\[-1ex]
    &\qquad\qquad D,p \la f(A,B;o)\\[-1ex]
    &\qquad\qquad q\la p + 1\\[-1ex]
    &\qquad\qquad E \la D * B\\[-1ex]
    &\qquad \return E,q\\[-1ex] &\return C,n
  \end{aligned}
\end{equation}
that computes $AB^m$, and also outputs $m$ (after some silly
computation).

The only valid linearization of this algorithm is obtained by fixing
$B$ and $m$. We define $f(\bullet,B;m)$ as the program
\begin{equation}
  \begin{aligned}
   &\proc f(\bullet,B;m)\\[-1ex]
   &\qquad n \la \talif (m \le 0)\\[-1ex]
   &\qquad \return n \talelse\\[-1ex]
   &\qquad\qquad o \la m-1\\[-1ex]
   &\qquad\qquad p \la f(\bullet,B;o)\\[-1ex]
   &\qquad\qquad q \la p + 1\\[-1ex]
   &\qquad \return q\\[-1ex]
   &\return n
   \text{,}
  \end{aligned}
\end{equation}
and we define by $\mathcal{T}(f(\bullet,B;m))$ its trace, i.e. the
list of the values taken by the variables appearing in it and in the
called functions.

Then we define the partial evaluation $f_{B,m}$ as 
\begin{equation}
  \begin{aligned}
    &\proc f_{B,m}(A)\\[-1ex]
    &\qquad C \la \talif (m\le 0)\\[-1ex]
    &\qquad \return A \talelse\\[-1ex]
    &\qquad\qquad D \la f_{B,o}(A)\\[-1ex]
    &\qquad\qquad E \la D * B\\[-1ex]
    &\qquad \return E\\[-1ex]
    &\return C\text{.}
  \end{aligned}
\end{equation}
Observe that in order to compute its result, $f_{B,m}$ must know the
value of the variable $o$, so that it can make the call to
$f_{B,o}$. This value can be found in
$\mathcal{T}(f(\bullet,B;m))$. Also observe that
$\mathcal{T}(f(\bullet,B;o))$ is contained in
$\mathcal{T}(f(\bullet,B;m))$, thus one single trace is enough to
compute a partial evaluation, even if the code contains function
calls.

Now $f_{B,m}$ is an algebraic program, thus it can be transposed using
the techniques of Section~\ref{sec:stra-line-progr}. The result is
\begin{equation}
  \begin{aligned}
    &\proc \dual{f}_{B,m}(C)\\[-1ex]
    &\qquad C \la \talif (m\le 0)\\[-1ex]
    &\qquad \return C \talelse\\[-1ex]
    &\qquad\qquad D \la C * B\\[-1ex]
    &\qquad\qquad A \la \dual{f}_{B,o}(D)\\[-1ex]
    &\qquad \return A\\[-1ex]
    &\return C\text{.}
  \end{aligned}
\end{equation}
Thus the full transposition of $f(A,B;m)$ can be obtained by
concatenating $f(\bullet,B;m)$ and $\dual{f}_{B,m}$ (and giving the
full trace of the first to the second).




% Local Variables:
% mode:flyspell
% ispell-local-dictionary:"american"
% mode:TeX-PDF
% mode:reftex
% TeX-master: "../these"
% End:
%

\section{A word about automatic differentiation}
\label{sec:word-about-automatic}
The transposition principle has often been viewed as a special case of
the reverse mode in automatic differentiation
\cite{KY88,Ka2K,BoLeSc03}. This is somewhat ironic as the whole idea
of automatic differentiation can elegantly be derived in the
arithmetic circuit model and reverse mode in particular is just an
application of the transposition principle \cite{GG05}. It is probable
that the need for efficient AD tools in many scientific areas other
than mathematics and computer science is responsible for such reversal
of roles.

In this section we show how AD can be expressed in the arithmetic
circuit model and then discuss the main differences between the AD
tools and our approach. A more complete study on the differentiation
of circuits and on how the transposition principle relates the
gradient to the differential can be found in \cite{GG05,Ser08}, of
which this Section is a simplification.

To simplify the presentation, we consider a basis $\mathcal{B}$ over
$\R$ made exclusively of everywhere continuously derivable functions
(w.r.t the standard metric of the Euclidean space $\R^n$). What we
give here is a technique to approximate a circuit over
$(\R,\mathcal{B})$ by a ``linear'' circuit.

\begin{definition}[Derivative of a circuit]
  Let $C$ be a circuit over $(\R,\mathcal{B})$ with $n$ inputs and let
  $x\in\R^n$. For any function $f\in\mathcal{B}$, we note by $J_f$ its
  Jacobian. Then the \emph{derivative} of $C$ at $x$, noted $\diff_x
  C$ is the arithmetic circuit where any $v\in V$ with $\beta(v)=f$
  and incident edges $e_1,\ldots,e_m$ has been substituted by a $v'$
  with
  \begin{equation}
    \label{eq:derivative}
    \beta(v')=J_f\left(\eval_{e_1}(x),\ldots,\eval_{e_m}(x)\right)
    \text{ .}
  \end{equation}
\end{definition}

\begin{figure}[!ht]
  \centering
  \begin{tikzpicture}
    \tikzstyle{node}=[circle,thick,draw=black,minimum size=4mm]
    \tikzstyle{arg}=[rectangle,thin,draw=black,minimum size=4mm]
    
    \begin{scope}
      \node[arg](in1){$x_1$};
      \node[arg,right of=in1](in2){$x_2$};
      \node[arg,right of=in2](in3){$x_3$};
      \node[node,below of=in1,xshift=5mm](times1){$*$};
      \node[node,below of=times1,xshift=5mm](times2){$*$};
      \node[arg,below of=times2](out){$y_1$};

      \path[->]
      (in1) edge (times1)
      (in2) edge (times1)
      (times1) edge (times2)
      (in3) edge (times2)
      (times2) edge (out);
    \end{scope}

    \begin{scope}[xshift=4cm]
      \node[arg](in1){$x_1$};
      \node[arg,right of=in1](in2){$x_2$};
      \node[arg,right of=in2](in3){$x_3$};
      \node[node,below of=in1,xshift=5mm](times1){\tiny$(b,a)$};
      \node[node,below of=times1,xshift=5mm](times2){\tiny$(c,ab)$};
      \node[arg,below of=times2](out){$y_1$};

      \path[->]
      (in1) edge (times1)
      (in2) edge (times1)
      (times1) edge (times2)
      (in3) edge (times2)
      (times2) edge (out);
    \end{scope}
  \end{tikzpicture}
  \caption{A circuit and its derivative at the point $x=(a,b,c)$.}
  \label{fig:derivative}
\end{figure}

Taking the derivative of a circuit at $x$ amounts to chose for each
node the best linear approximation at the point where it is
evaluated. It is clear that this yields the best linear approximation
for the circuit at $x$.

\begin{proposition}
  $\eval_{\diff_xC}=J_{\eval_C}(x)$.
\end{proposition}

It is also clear that $\diff_xC$ is defined over a basis that is
exclusively made of matrices with coefficients in $\R$, in other words
$\diff_xC$ is defined in $RMod{\R}$. We have thus defined a
transformation from black-box derivable functions to black-box
matrices.

Now the black-box $\diff_xC$ can be queried by black-box algorithms to
obtain information about the Jacobian. The most simple application is
to compute the directional derivative in $x$ along a direction $u$:
for this task it suffices to evaluate the circuit once since
$\eval_{\diff_xC}(u)$ is the desired value. Computing the derivative
along $n$ linearly independent directions yields the whole Jacobian
and this corresponds to the direct mode in automatic
differentiation\footnote{To be more precise, direct mode automatic
  differentiation constructs $\diff_xC$ and evaluates the $n$
  directions in parallel, thus reducing the amount of storage
  needed.}.

When the circuit has many inputs but only one output, there is a more
convenient way to get the whole gradient with only one black-box
query: $\diff_xC$ computes a linear form whose coefficients are
exactly the coefficients of the gradient, thus the dual circuit
$\dual{(\diff_xC)}$ computes the transposed form\footnote{We haven't
  actually proven the transposition theorem for an arbitrary basis,
  but it should be clear how to generalize it.}, or column vector. The
single query $\eval_{\dual{(\diff_xC)}}(1)$ yields this vector. This
is exactly what is called ``reverse mode'' in automatic
differentiation.

Note however that one is not limited to direct or reverse mode: any
black-box algorithm can be combined with the derivative circuit to
obtain information on the original function. For example Wiedemann's
algorithm \cite{Wie86} can be used to determine if the function is
invertible around $x$ and directional derivatives of the inverse can
be computed.

Of course, direct and reverse automatic differentiation can be defined
by the more classical chain rule, and then the transposition theorem
can be derived as a special case of the reverse mode by observing
that, when all the nodes of the circuit are linear maps, $C=\diff_xC$
for any $x$. After all, the code transformation techniques given in
\cite{BoLeSc03} and further developed in Section \ref{sec:} were
already invented by researchers in AD \cite{GVM91}, though not often
implemented. 

This being said, why treat transposition separately?  The answer is
manifold and we only list here some key points.
\begin{itemize}
\item AD is often interested in recovering the full Jacobian, instead
  of just having a black-box for it. For an $n\times m$ matrix, this
  requires $n$ queries in direct mode or $m$ queries in reverse
  mode. In both cases, AD tools do more work than what we would like
  to.
\item Many AD tools do not optimize the computation of $\diff_xC$ for
  the case where nodes are linear and still compute the whole
  circuit. In particular, many AD tools generate a graph
  representation of an arithmetic circuit from a program instead of
  directly transposing the code. This adds a constant overhead to the
  case of transposition where simply $\diff_xC=C$.
\item If the circuit $\diff_xC$ is computed, it must be fully stored
  in memory for reverse mode. This may seem innocuous as $\diff_xC$
  has the same size as $C$, but consider programs that compute
  $\eval_C$ by means of for loops or other iterative constructs: while
  the evaluation of $C$ is compact and cheap, the evaluation of
  $\diff_xC$ possibly requires to introduce a new variable for each
  iteration of the loop. Depending on the implementation, this may
  lead to code or storage bloat. In the case of transposition, this
  never happens since for loops are directly reversed (at least when
  all the variables are linear). Griewank \cite{Gri92} gives a
  time/memory compromise that permits to keep both storage and time in
  a factor of $\log n$ from the original program, but this is still
  not as good as transposition.
\item Our approach is more general in that it permits to automatically
  treat functions that depend both on linear and non-linear arguments
  without any help from the user. Thanks to this, we are able to treat
  recursive functions, while, to the extent of our knowledge, no AD
  tool can.
\item Our approach is algebraic and permits to prove bounds on the
  algebraic complexity of the generated programs, while AD tools
  usually only deal with floating point numbers. More generally, AD
  languages are usually less rich than \tAL.
\end{itemize}




% Local Variables:
% mode:flyspell
% ispell-local-dictionary:"american"
% mode:TeX-PDF
% mode:reftex
% TeX-master: "../these"
% End:
%

\chapter{Linearity inference of karatsuba multiplication}
\label{cha:line-infer-karats}
\lstset{language=haskell}

\begin{lstlisting}
data L = L Integer
data S = S Integer

class Ring r where
  zero :: r
  (<+>) :: r -> r -> r
  (<*>) :: r -> S -> r
  neg :: r -> r
  
class Ring r => Module m r | m -> r where
  zeroM :: m
  (<<*) :: m -> S -> m
  (>>>) :: m -> Integer -> r
  (<<<) :: r -> Integer -> m
  (<++>) :: m -> m -> m
  add :: m -> m -> Integer -> m
  add a b n = foldl (<++>) zeroM 
              [((a>>>i) <+> (b>>>i))<<<i | i <- [1..n]]
  
instance Ring L where
  zero = L 0
  (L x) <+> (L y) = L (x+y)
  (L x) <*> (S y) = L (x*y)
  neg (L x) = L (-x)
  
instance Ring S where
  zero = S 0
  (S x) <+> (S y) = S (x+y)  
  (S x) <*> (S y) = S (x*y)
  neg (S x) = S (-x)

one = S 1

instance Ring r => Module [r] r where
  zeroM = [zero]
  [] <<* x = []
  (x:xs) <<* y = (x <*> y):(xs <<* y)
  [] >>> i = zero
  (x:xs) >>> i =
    if i < 1 then zero else if i == 1 then x else xs >>> (i-1)
  x <<< i = if i <= 1 then [x] else zero:(x <<< (i-1))
  [] <++> [] = []
  (x:xs) <++> [] = x:(xs <++> [])
  [] <++> (y:ys) = y:([] <++> ys)
  (x:xs) <++> (y:ys) = (x <+> y):(xs <++> ys)
  add [] [] n = []
  add [] (y:ys) n =
    if n > 0 then y:(add [] ys (n-1)) else []
  add (x:xs) [] n =
    if n > 0 then x:(add xs [] (n-1)) else []
  add (x:xs) (y:ys) n =
    if n > 0 then (x<+>y):(add xs ys (n-1)) else []


-- Karatsuba multiplication : the system will infer
-- shift :: Ring r => [r] -> Integer -> [r]
-- split :: Ring r => [r] -> Integer -> ([r], [r])
-- kara :: Ring r => [r] -> [S] -> Integer -> [r]

shift x n = if n <= 0 then x else shift (zero:x) (n-1)

split [] n = ([], [])
split (x:xs) n =
  if n <= 0
  then ([], x:xs)
  else let (a, b) = split xs (n-1) in (x:a, b)

kara [] y n = []
kara x [] n = []
kara x y n =
  if n <= 0
  then []
  else if n == 1 
       then [(x!!0) <*> (y!!0)]
       else 
         let h = n `div` 2 in
         let (a0, a1) = split x h in
         let (b0, b1) = split y h in
         let x0 = kara a0 b0 h in
         let x2 = kara a1 b1 (n-h) in
         let xx1 = kara (a1 <++> a0) (b1 <++> b0) (n-h) in
         let x1 = xx1 <++> ((x0 <++> x2) <<* (neg one)) in
         (shift x2 n) <++> (shift x1 h) <++> x0
\end{lstlisting}                    
                  

% Local Variables:
% mode:flyspell
% ispell-local-dictionary:"american"
% mode:TeX-PDF
% mode:reftex
% TeX-master: "../these"
% End:
%




%%% Local Variables: 
%%% mode:flyspell
%%% ispell-local-dictionary:"american"
%%% mode: TeX-PDF
%%% mode: reftex
%%% TeX-master: "../these"
%%% End: 

\part{Fast arithmetics using univariate representations}
\label{part:fast-arithm-using}
\chapter{Trace computations}
\label{cha:trace-computations}
We let $I$ be a zero-dimensional ideal of $\K[\lst{x}]$ and
$\algeb{A}=\K[\lst{x}]/I$.  To simplify the exposition, from now on we
assume that $I$ is radical, this is equivalent to all the points of
$V(I)=\{a\in\clot{\K}^n|f(a)=0, \forall f\in I\}$ being simple. We
address the reader interested in the case of arbitrary multiplicity to
\cite{mourrain+elkadi}.

To better understand the structure of $\algeb{A}$ it will be important
to study the variety $V(I)$. We denote by $\clot{I}$ the ideal of
$\clot{\K}[\lst{x}]$ generated by $I$ and by $\clot{\algeb{A}}$ the
quotient ring $\clot{\K}[\lst{x}]/\clot{I}$.

\begin{lemma}
  $\clot{\algeb{A}}$ is a finite dimensional $\clot{\K}$-vector
  space. Its dimension is the same as $\algeb{A}$'s dimension as
  $\K$-vector space.
\end{lemma}

In what follows, we suppose that $V(I)$ has cardinality $d$ and we
denote its points by $\zeta_i\in\clot{\K}^n$ for $1\le i\le d$.

\begin{proposition}
  The number of points of $V(I)$ equals the dimension of
  $\clot{\algeb{A}}$ as vector space.
\end{proposition}
\begin{proof}
  Since $\clot{I}$ is radical and zero-dimensional, its primary
  decomposition is
  \begin{equation}
    \label{eq:1}
    \clot{I}=Q_1\cap \cdots \cap Q_d
    \text{,}
  \end{equation}
  where $Q_i$ is the ideal vanishing on $\zeta_i$.
  
  The $Q_i$'s are maximal and pairwise coprime
  (i.e. $Q_i+Q_j=\clot{\K}[\lst{x}]$ whenever $i\ne j$), hence, by the
  Chinese remainder theorem,
  \begin{equation}
    \label{eq:3}
    \clot{\K}[\lst{x}]/Q_1\cap\cdots\cap Q_d \isom \bigoplus_{i=1}^d\clot{\K}[\lst{x}]/Q_i
    \text{.}
  \end{equation}

  But $Q_i$ is maximal, hence $\clot{\K}[\lst{x}]/Q_i$ is an algebraic
  field extension of $\clot{\K}$. Since $\clot{\K}$ is algebraically
  closed, $\clot{\K}[\lst{x}]/Q_i=\clot{\K}$, and $\clot{\algeb{A}}$ has
  dimension $d$ as expected.
\end{proof}


\begin{example}
  Consider the ideal $I=(y-3,x^2-y)$ of $Q[x,y]$. This ideal is prime
  and its variety has no $\Q$-rational points. Since $G=\{y-3,x^2-y\}$
  is a Gröbner basis for $I$ (for grevlex), elements of
  $\algeb{A}=Q[x,y]/I$ are uniquely represented by their normal form
  modulo $G$; for example
  \[x^5y + 3xy + 1 \equiv 36x + 1 \mod I\text{.}\] By analyzing the
  leading monomials of $G$, it is straightforward to realize that all
  normal forms modulo $G$ have degree at most $1$ in $x$ and degree
  $0$ in $y$, thus $\algeb{A}$ has dimension $2$ as vector space.

  Indeed, the variety $V(I)$ consists of two points:
  \[V(I)=\left\{(\sqrt{3},3), (-\sqrt{3},3)\right\}\subset\clot{\Q}^2\text{.}\]
  Hence, $\clot{I}=(x-\sqrt{3},y-3)\cap(x+\sqrt{3},y-3)$ and
  \[\clot{\algeb{A}}\isom \clot{\Q}/(x-\sqrt{3},y-3) \oplus
  \clot{\Q}/(x+\sqrt{3},y-3)\text{.}\] In particular, the element
  $36x+1$ of $\clot{\algeb{A}}$ is mapped to
  \[(1+36\sqrt{3},1-36\sqrt{3})\] by this isomorphism. The reader will
  have noticed that $\algeb{A}$ is isomorphic to $\Q(\sqrt{3})$ as a ring.
\end{example}

We set 
\begin{equation}
  \label{eq:4}
  \clot{\algeb{A}}_i\eqdef \clot{\K}[\lst{x}]/Q_i
  \text{,}
\end{equation}
then by Eq. \eqref{eq:3} 
\begin{equation}
  \label{eq:5}
  \clot{\algeb{A}} = \bigoplus_{i=1}^d\clot{\algeb{A}_i}
  \text{.}
\end{equation}

Now, the $\clot{\algeb{A}}_i$'s are subalgebras of $\clot{\algeb{A}}$
isomorphic to $\clot{\K}$. We denote by $\basis{e}_i$ the unit element
of $\clot{\algeb{A}}_i$, then
\begin{equation}
  \label{eq:6}
  \begin{aligned}
    \basis{e}_i^2 &= \basis{e}_i\text{,}\\
    \basis{e}_i\basis{e}_j &= 0\text{.}
  \end{aligned}
\end{equation}
Hence $(\basis{e}_1,\ldots,\basis{e}_d)$ is a basis of
$\clot{\algeb{A}}$ made of orthogonal idempotents.

\begin{example}
  Continuing the previous example, 
  \[\clot{\algeb{A}}_1 = \clot{\Q}/(x-\sqrt{3},y-3)
  \quad\text{and}\quad
  \clot{\algeb{A}}_2 = \clot{\Q}/(x+\sqrt{3},y-3)
  \text{.}\]
  The idempotents are given by
  \[\basis{e}_1 = (3+\sqrt{3}x)/6
  \quad\text{and}\quad \basis{e}_2 = (3-\sqrt{3}x)/6 \text{.}\] The
  verification of Eq. \eqref{eq:6} is straightforward. In particular
  \[36x + 1 = (1+36\sqrt{3})\basis{e}_1 + (1-36\sqrt{3})\basis{e}_2
  \text{.}\]
\end{example}

For any $f\in\clot{\K}[\lst{x}]$, we denote by $f(\zeta_i)$ the
evaluation of $f$ at $\zeta_i\in V(I)$. $f(\zeta_i)$ only depends on
the class of $f$ in $\algeb{\clot{A}}$, thus for
$a\in\clot{\algeb{A}}$, we define $a(\zeta_i)$ as the evaluation at
$\zeta_i$ of an arbitrary representative of the class $a$.

For any $a\in\clot{\algeb{A}}$, its class in $\clot{\algeb{A}}_i$ is $a(\zeta_i)$,
by Eq. \eqref{eq:4}. Hence
\begin{equation}
  \label{eq:2}
  a = \sum_{i=1}^da(\zeta_i)\basis{e}_i
  \text{.}
\end{equation}

The basis $(\basis{e}_1,\ldots,\basis{e}_d)$ is a very practical one
to represent elements of $\clot{\algeb{A}}$. Unfortunately, in the
general case the idempotents $\basis{e}_i$ may not be elements of
$\algeb{A}$, as the previous example shows; thus, using such basis
comes at the cost of lifting coefficients in $\clot{\K}$. In order to
find a basis better suited to represent elements of $\algeb{A}$, we
shall study the dual of the algebra $\clot{\algeb{A}}$.


\section{Dual}
\label{sec:dual}
We shall denote by $\dual{\algeb{A}}$ the dual space of $\algeb{A}$,
that is the space of $\K$-linear forms on $\algeb{A}$. Similarly, we
shall denote by $\dual{\clot{\algeb{A}}}$ the dual space of
$\clot{\algeb{A}}$.

The map
\begin{equation}
  \label{eq:8}
  \begin{aligned}
  \basis{1}_{\zeta_i} : \clot{\algeb{A}} &\ra \clot{\K}\\
  a &\mapsto a(\zeta_i)
  \end{aligned}
\end{equation}
is linear; in particular
\begin{equation}
  \label{eq:9}
  \basis{1}_{\zeta_i}(\basis{e}_j) =
  \begin{cases}
    1 &\text{if $i=j$,}\\
    0 &\text{if $i\ne j$.}
  \end{cases}
\end{equation}
Hence $(\basis{1}_{\zeta_1},\ldots,\basis{1}_{\zeta_d})$ is the basis
of $\dual{\clot{\algeb{A}}}$ dual to $(\basis{e}_1,\ldots,\basis{e}_d)$.

The space $\dual{\algeb{A}}$ has a natural $\algeb{A}$-module
structure under the law
$\cdot:\algeb{A}\times\dual{\algeb{A}}\ra\dual{\algeb{A}}$ defined by
\begin{equation}
  \label{eq:10}
  \begin{aligned}
    a\cdot\ell : \algeb{A} &\ra \K\\
    b &\mapsto \ell(ab)
    \text{.}
  \end{aligned}
\end{equation}
Similarly $\dual{\clot{\algeb{A}}}$ has an $\clot{\algeb{A}}$-module
structure under an analogous law.

\begin{proposition}
  \label{th:gorenstein}
  $\dual{\clot{\algeb{A}}}$ and $\clot{\algeb{A}}$ are isomorphic as
  $\clot{\algeb{A}}$-modules under the mapping
  $\rho:\basis{e}_i\mapsto\basis{1}_{\zeta_i}$ for $1\le i\le d$.
\end{proposition}
\begin{proof}
  The mapping is clearly a vector space isomorphism, we only need to
  prove that it is a morphism of $\clot{\algeb{A}}$-modules. We want
  to prove that for any $a,b\in\clot{\algeb{A}}$
  \[\rho(ab) = a\cdot\rho(b)\text{.}\]
  It suffices to prove this on the elements of the bases
  $(\basis{e}_1,\ldots,\basis{e}_d)$ and
  $(\basis{1}_{\zeta_1},\ldots,\basis{1}_{\zeta_d})$.

  On one hand
  \begin{equation}
    \label{eq:12}
    \rho(\basis{e}_i\basis{e}_j)=
    \begin{cases}
      \rho(0)=0 &\text{if $i\ne j$,}\\
      \rho(\basis{e}_i)=\basis{1}_{\zeta_i} &\text{if $i=j$.}
    \end{cases}
  \end{equation}
  On the other hand, $\basis{e_i}\cdot\basis{1_{\zeta_j}}$ is the form
  that associates to any $c\in\clot{\algeb{A}}$ the element
  \begin{equation}
    \label{eq:13}
    (\basis{e_i}c)(\zeta_j) = \basis{e}_i(\zeta_j)c(\zeta_j) = 
    \begin{cases}
      0 &\text{if $i\ne j$,}\\
      c(\zeta_j) &\text{if $i=j$,}
    \end{cases}
  \end{equation}
  where the last equality comes from \eqref{eq:9}. Hence
  \begin{equation}
    \label{eq:14}
    \basis{e}_i\cdot\basis{1}_{\zeta_j}=
    \begin{cases}
      0 &\text{if $i\ne j$,}\\
      \basis{1}_{\zeta_i} &\text{if $i=j$.}
    \end{cases}
  \end{equation}
\end{proof}

\begin{note}
  We have thus identified $\dual{\clot{\algeb{A}}}$ to
  $\clot{\algeb{A}}$ as $\clot{\algeb{A}}$-modules, this implies that
  $\clot{\algeb{A}}$ is a Gorenstein algebra \cite[Chapter
  8]{mourrain+elkadi}. The theory of Gorenstein algebras is much
  deeper than the exposition we give here, and giving a complete
  account of it would be beyond the scope of this
  document. Nevertheless, we will eventually point out the
  relationships between the results proven here and the general
  theory.
\end{note}

Since $1$ generates $\clot{\algeb{A}}$ as an $\clot{\algeb{A}}$-module, the form
\begin{equation}
  \label{eq:7}
  \Tr \eqdef \rho(1) = \sum_i\basis{1}_{\zeta_i}
\end{equation}
generates $\dual{\clot{\algeb{A}}}$ as an $\clot{\algeb{A}}$-module.
$\rho(1)$ will play an important role in the sequel; it is called the
\emph{trace form}, the reason for this will be clear soon.

The bilinear form on $\dual{\clot{\algeb{A}}}\times\clot{\algeb{A}}$
defined by
\begin{equation}
  \label{eq:11}
  \braket{\ell}{a} = \ell(a)
\end{equation}
is non-singular by definition (see Section
\ref{sec:linear-algebra:duality}). By means of the isomorphism $\rho$,
we can transport this to a bilinear form on
$\clot{\algeb{A}}\times\clot{\algeb{A}}$: we define
\begin{equation}
  \label{eq:15}
  \braket{a}{b}=\rho(a)(b)
  \text{.}
\end{equation}
By Proposition \ref{th:gorenstein}, by Eq. \eqref{eq:10} and by the
equality
\begin{equation}
  \label{eq:16}
  \rho(a)(b) = \sum_i a(\zeta_i)b(\zeta_i)
  \text{,}
\end{equation}
we deduce that
\begin{equation}
  \label{eq:17}
  \braket{a}{b} = \rho(a)(b) = ab\Tr(1) = a\cdot\Tr(b) = \Tr(ab)
  \text{.}
\end{equation}
is a non-singular form on $\clot{\algeb{A}}\times\clot{\algeb{A}}$
that identifies $\clot{\algeb{A}}$ to its dual.

\section{Multiplication}
\label{sec:multiplication}

Let $a\in\clot{\algeb{A}}$ and consider the linear map
\begin{equation}
  \label{eq:18}
  M_a:a \mapsto ab
  \text{.}
\end{equation}

\begin{theorem}[Stickelberger's theorem]
  \label{th:stickelberger}
  The element $\basis{e_i}$ is an eigenvector of $M_a$ associated to
  the eigenvalue $a(\zeta_i)$. The characteristic polynomial of $M_a$
  is
  \[\prod_{i=1}^d(X-a(\zeta_i))\text{.}\]
\end{theorem}
\begin{proof}
  Using Eq. \eqref{eq:2}, we have
  \begin{equation}
    \label{eq:19}
    M_a(\basis{e}_i) = a\basis{e}_i = \sum_ja(\zeta_j)\basis{e}_i\basis{e}_j
    \text{.}
  \end{equation}
  By Eq.~\eqref{eq:6}, we deduce
  \begin{equation}
    \label{eq:20}
    M_a(\basis{e}_i) = a(\zeta_i)\basis{e}_i
    \text{.}
  \end{equation}

  Since the $\basis{e}_i$'s form a basis of $\clot{\algeb{A}}$ as a
  vector space, $M_a$ is diagonalizable and its eigenvalues are the
  $a(\zeta_i)$'s, each counted once.
\end{proof}

\begin{definition}[Trace, norm]
  \label{def:trace}
  We define the \emph{trace} of $a$ as
  \[\Tr(a) = \Tr(M_a)\]
  and its \emph{norm} as
  \[\Norm(a) = \det(M_a)\text{.}\]
\end{definition}
Then, the following corollary is easily derived.

\begin{corollary}
  \label{th:stickelberger-trace-det}
  One has
  \begin{align}
    \label{eq:23}
    \Tr(a) &= \sum_{i=1}^da(\zeta_i)\\
    \label{eq:24}
    \Norm(a) &= \prod_{i=1}^da(\zeta_i)
  \end{align}
\end{corollary}

By Eqs.~\eqref{eq:23} and~\eqref{eq:7}, it is clear that
$\Tr(a)=\rho(1)(a)$, which justifies the notation we employed in the
last section.

When $a\in\algeb{A}$, the characteristic polynomial of $M_a$ has
coefficients in $\K$. This implies that $\rho(1)\in\dual{\algeb{A}}$.



% Local Variables:
% mode:flyspell
% ispell-local-dictionary:"american"
% mode:TeX-PDF
% mode:reftex
% TeX-master: "../these"
% End:



\chapter{Artin-Schreier towers}
\label{cha:artin-schr-towers}
\section{Introduction}

\paragraph*{\bf Definitions.} If $\U$ is a field of characteristic $p$,
polynomials of the form $P=X^p - X - \alpha$, with $\alpha \in \U$,
are called {\em Artin-Schreier polynomials}; a field extension
$\U'/\U$ is {\em Artin-Schreier} if it is of the form $\U' = \U[X]/P$,
with $P$ an Artin-Schreier polynomial.

An {\em Artin-Schreier tower} of height $k$ is a sequence of
Artin-Schreier extensions $\U_i / \U_{i-1}$, for $1\le i \le k$; it is
denoted by $(\U_0, \ldots, \U_k)$. In what follows, we only consider
extensions of finite degree over $\F_p$. Thus, $\U_i$ is of degree
$p^i$ over $\U_0$, and of degree $p^id$ over $\F_p$, with
$d=[\U_0:\F_p]$.

The importance of this concept comes from the fact that all Galois
extensions of degree $p$ are Artin-Schreier. As such, they arise
frequently, e.g., in number theory (for instance, when computing
$p^k$-torsion groups of Abelian varieties over $\F_p$). The need for
fast arithmetics in these towers is motivated in particular by
applications to isogeny computation and point-counting in cryptology,
as in~\cite{Couveignes96}.

\paragraph*{\bf Our contribution.} The purpose of this paper is to
give fast algorithms for arithmetic operations in Artin-Schreier
towers. Prior results for this task are due to Cantor~\cite{Can89} and
Couveignes~\cite{Couveignes00}. However, the algorithms
of~\cite{Couveignes00} need as a prerequisite a fast multiplication
algorithm in some towers of a special kind, called ``Cantor towers''
in~\cite{Couveignes00}. Such an algorithm is unfortunately not in the
literature, making the results of~\cite{Couveignes00} non practical.

This paper fills the gap. Technically, our main algorithmic
contribution is a fast change-of-basis algorithm; it makes it possible
to obtain fast multiplication routines, and by extension completely
explicit versions of all algorithms of~\cite{Couveignes00}. Along the
way, we also extend constructions of Cantor to the case of a general
finite base field $\U_0$, where Cantor had $\U_0=\F_p$.  We present
our implementation, in a library called \texttt{FAAST}, based on
Shoup's \texttt{NTL}~\cite{NTL}. As an application, we put to practice
Couveignes' isogeny computation algorithm~\cite{Couveignes96} (or,
more precisely, its refined version presented in~\cite{DeFeo10}).

\paragraph*{\bf Complexity notation.} We count time complexity
in number of operations in $\F_p$. Then, notation being as before,
optimal algorithms in $\U_k$ would have complexity $O(p^kd)$; most of
our results are (up to logarithmic factors) of the form
$O(p^{k+\alpha} d^{1+\beta})$, for small constants $\alpha,\beta$ such as
$0,1,2$ or $3$.

Many algorithms below rely on fast multiplication; thus, we let $\Mult
: \N \rightarrow \N$ be a {\em multiplication function}, such that
polynomials in $\F_p[X]$ of degree less than $n$ can be multiplied in
$\Mult(n)$ operations, under the conditions of~\cite[Ch.~8.3]{vzGG}.
Typical orders of magnitude for $\Mult(n)$ are $O(n^{\log_2(3)})$ for
Karatsuba multiplication or $O(n\log (n) \log\log (n))$ for FFT
multiplication. Using fast multiplication, fast algorithms are
available for Euclidean division or extended GCD~\cite[Ch.~9 \&
11]{vzGG}.

The cost of {\em modular composition}, that is, of computing $F(G)
\bmod H$, for $F,G,H\in\F_p[X]$ of degrees at most $n$, will be
written $\ModComp(n)$. We refer to~\cite[Ch.~12]{vzGG} for a
presentation of known results in an algebraic computational model: the
best known algorithms have subquadratic (but superlinear) cost in
$n$. Note that in a boolean RAM model, the algorithm of~\cite{KeUm08}
takes quasi-linear time.

For several operations, different algorithms will be available, and
their relative efficiencies can depend on the values of $p$, $d$ and
$k$. In these situations, we always give details for the case where
$p$ is small, since cases such as $p=2$ or $p=3$ are especially useful
in practice. Some of our algorithms could be slightly
improved, but we usually prefer giving the simpler solutions.

\paragraph*{\bf Previous work.} As said above, this paper
builds on former results of Cantor~\cite{Can89} and
Couveignes~\cite{Couveignes00,Couveignes96}; to our knowledge, prior
to this paper, no previous work provided the missing ingredients to
put Couveignes' algorithms to practice. Part of Cantor's
  results were independently discovered by Wang and Zhu~\cite{WaZh88}
and have been extended in another direction (fast polynomial
multiplication over arbitrary finite fields) by von zur Gathen and
Gerhard~\cite{GaGe96} and Mateer~\cite{GaMa08}.

This paper is an expanded version of the conference
paper~\cite{DeSc09}. We provide a more thorough description of the
properties of Cantor towers (Section~\ref{sec:fast-tower}),
improvements to some algorithms (e.g. the Frobenius or pseudo-trace
computations) and a more extensive experimental section.

\paragraph*{\bf Organization of the paper.}
Section~\ref{sec:arithmetics} consists in preliminaries: trace
computations, duality, basics on Artin-Schreier extensions. In
Section~\ref{sec:fast-tower}, we define a specific Artin-Schreier
tower, where arithmetic operations will be fast. Our key
change-of-basis algorithm for this tower is in
Section~\ref{sec:level-embedding}. In
Sections~\ref{sec:pseudotrace-frobenius}
and~\ref{sec:couveignes-algorithm}, we revisit Couveignes' algorithm
for isomorphism between Artin-Schreier towers~\cite{Couveignes00} in
our context, which yields fast arithmetics for {\em any}
Artin-Schreier tower. Finally, Section~\ref{sec:benchmarks} presents
our implementation of the \texttt{FAAST} library and gives
experimental results obtained by applying our algorithms to
Couveignes' isogeny algorithm~\cite{Couveignes96} for elliptic curves.




% Local Variables:
% mode:flyspell
% ispell-local-dictionary:"american"
% mode: TeX-PDF
% mode: reftex
% TeX-master: "../these"
% End:
%
% LocalWords:  Schreier Artin pseudotrace frobenius bivariate memoization
% LocalWords:  isogeny Couveignes

\section{Preliminaries}
\label{sec:arithmetics}

As a general rule, variables and polynomials are in upper
case; elements algebraic over $\F_p$ (or some other field, that will
be clear from the context) are in lower case.
 
%%%%%%%%%%%%%%%%%%%%%%%%%%%%%%%%%%%%%%%%%%%%%%%%%%%%%%%%%%%%

\subsection{Element representation}\label{ssec:rep}

Let $Q_0$ be in $\F_p[X_0]$ and let $(G_i)_{0 \le i < k}$ be
a sequence of polynomials over $\F_p$, with $G_i$ in
$\F_p[X_0,\dots,X_i]$. We say that the sequence $(G_i)_{0\le i <k}$
{\em defines the tower} $(\U_0,\dots,\U_k)$ if for $i \ge 0$, 
$\U_i=\F_p[X_0,\dots,X_i]/K_i$, where $K_i$ is
the
ideal generated by
$$\left | \begin{array}{l}
P_i=X_i^p-X_i -G_{i-1}(X_0,\dots,X_{i-1})\\
~~~\,~\vdots\\
P_1=X_1^p-X_1-G_0(X_0)\\
Q_0(X_0)
\end{array}\right .$$
in $\F_p[X_0,\dots,X_i]$, and if $\U_i$ is a field. The residue class of
$X_i$ (resp. $G_i$) in $\U_i$, and thus in $\U_{i+1},\dots$, is
written $x_i$ (resp. $\gamma_i$), so that we have
$x_i^p-x_i=\gamma_{i-1}$.

Finding a suitable $\F_p$-basis to represent elements of a tower
$(\U_0,\dots,\U_k)$ is a crucial question. If $d=\deg(Q_0)$, a natural
basis of $\U_i$ is the multivariate basis $\bB_i=\{x_0^{e_0} \cdots
x_i^{e_i}\}$ with $0 \le e_0 < d$ and $0\le e_j < p$ for $1 \le j \le
i$. However, in this basis, we do not have very efficient arithmetic
operations, starting from multiplication. Indeed, the natural
approach to multiplication in $\bB_i$ consists in a polynomial
multiplication, followed by reduction modulo $(Q_0,P_1,\dots,P_i)$;
however, the initial product gives a polynomial of partial degrees
$(2d-2,2p-2,\dots,2p-2)$, so the number of monomials appearing is not
linear in $[\U_i:\F_p]=p^id$.  See~\cite{LiMoSc07} for details.

As a workaround, we introduce the notion of a {\em primitive tower},
where for all $i$, $x_i$ generates $\U_i$ over $\F_p$. In this case,
we let $Q_i\in \F_p[X]$ be its minimal polynomial, of degree
$p^id$. In a primitive tower, unless otherwise stated, we represent
the elements of $\U_i$ on the $\F_p$-basis
$\bC_i=(1,x_i,\dots,x_i^{p^id-1})$.

To stress the fact that $v\in\U_i$ is represented on the basis
$\bC_i$, we write $v\wrt\U_i$. In this basis, assuming $Q_i$ is known,
additions and subtractions are done in time $p^id$, multiplications in
time $O(\Mult(p^id))$~\cite[Ch.~9]{vzGG} and inversions in time
$O(\Mult(p^id)\log(p^id))$~\cite[Ch.~11]{vzGG}.

Remark that having fast arithmetic operations in $\U_i$ enable us to
write fast algorithms for polynomial arithmetic in $\U_i[Y]$, where
$Y$ is a new variable. Extending the previous notation, let us write
$A \wrt\U_i[Y]$ to indicate that a polynomial $A \in \U_i[Y]$ is
written on the basis $(x_i^\alpha Y^\beta)_{0 \le \alpha < p^id, 0 \le
  \beta}$ of $\U_i[Y]$.  Then, given $A,B \wrt \U_i[Y]$, both of
degrees less than $n$, one can compute $AB \wrt \U_i[Y]$ in time
$O(\Mult(p^id n))$ using Kronecker's
substitution~\cite[Lemma~2.2]{GaSh92}.

One can extend the fast Euclidean division algorithm to this context,
as Newton iteration reduces Euclidean division to polynomial
multiplication. The analysis of~\cite[Ch.~9]{vzGG} implies that
Euclidean division of a degree $n$ polynomial $A \wrt \U_i[Y]$ by a
monic degree $m$ polynomial $B \wrt \U_i[Y]$, with $m \le n$, can be
done in time $O(\Mult(p^id n))$.

Finally, fast GCD techniques carry over as well, as they are based on
multiplication and division. Using the analysis
of~\cite[Ch.~11]{vzGG}, we see that the extended GCD of two monic
polynomials $A,B \wrt \U_i[Y]$ of degree at most $n$ can be computed
in time $O(\Mult(p^id n \log(n)))$.

%%%%%%%%%%%%%%%%%%%%%%%%%%%%%%%%%%%%%%%%%%%%%%%%%%%%%%%%%%%%

\subsection{Trace and pseudotrace}\label{ssec:tpt}


We continue with a few useful facts on traces. Let $\U$ be a field and
let $\U'=\U[X]/Q$ be a separable field extension of $\U$, with
$\deg(Q)=n$. For $a \in \U'$, the {\em trace} $\Tr(a)$ is the trace of
the $\U$-linear map $M_a$ of multiplication by $a$ in $\U'$.

The trace is a $\U$-linear form; in other words, $\Tr$ is in the dual
space $\dual{\U'}$ of the $\U$-vector space $\U'$; we write it
$\Tr_{\U'/\U}$ when the context requires it. In finite fields, we
also have the following well-known properties:
\begin{align}
  \tag{$\bP_1$} &\begin{array}{c}  
  \Tr_{\F_{q^n}/\F_q}: a \mapsto \sum_{\ell=0}^{n -
    1}a^{q^\ell} \text{,}
  \end{array}\\
  \tag{$\bP_2$}\label{eq:trcomp}
  &\Tr_{\F_{q^{mn}}/\F_q} = \Tr_{\F_{q^m}/\F_q} \circ
  \Tr_{\F_{q^{mn}}/\F_{q^m}}\text{.}
\end{align}

Besides, if $\U'/\U$ is an Artin-Schreier extension generated by a
polynomial $Q$ and $x$ is a root of $Q$ in $\U'$, then
\begin{equation}
  \tag{$\bP_3$}\label{eq:pd} \Tr_{\U'/\U}(x^j) = 0~ \text{for}~j
  <p-1; \quad \Tr_{\U'/\U}(x^{p-1}) = -1\text{.}
\end{equation}
Following~\cite{Couveignes00}, we also use a generalization of the
trace. The $n$th {\em pseudotrace} of order $m$ is the
$\F_{p^m}$-linear operator
\begin{equation*}
\begin{array}{c}  \PTr_{(n,m)}: a \mapsto \sum_{\ell=0}^{n-1}a^{p^{m\ell}};\end{array}
\end{equation*}
for $m=1$, we call it the $n$th pseudotrace and write $\PTr_n$.

In our context, for $n=[\U_i:\U_j]=p^{i-j}$ and $m=[\U_j:\F_p]=p^jd$,
$\PTr_{(n,m)}(v)$ coincides with $\Tr_{\U_{i}/\U_j}(v)$ for $v$ in
$\U_i$; however $\PTr_{(n,m)}(v)$ remains defined for $v$ not in
$\U_i$, whereas $\Tr_{\U_{i}/\U_j}(v)$ is not.

%%%%%%%%%%%%%%%%%%%%%%%%%%%%%%%%%%%%%%%%%%%%%%%%%%%%%%%%%%%%

\subsection{Duality}\label{ssec:duality}

Finally, we discuss two useful topics related to duality,
starting with the transposition of algorithms.

Introduced by Kaltofen and Shoup, the \emph{transposition principle}
relates the cost of computing an $\F_p$-linear map $f:\ V \to W$ to
that of computing the transposed map $\dual{f}:\ \dual{W} \to
\dual{V}$.  Explicitly, from an algorithm that performs an $r \times
s$ matrix-vector product $b \mapsto M b$, one can deduce the existence
of an algorithm with the same complexity, up to $O(r+s)$, that
performs the transposed product $c \mapsto M^t c$;
see~\cite{BuClSh97,Kaltofen00,BoLeSc03}. However, making the
transposed algorithm explicit is not always straightforward; we will
devote part of Section~\ref{sec:level-embedding} to this issue.

We give here first consequences of this principle,
after~\cite{Sho94,Shoup99,BoLeSc03}. Consider a degree $n$ field
extension $\U \to \U'$, where $\U'$ is seen as an $\U$-vector
space. For $w$ in $\U'$, recall that $M_w: \U'\rightarrow\U'$ is the
multiplication map $M_w(v) = vw$.  Its dual $\dual{M_w}: \dual{\U'}
\rightarrow \dual{\U'}$ acts on $\ell\in\dual{\U'}$ by
$\dual{M_w}(\ell)(v) = \ell\left(M_w(v)\right) = \ell(vw)$ for $v$ in
$\U'$. We prefer to denote the linear form $\dual{M_w}(\ell)$ by
$w\cdot\ell$, keeping in mind that $(w\cdot\ell)(v) = \ell(vw)$.

Suppose then that $\bD$ is a $\U$-basis of $\U'$, in which we can
perform multiplication in time $T$. Then by the transposition
principle, given $w$ on $\bD$ and $\ell$ on the dual basis
$\dual{\bD}$, we can compute $w\cdot \ell$ on the dual basis
$\dual{\bD}$ in time $T+O(n)$.  This was discussed already
in~\cite{Shoup99,BoLeSc03}, and we will get back to this in
Section~\ref{sec:level-embedding}.

Suppose finally that $\U'$ is separable over $\U$ and that $b\in \U'$
generates $\U'$ over $\U$; we will denote by $Q \in \U[X]$ the minimal
polynomial of $b$. Given $w$ in $\U'$, we want to find an expression
$w=A(b)$, for some $A \in \U[X]$. Hereafter, for $P \in \U[X]$ of
degree at most $e$, we write $\rev_e(P)=X^eP(1/X) \in \U[X]$. Then,
recalling that $n=[\U':\U]$, we define $\ell=w\cdot\Tr_{\U'/\U} \in
\dual{\U'}$ and
\begin{equation}
  \label{eq:MN}
  M = \sum_{j < n}\ell(b^j)X^j,\quad N = M\rev_{n}(Q) \bmod X^n.
\end{equation}
This construction solves our problem: Theorem~3.1
in~\cite{Rouillier99} shows that $w=A(b)$, with $A=\rev_{n-1}(N)
{Q'}^{-1} \bmod Q$. We will hereafter denote by
$\alg{FindParameterization}(b,w)$ a subroutine that computes this
polynomial $A$; it follows closely a similar algorithm given
in~\cite{Sho94}. Since this is the case we will need later on, we give
details for the case where $Q$ is Artin-Schreier (so $n=p$): then,
$Q'=-1$, so no work is needed to invert it modulo $Q$.

In the following algorithm, we suppose that $\U'$ is presented as
$\U'=\U[X]/P$, where $P$ is Artin-Schreier. We let $x$ be the residue
class of $X$ in $\U'$.
\begin{algorithm}{FindParameterization}
  {$w \in \U'$ written as $w_0 + \cdots + w_{p-1} x^{p-1}$,  
   $b \in \U'$ written as $b_0 + \cdots + b_{p-1} x^{p-1}$}
  {A polynomial $A$ of degree less than $p$ such that $w=A(b)$}
\item\label{alg:para:trmul} let $\ell = w \cdot\Tr_{\U'/\U}$
\item\label{alg:para:trmodcomp} let $M= \sum_{j < p}\ell(b^j)X^j$
\item\label{alg:para:multrunc} let $N = M \rev_{p}(Q) {\sf ~mod~} X^{p}$
\item\label{alg:para:mulmod} return $-\rev_{p-1}(N)$
\end{algorithm}
\begin{proposition}
  \label{th:findparameterization}
  If $Q$ is Artin-Schreier, the cost of $\alg{FindParameterization}$ is
  $O(p^2)$ operations $(+,\times)$ in $\U$.
\end{proposition}
\begin{proof} By~\ref{eq:pd}, the representation of $\Tr_{\U'/\U}$ in
$\U'^\ast$ is simply $(0,\ldots,0,-1)$. Then by the discussion above,
if $T$ is the cost of multiplying two elements of $\U'$ in the basis
$(1,\ldots,x^{p-1})$, step~\ref{alg:para:trmul} costs $T + O(p)$; this
stays in $O(p^2)$ by taking a naive
multiplication. Step~\ref{alg:para:trmodcomp} fits into the same
bound, by the proof of~\cite[Th.~4]{Sho94}. Taking the $\rev$'s in
steps~\ref{alg:para:multrunc} and~\ref{alg:para:mulmod} is just
reading the polynomials from right to left, thus this costs no
arithmetic operation. Finally, step~\ref{alg:para:multrunc} features a
polynomial multiplication truncated to the order $p$, this costs
$O(p^2)$ operations by a naive algorithm.\end{proof}

Note that this cost can be improved with respect to $p$, by using fast
modular composition as in~\cite{Sho94}; we do not give details, as this
would not improve the overall complexity of the algorithms of the next
sections.


% Local Variables:
% mode:flyspell
% ispell-local-dictionary:"american"
% End:
%
% LocalWords:  Schreier Artin pseudotrace frobenius bivariate memoization

\section{A \emph{nice} tower}
\label{sec:fast-tower}

This section is devoted to the proof of the following theorem which
extends a result by Cantor~\cite{Can89}.

\begin{theorem}
  \label{th:cantor}
  Let $d=[\U_0 :\F_p]$ be prime to $p$ and let $x_0$ be a generator of
  $\U_0$ over $\F_p$ such that $\Tr_{\U_0/\F_p}(x_0)\ne0$. Define
  $\gamma_i$, $\AS{P}_i$ and $x_i$ as
  \begin{eqnarray*}
    \gamma_0 &=& x_0\\
    \gamma_1 &=& 
    \begin{cases}
      x_1        &\text{if $p=2$,}\\
      x_1^{2p-1}      &\text{if $p>2$,}
    \end{cases} \\
    \gamma_i &=&  x_i^{2p-1} \quad i \ge 2\\
    \\
    \AS{P}_{i}(X) &=& X^p - X - \gamma_i\\
    x_{i+1} &&\text{ a root of } \AS{P}_i
  \end{eqnarray*}
  Then, for $i=0,\dots,k-1$, the polynomial $\AS{P}_i$ is irreducible
  over $\U_i$ and $x_{i+1}$ generates $\U_{i+1}$ over $\F_p$.
\end{theorem}

In \cite[Theorem 1.2]{Can89} Cantor proves the case $\U_0 = \F_p$ and
$x_0=1$. His statement is more generic than ours because it asks for
less restrictive conditions on the $\gamma_i$'s, but such greater
generality has to be paid off with a more complicated proof. We would
like to discuss the relationship between his proof and ours.

In order to prove his theorem, Cantor introduces the linear operator
$S:x\mapsto x^p-x$, remarks that its kernel is $\F_p$ and studies the effects
of its application on the $x_i$'s. In order to extend such a technique
to the case $\U_0\ne\F_p$, we need consider a similar operator and we
want its kernel to be $\U_0$. A natural choice is then
\[S:x\mapsto x^{p^d}-x\text{.}\]
As in Cantor's proof, in order to study the action of $S$ one needs to
see $\U_k$ as a vector space over $\U_0$ and consider some linear
subspaces that are not necessarily fields. There is a way of getting
rid of such an abundance of linear subspaces by considering only some
iterated applications of the operator $S$. Indeed
\[S^{p^i-p^j} = \PTr_{(p^i,dp^j)}\]
where $\PTr$ is the operator we will introduce in definition
\ref{def:pseudotrace}. This way, the only subspaces of $\U_k$ one
needs to consider are the $\U_i$'s and the proof is made extremely
simple by the particular conditions we asked on the $\gamma_i$'s.

We now give the proof of the theorem. We start with a classic lemma.

\begin{lemma}
  \label{Lemma:trace-AS}
  Let $\LK/\K$ be an Artin-Schreier extension and let $x$ be a
  generator of $\LK$ over $\K$ whose minimal polynomial is
  Artin-Schreier, then
  \begin{equation*}
    \Tr_{\LK/\K}(x^i) = \begin{cases}
      0  & \text{if $i=0,\ldots,p-2$,}\\
      -1 & \text{if $i=p-1$.}
    \end{cases}
  \end{equation*}
\end{lemma}
\begin{proof}
  Consider the formal power series of $\Tr_{\LK/\K}(x^i)$
  \begin{equation*}
    \sum_{i\ge0}\Tr(x^i)T^i = \sum_{i\ge0}\sum_{\zeta\in B}\zeta^iT^i
    \text{,}
  \end{equation*}
  where $B$ is the set of the conjugates of $x$ over $\K$ and the
  equality follows from the fact that the conjugation distributes over
  the product. By swapping sums
  \begin{equation*}
    \sum_{i\ge0}\Tr(x^i)T^i = \sum_{\zeta\in B}\frac{1}{1-\zeta T} =
    \frac{\sum_{\zeta\in B}\prod_{\zeta'\ne\zeta}(1-\zeta'T)}
         {\prod_{\zeta\in B}(1-\zeta T)} \text{.}
  \end{equation*}
  
  Let $\AS{P} = X^p-X-\alpha$ be the minimal polynomial of $x$ over
  $\K$, remark the following equalities
  \begin{equation*}
    \AS{P}(T) = \prod_{\zeta\in B}(T-\zeta) \;\text{,} \qquad
    \frac{\partial\AS{P}}{\partial X}(T) =
    \sum_{\zeta\in B}\prod_{\zeta'\ne\zeta}(T-\zeta') = -1\text{.}
  \end{equation*}
  Notice that the two polynomials have degrees in $T$ respectively $p$
  and $p-1$. Then, by taking reverse polynomials (see also section
  \ref{sec:level-embedding}, definition \ref{def:rev}), we have
  \begin{equation*}
    \sum_{i\ge0}\Tr(x^i)T^i =
    \frac{T^{p-1}\frac{\partial\AS{P}}{\partial X}(\frac{1}{T})}
      {T^p\AS{P}(1/T)} = \frac{-T^{p-1}}{1 - T^{p-1} - \alpha T^p}
      \text{.}
  \end{equation*}
  
  Hence we see that the first non-null coefficient of the formal power
  series is the $(p-1)$-th one and its value is exactly $-1$.
\end{proof}

We first address the irreducibility of $\AS{P}_i$.

\begin{lemma}
  \label{Lemma:irreducibility}
  For $i=0,\dots,k-1$ the polynomial $\AS{P}_i$ is
  irreducible and $x_{i+1}$ generates $\U_{i+1}$ over $\U_i$.
\end{lemma}
\begin{proof}
  We will show by induction on $i$ that $\Tr_{\U_i/\F_p}\ne0$ for
  every $i$, then, by theorem \ref{th:asfundamental}, $\AS{P}_i$ is
  irreducible and $x_{i+1}$ generates $\U_{i+1}$ over $\U_i$.
  
  For $i=0$ this is true by hypothesis. For $i \ge 1$, $x_i$ generates
  $\U_i$ over $\U_{i-1}$ by induction hypothesis and its minimal
  polynomial $\AS{P}_{i-1}$ is evidently Artin-Schreier. Then $x_i$
  meets the hypotheses of lemma \ref{Lemma:trace-AS}.

  Suppose that $p\ge3$, then
  \begin{equation*}
    \gamma_i=x_i^{2p-1}  = x_i^px_i^{p-1} =
    \gamma_{i-1} + x_i + x_i^{p-1} \gamma_{i-1}\text{.}
  \end{equation*}
  By the linearity of the trace and by lemma \ref{Lemma:trace-AS}
  \begin{equation}
    \label{eq:trace>2}
    \Tr_{\U_i/\U_{i-1}}(\gamma_i) = -\gamma_{i-1}
  \end{equation}
  and by the composition of traces $\Tr_{\U_i/\F_p}(\gamma_i) =
  -\Tr_{\U_{i-1}/\F_p}(\gamma_{i-1})$ is different from $0$ by induction
  hypothesis.

  When $p=2$, and $i=1$, using the same technique yields
  \begin{equation}
    \label{eq:trace=2,1}
    \Tr_{\U_1/\F_2}(\gamma_1)= \Tr_{\U_1/\F_2}(x_1)=
    \Tr_{\U_0/\F_2}(\Tr_{\U_1/\U_0}(x_1))= \Tr_{\U_0/\F_2}(1)\text{,}
  \end{equation}
  which is different from $0$ because $[\U_0:\F_2]$ is odd.

  For $p=2$ and $i > 1$, we get $x_i^{3} = \gamma_{i-1} + x_i (1+
  \gamma_{i-1})$ and as a consequence
  \begin{equation}
    \label{eq:trace=2}
    \Tr_{\U_i/\U_{i-1}}(\gamma_i)=\Tr(x_i^3)=1+\gamma_{i-1}
    \text{.}
  \end{equation}
  Since $\U_{i-1}$ has even degree over $\F_2$, we deduce
  $\Tr_{\U_{i-1}/\F_2}(1)=0$, and thus
  \[\Tr_{\U_{i}/\F_2}(\gamma_i)=\Tr_{\U_{i-1}/\F_2}(\gamma_{i-1})\]
  and we conclude by induction.
\end{proof}

As a consequence of the proof of the lemma, we have the following

\begin{corollary}
  \label{coro:trace}
  For $0\le j < i \le k-1$ we have
  \begin{equation*}
    \Tr_{\U_i/\U_{j}}(\gamma_i) = \begin{cases}
      (-1)^{i-j}\gamma_{j}  &\text{if $p\ge3$,}\\
      1+\gamma_{j}         &\text{if $p=2$ and $j\ge1$,}\\
      1                    &\text{if $p=2$ and $j=0$.}
    \end{cases}
  \end{equation*}
\end{corollary}
\begin{proof}
  The case $p\ge3$ follows from equation \eqref{eq:trace>2} by
  induction on $j$. The case $p=2$ follows from equations
  \eqref{eq:trace=2,1} and \eqref{eq:trace=2} and the fact that
  $\Tr_{\U_i/\F_p}(1)=0$ for $i\ge1$.
\end{proof}

We have just showed that $\U_{i+1} = \F_p[x_0,\ldots,x_i]$; the next
step is to show $\U_{i+1}=\F_p[x_i]$. To do so we introduce some more
notation.

\begin{definition}[Pseudotrace]
  \label{def:pseudotrace}
  The $i$-th pseudotrace of order $j$ is the linear operator
  defined by
  \begin{equation*}
    \PTr_{(i,j)} \;:\; x \mapsto \sum_{l=0}^{i-1}x^{p^{jl}}\text{.}
  \end{equation*}
  When $j=1$ we simply call it the $i$-th pseudotrace and we note
  $\PTr_i$.
\end{definition}

Notice that the pseudotrace is defined by the polynomial
$\PTr_{(i,j)}(X)$ in $\F_p[X]$. It is a ``linearized'' polynomial and
it has many special properties, the most important of which is being
$\F_{p^j}$-linear in its argument. See \cite[3.4]{LN} for more
details. It is also important to notice that $\PTr_{(p^i,dp^j)}$
coincides with $\Tr_{\U_{i+j}/\U_j}$ over elements of $\U_{i+j}$. For
the sake of convenience, we will note $\PTr_{\U_{i+j}/\U_{j}}$ instead
of $\PTr_{(p^i,dp^j)}$.

\begin{lemma}
  \label{Lemma:primitive>2}
  If $p\ge3$, $x_i$ generates $\U_i$ over $\F_p$ for $i=0,\ldots,k$.
\end{lemma}
\begin{proof}
  Induction on $i$. For $i=0$ this is true by hypothesis.

  For $i\ge1$, $x_i$ generates $\U_i$ over $\U_{i-1}$ by lemma
  \ref{Lemma:irreducibility} and $x_{i-1}$ generates $\U_{i-1}$ over
  $\F_p$ by induction hypothesis, so that $\U_i =
  \F_p[x_{i-1},x_i]$. Then it suffices to show that there is a
  polynomial $P\in\F_p[X]$ such that $P(x_i) = x_{i-1}$.

  For $i=1$ we know by definition that $x_1^p-x_1=x_0$, so that the
  polynomial $X^p-X$ does the trick.

  For $i\ge2$, consider the following equality, given by the linearity
  of the pseudotrace
  \begin{equation*}
    \PTr_{\U_{i-1}/\U_{i-2}}(x_i)^p - \PTr_{\U_{i-1}/\U_{i-2}}(x_i) =
    \PTr_{\U_{i-1}/\U_{i-2}}(x_i^p-x_i)\text{.}
  \end{equation*}
  We know by definition that $x_i^p-x_i=\gamma_{i-1}$, but
  $\gamma_{i-1}$ is in $\U_{i-1}$, so that the pseudotrace coincides
  with the trace as remarked above. We have
  \begin{equation*}
    \PTr_{\U_{i-1}/\U_{i-2}}(x_i)^p - \PTr_{\U_{i-1}/\U_{i-2}}(x_i) =
    \Tr_{\U_{i-1}/\U_{i-2}}(\gamma_{i-1}) = -\gamma_{i-2} \text{,}
  \end{equation*}
  where the last equality comes from corollary \ref{coro:trace}.

  We have shown that $\PTr_{\U_{i-1}/\U_{i-2}}(x_i)$ is a root of
  ${X^p-X+\gamma_{i-2}}$. But ${x_{i-1}^p-x_i=\gamma_{i-2}}$, hence we
  have the equality
  \begin{equation*}
    \PTr_{\U_{i-1}/\U_{i-2}}(x_i) = -x_{i-1}+\beta_i
    \qquad\text{with $\beta_i\in\F_p$.}
  \end{equation*}
  We conclude by observing that the polynomial
  $-\PTr_{\U_{i-1}/\U_{i-2}}(X)+\beta_i$ is in $\F_p[X]$ and sends
  $x_i$ over $x_{i-1}$.
\end{proof}

The case $p=2$ uses the same technique, but requires a bit more
work. Recall that $\F_4$ is the field $\{0,1,\omega,\omega+1\}$, where
$\omega$ and $\omega+1$ are two primitive cubic roots of the unit and
their minimal polynomial is $X^2-X-1$. Notice that $\F_4$ is not a
subfield of $\U_0$ because $d$ is odd, but it is a subfield of $\U_1$.

\begin{lemma}
  \label{Lemma:primitive=2-F4}
  If $p=2$, $x_i$ generates $\U_i$ over $\F_4$ for $i=1,\ldots,k$.
\end{lemma}
\begin{proof}
  Induction on $i$. For $i=1$, $\U_1 = \F_2[x_0,x_1]$, but
  $x_1^2-x_1=x_0$, so $\U_1 = \F_2[x_1]$. Since $\F_4\subset\U_1$,
  $x_1$ generates $\U_1$ over $\F_4$.
  
  For $i=2$, $\U_2=\F_4[x_1,x_2]$ by lemma \ref{Lemma:irreducibility}
  and induction hypothesis. But $x_2^2-x_2=x_1$, so that
  $\U_2=\F_4[x_2]$.

  For $i\ge3$, like in lemma \ref{Lemma:primitive>2},
  \begin{equation*}
    \PTr_{\U_{i-1}/\U_{i-2}}(x_i)^2 - \PTr_{\U_{i-1}/\U_{i-2}}(x_i) =
    \Tr_{\U_{i-1}/\U_{i-2}}(\gamma_{i-1}) = 1+\gamma_{i-2}
  \end{equation*}
  so that $\PTr_{\U_{i-1}/\U_{i-2}}(x_i)$ is a root of
  $X^2-X-(1+\gamma_{i-2})$. We deduce
  \[\PTr_{\U_{i-1}/\U_{i-2}}(x_i) = x_{i-1} + \omega_i\text{,}\]
  where $\omega_i$ is one of the two primitive roots of the unity,
  hence in $\F_4$. Then $\PTr_{\U_{i-1}/\U_{i-2}}(X) - \omega_i$ is in
  $\F_4[X]$ and sends $x_i$ over $x_{i-1}$.
\end{proof}

\begin{lemma}
  \label{Lemma:primitive=2}
  If $p=2$, $x_i$ generates $\U_i$ over $\F_2$ for $i=0,\ldots,k$.
\end{lemma}
\begin{proof}
  For $i=0$ this is true by hypothesis and for $i=1$ we already showed
  in the proof of lemma \ref{Lemma:primitive=2-F4} that
  $\U_1=\F_2[x_1]$.

  For $i\ge2$ we know by lemma \ref{Lemma:primitive=2-F4} that
  $\U_i=\F_2[x_i,\omega]$ where $\omega$ is a primitive root of
  unity. We then look for a polynomial in $\F_2[X]$ that sends $x_i$
  over $\omega$.

  By the same technique as in lemmas \ref{Lemma:primitive>2} and
  \ref{Lemma:primitive=2-F4}
  \begin{equation*}
    \PTr_{\U_{i-1}/\U_0}(x_i)^2-\PTr_{\U_{i-1}/\U_0}(x_i) = 
    \Tr_{\U_{i-1}/\U_0}(\gamma_{i-1}) = 1 \text{,}
  \end{equation*}
  where the last equality comes from corollary \ref{coro:trace}.
  
  Then $\PTr_{\U_{i-1}/\U_0}(x_i)$ is a root of $X^2-X-1$, hence
  \begin{equation*}
    \PTr_{\U_{i-1}/\U_0}(x_i) = \omega + \beta_i
    \qquad\text{with $\beta_i\in\F_2$.}
  \end{equation*}
  Then the polynomial $\PTr_{\U_{i-1}/\U_0}(X) - \beta_i$ is in
  $\F_2[X]$ and sends $x_i$ over $\omega$.
\end{proof}


% Local Variables:
% mode:flyspell
% ispell-local-dictionary:"british"
% End:
%
% LocalWords:  Schreier Artin pseudotrace frobenius bivariate memoization

\section{Building up the tower}
\label{sec:building-up}

Using theorem \ref{th:cantor} we can build an Artin-Schreier tower for
which all of the arithmetic operations listed in Section
\ref{sec:arithmetics} have quasi-optimal algorithms. In Section
\ref{sec:couveignes-algorithm} we develop a method to bring such fast
arithmetics to any Artin-Schreier tower.

We want to build the tower defined by $\gamma_0,\gamma_1,\ldots$ as in
theorem \ref{th:cantor}. We will give algorithms for the arithmetics
in the same order as in Section \ref{sec:arithmetics}.

\subsection{Representation}
Suppose we are given the polynomial $Q_0$ defining $\U_0$ as the field
$\F_p[X_0]/Q_0(X_0)$ and let $x_0$ be the class of $X_0$ in $\U_0$,
then $x_0$ generates $\U_0$ over $\F_p$. If
$\Tr_{\U_0/\F_p}(x_0)\ne0$, the hypotheses of theorem \ref{th:cantor}
are met and we can start building the tower. If this is not the case,
though, define $x_0'=x_0+1$, then $x_0'$ obviously generates $\U_0$
over $\F_p$. Moreover, since $d$ is prime to $p$,
$\Tr_{\U_0/\F_p}(1)\ne0$ and $x_0'$ meets the required
hypotheses.

What we need now is to find $Q_0'$, the minimal polynomial of
$x_0'$. It is easy to observe that $x_0'$ is a root of $Q_0(X-1)$ and
since $Q_0(X-1)$ is monic of degree $d$ and $x_0'$ generates $U_0$,
then $Q_0'(X) = Q_0(X-1)$. To compute $Q_0'$ we use the algorithm
\alg{Compose\_X-1}.

\begin{figure}[!h]
  \begin{algorithm}
    {Compose\_X-1}
    {$P\in\F_p[X]$,}
    {$P(X-1)$.}
  \item If $\deg(P) < 1$, return $P$.
  \item Else, let $P = P_0 + X^{p^{i-1}}P_1$ with $p^{i-1}\le\deg(P)<p^i$,
    \begin{enumerate}
      \item Compute $P_0' =$ Compose\_X-1($P_0$) ;
      \item Compute $P_1' =$ Compose\_X-1($P_1$) ;
      \item \label{alg:compose-x-1:sum}Compute and return $P_0' - P_1'
        + X^{p^{i-1}}P_1'$.
    \end{enumerate}
  \end{algorithm}
\end{figure}

\begin{theorem}
  \label{th:x-1}
  The algorithm \alg{Compose\_X-1} is correct. If its input has degree
  less than $n$, it computes its result in $O(pn\log_pn)$
  $\F_p$-operations.\footnote{A similar algorithm that splits $P$ into
    $p$ polynomials of the same degree and then uses Horner's rule to
    obtain the result does exactly the same number of additions and
    polynomial shifts as this one. Indeed, splitting $P$
    asymmetrically and then composing as in step
    \ref{alg:compose-x-1:sum}, is just a concealed Horner's rule. This
    accounts for the supplementary $p$ factor in the complexity,
    compared to the quasi-optimal complexity $O(n\log_pn)$.}
\end{theorem}
\begin{proof}
  The correctness is evident from the equation
  \begin{equation*}
    P_0(X-1) + (X-1)^{p^{i-1}}P_1(X-1) =
    P_0(X-1) + (X^{p^{i-1}}-1)P_1(X-1)\text{.}
  \end{equation*}

  We note by $C$ the complexity of the algorithm. We start by
  analysing $C(p^i-1)$. If $i=0$, $C(0) = 1\in O(1)$.

  For $i\ge1$, 
  \begin{align*}
    \deg(P_0)&\le p^{i-1}-1\text{,}\\
    \deg(P_1)&\le p^{i-1}(p-1)-1\text{.}
  \end{align*}
  Then step \ref{alg:compose-x-1:sum} requires one polynomial shift by
  $p^{i-1}$ and $p^{i-1}(p-1)$ scalar sums (since $P_0'$ and
  $X^{p^i}P_1'$ have no common non-zero coefficient). The overall
  complexity of step \ref{alg:compose-x-1:sum} is then $O(p^i)$, so
  that we have
  \[C(p^i-1) = C(p^{i-1}-1) + C(p^{i-1}(p-1)-1) + O(p^i)\text{.}\]

  If we analyse the recursive call on $P_1$, we see that the algorithm
  splits it in two polynomials of degree $p^{i-1}-1$ and
  $p^{i-1}(p-2)-1$ and that step \ref{alg:compose-x-1:sum} costs one
  polynomial shift by $p^{i-1}$ and $p^{i-1}(p-2)$ scalar sums. By
  continuing to follow the recursive call on $P_1$ until it has degree
  $p^{i-1}-1$, one sees that
  \[C(p^i-1) = pC(p^{i-1}-1) + pO(p^i)\text{.}\]
  From which we deduce $C(p^i-1) = O(p^{i+1}i)$.

  Let now $cp^i\le n<(c+1)p^i$ with $1\le c<p$. By the above approach
  we see that
  \[C(n) \le C((c+1)p^i-1)) = (c+1)C(p^i-1) + (c+1)O(p^{i+1})\text{,}\]
  so that from what we have shown before
  \begin{equation*}
    C(n) \le O\left((c+1)p^{i+1}i + (c+1)p^{i+1}\right) = 
    O\left(cp^{i+1}i\right)
    \text{.}
  \end{equation*}
  By observing that $cp^i\le n$ and that $i\le\log_pn$ we conclude
  that $C(n) \le O(pn\log_pn)$.
\end{proof}

Now we can start building up the tower. What we need is the sequence
$Q_1,Q_2,\ldots$ of the minimal polynomials of $x_1,x_2,\ldots$ over
$F_p$. For $i=1$, we know that $x_1^p-x_1=\gamma_0=x_0$, so that $x_1$
is a root of $Q_0(X^p-X)$ (if $\gamma_0=x_0+1$ we use $Q_0'$
instead). Since $Q_0(X^p-X)$ is monic and has degree $pd$ and $x_1$ is
a primitive element over $\F_p$, we deduce $Q_1(X)=Q_0(X^p-X)$. The
same considerations hold for $i=2$ when $p=2$. To compute $Q_i(X^p-X)$
we use the algorithm \alg{Compose\_X$^p$-X}.

\begin{figure}[!h]
  \begin{algorithm}
    {Compose\_X$^p$-X}
    {$P\in\F_p[X]$,}
    {$P(X-1)$.}
  \item If $\deg(P) < 1$, return $P$.
  \item Else, let $P = P_0 + X^{p^{i-1}}P_1$ with $p^{i-1}\le\deg(P)<p^i$,
    \begin{enumerate}
      \item Compute $P_0' =$ Compose\_X$^p$-X($P_0$) ;
      \item Compute $P_1' =$ Compose\_X$^p$-X($P_1$) ;
      \item \label{alg:compose-xp-x:sum}Compute and return $P_0' -
        X^{p^{i-1}}P_1' + X^{p^i}P_1'$.
    \end{enumerate}
  \end{algorithm}
\end{figure}

\begin{theorem}
  The algorithm \alg{Compose\_X$^p$-X} is correct. If its input has
  degree less than $n$, it computes its result in $O(p^2n\log_pn)$
  $\F_p$-operations.
\end{theorem}
\begin{proof}
  The correctness is evident from the equation
  \begin{equation*}
    P_0(X^p-X) + (X^p-X)^{p^{i-1}}P_1(X^p-X) =
    P_0(X^p-X) + (X^{p^i}-X^{p^{i-1}})P_1(X^p-X)\text{.}
  \end{equation*}

  We note by $C$ the complexity of the algorithm. As in the proof of
  theorem \ref{th:x-1}, we start by studying $C(p^i-1)$.If $i=0$,
  $C(0) = 1\in O(1)$.

  For $i\ge1$,
  \begin{align*}
    \deg(P_0')&\le p^i-p\text{,}\\
    \deg(P_1')&\le p^i(p-1)-p\text{.}
  \end{align*}
  Then step \ref{alg:compose-xp-x:sum} requires two polynomial shifts
  by $p^{i-1}$ and $p^i$ and $(p^i-1)(p-1)$ scalar sums (since $P_0'$
  and $X^{p^i}P_1'$ have no common non-zero coefficient). The overall
  complexity of step \ref{alg:compose-x-1:sum} is then $O(p^{i+1})$,
  so that we have
  \[C(p^i-1) = C(p^{i-1}-1) + C(p^{i-1}(p-1)-1) + O(p^{i+1})\text{.}\]

  Like in the proof of theorem \ref{th:x-1}, we analyse the rightmost
  branching of the recursion tree at depth $p$ and we find that
  \[C(p^i-1) = pC(p^{i-1}-1) + pO(p^{i+1})\text{,}\]
  From which we deduce $C(p^i-1) = O(p^{i+2}i)$.

  Let now $cp^i\le n<(c+1)p^i$ with $1\le c<p$. As in theorem
  \ref{th:x-1}
  \begin{equation*}
    C(n) \le C((c+1)p^i-1)) \le O\left((c+1)p^{i+2}i + (c+1)p^{i+2}\right) = 
    O\left(cp^{i+2}i\right)
    \text{.}
  \end{equation*}
  By observing that $cp^i\le n$ and that $i\le\log_pn$ we conclude
  that $C(n) \le O(p^2n\log_pn)$.
\end{proof}

Before constructing the other levels of the tower, we will have to do
some more theory.

In what follows, $n$ is prime to $p$, $\Cyclo_n$ is the $n$-th
cyclotomic polynomial and $\omega$ one of its root (thus a primitive
$n$-th root of unity). It is well known (see \cite[Section 2.4]{LN})
that $\deg(\Cyclo_n)$ has degree $\euler(n)$, and that its roots are
the generators of the (multiplicative) cyclic group of the $n$-th
roots of unity. It is also known that $\Cyclo_n$ factors over $\F_p$
in $\euler(n)/d$ distinct factors of degree $d$, where $d$ is the
least positive integer such that $p^d=1\bmod n$.

It is less well known, yet fundamental, that $\F_p[X]/P(X)$, where $P$
is any divisor (not necessary irreducible) of $\Cyclo_n$, is a
\emph{polynomially cyclic $\F_p$-algebra} in the sense of Mih\u
ailescu and Vuletescu \cite{MV}. This is an easy consequence of the
factorisation of $\Cyclo_n$ and of \cite[Theorem 2.4]{MV}.

Mih\u ailescu and Vuletescu show that for any polynomially cyclic
$\K$-algebra $A=\K[X]/f(X)$, there is an automorphism $\nu$ such that
for any root $\omega\in A$ of $f$,
\[\nu(\omega),\nu^2(\omega),\ldots,\nu^{\deg(f)}(\omega)\]
are all distinct roots of $f$. The cyclic group generated by $\nu$ has
order $\deg(f)$ and is noted $\Gal(A/\K)$.

\cite[Theorem 3.2]{MV} says that $\Fix(\Gal(A/\K)) = \K$, thus in
order to show that an element of $A$ is in $\K$, one may show that it
is fixed under the action of $\Gal(A/\K)$. This argument is easily
extended to show that polynomials in $A[Y]$ are in $\K[Y]$.

Going back to the case $A_{\Cyclo_n} = \F_p[X]/\Cyclo_n(X)$, if we
note by $\mathcal{S}_{\Cyclo_n}$ the group of permutations of the
roots of $\Cyclo_n$, then $\Gal(A_{\Cyclo_n}/\K)$ is the largest
subgroup of $\mathcal{S}_{\Cyclo_n}$ that is compatible with the group
structure of the $n$-th roots of unity. When $A = \F_p[X]/P(X)$ with
$P$ a divisor of $\Cyclo_n$, \cite[Theorem 5.2]{MV} tells us that
$\Gal(A/\K) = \Gal(A_{\Cyclo_n}/\K)/H$ where $H$ is the subgroup of
$\Gal(A_{\Cyclo_n}/\K)$ letting the roots of $P$ stable. Such a group
is isomorphic to the Galois group of $A/\K$ when $P$ is irreducible.

Using these tools, we can easily show the following lemma

\begin{lemma}
  \label{lemma:poly-cyclic}
  Let $P$ be a divisor of $\Cyclo_n$, let $A = \F_p[X]/P(X)$ and let
  $\omega\in A$ be a root of $P$. Let $Q\in\F_p[Y]$ and define
  \[Q'(Y) = \prod_{i=0}^{n-1}Q(\omega^iY)\text{.}\]
  \begin{enumerate}
  \item $Q'(Y)$ is in $\F_p[Y]$,
  \item $Q'(Y) = q'(Y^n)$ for a polynomial $q'\in\F_p[Y]$.
  \end{enumerate}
\end{lemma}
\begin{proof}
  1. Let $\nu\in\Gal(A/\K)$. Since $\nu\in\mathcal{S}_{\Cyclo_n}$ and
  since $\omega$ is a generator of the cyclic group of the $n$-th
  roots of unity, $\nu(\omega) = \omega^j$ for a certain $j$ prime to
  $n$. Then
  \begin{equation*}
    \nu(Q'(Y)) = \prod_{i=0}^{n-1}Q(\nu(\omega^i)Y) =
    \prod_{i=0}^{n-1}Q(\omega^{ij}Y) = \prod_{i'=0}^{n-1}Q(\omega^{i'}Y) =
    Q'(Y)\text{.}
  \end{equation*}
  Thus $Q'$ is stable under the action of $\Gal(A/\K)$ and we conclude
  that it is in $\F_p[Y]$.

  2. Observe the following identity
  \begin{equation*}
    Q'(\omega Y) = \prod_{i=0}^{n-1}Q(\omega^{i+1}Y) =
    \prod_{i'=1}^{n}Q(\omega^{i'}Y) = Q'(Y) \text{.}
  \end{equation*}
  If $Q'(Y) = \sum a_jY^j$, the above equality tells that $a_j =
  a_j\omega^j$ for all $j$, hence 
  \[\omega^j = 1 \;\Leftrightarrow\; a_j \text{ is invertible.}\]
  But $\omega$ is a primitive root of unity, then
  \[\omega^j = 1 \;\Leftrightarrow\; n|j\]
  which implies the thesis.
\end{proof}

Suppose now that we know $Q_i$, the minimal polynomial of $x_i$ over
$\F_p$, and consider the following polynomials in $\F_p[Y]$
\[q_i'(Y^{2p-1}) = Q_i'(Y) = \prod_{i=0}^{2p-2}Q_i(\omega^i Y)\text{,}\]
where $\omega$ is a primitive $2p-1$-th root of unity in a
polynomially cyclic algebra.

$x_i$ is a root of $Q_i$, hence it is a root of $Q_i'$ too ; we deduce
that $\gamma_i=x_i^{2p-1}$ is a root of $q_i'$. If we define
\[Q_{i+1}(Y) = q_i'(Y^p-Y)\text{,}\]
we see that $x_{i+1}$ is a root of $Q_{i+1}$. It is easy to see that
$Q_{i+1}$ is a monic polynomial of degree $p^{i+1}d$, hence it is the
minimal polynomial of $x_{i+1}$ over $\F_p$.

\begin{theorem}
  To compute $Q_{i+1}$ knowing $Q_i$ it takes $O\left(\Mult(p^{i+2}d)\log p
  + p^{i+2}d\log_p(p^id)\right)$ $\F_p$-operations.
\end{theorem}
\begin{proof}
  In order to compute the polynomial $q_i'$ we need to work in the
  algebra $A=\F_p[X]/P(X)$ where $P$ is a divisor of
  $\Cyclo_{2p-1}$. Our choice\footnote{It is evident that
    $\euler(2p-1)$ is about the same number of bits as $p$, thus
    arithmetics in $A$ may be slow for too large $p$'s. Factoring
    $\Cyclo_{2p-1}$ usually yields polynomials with much less degree
    and it would be interesting in general to compute modulo such
    polynomials. The state of the art factoring techniques, though,
    don't have a satisfactory asymptotic complexity, this is why we
    don't try to factor the cyclotomic polynomial. From a practical
    point of view, the computation of $P$ has to be done once for all
    for the whole tower, while arithmetics in $A$ have to be performed
    at each time we add up a level to it. It may be interesting, then,
    to use a quick probabilistic factoring method to reduce the degree
    of $P$ at the beginning of the computation. Since the choice of
    $P$ only depends on the characteristic, such polynomials may even
    be stored in a file for small $p$'s.} is to pick
  $P=\Cyclo_{2p-1}$. Using the algorithm in \cite{}, it takes
  $O(\Mult(p)\log p)$ operations to compute $\Cyclo_{2p-1}$ and this
  can be done once for all for any tower in characteristic $p$.
  
  Once we know $\Cyclo_{2p-1}$, the multiplication of polynomials of
  degree $n$ in $A[Y]$ can be performed in $\Mult(n\euler(2p-1)) =
  O(\Mult(np))$ $\F_p$-operations by Kronecker substitution. The
  overall cost of computing $Q_i'$ is then $O(\Mult(p^{i+2}d)\log p)$
  using a subproduct tree approach. To compute $Q_{i+1}$ we then use
  algorithm \alg{Compose\_X$^p$-X} which takes $O(p^{i+2}d\log_p(p^id))$
  operations.
\end{proof}

\subsection{Arithmetics}
Now that we are able to represent the elements of $\U_i$ as
polynomials in $\F_p[X_i]$ modulo $Q_i$, we can perform all the
standard operations such as addition, multiplication, powering, traces
and polynomial arithmetics using the classical algorithms sketched in
Section \ref{sec:arithmetics}.

Level embedding, iterated Frobenius, pseudotraces and tower
isomorphism need special treating and they are going to be the object
of Sections \ref{sec:level-embedding}, \ref{sec:pseudotrace-frobenius} and
\ref{sec:couveignes-algorithm} respectively.




% Local Variables:
% mode:flyspell
% ispell-local-dictionary:"british"
% End:
%
% LocalWords:  Schreier Artin pseudotrace frobenius bivariate memoization monic
% LocalWords:  Horner Horner's cyclotomic polynomially automorphisms
% LocalWords:  automorphism

%% these.tex
%% Copyright 2010 Luca De Feo
%% All rights reserved


\section{Level embedding}
\label{sec:level-embedding}

We discuss here change-of-basis algorithms for the tower $(\U_0,
\ldots, \U_k)$ of the previous section; these algorithms are needed
for most further operations. We detail the main case where $P_i =
X_i^p - X_i - X_{i-1}^{2p-1}$; the case $P_1= X_1^p - X_1 - X_0$ (and
$P_2=X_2^2+X_2+X_1$ for $p=2$ and $d$ odd) is easier.

Recall the two families of $\F_p$-bases we have defined so far:
\begin{align}
  \pdfmctwo{Explicit indices for the basis.}
  \basis{B}_i &=
  \{x_0^{e_0} \cdots x_i^{e_i} \;|\; 0 \le e_0 < d,\; 0\le e_j < p 
  \text{ for $j>0$}\}
  \text{,}\\
  \basis{C}_i&=\{x_i^a \;|\; 0\le a<p^id\}
  \text{.}  
\end{align}
The first one arises naturally when constructing the tower as a
succession of Artin-Schreier extensions, and we expect our inputs to
be given in such basis. Furthermore, lifting in $\basis{B}_j$ an
element written on $\basis{B}_i$ for $i<j$ is immediate in this basis,
and so is the inverse operation. The basis $\basis{C}_i$, on the
other hand, is practical for multiplication, inversion, etc., but it
is not evident how to lift elements.

We shall thus need algorithms to change between these two bases. Since
$x_i$ is clearly a separating element for the variety $V(K_i)$ (see
Chapter~\ref{cha:trace-computations}), we will use
Proposition~\ref{th:uni-multi-uni} to go from $\basis{B}_i$ to
$\basis{C}_i$, but we shall need an algorithm for the inverse map first.

Instead of converting from $\basis{C}_i$ to $\basis{B}_i$ in one shot,
we will pass through some intermediate bivariate bases to keep the
complexity low. By Theorem~\ref{th:cantor}, $\U_i$ equals
$\F_p[X_{i-1},X_i]/I$, where the ideal $I$ admits the following
Gr{\"o}bner bases, for respectively the lexicographic orders
$X_i>X_{i-1}$ and $X_{i-1}>X_i$:
\begin{equation}
  \left |
  \begin{array}{rl}
    X_i^p - X_i - X_{i-1}^{2p-1} \\
    Q_{i-1}(X_{i-1})         
  \end{array}
\right.
  \quad \text{and}\quad
  \left |
  \begin{array}{rl}
    X_{i-1} - R_i(X_i) \\
    Q_i(X_i),
  \end{array}
\right.
\end{equation}
with $R_i$ in $\F_p[X_i]$. Both Gröbner bases are triangular and
bivariate, one can go from one to the other using the algorithms of
Pascal and Schost~\cite{pascal+schost06}, in fact most of the ideas of
this section are inspired by their paper.

Since $\deg(Q_{i-1})=p^{i-1}d$ and $\deg(Q_{i})=p^id$, we associate
the following $\F_p$-bases of $\U_i$ to each system:
\begin{align}
  \pdfmctwo{Explicit indices for the basis.}
  \basis{D}_i &= \{x_{i-1}^ax_i^b \;|\; 0\le a<p^{i-1}d,\; 0\le b<p\}
  \text{,}\\
  \label{eq:bases}
 \basis{C}_i&=\{x_i^a \;|\; 0\le a<p^id\}
 \text{.}  
\end{align}
We describe an algorithm called \titleref{alg:push-down} which takes $v$
written on the basis $\basis{C}_i$ and returns its coordinates on the
basis $\basis{D}_i$. Then, using Proposition~\ref{th:uni-multi-uni},
we will be able to describe the inverse operation, called
\titleref{alg:liftup}.
In other words, \titleref{alg:push-down} inputs $v\wrt\U_i$ and
outputs the representation of $v$ as
\begin{equation}
  \label{eq:vectorspace}
  v = v_0 + v_1x_i + \cdots + v_{p-1}x_i^{p-1}, \quad\text{with all~} v_j \wrt \U_{i-1}
\end{equation}
and \titleref{alg:liftup} does the opposite.

Then, the change from $\basis{C}_i$ to $\basis{B}_i$ is done by
repeatedly applying \titleref{alg:push-down}, and the opposite is obtained by
repeatedly applying \titleref{alg:liftup}.

Hereafter, we let $\Lift:\N-\{0\} \to \N$ be such that both
\titleref{alg:push-down} and
\titleref{alg:liftup} can be performed in time
$\Lift(i)$; to simplify some expressions appearing later on, we add
the mild constraints that $p\,\Lift(i) \le \Lift(i+1)$ and
$p\,\Mult(p^{i}d) \in O(\Lift(i))$.  To reflect the behavior of the
implementation, we also allow precomputations. These precomputations
are performed when we build the tower; further details are at the end
of this section.
\begin{theorem}\label{theo:L}
  One can take $\Lift(i)$ in $O( p^{i+1}d\log_p(p^id)^2 \ + \
\,p\,\Mult(p^{i}d))$.
\end{theorem}
Remark that the input and output have size $p^id$; using fast
multiplication, the cost is linear in $p^{i+1}d$, up to logarithmic
factors. The rest of this section is devoted to proving this theorem.
\titleref{alg:push-down} is a divide-and-conquer process, adapted to the shape
of our tower; \titleref{alg:liftup} is a special case of
Proposition~\ref{th:uni-multi-uni}, the power projection will be
obtained using the transposed version of \titleref{alg:push-down}.

As said before, the algorithms of this section (and of the following
ones) use precomputed quantities. To keep the pseudo-code simple, we
do not explicitly list them in the inputs of the algorithms;
we show, later, that the precomputation is fast too.

%%%%%%%%%%%%%%%%%%%%%%%%%%%%%%%%%%%%%%%%%%%%%%%%%%%%%%%%%%%%

\subsection{Modular multiplication}
\label{ssec:mulmod}

We first discuss a routine for multiplication by $X_i^{p^n}$
in $\F_p[Y,X_i]/(X_i^p-X_i-Y)$, and its transpose. We start by
remarking that 
\begin{equation}
  \label{eq:Kn}
  X_i^{p^n}=X_i+R_n \mod (X_i^p-X_i-Y) \qquad\text{with }
  R_n = \sum_{j=0}^{n-1} Y^{p^j}
  \text{.}
\end{equation}
Then, precisely, for $k$ in $\N$, we are interested in the operation
\begin{equation}
  \tag{\alg{MulMod}}
  \label{eq:89}
  \text{\alg{MulMod}}_{k,n}: A \mapsto (X_i+R_n)A \mod (X_i^p-X_i-Y)
  \text{,}  
\end{equation}
with $A\in \F_p[Y,X_i]$, $\deg_Y(A) < k$ and $\deg_{X_i}(A) <p$.

Since $R_n$ is sparse, it is advantageous to use the naive algorithm;
besides, to make transposition easy, we explicitly give the matrix of
\titleref{eq:89}. Let $m_0$ be the
$(k+p^{n-1})\times k$ matrix having $1$'s on the diagonal only, and
for $\ell \le p^{n-1}$, let $m_\ell$ be the matrix obtained from $m_0$
by shifting the diagonal down by $\ell$ places. Let finally $m'$ be
the sum $\Sigma_{j=0}^{n-1} m_{p^j}$. Then one verifies that the
matrix of $\text{\titleref{eq:89}}_{k,n}$ is
\begin{equation}
  \pdfmctwo{Detailed last column of the matrix.}
  \begin{bmatrix}
    m'  &     &        &        &        & m_1 \\
    m_0 & m'  &        &        &        & m_0 \\
        & m_0 & m'     &        &        & 0   \\
        &     & \ddots & \ddots &        & \vdots\\
        &     &        & \ddots & \ddots & 0\\
        &     &        &        & m_0    & m'
  \end{bmatrix}
  \text{,}
\end{equation}
with columns and rows indexed by 
\begin{equation}
  \label{eq:90}
  (X_i^j,\dots,Y^{k-1}X_i^j)_{j < p}
  \quad\text{and}\quad
  (X_i^j,\dots,Y^{k+p^{n-1}-1}X_i^j)_{j < p}  
\end{equation}
respectively.  Since this matrix has $O(pnk)$ non-zero entries, we can
compute both \titleref{eq:89} and its dual
$\dual{\titleref{eq:89}}$ in time $O(pnk)$.


%%%%%%%%%%%%%%%%%%%%%%%%%%%%%%%%%%%%%%%%%%%%%%%%%%%%%%%%%%%% 

\subsection{Push-down}\label{sec:level-embedding:push-down}

The input of \titleref{alg:push-down} is $v \wrt \U_i$, that is, given on the
basis $\basis{C}_i$; we see it as a polynomial $V \in \F_p[X_i]$ of degree
less than $p^id$. The output is the normal form of $V$ modulo
$X_i^p-X_i-X_{i-1}^{2p-1}$ and $Q_{i-1}(X_{i-1})$. We first use a
divide-and-conquer subroutine to reduce $V$ modulo
$X_i^p-X_i-X_{i-1}^{2p-1}$; then, the result is reduced modulo
$Q_{i-1}(X_{i-1})$ coefficient-wise.

To reduce $V$ modulo $X_i^p-X_i-X_{i-1}^{2p-1}$, we first compute 
\begin{equation}
  \label{eq:91}
  W=V \bmod (X_i^p-X_i-Y)
  \text{,} 
\end{equation}
then we replace $Y$ by $X_{i-1}^{2p-1}$ in $W$.  Because our algorithm
will be recursive, we let $\deg(V)$ be arbitrary; then, we have the
following estimate for $W$.

\begin{lemma}
  \label{th:push-down-degree} We have $\deg_Y(W)\le \deg(V)/p$.
\end{lemma}
\begin{proof}
  Consider the matrix $M$ of multiplication by $X_i^p$ modulo
  $X_i^p-X_i-Y$; it has entries in $\F_p[Y]$. Due to the sparseness of
  the modulus, one sees that $M$ has degree at most $1$, and so $M^k$
  has coefficients of degree at most $k$. Thus, the remainders of
  $X_i^{pk},\dots,X_i^{pk+p-1}$ modulo $X_i^p-X_i-Y$ have degree at
  most $k$ in $Y$.
\end{proof}


We compute $W$ by a recursive subroutine \titleref{alg:push-down-rec}, similar
to \titleref{alg:compose}. As before, we let $c,n$ be such that $1\le c<p$ and
$\deg(V) < (c+1)p^n$, so that we have
$$V=V_0+ V_1X_i^{p^n}+\cdots+V_c X_i^{cp^n},$$ with all $V_j$ in
$\F_p[X_i]$ of degree less than $p^n$. First, we recursively reduce
$V_0,\dots,V_c$ modulo $X_i^p-X_i-Y$, to obtain bivariate
polynomials $W_0,\dots,W_{c}$. Let $R_n$ be the polynomial defined in
Equation~\eqref{eq:Kn}. Then, we get $W$ by computing
$\Sigma_{j=0}^c W_j(X_i+R_n)^j$ modulo $X_i^p-X_i-Y$,
using Horner's scheme as in \titleref{alg:compose}. Multiplications by
$X_i+R_n$ modulo $X_i^p-X_i-Y$ are done using \titleref{eq:89}.

\begin{algorithm}
  \caption{\alg{Push-down-rec}}
  \label{alg:push-down-rec}
  \begin{algorithmic}[1]
    \REQUIRE $V\in \F_p[X_i]$ and $c,n\in\N$.
    \ENSURE $W \in\F_p[Y,X_i]$.
    \STATE if $n=0$ return $V$;
    \STATE write $V=\sum_{j=0}^{c} V_j X_i^{jp^n}$, with $V_j \in \F_p[X_i], \deg V_j<p^n$;
    \STATE for $j\in [0,\dots,c]$, let $W_j=$ \titleref{alg:push-down-rec}$(V_j,p-1,n-1)$;
    \STATE $W=0$;
    \STATE\label{pd:loop} for $j\in [c,\dots,0]$, let $W=\text{\titleref{eq:89}}_{(c+1)p^{n-1},n}(W) + W_j$;
    \STATE return $W$.
  \end{algorithmic}
\end{algorithm}

\begin{algorithm}
  \caption{\alg{Push-down}}
  \label{alg:push-down}
  \begin{algorithmic}[1]
    \REQUIRE $v\wrt \U_i$.
    \ENSURE $v$ written as $v_0+\cdots+v_{p-1}x_i^{p-1}$ with $v_j \wrt \U_{i-1}$.
    \STATE let $V$ be the canonical preimage of $v$ in $\F_p[X_i]$;
    \STATE let $n=\lfloor \log_p(p^id-1) \rfloor$ and $c=(p^id-1)\text{ div } p^n$;
    \STATE let $W =$ \titleref{alg:push-down-rec}$(V,c,n)$;
    \STATE let $Z =$ \alg{Evaluate}$(W,[X_{i-1}^{2p-1},X_i])$;
    \STATE \label{step:pd:mod} let $Z = Z \bmod Q_{i-1}$;
    \STATE \label{step:pd:return} return the residue class of $Z \bmod (X_i^p - X_i - X_{i-1}^{2p-1},Q_{i-1})$.
  \end{algorithmic}
\end{algorithm}

\begin{proposition}\label{prop:pd}
  Algorithm \titleref{alg:push-down} is correct and takes time 
  \begin{equation}
    \label{eq:92}
    O(p^{i+1}d
    \log_p(p^id)^2 + p\,\Mult(p^id))
    \text{.}
  \end{equation}
\end{proposition}
\begin{proof}
  Correctness is straightforward; note that at step~\ref{pd:loop} of
  \titleref{alg:push-down-rec},
  $\deg_Y(W) < (c+1)p^{n-1}$, so our call to
  \titleref{eq:89} is justified. By
  the claim of Subsection~\ref{ssec:mulmod} on the cost of
  \titleref{eq:89}, the total time spent in that loop is
  $O(nc^2p^n)$. As in Theorem~\ref{theo:comp}, we deduce that the time
  spent in \titleref{alg:push-down-rec} is $O(n^2c^2p^n)$.

  In \titleref{alg:push-down}, we have $cp^n< p^id$ and $n<\log_p (p^id)$, so
  the previous cost is seen to be $O(p^{i+1}d
  \log_p(p^id)^2)$. Reducing one coefficient of $Z$ modulo $Q_{i-1}$
  takes time $O(\Mult(p^id))$, so step~\ref{step:pd:mod} has cost
  $O(p\,\Mult(p^id))$. Step~\ref{step:pd:return} is free, since at
  this stage $Z$ is already reduced. 
\end{proof}

%%%%%%%%%%%%%%%%%%%%%%%%%%%%%%%%%%%%%%%%%%%%%%%%%%%%%%%%%%%%
%%%%%%%%%%%%%%%%%%%%%%%%%%%%%%%%%%%%%%%%%%%%%%%%%%%%%%%%%%%%
%%%%%%%%%%%%%%%%%%%%%%%%%%%%%%%%%%%%%%%%%%%%%%%%%%%%%%%%%%%%

\subsection{Transposed push-down}

Before giving the details for \titleref{alg:liftup}, we discuss here the
transpose of \titleref{alg:push-down}.  As in
Section~\ref{sec:from-univ-bivar}, \titleref{alg:push-down} is the same thing
as the map
\begin{equation}
  \label{eq:81}
  \begin{aligned}
    \ev_{x_i}:\F_p[T] &\ra \U_i^{\basis{D}_i}\text{,}\\ 
    g&\mapsto g(x_i)\text{.}
  \end{aligned}
\end{equation}
So its transpose is the map
\begin{equation}
  \label{eq:82}
  \begin{aligned}
    \proj_{x_i}:(\dual{\U_i})^{\dual{\basis{D_i}}}&\ra\F_p[[1/T]]\text{,}\\
    \ell&\mapsto\sum_{j\ge0}\frac{\ell(x_i^j)}{T^j}\text{.}
  \end{aligned}
\end{equation}

\titleref{alg:push-down} is an \hyperref[sec:r-algebraic-transforms]{algebraic
  transform}, thus, applying Theorem~\ref{th:tellegen-R-algeb}, the
transposed algorithm is obtained by reversing the initial algorithm
step by step, and replacing subroutines by their transposes. The
overall cost remains the same; we review here the main
transformations.

As usual, we identify the dual of the space $\F_p[Y,X_i]$ to
$\F_p[[1/Y,1/X_i]]$. Thus linear forms given as input to the algorithm
are written as series
\begin{equation}
  \label{eq:84}
  L=\sum_{a,b\ge0}\frac{\ell_{a,b}}{Y^aX_i^b}
  \text{.}
\end{equation}
We do the same for $\F_p[X_i]$ and $\F_p[X_{i-1},X_i]$.

The initial loop at step~\ref{pd:loop} is a Horner scheme; the
transposed loop is run backward, and its core becomes $L_j=L\bmod
Y^{1-n}$ and $L=\dual{\text{\titleref{eq:89}}_{(c+1)p^{n-1},n}}(L)$; a small
simplification yields the pseudo-code we give.  In
\titleref{alg:push-down}, after calling \titleref{alg:push-down-rec},
we evaluate $W$ at $[X_{i-1}^{2p-1},X_i]$: the transposed operation
$\dual{\text{\alg{Evaluate}}}$ is the map
\begin{equation}
  \label{eq:83}
  \sum_{a,b} \frac{\ell_{a,b}}{X_{i-1}^a X_i^b} \mapsto
  \sum_{a,b} \frac{\ell_{(2p-1)a,b}}{Y^a X_i^b}
  \text{.} 
\end{equation}
Then, originally, we perform a Euclidean division by $Q_{i-1}$ on
$Z$. The \index{transposed~modular~reduction}transposed algorithm
$\dual{\bmod}$ amounts to compute the values of a sequence linearly
generated by the polynomial $Q_{i-1}$ from its first $p^{i-1}d$ values
(see Section~\ref{sec:transp-eucl-divis}).

\begin{algorithm}
  \caption{\alg{Push-down-rec}$\dual{}$}
  \label{alg:push-down-rec-star}
  \begin{algorithmic}[1]
    \REQUIRE $L\in\F_p[[1/Y,1/X_i]]$ and $c,n\in\N$.
    \ENSURE $M\in \F_p[[1/X_i]]$.
    \STATE If $n=0$ return $L$;
    \FORALL{\label{pdt:loop} $j\in [c,\dots,0]$}
    \STATE let $L_j = L \bmod Y^{1-n}$;
    \STATE let $M_j=$ \titleref{alg:push-down-rec-star}$(L_j,p-1,n-1)$;
    \STATE let $L = \dual{\text{\titleref{eq:89}}_{(c+1)p^{n-1},n}}(L)$;
    \ENDFOR
    \STATE return $\sum_{j=0}^{c} \frac{M_j}{X_i^{jp^n}}$.
  \end{algorithmic}
\end{algorithm}

\begin{algorithm}
  \caption{\alg{Push-down}$\dual{}$}
  \label{alg:push-down-star}
  \begin{algorithmic}[1]
    \REQUIRE $L\in \F_p[[1/X_{i-1},1/X_i]]$.
    \ENSURE $M \in \F_p[[1/T]]$.
    \STATE let $n=\lfloor \log_p(p^id-1) \rfloor$ and $c=(p^id-1) \text{ div } p^n$;
    \STATE let $P=\dual{\bmod}(L,Q_{i-1})$;
    \STATE let $M = \dual{\text{\alg{Evaluate}}}(P,[X_{i-1}^{2p-1},X_i])$;
    \STATE return \titleref{alg:push-down-rec-star}$(M,c,n)$;
  \end{algorithmic}
\end{algorithm}


%%%%%%%%%%%%%%%%%%%%%%%%%%%%%%%%%%%%%%%%%%%%%%%%%%%%%%%%%%%%
%%%%%%%%%%%%%%%%%%%%%%%%%%%%%%%%%%%%%%%%%%%%%%%%%%%%%%%%%%%%
%%%%%%%%%%%%%%%%%%%%%%%%%%%%%%%%%%%%%%%%%%%%%%%%%%%%%%%%%%%%

\subsection{Lift-up}
\label{sec:level-embedding:lift-up}

Let $v$ be given on the basis $\basis{D}_i$ and $W$ its canonical
preimage in $\F_p[X_{i-1},X_i]$.  The lift-up algorithm finds $V$ in
$\F_p[X_i]$ such that
\begin{equation}
  \label{eq:93}
  W=V \mod
  (X_i^p-X_i-X_{i-1}^{2p-1},Q_{i-1})
\end{equation}
and outputs the residue class of $V$ modulo $Q_i$. Hereafter, we
assume that both $Q_i'^{-1} \bmod Q_i$ and the values $\rho_i(1)$ of
the trace $\Tr_{\U_i/\F_p}$ on the basis $\basis{D}_i$ are known.  See
the discussion below.

\paragraph{Lift-up}
We use Proposition~\ref{th:uni-multi-uni} to write $v$ as a
polynomial in $x_i$. To do this we proceed as in steps~\ref{alg:rur:4}
and~\ref{alg:rur:5} of \titleref{alg:rur}.  To compute the power projection we
could use transposed bivariate modular composition as
in~\cite{shoup99}; it is however more efficient to use
\titleref{alg:push-down-star}.

\begin{algorithm}
  \caption{\alg{Lift-up}}
  \label{alg:liftup}
  \begin{algorithmic}[1]
    \REQUIRE $v$ written as $v_0+\cdots+v_{p-1}x_i^{p-1}$ with $v_j \wrt \U_{i-1}$.
    \ENSURE $v\wrt \U_i$.
    \STATE \label{alg:lift-up:transmul} let $\ell =$ \alg{TransposedMul}$(v,\,\rho_i(1))$;
    \STATE \label{alg:lift-up:pow} let $M=\frac{1}{T}$\titleref{alg:push-down-star}$(\ell)$;
    \STATE \label{alg:lift-up:mult} let $V = Q_iM \bmod T^{p^id}$;
    \STATE \label{alg:lift-up:mulmod} return $v=V(x_i)Q_i(x_i)^{-1} = V {Q_i'}^{-1} \bmod Q_i$.
  \end{algorithmic}
\end{algorithm}

\begin{proposition}\label{prop:lu}
  Algorithm \titleref{alg:liftup} is correct and takes time
  \begin{equation}
    \label{eq:85}
    O(p^{i+1}d\log_p(p^id)^2+p\,\Mult(p^{i}d))
    \text{.}    
  \end{equation}
\end{proposition}
\begin{proof}
  Correctness is a consequence of Theorem~\ref{th:rur} and of the
  algorithm given in Section~\ref{sec:from-univ-bivar}.

  \alg{TransposedMul} implements the
  \index{transposed~modular~multiplication}transposed bivariate
  modular multiplication; an algorithm of cost $O(\Mult(p^id))$ for
  this is in~\cite[Corollary~2]{pascal+schost06} (see also
  Section~\ref{sec:transp-algor}).  The last subsection showed that
  step~\ref{alg:lift-up:pow} has the same cost as
  \titleref{alg:push-down}. Then, the costs of
  steps~\ref{alg:lift-up:mult} and~\ref{alg:lift-up:mulmod} are
  $O(\Mult(p^id))$.
\end{proof}

Propositions~\ref{prop:pd} and~\ref{prop:lu} prove
Theorem~\ref{theo:L}.


\paragraph{Precomputations}
The precomputations, that are done at the construction of $\U_i$, are
as follows.  First, we need the values of the trace on the basis
$\basis{D}_i$. By~\eqref{eq:trcomp} we know that
\begin{equation}
  \label{eq:86}
  \Tr_{\U_i/\F_p}(x_{i-1}^ax_i^b) = 
  \Tr_{\U_{i-1}/\F_p}\circ\Tr_{\U_i/\U_{i-1}}(x_{i-1}^ax_i^b)
  \text{,}
\end{equation}
then, by \eqref{eq:pd}
\begin{equation}
  \label{eq:87}
  \Tr_{\U_i/\U_{i-1}}(x_{i-1}^ax_i^b) =
  \begin{cases}
    0 &\text{for $0\le b < p-1$,}\\
    -x_{i-1}^a &\text{for $b=p-1$.}
  \end{cases}
\end{equation}
Thus the values of $\Tr_{\U_i/\F_p}$ on the basis $\basis{D}_i$ are 
\begin{equation}
  \label{eq:88}
  0, \ldots, 0, -\Tr_{\U_{i-1}/\F_p}(1), -\Tr_{\U_{i-1}/\F_p}(x_{i-1}), \ldots, -\Tr_{\U_{i-1}/\F_p}(x_{i-1}^{p^{i-1}d-1})
  \text{.}
\end{equation}
They can be computed in time $O(\Mult(p^{i-1}d))$ using
Lemma~\ref{th:multi-newton-sums}.

Then, we need ${Q_i'}^{-1} \bmod Q_i$; this takes time $O(\Mult(p^id)
\log(p^id))$ by fast extended GCD computation.  These precomputations
save logarithmic factors at best, but are useful in practice.


% Local Variables:
% mode:flyspell
% ispell-local-dictionary:"american"
% mode: TeX-PDF
% mode: reftex
% TeX-master: "../these"
% End:
%
% LocalWords:  Schreier Artin pseudotrace frobenius bivariate memoization
% LocalWords:  precomputations precomputation

\section{Frobenius and pseudotrace}
\label{sec:pseudotrace-frobenius}

In this section, we describe algorithms computing Frobenius
and pseudotrace operators, specific to the tower of
Section~\ref{sec:fast-tower}; they are the keys to the algorithms of
the next section.

The algorithms in this section and the next one closely follow
Couveignes'~\cite{couveignes00}. However, the latter assumed the
existence of a quasi-linear time algorithm for multiplication in some
specific towers in the multivariate basis $\basis{B}_i$ of
Subsection~\ref{ssec:rep}. To our knowledge, no such algorithm
exists. We use here the univariate basis $\basis{C}_i$ introduced
previously, which makes multiplication straightforward. However,
several push-down and lift-up operations are now required to
accommodate the recursive nature of the algorithm.

Our main purpose here is to compute the pseudotrace
\begin{equation}
  \label{eq:74}
  \PTr_{p^jd}:x\mapsto\sum_{\ell=0}^{p^jd-1}x^{p^{\ell}}
  \text{;}  
\end{equation}
we already gave an algorithm for this task in
Section~\ref{sec:modular-composition}, but in our context we can do
better. We start by describing how to compute values of the iterated
Frobenius operator $x \mapsto x^{p^n}$ by a recursive descent in the
tower.

We focus on computing the iterated Frobenius for $n<d$ or $n=p^jd$. In
both cases, similarly to~\eqref{eq:Kn}, we have:
\begin{gather}
  \label{eq:frobeniussum}
  x_i^{p^{n}} = x_i + \beta_{i-1,n}, \quad\text{with}\quad \beta_{i-1,n}=\PTr_{n}(\gamma_{i-1}).
\end{gather}
Assuming $\beta_{i-1,n}$ is known, the recursive step of the Frobenius
algorithm follows: starting from $v\wrt\U_i$, we first write
$v=v_0+\cdots+v_{p-1}x_i^{p-1}$, with $v_h\wrt\U_{i-1}$; by
\eqref{eq:frobeniussum} and the linearity of the Frobenius, we deduce
that
\begin{equation}
  \label{eq:frobeniuscomp}
\begin{array}{c}
v^{p^n}
  =\sum_{h=0}^{p-1} v_h^{p^n} \left(x_i + \beta_{i-1,n}\right)^{h}.
\end{array}
\end{equation}
Then, we compute all $v_h^{p^n}$ recursively; the final sum is
computed using Horner's scheme. Remark that this equations are not
limited to the case where $n<d$ or of the form $p^jd$: an arbitrary
$n$ would do as well. However, we impose this limitation since these
are the only values we need to compute $\PTr_{p^jd}$.

In the case $n=p^jd$, any $v \in \U_j$ is left invariant by this
Frobenius map, thus we stop the recursion when $i=j$, as there is
nothing left to do. In the case $n<d$, we stop the recursion when
$i=0$ and apply (see Section~\ref{sec:modular-composition}). We
summarize the two variants in one unique algorithm
\alg{IterFrobenius}.

\begin{algorithm}
  \caption{IterFrobenius} 
  \begin{algorithmic}[1]
    \REQUIRE $v$, $i$, $n$ with $v\wrt\U_i$ and $n<d$ or $n=p^jd$.
    \ENSURE $v^{p^n} \wrt \U_i$.
    \STATE \label{alg:frob:base} if $n=p^jd$ and $i \le j$, return $v$;
    \STATE \label{alg:frob:base2} if $i=0$, return $v^{p^n}$;
    \STATE \label{alg:frob:push} let $v_0 + v_1 x_i + \dots + v_{p-1} x_i^{p-1}=\text{\alg{Push-down}}(v)$;
    \STATE \label{alg:frob:rec} for $h \in [0,\dots,p-1]$, let $t_h = \alg{IterFrobenius}(v_h, i-1, n)$;
    \STATE let $F=0$;
    \STATE\label{alg:frob:T} for $h \in [p-1,\dots,0]$, let $F = t_h +  (x_i+\beta_{i-1,n})F$;
    \STATE \label{alg:frob:lift} return $\text{\alg{Lift-up}}(F)$.
  \end{algorithmic}
\end{algorithm}

As mentioned above, the algorithm requires the values $\beta_{i',n}$
for $i'<i$: we suppose that they are precomputed (the discussion of
how we precompute them follows).  To analyze costs, we use the
function $\Lift$ of Section~\ref{sec:level-embedding}.
\begin{theorem}
  \label{th:b-ifrob}
  On input $v\wrt\U_i$ and $n=p^jd$, algorithm \alg{IterFrobenius}
  correctly computes $v^{p^n}$ and takes time
  \begin{equation}
    \label{eq:95}
    O((i-j)\Lift(i))
    \text{.}
  \end{equation}
\end{theorem}
\begin{proof}
  Correctness is clear. We note $\Frob(i,j)$ for the complexity on
  inputs as in the statement; then $\Frob(0,j)=\cdots=\Frob(j,j)=0$
  because step~\ref{alg:frob:base} comes at no cost. For $i>j$, each
  pass through step~\ref{alg:frob:T} involves a multiplication by
  $x_i+\beta_{i-1,n}$, of cost of $O(p\Mult(p^{i-1}d))$, assuming
  $\beta_{i-1,n}\wrt \U_{i-1}$ is known. Altogether, we deduce the
  recurrence relation
  \begin{equation}
    \label{eq:75}
    \Frob(i,j) \le
    p\,\Frob(i-1,j)+2\,\Lift(i)+O(p^2\Mult(p^{i-1}d))
    \text{,}
  \end{equation}
  so $\Frob(i,j) \le p\,\Frob(i-1,j)+O(\Lift(i)),$ by assumptions on
  $\Mult$ and $\Lift$.  The conclusion follows, again by assumptions
  on $\Lift$.
\end{proof}

\begin{theorem}
  \label{th:l-ifrob}
  On input $v\wrt\U_i$ and $n<d$, algorithm \alg{IterFrobenius}
  correctly computes $v^{p^n}$ and takes time 
  \begin{equation}
    \label{eq:94}
    O(p^i\ModComp(d)\log (n) + i\Lift(i))
    \text{.}    
  \end{equation}
\end{theorem}
\begin{proof}
  The analysis is identical to the previous one, except that
  step~\ref{alg:frob:base2} is now executed instead of
  step~\ref{alg:frob:base} and this costs $O(\ModComp(d)\log (n))$ by
  the algorithm of Section~\ref{sec:modular-composition}. The
  conclusion follows by observing that step~\ref{alg:frob:base2} is
  repeated $p^i$ times.
\end{proof}

Next, we compute pseudotraces. We use the following relations, whose
verification is straightforward:
\begin{equation}
  \PTr_{n+m}(v) =
  \PTr_{n}(v) + \PTr_{m}(v)^{p^n}
  \text{,}\qquad
  \PTr_{nm}(v) =
  \sum_{h=0}^{m-1}\PTr_{n}(v)^{p^{hn}}
  \text{.}
\end{equation}
We give two \emph{divide-and-conquer} algorithms that do a slightly
different \emph{divide} step; each of them is based on one of the
previous formulas. The first one, \alg{LittlePseudotrace}, is meant to
compute $\PTr_d$. It follows a binary divide-and-conquer scheme
similar to the algorithm in Section~\ref{sec:modular-composition}. The
second one, \alg{Pseudotrace}, computes $\PTr_{p^jd}$ for $j>0$. It
uses the previous formula with $n=p^{j-1}d$ and $m=p$, computing
Frobenius-es for such $n$; when $j=0$, it invokes the first algorithm.


\begin{algorithm}
  \caption{LittlePseudotrace}
  \begin{algorithmic}[1]
    \REQUIRE $v$, $i$, $n$ with $v\wrt\U_i$ and $0<n\le d$.
    \ENSURE $T_{n}(v) \wrt \U_i$.
    \STATE \label{alg:lpseudo:base} if $n = 1$ return $v$;
    \STATE \label{alg:lpseudo:half} let $m = \lfloor n/2 \rfloor$;
    \STATE \label{alg:lpseudo:rec} let $t=$ {\sf LittlePseudotrace}($v$,
    $i$, $m$);
    \STATE \label{alg:lpseudo:frob} let $t=t+$ {\sf IterFrobenius}($t$, $i$, $m$);
    \STATE \label{alg:lpseudo:odd} if $n$ is odd, let $t=t+$ {\sf
      IterFrobenius}($v$, $i$, $n$);
    \STATE return $t$.
  \end{algorithmic}
\end{algorithm}

\begin{algorithm}
  \caption{Pseudotrace}
  \begin{algorithmic}[1]
    \REQUIRE $v$, $i$, $j$ with $v\wrt\U_i$.
    \ENSURE $T_{p^jd}(v) \wrt \U_i$.
    \STATE \label{alg:pseudo:base} if $j = 0$ return {\sf LittlePseudotrace}($v$, $d$);
    \STATE \label{alg:pseudo:rec} $t_0=${\sf Pseudotrace}($v, i, j-1$);
    \STATE \label{alg:pseudo:frob}for $h\in [1,\dots,p-1]$, let $t_h=\text{\alg{IterFrobenius}}(t_{h-1}, i, j-1)$;
    \STATE \label{alg:pseudo:sum}return $t_0 + t_1 + \cdots + t_{p-1}$.
  \end{algorithmic}
\end{algorithm}

\begin{theorem}
  \label{th:l-pseudo}
  Algorithm \alg{LittlePseudotrace} is correct and takes time
  \begin{equation}
    \label{eq:96}
    O(p^i\ModComp(d)\log^2 (n) + i \Lift(i)\log (n))
    \text{.}
  \end{equation}
\end{theorem}
\begin{proof}
  Correctness is clear. For the cost analysis, we write $\Ptr(i,n)$
  for the cost on input $i$ and $n$, so $\Ptr(i,1)=O(1)$.  For $n>1$,
  step \ref{alg:lpseudo:rec} costs $\Ptr(i,\lfloor n/2 \rfloor)$,
  steps~\ref{alg:lpseudo:frob} and~\ref{alg:lpseudo:odd} cost both
  \begin{equation}
    \label{eq:100}
    O(p^i\ModComp(d)\log^2 (n) + i \Lift(i))
  \end{equation}
  by Theorem~\ref{th:l-ifrob}. This gives
  \begin{equation}
    \label{eq:98}
    \Ptr(i,n) = \Ptr(i,\lfloor
    n/2\rfloor) +O(p^i\ModComp(d)\log^2 (n) + i \Lift(i))
    \text{,}
  \end{equation}
  and thus
  \begin{equation}
    \label{eq:99}
    \Ptr(i,n) \in O(p^i\ModComp(d)\log^2 (n) + i \Lift(i)\log n)
    \text{.}
  \end{equation}
\end{proof}


\begin{theorem}
  \label{th:b-pseudo}
  Algorithm \alg{Pseudotrace} is correct and takes time
  \begin{equation}
    \label{eq:97}
    \Ptr(i)=O((pi+\log (d))i\Lift(i)+p^i\ModComp(d)\log^2 (d))    
  \end{equation}
  for $j \le
  i$.
\end{theorem}
\begin{proof}
  Correctness is clear. For the cost analysis, we write $\Ptr(i,j)$
  for the cost on input $i$ and $j$, so theorem~\ref{th:l-pseudo}
  gives 
  \begin{equation}
    \label{eq:101}
    \Ptr(i,0)=O(p^i\ModComp(d)\log^2 (d) + i \Lift(i)\log(d))
    \text{.} 
  \end{equation}
  For $j>0$, step \ref{alg:pseudo:rec} costs $\Ptr(i,j-1)$,
  step~\ref{alg:pseudo:frob} costs $O(p i \Lift(i))$ by
  Theorem~\ref{th:b-ifrob} and step~\ref{alg:pseudo:sum} costs
  $O(p^{i+1}d)$. This gives 
  \begin{equation}
    \label{eq:102}
    \Ptr(i,j) = \Ptr(i,j-1) +O(p i \Lift(i))
    \text{,}
  \end{equation}
  and thus 
  \begin{equation}
    \label{eq:103}
    \Ptr(i,j) \in O(pij{\sf L}(i) + \Ptr(i,0))
    \text{.}
  \end{equation}
\end{proof}

The cost is thus $O(p^{i+2}d+p^i\ModComp(d))$, up to logarithmic
factors, for an input and output size of $p^id$: this time, due to
modular compositions in $\U_0$, the cost is not linear in $d$.

Finally, let us discuss precomputations. On input $v$, $i$, $d$, the
algorithm \alg{LittlePseudotrace} makes less than $2\log d$ calls to
\alg{IterFrobenius($x$,$i$,$n$)} for some value $x\in\U_i$ and for
$n\in N$ where the set $N$ only depends on $d$. When we construct
$\U_{i+1}$, we compute (only) all $\beta_{i,n}=\PTr_{n}(\gamma_i)
\wrt\U_i$, for increasing $n\in N$, using the \alg{LittlePseudotrace}
algorithm. The inner calls to \alg{IterFrobenius} only use
pseudotraces that are already known. Besides, a single call to
\alg{LittlePseudotrace}$(\gamma_i,i,d)$ actually computes {\em all}
$\PTr_{n}(\gamma_i)$ in time 
\begin{equation}
  \label{eq:104}
  O(p^i\ModComp(d)\log^2 d + i
  \Lift(i)\log d)
  \text{.}
\end{equation}
Same goes for the precomputation of all
$\beta_{i,p^jd}=\PTr_{p^jd}(\gamma_i) \wrt\U_i$, for $j\le i$, using
the \alg{Pseudotrace} algorithm: this costs $\Ptr(i)$. Observe that in
total we only store $O(k^2 + k\log d)$ elements of the tower, thus the
space requirements are quasi-linear.

\begin{remark}
  A dynamic programming version of~\alg{LittlePseudotrace} as
  in Section~\ref{sec:modular-composition} would only precompute
  $\beta_{i,2^e}$ for $2^e<d$, thus reducing the storage from $2\log
  d$ to $\lfloor\log d\rfloor$ elements. This would also allow to
  compute $\PTr_n$ for any $n<d$ without needing any further
  precomputation. Using this algorithm and a decomposition of $n>d$ as
  $n=r+\sum_jc_jp^jd$ with $r<d$ and $c_j<p$, one could also compute
  $T_{n}$ and $x^{p^n}$ for any $n$ at essentially the same cost. We
  omit these improvements since they are not essential to the next
  section.
\end{remark}

% Local Variables:
% mode:flyspell
% ispell-local-dictionary:"american"
% TeX-master: "../these"
% mode: TeX-PDF
% mode: reftex
% End:
%
% LocalWords:  Schreier Artin pseudotrace frobenius bivariate memoization
% LocalWords:  precomputed precomputation precompute precomputations Couveignes
% LocalWords:  univariate pseudotraces

\section{Arbitrary towers}
\label{sec:couveignes-algorithm}

Finally, we bring our previous algorithms to an arbitrary tower, using
Couveignes' isomorphism algorithm~\cite{Couveignes00}. As in the
previous section, we adapt this algorithm to our context, by adding
suitable push-down and lift-up operations.

Let $Q_0$ be irreducible of degree $d$ in $\F_p[X_0]$, such that
$\Tr_{\U_0/\F_p}(x_0)\ne0$, with as before
$\U_0=\F_p[X_0]/Q_0$. We let $(G_i)_{0 \le i < k}$ and
$(\U_0,\ldots,\U_k)$ be as in Section~\ref{sec:fast-tower}.

We also consider another sequence $(G'_i)_{0 \le i < k}$, that defines
another tower $(\U'_0,\ldots,\U'_k)$.  Since $(\U'_0,\ldots,\U'_k)$ is
not necessarily primitive, we fall back to the multivariate basis of
Subsection~\ref{ssec:rep}: we write elements of $\U'_i$ on the basis
$\bB'_i=\{{x'_0}^{e_0} \cdots {x'_i}^{e_i}\}$, with $x_0=x'_0$, $0 \le
e_0 < d$ and $0\le e_j < p$ for $1 \le j \le i$.
%% We assume that all
%% polynomials $G'_i$ satisfy $\deg(G'_i,X_0)< d$ and $\deg(G'_i,X_j)< p$
%% for $j \le i$, so that writing $\gamma'_i$ on the basis $\bB'_i$
%% requires no operation.

To compute in $\U'_i$, we will use an isomorphism $\U'_i \to \U_i$.
Such an isomorphism is determined by the images
$\bs_i=(s_0,\dots,s_i)$ of $(x'_0,\dots,x'_i)$, with $s_i \wrt \U_i$
(we always take $s_0=x_0$). This isomorphism, denoted by
$\sigma_{\bs_i}$, takes as input $v$ written on the basis $\bB'_i$ and
outputs $\sigma_{\bs_i}(v)\wrt \U_i$.

To analyze costs, we use the functions $\L$ and $\Ptr$ introduced in
the previous sections. We also let $2 \le \omega \le 3$ be a feasible
exponent for linear algebra over $\F_p$~\cite[Ch.~12]{vzGG}.
\begin{theorem}\label{theo:main}
  Given $Q_0$ and $(G'_i)_{0 \le i < k}$, one can find
  $\bs_k=(s_0,\dots,s_k)$ in time $O(d^\omega k + \Ptr(k) +
  \Mult(p^{k+1} d) \log(p))$. Once they are known, one can apply
  $\sigma_{\bs_k}$ and $\sigma_{\bs_k}^{-1}$ in time $O(k\, \L(k))$.
\end{theorem}
Thus, we can compute products, inverses, etc, in $\U'_k$ for
the cost of the corresponding operation in $\U_k$, plus $O(k\,
\L(k))$.

%%%%%%%%%%%%%%%%%%%%%%%%%%%%%%%%%%%%%%%%%%%%%%%%%%%%%%%%%%%%

\subsection{Solving Artin-Schreier equations} 

As a preliminary, given $\alpha\wrt \U_i$, we discuss how to
solve the Artin-Schreier equation $X^p-X=\alpha$ in $\U_i$. We assume
that $\Tr_{\U_i/\F_p}(\alpha)=0$, so this equation has solutions in
$\U_i$.

Because $X^p-X$ is $\F_p$-linear, the equation can be directly solved
by linear algebra, but this is too costly. In~\cite{Couveignes00},
Couveignes gives a solution adapted to our setting, that reduces the
problem to solving Artin-Schreier equations in $\U_0$. Given a solution
$\delta\in\U_i$ of the equation $X^p - X = \alpha$, he observes that 
any solution $\mu$ of
\begin{equation}
  \label{eq:approximateAS}
  X^{p^{p^{i-1}d}} - X = \eta, \quad\text{with}\quad \eta=\PTr_{p^{i-1}d}(\alpha).
\end{equation}
is of the form $\mu=\delta - \Delta$ with $\Delta\in\U_{i-1}$, hence
$\Delta$ is a \mbox{root of}
\begin{equation}
  \label{eq:approximant}
  X^p-X-\alpha+\mu^p-\mu.
\end{equation}
This equation has solutions in $\U_{i-1}$ by hypothesis and hence it
can be solved recursively. First, however, we tackle the problem of
finding a solution of~\eqref{eq:approximateAS}.

For this purpose, observe that the left hand side
of~\eqref{eq:approximateAS} is $\U_{i-1}$-linear and its matrix on the
basis $(1,\ldots,x_i^{p-1})$ is
\begin{equation*}
  \label{eq:approximate-matrix}
  \begin{bmatrix}
    0 & \binom{1}{0}\beta_{i-1,p^{i-1}d} & \hdots & \binom{p-1}{0}\beta_{i-1,p^{i-1}d}^{p-1} \\
      & \ddots          &        & \vdots               \\
      &                 & 0      &\binom{p-1}{p-2}\beta_{i-1,p^{i-1}d} \\
      &                 &        & 0
  \end{bmatrix}
\end{equation*}
Then, algorithm \alg{ApproximateAS} finds the required solution.



%% We use an algorithm called \alg{Summation}; given $E$ in
%% $\U_{i-1}[X]$ of degree less than $p-1$ and $\beta$ in $\F_p$, it
%% returns the unique $D$ in $\U_{i-1}[X]$ of degree less than $p$ such
%% that $D(X+\beta)-D(X)=E(X)$ and $D(0)=0.$ Rescaling by $\beta$, we
%% can assume that $\beta=1$. Then, the algorithm converts to the falling
%% factorial basis, sums and converts back~\cite[Ch.~21]{vzGG}, using
%% $O(p^2)$ operations $(+,\times)$ in $\U_{i-1}$ (better solutions are
%% known, but do not improve the overall runtime in Theorem~\ref{th:approximateAS}).


\begin{algorithm}
  \caption{ApproximateAS} 
  \begin{algorithmic}[1]
    \REQUIRE $\eta\wrt\U_i$ such that~\eqref{eq:approximateAS} has a solution.
    \ENSURE $\mu\wrt\U_i$ solution of~\eqref{eq:approximateAS}.
    \STATE let $\eta_0 + \eta_1 x_i + \dots + \eta_{p-2} x_i^{p-2}=\text{\alg{Push-down}}(\eta)$
    \FORALL {\label{alg:AAS:loop} $j\in[p-1,\ldots,1]$}
    \STATE let $\mu_j =
   \frac{1}{jT}\left(\eta_{j-1} -
  \sum_{h=j+1}^{p-1}\binom{h}{j-1}\beta_{i-1,p^{i-1}d}^{h-j+1}\mu_h\right)$
\ENDFOR
\STATE return $\text{\alg{Lift-up}}(\mu_1 x_i + \ldots + \mu_{p-1} x_i^{p-1})$
\end{algorithmic}
\end{algorithm}

%% \begin{itemize}
%%   {ApproximateAS} 
%%   {$\eta\wrt\U_i$ such that~\eqref{eq:approximateAS} has a solution.}
%%   {$\mu\wrt\U_i$ solution of~\eqref{eq:approximateAS}.}
%% \STATE let $\eta_0 + \eta_1 x_i + \dots + \eta_{p-2} x_i^{p-2}=\text{\alg{Push-down}}(\eta)$
%% \STATE \label{alg:AAS:loop}for $j\in[p-1,\ldots,1]$,\\ let $\mu_j =
%%    \frac{1}{jT}\left(\eta_{j-1} -
%%   \sum_{h=j+1}^{p-1}\binom{h}{j-1}\beta_{i-1,p^{i-1}d}^{h-j+1}\mu_h\right)$
%% \STATE return $\text{\alg{Lift-up}}(\mu_1 x_i + \ldots + \mu_{p-1} x_i^{p-1})$
%% \end{itemize}

\begin{theorem}
  \label{th:approximateAS}
  Algorithm \alg{ApproximateAS} is correct and takes time $O(\L(i))$.
\end{theorem}

\begin{proof} Correctness is clear from Gaussian elimination.  For the cost
analysis, remark that $\beta_{i-1,p^{i-1}d}$ has already been
precomputed to permit iterated Frobenius and pseudotrace
computations. Step~\ref{alg:AAS:loop} takes $O(p^2)$ additions and
scalar operations in $\U_{i-1}$; the overall cost is dominated by that
of the push-down and lift-up by assumptions on $\L$. \end{proof}

%% \begin{proof} Write $\mu=\Sigma_{j=0}^{p-1}\mu_j x_i^j$ and
%% $\eta=\Sigma_{j=0}^{p-1}\eta_j x_i^j$, with $\mu_j$ and $\eta_j \wrt
%% \U_{i-1}$. Since all $\mu_j$ are invariant by the $p^{i-1}d$-power of
%% the Frobenius, we get from Equation~\eqref{eq:frobeniussum}
%% \begin{equation*}
%%     \begin{array}{c}
%% {\mu}^{p^{p^{i-1}d}}=\sum_{j=0}^{p-1} \mu_j x_i^{jp^{p^{i-1}d}} = \sum_{j=0}^{p-1}  \mu_j(x_i +\beta_{i-1,p^{i-1}d}))^j,
%%     \end{array}
%% \end{equation*}
%% where $\beta_{i-1,p^{i-1}d}$ is the trace of $\gamma_{i-1}$ over $\F_p$.
%% Thus, $\mu$ is a solution of~\eqref{eq:approximateAS} if and only if
%% the polynomials $D=\sum_{j=0}^{p-1} \mu_j X^j$ and $E=\sum_{j=0}^{p-1}
%% \eta_j X^j$ satisfy $D(X+\beta_{i-1,p^{i-1}d}) - D(X) = E(X).$ Remark that
%% $\eta_{p-1}$ must be $0$ and that $\mu_0$ plays no role in the
%% solution, so we take $\mu_0=0$. For the cost analysis, remark that
%% $\beta_{i-1,p^{i-1}d}$ has already been precomputed to permit iterated
%% Frobenius and pseudotrace computations (see Section
%% \ref{sec:pseudotrace-frobenius}). Then, the overall cost is dominated
%% by that of the push-down and lift-up by assumptions on $\L$. \end{proof}

\smallskip

Writing the recursive algorithm is now straightforward. To solve
Artin-Schreier equations in $\U_0$, we use a naive algorithm based on
linear algebra, written $\alg{NaiveSolve}$.


\begin{algorithm}
  \caption{Artin-Schreier}
  \begin{algorithmic}[1]
    \REQUIRE $\alpha,i$ such that $\alpha\wrt\U_i$ and $\Tr_{\U_i/\F_p}(\alpha)=0$.
    \ENSURE $\delta\wrt\U_i$ such that $\delta^p-\delta=\alpha$.
    \STATE \label{alg:cou:base}if $i=0$, return $\alg{NaiveSolve}(X^p-X-\alpha)$
    \STATE \label{alg:cou:pseudo} let $\eta = \alg{Pseudotrace}(\alpha, i,i-1)$
    \STATE \label{alg:cou:push-beta} let $\mu= \alg{ApproximateAS}(\eta)$
    \STATE \label{alg:cou:push-alpha} let $\alpha_0=\text{\alg{Push-down}}(\alpha-\mu^p+\mu)$
    \STATE \label{alg:cou:rec} let $\Delta =\text{\alg{Artin-Schreier}}(\alpha_0,i-1)$
    \STATE \label{alg:cou:lift} return $\mu+\text{\alg{Lift-up}}(\Delta)$
  \end{algorithmic}
\end{algorithm}


\begin{theorem}\label{theo:AS}
  Algorithm \alg{Artin-Schreier} is correct and takes time $O(d^\omega
  + \Ptr(i))$.
\end{theorem}
\begin{proof} Correctness follows from the previous discussion.  For the
complexity, note ${\sf AS}(i)$ the cost for $\alpha\wrt\U_i$. The cost
${\sf AS}(0)$ of the naive algorithm is $O(\Mult(d)\log(p) +
d^\omega)$, where the first term is the cost of computing $x_0^p$ and
the second one the cost of linear algebra.

When $i\ge1$, step \ref{alg:cou:pseudo} has cost $\Ptr(i)$, steps
\ref{alg:cou:push-beta}, \ref{alg:cou:push-alpha} and
\ref{alg:cou:lift} all contribute $O(\L(i))$ and step
\ref{alg:cou:rec} contributes ${\sf AS}(i-1)$. The most important
contribution is at step \ref{alg:cou:pseudo}, hence ${\sf AS}(i) =
{\sf AS}(i-1) + O(\Ptr(i))$. The assumptions on~$\L$ imply that the
sum $\Ptr(1) + \cdots + \Ptr(i)$ is $O(\Ptr(i))$. \end{proof}

%%%%%%%%%%%%%%%%%%%%%%%%%%%%%%%%%%%%%%%%%%%%%%%%%%%%%%%%%%%%

\subsection{Applying the isomorphism}

We get back to the isomorphism question. We assume that
$\bs_i=(s_0,\dots,s_i)$ is known and we give the cost of applying
$\sigma_{\bs_i}$ and its inverse.  We first discuss the forward
direction.

As input, $v \in \U'_i$ is written on the multivariate basis $\bB'_i$
of $\U'_i$; the output is $t=\sigma_{\bs_i}(v) \wrt \U_i$. As before,
the algorithm is recursive: we write $v=\Sigma_{j <p}
v_j(x'_0,\dots,x'_{i-1}) {x'_i}^j$, whence
$$\begin{array}{c}\sigma_{\bs_i}(v)\ =\ \sum_{j
  <p} \sigma_{\bs_i}(v_j) s_i^j\ =\ \sum_{j
  <p} \sigma_{\bs_{i-1}}(v_j) s_i^j
\end{array}
;$$ the sum is computed by Horner's scheme.
To speed-up the computation, it is better to
perform the latter step in a bivariate basis, that is, through a
push-down and a lift-up.



Given $t \wrt \U_i$, to compute $v=\sigma_{\bs_i}^{-1}(t)$, we run the
previous algorithm backward. We first push-down $t$, obtaining $t=t_0
+ \cdots + t_{p-1}x_i^{p-1}$, with all $t_j \wrt \U_{i-1}$. Next, we
rewrite this as $t=t'_0+\cdots + t'_{p-1}s_i^{p-1}$, with all $t'_j
\wrt \U_{i-1}$, and it suffices to apply $\sigma_{\bs_i}^{-1}$ (or
equivalently $\sigma_{\bs_{i-1}}^{-1}$) to all $t'_i$. The non-trivial
part is the computation of the $t'_j$: this is done by applying the
algorithm \alg{FindParameterization} mentioned in
Subsection~\ref{ssec:duality}, in the extension $\U_i=
\U_{i-1}[X_i]/P_i$.



\begin{algorithm}
  \caption{ApplyIsomorphism} 
  \begin{algorithmic}[1]
    \REQUIRE $v,i$ with $v\in \U'_i$ written on the basis $\bB'_i$.
    \ENSURE $\sigma_{\bs_i}(v) \wrt \U_i$.
    \STATE if $i=0$ then return $v$
    \STATE write $v=\Sigma_{j <p} v_j(x'_0,\dots,x'_{i-1}) {x'_i}^j$
    \STATE let $s_{i,0}+\cdots+s_{i,p-1}x_i^{p-1}=\text{\alg{Push-down}}(s_i)$
    \STATE for $j \in [0,\dots,p-1]$ let $t_j=\alg{ApplyIsomorphism}(v_j,i-1)$
    \STATE let $t=0$
    \STATE  for $j \in [p-1,\dots,0]$ let $t=(s_{i,0}+\cdots+s_{i,p-1}x_i^{p-1})t+t_j$
    \STATE return $\text{\alg{Lift-up}}(t)$
  \end{algorithmic}
\end{algorithm}
\begin{algorithm}
  \caption{ApplyInverse} 
  \begin{algorithmic}[1]
    \REQUIRE $t,i$ with $t \wrt \U_i$.
    \ENSURE $\sigma_{\bs_i}^{-1}(t)\in \U'_i$ written on the basis $\bB'_i$.
    \STATE if $i=0$ then return $t$
    \STATE let $t_0 + \cdots + t_{p-1}x_i^{p-1} = \text{\alg{Push-down}}(t)$
    \STATE let $s_{i,0}+\cdots+s_{i,p-1}x_i^{p-1}=\text{\alg{Push-down}}(s_i)$
    \STATE let $t'_0 + \cdots + t'_{p-1}X^{p-1} = \alg{FindParameterization}
  (t_0 + \cdots + t_{p-1}x_i^{p-1}, s_{i,0}+\cdots+s_{i,p-1}x_i^{p-1})$
  \STATE return $\Sigma_{j < p} \alg{ApplyInverse}(t'_j, i-1) {x'_i}^j$
\end{algorithmic}
\end{algorithm}


\begin{proposition}\label{Prop:apply}
  Algorithms \alg{ApplyIsomorphism} and  \alg{Ap\-plyInverse} are
correct and both take time $O(i\L(i))$.
\end{proposition}
\begin{proof} In both cases, correctness is clear, since the algorithms
translate the former discussion. As to complexity, in both cases, we
do $p$ recursive calls, $O(1)$ push-downs and lift-ups, and a few
extra operations: for \alg{ApplyIsomorphism}, these are $p$
multiplications / additions in the bivariate basis ${\bf D}_i$ of
Section~\ref{sec:level-embedding}; for \alg{ApplyInverse}, this is
calling the algorithm \alg{FindParameterization} of
Subsection~\ref{ssec:duality}.  The costs are $O(p\Mult(p^id))$ and
$O(p^2\Mult(p^{i-1}d))$, which are in $O(\L(i))$ by assumption on
$\L$. We conclude as in Theorem~\ref{th:b-ifrob}.\end{proof}

%%%%%%%%%%%%%%%%%%%%%%%%%%%%%%%%%%%%%%%%%%%%%%%%%%%%%%%%%%%%

\subsection{Proof of Theorem~\ref{theo:main}}

\noindent Finally, assuming that only $(s_0,\dots,s_{i-1})$ are known,
we describe how to determine $s_i$. Several choices are possible: the
only constraint is that $s_i$ should be a root of
$X_i^p-X_i-\sigma_{\bs_i}(\gamma'_{i-1})=X_i^p-X_i-\sigma_{\bs_{i-1}}(\gamma'_{i-1})$ in $\U_i$. 

Using Proposition~\ref{Prop:apply}, we can compute
$\alpha=\sigma_{\bs_{i-1}}(\gamma'_{i-1}) \wrt\U_{i-1}$ in time
$O((i-1)\L(i-1)) \subset O(i\L(i))$.  Applying a lift-up to $\alpha$,
we are then in the conditions of Theorem~\ref{theo:AS}, so we can find
$s_i$ for an extra $O(d^\omega + \Ptr(i))$ operations.

We can then summarize the cost of all precomputations: to the cost of
determining $\bs_i$, we add the costs related to the tower
$(\U_0,\dots,\U_i)$, given in
Sections~\ref{sec:fast-tower},~\ref{sec:level-embedding}
and~\ref{sec:pseudotrace-frobenius}. After a few simplifications, we
obtain the upper bound $O( d^\omega + \Ptr(i) + \Mult(p^{i+1} d)
\log(p)).$ Summing over $i$ gives the first claim of the theorem. The
second is a restatement of Proposition~\ref{Prop:apply}.

%% ----AS-----   -----------------build------------------  ------level--------   -PT-
%% d^3 + PT(i) + M(p^i+2 d) log(p) + p^i+2 d log_p(p^i d) + M(p^id) \log(p^id) + PT(i)

% Local Variables:
% mode:flyspell
% ispell-local-dictionary:"american"
% End:
%
% LocalWords:  Schreier Artin pseudotrace frobenius bivariate memoization
% LocalWords:  precomputed precomputation precompute precomputations Couveignes

\section{Experimental results}
\label{sec:artin-benchmarks}

We discuss here the implementation of the algorithms of this Chapter
and some experimental results.

\paragraph{Implementation}
We packaged the algorithms of this paper in a \texttt{C++} library
called \texttt{FAAST} and made it available under the terms of the
\texttt{GNU GPL} software license from
\url{http://www.lix.polytechnique.fr/Labo/Luca.De-Feo/FAAST/}.

\texttt{FAAST} is implemented on top of the \texttt{NTL}
library~\cite{shoup2003ntl} which provides the basic univariate
polynomial arithmetic needed here. Our library handles three NTL
classes of finite fields: {\tt GF2} for $p=2$, {\tt zz\_p} for
word-size $p$ and {\tt ZZ\_p} for arbitrary $p$; this choice is made
by the user at compile-time through the use of \texttt{C++} templates
and the resulting code is thus quite efficient.  Optionally,
\texttt{NTL} can be combined with the \texttt{gf2x}
package~\cite{gf2x} for better performance in the $p=2$ case, as we
did in our experiments.

All the algorithms of
Sections~\ref{sec:fast-tower}--\ref{sec:pseudotrace-frobenius} are
faithfully implemented in \texttt{FAAST}. The algorithms
\alg{ApplyIsomorphism} and \alg{ApplyInverse} have slightly different
implementations \texttt{toUnivariate()} and \texttt{toBivariate()}
that allow more flexibility. Instead of being recursive algorithms
doing the change to and from the multivariate basis
$\basis{B}'_i=\{{x_0'}^{e_0}\cdots {x_i'}^{e_i}\}$, they only
implement the change to and from the bivariate basis
$\basis{D}'_i=\{{x_{i-1}}^{e_{i-1}}{x_i'}^{e_i}\}$ with $0\le
e_{i-1}<p^{i-1}d$ and $0\le e_i<p$. Equivalently, this amounts to
switch between the representations
\begin{equation}
  \wrt\U_i \quad\text{and}\quad
  \wrt\U_{i-1}[X_i']/(X_i'^p-X_i'-\gamma_{i-1}')
  \text{.}
\end{equation}
The same result as one call to \alg{ApplyIsomorphism} or
\alg{ApplyInverse} can be obtained by $i$ calls to
\texttt{toUnivariate()} and \texttt{toBivariate()}
respectively. However, in the case where several generic Artin-Schreier
towers, say $(\U_0',\ldots,\U_k')$ and $(\U_0'',\ldots,\U_k'')$, are
built using the algorithms of Section \ref{sec:couveignes-algorithm},
this allows to \emph{mix} the representations by letting the user
chose to switch to any of the bases $\{y_0^{e_0}\cdots y_i^{e_i}\}$
where $y_i$ is either $x_i'$ or $x_i''$. In other words this allows
the user to \emph{zig-zag} in the lattice of finite fields as in
Figure~\ref{fig:lattice}.

\begin{figure}
  \centering
  \begin{equation*}
    \xymatrix@C=1cm{
      & v\wrt\U_k \ar@{<--}@(r,u)[dr] \\
      \U_k'\ar@{-}[r]^{\sigma'} & \U_k\ar@{-}[d] & \U_k''\ar@{-}[l]_{\sigma''} \ar@{--}@(d,u)[dll]\\
      \U_{k-1}'\ar@{-}[r]^{\sigma'} \ar@{--}@(d,u)[drr] & \U_{k-1}\ar@{.}[d] & \U_{k-1}''\ar@{-}[l]_{\sigma''}\\
      \U_1'\ar@{-}[r]^{\sigma'} & \U_1\ar@{-}[d] & \U_1''\ar@{-}[l]_{\sigma''}\ar@{-->}[d]\\
      & \U_0 & *[r]{v\wrt\{{x_0}^{e_0}{x_1''}^{e_1}\cdots {x_{k-1}'}^{e_{k-1}}{x_k''}^{e_k}\}}
    }
  \end{equation*}
  \caption{An example of conversion from the univariate basis to a
    mixed multivariate basis.}
  \label{fig:lattice}
\end{figure}

Besides the algorithms presented in this paper, \texttt{FAAST} also
implements some algorithms described in
Chapter~\ref{cha:algor-small-char} for minimal polynomials, evaluation
and interpolation, as they are required for the isogeny computation
algorithm described there.

\paragraph{Experimental results.} We compare our timings with
those obtained in Mag\-ma~\cite{MAGMA} for similar questions.  All
results are obtained on an Intel Xeon E5430 (2.6GHz).


The experiments for the \texttt{FAAST} library were only made for the
classes \texttt{GF2} and \texttt{zz\_p}. The class \texttt{ZZ\_p} was
left out because all the primes that can be reasonably handled by our
library fit in one machine-word. In Magma, there exist several ways to
build field extensions:
\begin{description*}
\item [$\bullet$ {\tt quo<U|P>}] builds the quotient of the
  univariate polynomial ring $U$ by  $P \in U$
  (written magma(1) hereafter);
\item [$\bullet$ {\tt ext<k|P>}] builds the extension of the field $k$ by $P \in
  k[X]$ (written magma(2));
\item [$\bullet$ {\tt ext<k|p>}] builds an extension of degree $p$ of $k$
  (written magma(3)).
\end{description*}
We made experiments for each of these choices where this makes sense.

The parameters to our algorithms are $(p,d,k)$. Thus, our experiments
describe the following situations:

\begin{itemize}
\item {\em Increasing the height $k$.} Here we take $p=2$ and $d=1$ (that is,
  $\U_0=\F_2$); the $x$-coordinate gives the number of levels we
  construct and the $y$-coordinate gives timings in seconds, in {\em
    logarithmic} scale.

  This is done in Figure~\ref{fig:height}. We let the height of the
  tower increase and we give timings for (1) building the tower of
  Section~\ref{sec:fast-tower} and (2) computing an isomorphism with a
  random arbitrary tower as in Section~\ref{sec:couveignes-algorithm}.
  In the latter experiment, only the magma(2) approach was meaningful
  for Magma.
\item {\em Increasing the degree $d$ of $\U_0$.} Here we take $p=5$
  and we construct $2$ levels; the $x$-coordinate gives the degree $d
  = [\U_0:\F_p]$ and the $y$-coordinate gives timings in seconds.
  This is done in Figure~\ref{fig:p-d} (left).
\item {\em Increasing $p$.} Here we take $d=1$ (thus $\U_0=\F_p$) and
  we construct $2$ levels; the $x$-coordinate gives the characteristic
  $p$ and the $y$-coordinate gives timings in seconds.  This is done
  in Figure~\ref{fig:p-d} (right).
\end{itemize}



\begin{figure}
  \centering
%  \includegraphics[height=0.5\textwidth]{build1}
%  \includegraphics[height=0.5\textwidth]{iso1}
  
  \caption{Build time (left) and isomorphism time (right) with respect to tower height. Plot is in logarithmic scale.}
  \label{fig:height}
\end{figure}

\begin{figure}
  \centering
%  \includegraphics[height=0.5\textwidth]{build-d}
%  \includegraphics[height=0.5\textwidth]{build-p}
  
  \caption{Build times with respect to $d$ (left) and $p$ (right).}
  \label{fig:p-d}
\end{figure}

The timings of our code are significantly better for increasing height
or increasing $d$. Not surprisingly, for increasing $p$, the magma(1)
approach performs better than any other: the {\tt quo} operation
simply creates a residue class ring, regardless of the
(ir)reducibility of the modulus, so the timing for building two levels
barely depend on $p$. Yet, we notice that \texttt{FAAST} has
reasonable performances for characteristics up to about $p=50$.

In Tables~\ref{tab:arith-gf2} and~\ref{tab:arith-zzp} we provide some
comparative timings for the different arithmetic operations provided
by \texttt{FAAST}. The column ``Primitive'' gives the time taken to
build one level of the primitive tower (this includes the
precomputation of the data as described in
Subsection~\ref{sec:level-embedding:lift-up}); the other entries are
self-explanatory. Product and inversion are just wrappers around
\texttt{NTL} routines: in these operations we didn't observe any
overhead compared to the native \texttt{NTL} code. All the operations
stay within a factor of $30$ of the cost of multiplication, which is
satisfactory.

\begin{table}
  \centering
  \begin{tabular}{l | r | r | r | r | r | r | r}
    \small level & \small Primitive & \small Push-d. & \small Lift-up & \small Product & \small Inverse & \small apply $\sigma^{-1}$ & \small apply $\sigma$ \\
    \hline
     19 &  1.061 & 0.269 &  1.165 & 0.038 &  0.599 &  0.572 &  1.152\\
     20 &  2.381 & 0.538 &  2.554 & 0.076 &  1.430 &  1.146 &  2.333\\
     21 &  5.284 & 1.083 &  5.645 & 0.171 &  3.331 &  2.306 &  4.807\\
     22 & 11.747 & 2.202 & 12.595 & 0.430 &  7.730 &  4.811 & 10.051\\
     23 & 26.441 & 4.654 & 28.641 & 0.961 & 18.059 & 10.240 & 21.494\\
  \end{tabular}
  \caption{Some timings in seconds for arithmetics in a generic tower built over $\F_2$ using \texttt{GF2}.}
  \label{tab:arith-gf2}
\end{table}

\begin{table}
  \centering
  \begin{tabular}{l | r | r | r | r | r | r | r}
    \small level & \small Primitive & \small Push-d. & \small Lift-up & \small Product & \small Inverse & \small $\sigma^{-1}$ & \small  $\sigma$ \\
    \hline
    18 &   9.159 &  0.514 &   8.278 &  0.321 &   6.432 &  2.379 &   6.624\\
    19 &  21.695 &  1.130 &  20.388 &  1.083 &  14.929 &  6.289 &  18.202\\
    20 &  49.137 &  3.058 &  48.605 &  2.444 &  33.986 & 10.716 &  32.493\\
    21 & 122.252 &  7.476 & 123.369 &  5.307 &  92.827 & 26.437 &  76.780\\
    22 & 275.110 & 15.832 & 279.338 & 10.971 & 210.680 & 47.956 & 134.167\\
  \end{tabular}
  \caption{Some timings in seconds for arithmetics in a generic tower built over $\F_2$ using \texttt{zz\_p}.}
  \label{tab:arith-zzp}
\end{table}


Finally, we mention the cost of precomputation. The precomputation of
the images of $\sigma$ as explained in
Section~\ref{sec:couveignes-algorithm} is quite expensive; most of it
is spent computing pseudotraces. Indeed it took one week to precompute
the data in Figure~\ref{fig:height} (right), while all the other data
can be computed in a few hours. There is still space for some minor
improvement in \texttt{FAAST}, mainly tweaking recursion thresholds
and implementing better algorithms for small and moderate input
sizes. Yet we think that only a major algorithmic improvement could
consistently speed up this phase.



% Local Variables: 
% mode:flyspell
% ispell-local-dictionary:"american"
% TeX-master: "../these"
% mode: Tex-PDF
% mode: reftex
% End: 
%
% LocalWords:  univariate isogeny Couveignes isogenies Artin Schreier



% Local Variables:
% mode:flyspell
% ispell-local-dictionary:"american"
% mode:TeX-PDF
% mode:reftex
% TeX-master: "../these"
% End:

\part{Applications to isogeny computation and cryptology}
\label{part:appl-isog-comp}
\chapter{Elliptic curves and isogenies}
\label{cha:ellipt-curv-isog}
\section{Preliminaries on Isogenies}
\label{sec:preliminaries}

Let $E$ be an ordinary elliptic curve over the field $\F_q$. We
suppose it is given to us as the locus of zeroes of an affine
Weierstrass equation
\[y^2 + a_1xy + a_3y = x^3 + a_2x^2 + a_4x + a_6
\qquad a_1,\ldots,a_6\in\F_q\text{.}\]

\paragraph{Simplified forms} If $p>3$ it is well known that the curve
$E$ is isomorphic to a curve in the form
\begin{equation}
  \label{eq:weierstrass>3}
  y^2 = x^3 + ax + b
\end{equation}
and its $j$-invariant is $j(E) = \frac{1728(4a)^3}{16(4a^3 + 27b^2)}$.

When $p=3$, since $E$ is ordinary, it is isomorphic to a curve
\begin{equation}
  \label{eq:weierstrass=3}
  y^2 = x^3 + ax^2 + b
\end{equation}
and its $j$-invariant is $j(E) = -\frac{a^3}{b}$.

Finally, when $p=2$, since $E$ is ordinary, it is isomorphic to a curve
\begin{equation}
  \label{eq:weierstrass=2}
  y^2 + xy = x^3 + ax^2 + b
\end{equation}
and its $j$-invariant is $j(E) = \frac{1}{b}$.

These isomorphism are easy to compute and we will always assume that
the elliptic curves given to our algorithms are in such simplified
forms.

\paragraph{Isogenies}
Elliptic curves are endowed with the classic group structure through
the chord-tangent law. A group morphism having finite kernel is called
an \emph{isogeny}. Isogenies are regular maps, as such they can be
represented by rational functions. An isogeny is said to be
$\K$-rational if it is $\K$-rational as regular map; its degree is the
degree of the regular map.

One important property about isogenies is that they factor the
multiplication-by-$m$ map.

\begin{definition}[Dual isogeny]
  Let $\I : E \rightarrow E'$ be a degree $m$ isogeny. There exists an
  unique isogeny $\hat{\I} : E' \rightarrow E$, called the \emph{dual
    isogeny} such that
  \[\I\circ\hat{\I} = [m]_E \qquad\text{and}\qquad \hat{\I}\circ\I =
  [m]_{E'}\]
\end{definition}

As regular maps, isogenies can be separable, inseparable or purely
inseparable. In the case of finite fields, purely inseparable
isogenies are easily understood as powers of the frobenius map. Let
\[E^{(p)} : y^2 + a_1^pxy + a_3^py = x^3 + a_2^px^2 + a_4^px + a_6^p\]
then the map
\begin{align*}
  \frobisog : E &\rightarrow E^{(p)}\\
          (x,y) &\mapsto (x^p,y^p)
\end{align*}
is a degree $p$ purely inseparable isogeny. Any purely inseparable
isogeny is a composition of such frobenius isogenies.

Let $E$ and $E'$ be two elliptic curves defined over $\F_q$, by
finding an \emph{explicit isogeny} we mean to find an
($\F_q$-rational) rational function from $E(\clot{\F}_q)$ to
$E'(\clot{\F}_q)$ such that the map it defines is an isogeny.

\begin{figure}
  \centering
  \[\xymatrix{
    E \ar[r]^{[m]}\ar@/_1pc/[rrr]_{\I'} & E \ar[r]^\I & E' \ar[r]^{\frobisog^n} & E'^{(p^n)}\\
  }\]
  \label{fig:fact}
  \caption{Factorization of an isogeny. $\I'$ has kernel $E[m]\oplus\ker\I$.}
\end{figure}

Since an isogeny can be uniquely factored in the product of a
separable and a purely inseparable isogeny, we focus ourselves on the
problem of computing explicit separable isogenies. Furthermore one can
factor out multiplication-by-$m$ maps, thus reducing the problem to
compute explicit separable isogenies with cyclic kernel (see figure
\ref{fig:fact}).

In the rest of this paper, unless otherwise stated, by $\ell$-isogeny
we mean a separable isogeny with kernel isomorphic to $\Z/\ell\Z$.

\paragraph{Vélu formulae}
For any finite subgroup $G \subset E(\clot{\K})$, Vélu formulae
\cite{Vel71} give in a canonical way an elliptic curve $\bar{E}$ and
an explicit isogeny $\I:E\rightarrow \bar{E}$ such that
$\ker\I=G$. The isogeny is $\K$-rational if and only if the polynomial
vanishing on the abscissae of $G$ belongs to $\K[X]$.

In practice, if $E$ is defined over $\F_q$ and if
\[h(X) = \prod_{\substack{P\in G\\P\ne\0}}(X - x(P)) \in \F_q[X]\]
is known, Vélu formulae compute a rational function
\begin{equation}
  \label{eq:isog}
  \bar{\I}(x,y) = \left(\frac{g(x)}{h(x)}, \frac{k(x,y)}{l(x)}\right)  
\end{equation}
and a curve $\bar{E}$ such that $\bar{\I} : E\rightarrow\bar{E}$ is an
$\F_q$-rational isogeny of kernel $G$. A consequence of Vélu formulae
is
\begin{equation}
  \label{eq:velu-deg}
  \deg g = \deg h + 1 = \card{G}
  \text{.}
\end{equation}

Given two curves $E$ and $E'$, Vélu formulae reduce the problem of
finding an explicit isogeny between $E$ and $E'$ to that of finding
the kernel of an isogeny between them. Once the polynomial $h(X)$
vanishing on $\ker\I$ is found, the explicit isogeny is computed
composing Vélu formulae with the isomorphism between $\bar{E}$ and
$E'$ as in figure \ref{fig:velu}.

\begin{figure}
  \centering
  \[\xymatrix{
    E \ar[r]^{\bar{\I}} \ar[rd]^\I & \bar{E} \ar[d]^{\simeq}\\
    & E'
  }\]
  \caption{Using Vélu formulae to compute an explicit isogeny.}
  \label{fig:velu}
\end{figure}




% Local Variables:
% mode:flyspell
% ispell-local-dictionary:"british"
% mode:TeX-PDF
% TeX-master: "ec-isogeny"
% End:
%
% LocalWords:  Schreier Artin pseudotrace frobenius bivariate Joux Sirvent FFT
% LocalWords:  Couveignes isogenies Schoof isogeny cryptosystems Lercier
% LocalWords:  precomputation arithmetics polylogarithmic Karatsuba
% LocalWords:  endomorphisms


\chapter{Computing isogenies over finite fields}
\label{cha:algor-small-char}
%\section{Couveignes first algorithm}
%\section{Lercier}
%% these.tex
%% Copyright 2010 Luca De Feo
%% All rights reserved


Let $E$ and $E'$ be two elliptic curves defined over $\K$, by finding
an \emph{explicit isogeny} we mean to find a $\K$-rational function
from $E(\clot{\K})$ to $E'(\clot{\K})$ such that the map it defines is
an isogeny.

In this chapter we are interested in finding explicit isogenies of
ordinary elliptic curves over finite fields. In what follows $\F_q$
will be a finite field of characteristic $p$, and $d$ the positive
integer such that $q=p^d$.

\pdfmctwo{Given more details on "what's changed".}
Parts of this chapter and of the following have been published
in~\cite{df10}. However, the complexity analysis we give in
Proposition~\ref{th:lercier-sirvent} benefits from recent advances on
the computation of modular polynomials~\cite{sutherland10:modpol};
this in turn changes the relative ranking of the algorithms of this
chapter in terms of complexity. We also present a new algorithm in
Section~\ref{sec:bounded}.


\section{Overview}
\label{sec:history}

The problem of computing an explicit degree $\ell$ isogeny between two
given elliptic curves over $\F_q$ was originally motivated by point
counting methods based on Schoof's
algorithm~\cite{atkin88,elkies98,schoof95}. A review of the most
efficient algorithms to solve this problem is given
in~\cite{bostan+morain+salvy+schost08}, together with a new
quasi-optimal algorithm that we will review in Section~\ref{sec:bmss}.

All the algorithms of~\cite{bostan+morain+salvy+schost08} only work
when $\ell\ll p$. The case where $p$ is small compared to $\ell$ was
first treated by Couveignes in~\cite{couveignes94}, making use of
formal groups. The complexity of his method is $\tildO(\ell^3\log q)$ operations in
$\F_p$ assuming $p$ is constant, however it has an exponential
complexity in $\log p$.

Later, Lercier~\cite{lercier96} gave an algorithm specific to
characteristic $2$, that uses some linear properties of the problem to
build a linear system from whose solution the isogeny can be deduced.
Its complexity is conjectured to be $\tildO(\ell^3\log q)$ operations
in $\F_p$, but it has a much better constant factor
than~\cite{couveignes94}. At the moment we write, this is by many
orders of magnitude the fastest algorithm to solve practical instances
of the problem when $p=2$, thus being the \emph{de facto} standard for
cryptographic use.

Couveignes, again, proposed an algorithm in~\cite{couveignes96}
working for any $p$, based on the structure of the $p^k$ torsion of
ordinary elliptic curves. Using improvements
from~\cite{couveignes00,df+schost09,df10}, this algorithm has a
quadratic complexity in $\ell$. We review the original algorithm as
well as its improved variants in Sections~\ref{sec:C2}
to~\ref{sec:bounded}.

\pdfmcone{A little more crypto here.}
After the discovery of $p$-adic alternatives to Schoof's
algorithm~\cite{satoh00,fouquet+gaudry+harley00}, interest in computing
isogenies in small characteristic was lost for some time. However,
other cryptographic applications of isogenies soon appeared.  The
\index{GHS~attack}GHS attack uses Weil descent to reduce the
\index{discrete~logarihm~problem}discrete logarithm problem
(\index{DLP|see{discrete logarithm problem}}DLP) of an elliptic curve
over a binary field of composite degree to the DLP of an hyperelliptic
curve over a smaller field~\cite{gaudry+hess+smart02,GHS,hess03}. A
similar application is the reduction of the DLP of some genus $3$
hyperelliptic curves to the DLP of genus $3$ non-hyperelliptic
curves~\cite{smith09}. Isogeny graphs have been used to construct hash
functions~\cite{charles+lauter+goren09} and to compute Hilbert class
polynomials and modular
polynomials~\cite{sutherland10:hilbert,sutherland10:modpol}. New
cryptographic protocols based on isogenies have also been proposed:
Rostovtsef and Stolbunov~\cite{rostovtsev+stolbunov06} construct a
Diffie-Hellman key exchange based on a DLP-like problem in a cycle of
isogenous curves; Teske~\cite{mauer+menezes+teske01,teske06}
constructs a trapdoor cryptosystem by hiding a GHS-vulnerable curve
behind a random path of isogenies.

On the wave of the renewed interest for isogenies, two $p$-adic
algorithms were recently proposed by Joux and
Lercier~\cite{joux+lercier06} and Lercier and
Sirvent~\cite{lercier+sirvent08} to compute isogenies in arbitrary
characteristic. The former method has complexity $\tildO(\ell^2(1 +
\ell/p)\log q)$ operations in $\F_p$, which makes it well adapted to
the case where $p\sim\log q$.  The latter has complexity
$\tildO(\ell^2\log q)$ operations in $\F_p$, making it the best
algorithm to compute isogenies is small characteristics. We review the
second algorithm in Section~\ref{sec:lercier-sirvent}.


\section{Vélu formulas}
\label{sec:velu-formulas}


\begin{figure}
  \centering
  \[\xymatrix{
    E \ar[r]^{[m]}\ar@/_1pc/[rrr]_{\I'} & E \ar[r]^\I & E' \ar[r]^{\frobisog^n} & E'^{(p^n)}
  }\]
  \label{fig:fact}
  \caption{Factorization of an isogeny. $\I'$ has kernel $E[m]\oplus\ker\I$.}
\end{figure}

Since an isogeny can be uniquely factored in the product of a
separable and a purely inseparable isogeny, we focus on the problem of
computing explicit separable isogenies. Furthermore one can factor out
multiplication-by-$m$ maps, thus reducing the problem to compute
explicit separable isogenies with cyclic kernel (see figure
\ref{fig:fact}).

In the rest of this chapter, unless otherwise stated, by
$\ell$-isogeny we mean a separable isogeny with kernel isomorphic to
$\Z/\ell\Z$.


For any finite subgroup $G \subset E(\clot{\K})$, Vélu
formulas~\cite{velu71} give in a canonical way an elliptic curve
$\bar{E}$ and an explicit separable isogeny $\I:E\rightarrow \bar{E}$
such that $\ker\I=G$. The isogeny is $\K$-rational if and only if the
polynomial vanishing on the abscissas of $G$ belongs to $\K[X]$.

The isogeny computed by Vélu formulas is the map
\begin{multline}
  \label{eq:155}
  \I(P) = \left(x(P) + \sum_{Q\in G\diffset\{\0\}}x(P+Q) - x(Q),\right.\\
    \left.y(P) + \sum_{Q\in G\diffset\{\0\}}y(P+Q) - y(Q)\right)
  \text{.}
\end{multline}
Using the addition formulas it is straightforward to obtain the
coefficients of the curve $\bar{E}$ and the explicit isogeny.  For
simplicity, we do so only for the case $p\ge3$ and $E$ in the form
\begin{equation}
  \label{eq:160}
  E \;:\; y^2 =  x^3 + a_2x^2 + a_4x + a_6
\end{equation}
(note that this is always possible by
Proposition~\ref{th:simplified-weierstrass}). 

We set $G^\ast=G\diffset\{\0\}$. We denote by $f,f'$ the two
functions in $\K(E)$
\begin{equation}
  \label{eq:162}
  \begin{aligned}
    f(P) &= x(P)^3 + a_2x(P)^2 + a_4x(P) + a_6
    \text{,}\\
    f'(P) &= 3x(P)^2 + 2a_2x(P) + a_4
    \text{.}
  \end{aligned}
\end{equation}
From the \hyperref[eq:121]{addition formulas}, after some calculations
(see Appendix~\ref{cha:proof-velus-formulas} for an automatic proof of
this calculation), Eq.~\eqref{eq:155} is equivalent to
\begin{multline}
  \label{eq:161}
  \I(x,y) = \left(x + \sum_{Q\in G^\ast} \frac{f'(Q)}{x-x(Q)} + \frac{2f(Q)}{(x-x(Q))^2}\text{,}\right.\\
  \left.y + \sum_{Q\in G^\ast} -\frac{yf'(Q)}{(x-x(Q))^2} - \frac{4yf(Q)}{(x-x(Q))^3}\right)
  \text{.}
\end{multline}

Observe that if $Q\in G^\ast$ is a $2$-torsion point, then $f(Q)=0$;
while if $Q$ is not a $2$-torsion point, $x(Q)$ is counted twice in
the sum of the previous equation. Then, the denominator of $\I_x$ is
  \begin{equation}
    \label{eq:158}
    h(x) = \prod_{Q\in G^\ast}(x - x(Q))
    \text{.}
  \end{equation}
We set
\begin{equation}
  \label{eq:164}
  \begin{gathered}
    t = \sum_{Q\in G^\ast} f'(Q)\text{,}
    \qquad
    u = \sum_{Q\in G^\ast} 2f(Q)\text{,}
    \qquad
    w = u + \sum_{Q\in G^\ast} x(Q)f'(Q)\text{,}\\
    \frac{g(x)}{h(x)} = x + t\frac{h'(x)}{h(x)} - u\left(\frac{h'(x)}{h(x)}\right)'
    \text{,}
  \end{gathered}
\end{equation}
then Eq.~\eqref{eq:161} becomes
\begin{equation}
  \label{eq:159}
  \I(x,y) = \left(\frac{g(x)}{h(x)}, y\left(\frac{g(x)}{h(x)}\right)'\right)
  \text{,}
\end{equation}
and the isogenous curve has equation
\begin{equation}
  \label{eq:163}
  \bar{E}\;:\;y^2 = x^3 + a_2x^2 + (a_4-5t)x + a_6 - 4a_2t - 7w
  \text{.}
\end{equation}
Thus, from the knowledge of $h(x)$ one can deduce the isogeny and the
isogenous curve in $O(\Mult(\deg\I))$ operations in $\K$.

\begin{remark}
  Traditionally, Eqs.~\eqref{eq:164} and~\eqref{eq:159} are used to
  deduce the isogeny and the curve from $h(x)$ and its first three
  power sums.

  % Overfull in b5
  \pdfmctwo{The sentence "When the isogenous curve is of no interest"
    wasn't really useful. I stripped it.}  It is sometimes more
  convenient to use the reformulation given by Elkies~\cite{elkies98}
  \begin{equation}
    \label{eq:157}
    \frac{g(x)}{h(x)} = x + \sum_{Q\in G^\ast}x - x(Q) - \frac{f'(x)}{x-x(Q)} + \frac{2f(x)}{(x-x(Q))^2}
  \end{equation}
  (this equality is shown in Appendix~\ref{cha:proof-velus-formulas},
  too). This implies
  \begin{equation}
    \label{eq:165}
    \frac{g(x)}{h(x)} = \ell x - p_1 - f'(x)\frac{h'(x)}{h(x)} -
    2f(x)\left(\frac{h'(x)}{h(x)}\right)'
    \text{,}
  \end{equation}
  where $p_1$ is the first power sum of $h$.
\end{remark}

Given two curves $E$ and $E'$, Vélu formulas reduce the problem of
finding an explicit isogeny between $E$ and $E'$ to that of finding
the kernel of an isogeny between them. Once the polynomial $h(X)$
vanishing on $\ker\I$ is found, the explicit isogeny is computed
composing Vélu formulas with the isomorphism between $\bar{E}$ and
$E'$ as in figure \ref{fig:velu}.

\begin{figure}
  \centering
  \[\xymatrix{
    E \ar[r]^{\bar{\I}} \ar[rd]^\I & \bar{E} \ar[d]^{\simeq}\\
    & E'
  }\]
  \caption{Using Vélu formulas to compute an explicit isogeny.}
  \label{fig:velu}
\end{figure}




% Local Variables:
% mode:flyspell
% ispell-local-dictionary:"american"
% mode:TeX-PDF
% mode:reftex
% TeX-master: "../these"
% End:
%
% LocalWords:  Schreier Artin pseudotrace frobenius bivariate Joux Sirvent FFT
% LocalWords:  Couveignes isogenies Schoof isogeny cryptosystems Lercier
% LocalWords:  precomputation arithmetics polylogarithmic Karatsuba

%% these.tex
%% Copyright 2010 Luca De Feo
%% All rights reserved


\section{BMSS}
\label{sec:bmss}
In this section we present the BMSS
algorithm~\cite{bostan+morain+salvy+schost08} to compute isogenies of
degree $\ell\ne p$ in characteristic $0$ or $p\gg\ell$. It takes as
input the integer $\ell$ and two elliptic curves $E$ and $E'$ over a
finite field $\F_q$ defined by \emph{normalized models} (see
definitions \hyperref[def:canon-isog]{below}). It outputs the explicit
isogeny using $O(\Mult(\ell)\log\ell)$ operations in $\F_q$, or
$O(\Mult(\ell))$ in case the sum of the abscissas of the kernel of the
isogeny is known.

\pdfmcthree{Adapted to the case p=3.}  Because of the assumption on
the characteristic, we can assume $p\ne 2$ and the curves to be in
the form
\begin{equation}
  \label{eq:140}
  \begin{aligned}
    E \;&:\: y^2 = x^3 + a_2x^2 + a_4x + a_6\text{,}\\
    E'\;&:\; y^2 = x^3 + a_2'x^2 + a_4'x + a_6'\text{.}
  \end{aligned}
\end{equation}
Then, any isogeny $\I:E\ra E'$ of odd degree is of the form
\begin{equation}
  \label{eq:149}
  \I(x,y) = \left(\frac{g(x)}{h(x)},cy\left(\frac{g(x)}{h(x)}\right)'\right)
  \text{,}
\end{equation}
with $c\in\clot{\K}$, and $g,h$ monic polynomials in $\clot{\K}[X]$
(this is a consequence of \titleref{sec:velu-formulas}).

\begin{definition}[Normalized isogeny]
  \label{def:canon-isog}
  An explicit isogeny given by Eq.\ifafive\ \else~\fi\eqref{eq:149} is said to be
  \index{normalized~isogeny}\index{isogeny!normalized}\emph{normalized}
  if $c=1$. 

  Given two $\ell$-isogenous curves $E$ and $E'$, Weierstrass
  equations for them such that the explicit $\ell$-isogeny
  $\I:E\ra E'$ is normalized, are called
  \index{normalized~model}\emph{$\ell$-normalized models} for those
  elliptic curves.
\end{definition}

\pdfmctwo{Swapped sentences in this paragraph in order to stress that
  the differential equation satisfied by the isogeny comes from Vélu's
  formulas.} It is noteworthy that Vélu formulas output normalized
models and a normalized isogeny. Normalized models naturally arise in
point counting: in fact in the Schoof-Elkies-Atkin
algorithm~\cite{atkin88,elkies98,schoof95} one factors the modular
polynomial $\Modpol_\ell$ to obtain $j$-invariants of curves
$\ell$-isogenous to $E$. As a consequence of Vélu formulas, it is
possible to obtain normalized models for such curves from the
knowledge of the partial derivatives of $\Modpol_\ell$.  Details can
be found in~\cite{schoof95,morain95,elkies98,lercier-algorithmique}.

\pdfmcthree{Modified equation to work when p=3.}
Our goal is to compute the rational fraction $\frac{g(x)}{h(x)}$. From
the fact that $\I$ is normalized and from Eq.~\eqref{eq:149} we deduce
\ifafive
\begin{multline}
  \label{eq:166}
  (x^3 + a_2x^2 + a_4x + a_6){\left(\frac{g(x)}{h(x)}\right)'}^2 =\\
  \left(\frac{g(x)}{h(x)}\right)^3 + a_2'\left(\frac{g(x)}{h(x)}\right)^2 + a_4'\frac{g(x)}{h(x)} + a_6'
  \text{.}
\end{multline}
\else
\begin{equation}
  \label{eq:166}
  (x^3 + a_2x^2 + a_4x + a_6){\left(\frac{g(x)}{h(x)}\right)'}^2 =
  \left(\frac{g(x)}{h(x)}\right)^3 + a_2'\left(\frac{g(x)}{h(x)}\right)^2 + a_4'\frac{g(x)}{h(x)} + a_6'
  \text{.}
\end{equation}
\fi
The key idea is to find a power series solution to this differential
equation and then deduce the rational fraction.

\pdfmcthree{Modified equation to work when p=3.}  However, we do not
know the initial condition at $0$, and we cannot look for an expansion
at infinity either, because the degree of $g$ is greater than the
degree of $h$.  Instead we set
\begin{equation}
  \label{eq:167}
  S(x) = \sqrt{\frac{h(1/x^2)}{g(1/x^2)}}
  \quad\Leftrightarrow\quad
  \frac{g(x)}{h(x)} = \frac{1}{S(1/\sqrt{x})^2}
  \text{,}
\end{equation}
so that $S(x) = x + O(x^3)$ from the monicity of $g$ and $h$. Now
$S(x)$ satisfies the differential equation
\begin{equation}
  \label{eq:168}
  (a_6x^6 + a_4x^4 + a_2x^2 + 1){S'}^2 = 1 + a_2'S^2 + a_4'S^4 + a_6'S^6
  \text{,}
\end{equation}
hence we can use a Newton iteration to find a power series
solution. In~\cite[2.4]{bostan+morain+salvy+schost08}, a generic
iteration to solve Eq.~\eqref{eq:168} is used; here we present a more
efficient iteration due to Lercier and
Sirvent~\cite{lercier+sirvent08}.

\pdfmcthree{Modified equation to work when p=3.}
Let 
\begin{equation}
  \label{eq:169}
  G = \frac{1}{1 + a_2x^2 + a_4x^4 + a_6x^6}
  \;\text{,}\qquad
  H = 1 + a_2't^2 + a_4't^4 + a_6't^6
  \text{,}
\end{equation}
Lercier and Sirvent give an algorithm to find a solution in $\K[[x]]$
of any equation of the form
\begin{equation}
  \label{eq:170}
  {S'}^2 = (H\circ S)G
  \text{,}
\end{equation}
with $G\in\K[[x]]$ and $H\in\K[t]$.


\begin{algorithm}
  \caption{\label{alg:le-si-diff} Solve differential equation}
  \begin{algorithmic}[1]
    \REQUIRE $\mu>1$, $\alpha\in\K$, $\beta\in\K^\ast$, $H\in\K[t]$, $G\in\K[[x]]$.
    \ENSURE $S\in\K[[x]]$, solution to ${S'}^2=(H\circ S)G$ modulo $x^{2^\mu}$.
    \STATE let $U \la 1/\beta +O(x)$, $J \la 1/\sqrt{H(\alpha)} + O(x)$, $V \la \sqrt{H(\alpha)} + O(x)$;
    \STATE let $S \la \alpha + \beta x +  \frac{G'(0)H(\alpha) + G(0)H'(\alpha)\beta}{4\beta}x^2+O(x^3)$;
    \FORALL {$d\in\{2, 2^2, \ldots, 2^{\mu-1}\}$}
    \STATE $U \la U(2 - S' U) \mod x^d$;
    \STATE $V \la (V +  (H\circ S) J (2 - VJ))/2 \mod x^d$;
    \STATE $J \la J(2-VJ)) \mod x^d$;
    \STATE \label{alg:le-si-diff:int}$S \la S + V\displaystyle\int\left((H\circ S)G - {S'}^2\right)UJ/2 \mod x^{2d+1}$;
    \ENDFOR
    \STATE output $S$.
  \end{algorithmic}
\end{algorithm}

\begin{theorem}
  Let $\K$ be a field of characteristic $0$ or $p>2^\mu$. Let
  $\alpha,\beta,H,G$ be the inputs to algorithm~\ref{alg:le-si-diff}
  such that $G(0)H(\alpha)=\beta^2$. Then
  Algorithm~\ref{alg:le-si-diff} computes a solution to
  \begin{equation}
    \label{eq:171}
    {S'}^2 = (H\circ S)G
    \text{,}\quad
    S(0) = \alpha
    \text{,}\quad
    S'(0) = \beta
  \end{equation}
  modulo $x^{2^\mu}$ using $O(\Mult(2^{\mu}))$ operations in $\K$.
\end{theorem}
\begin{proof}
  The complete proof is quite long and can be found
  in~\cite{lercier+sirvent08}; here we just give a sketch of it.

  Let $t$ be a solution to Eq.~\eqref{eq:171} modulo $x^{d+1}$ and let
  $h$ be such that 
  \begin{equation}
    \label{eq:175}
    S = t + h \mod x^{2d+1}
    \text{,}
  \end{equation}
  so that $x^{d+1}$ divides $h$.  Then $x^{2d}$ divides ${h'}^2$ and,
  by Eq.~\eqref{eq:171}
  \begin{equation}
    \label{eq:176}
    2t'h' + {t'}^2 = G(x)H(t+h) \mod x^{2d}
    \text{.}
  \end{equation}
  Using the Taylor expansion of $H$ at $t$, we get the linearized
  differential equation
  \begin{equation}
    \label{eq:177}
    2t'h' + {t'}^2 = G(x)H(t) + G(x)H'(t)h
    \mod x^{2d}
  \end{equation}
  with initial condition $h(0)=0$. By Eq.~\eqref{eq:209}, this
  equation has solution
  \begin{equation}
    \label{eq:178}
    h = \frac{1}{J} \int \frac{(G(x)H(t) - {t'}^2)J}{2t'}\diff x
    \text{,}
  \end{equation}
  where $J$ is 
  \begin{equation}
    \label{eq:179}
    J=\exp\left(-\int\frac{G(x)H'(t)}{2t'}\diff x\right)
    \text{.}
  \end{equation}

  The key observation is that, in order to compute the above solution
  to precision $x^{2d+1}$, $J$ must only be known to precision
  $x^d$. But $t$ is a solution of~\eqref{eq:171} modulo $x^{d+1}$, thus 
  \begin{equation}
    \label{eq:172}
    \frac{G(x)H'(t)}{2t'} = \frac{H'(t)t'}{2H(t)} \mod x^d
    \text{,}
  \end{equation}
  hence
  \begin{equation}
    \label{eq:173}
    J = \exp\left(-\frac{1}{2}\log H(t)\right) = \frac{1}{\sqrt{H(t)}}
    \text{.}
  \end{equation}

  Then, at each iteration, the algorithm computes the quantities
  \begin{equation}
    \label{eq:174}
    S,\quad U = 1/S',\quad V = \sqrt{H\circ S},\quad J = 1/V\text{,}
  \end{equation}
  doubling the precision at each iteration. Since the only operations
  are integrals and multiplications of power series, the $i$-th
  iteration costs $O(\Mult(2^i))$ operations in $\K$, thus the last
  iteration dominates the complexity.
\end{proof}


Then, the algorithm to compute the isogeny goes as follows.  The power
series expansion of $S$ is computed to precision $4\ell$, then we set
\begin{equation}
  \label{eq:180}
  S(x) = xT(x^2)
  \text{,}\quad
  R(x) = \frac{1}{T(x)^2}
  \text{,}\quad\text{so that}\quad
  \frac{g(x)}{h(x)} = xR(1/x)
  \text{.}
\end{equation}
Finally, the rational fraction is recovered by rational fraction
reconstruction (see Section~\ref{sec:eucl-algor-rati}); the overall
complexity is dominated by this last step.

\pdfmcthree{Modified algorithm to work when p=3.}
\begin{algorithm}
  \caption{\alg{BMSS}}
  \label{alg:bmss}
  \begin{algorithmic}[1]
    \REQUIRE $\ell>1$, $\ell$-normalized models of $E$ and $E'$ .
    \ENSURE An isogeny $\I:E\ra E'$ of degree $\ell$.
    \STATE Compute $G(x) = 1/(1 + a_2x^2 + a_4x^4 + bx^6) \mod x^{4\ell-1}$;
    \STATE find $S(x)\bmod x^{4\ell-1}$ using Algorithm~\ref{alg:le-si-diff};
    \STATE let $T(x) = \sum_{i=0}^{2\ell-1}s_{2i+1}x^i$;
    \STATE compute $R(x) = 1/T(x)^2 \mod x^{2\ell-1}$;
    \STATE compute $\frac{g(x)}{h(x)}$ by \hyperref[sec:eucl-algor-rati]{rational fraction reconstruction}.
  \end{algorithmic}
\end{algorithm}

\begin{remark}
  \label{rk:bmss}
  Alternatively, if the sum of the abscissas of the kernel
  \begin{equation}
    \label{eq:182}
    p_1 = \sum_{Q\in G^\ast}x(Q)
  \end{equation}
  is known, we can avoid the rational fraction reconstruction.

  The idea is to recover the Newton sums $p_0,\ldots,p_{\ell-1}$ of
  $h$ from $\frac{g(x)}{h(x)}$. From Eq.~\eqref{eq:165} we deduce
  \begin{equation}
    \label{eq:181}
    \begin{aligned}
      &\frac{g(x)}{h(x)} = x + \sum_{i\ge1}\frac{h_i}{x^i},
      \quad\text{where, for any $i\ge1$}\\
      &h_i = (2i+1)p_{i+1} + 2ia_2p_i + (2i-1)a_4p_{i-1} + (2i-2)a_6p_{i-2}
      \text{.}
    \end{aligned}
  \end{equation}
  Thus, knowing $p_0=\ell-1$ and $p_1$ is enough to compute all the
  Newton sums up to $p_{\ell-1}$ using $O(\ell)$ operations (observe,
  in fact, that the equation for $h_1$ has only three non-zero
  summands).

  From the power sums, we can recover $h(x)$ using
  Remark~\ref{rk:newton-sums} in $O(\Mult(\ell))$ operations. Then,
  $g(x)$ is obtained simply multiplying $\frac{g(x)}{h(x)}$ by $h(x)$,
  again in $\Mult(\ell)$ operations.

  Using this approach, we gain a logarithmic factor compared to the
  \hyperref[sec:eucl-algor-rati]{rational fraction reconstruction};
  and the number of coefficients of $S(x)$ to compute goes down to
  $2\ell$. This is similar to the trade-off we had in
  Remark~\ref{rk:shoups-algorithm-1}.

  \pdfmctwo{p1 is not really for free, as I said before. It is just
    absorbed in the rest of the computation.}  The knowledge of $p_1$
  (i.e.\ the coefficient of $x^{\ell-2}$ in $h$) may seem a rather
  bizarre requirement; however, in the Schoof-Elkies-Atkin algorithm
  this information is obtained, together with the normalized model for
  $E'$, from the derivatives of the modular polynomial
  (see~\cite{elkies98,morain95}), and this is why this algorithm has
  been developed.
\end{remark}



\section{Lercier-Sirvent}
\label{sec:lercier-sirvent}
The integral at step~\ref{alg:le-si-diff:int} requires divisions by
all the integers in the interval $[1,\ldots,2^\mu]$, thus, when
$2^{\lceil\log_2(4\ell-1)\rceil}>p$, \titleref{alg:bmss}
encounters a division by $0$. A natural idea is to work in
characteristic $0$ by lifting the curves in the $p$-adics. However,
lifting the Weierstrass models of $E$ and $E'$, there is no guarantee
of obtaining a pair of $\ell$-normalized models, thus
\titleref{alg:bmss} cannot apply.

To circumvent this problem, Lercier and
Sirvent~\cite{lercier+sirvent08} use Elkies' formulas to obtain
normalized models in the $p$-adic, and then apply
\titleref{alg:bmss}. The algorithm is summarized below; it
requires $p\ge5$ and it makes computations in an unramified extension
of degree $d$ of $\Q_p$, denoted by $\Q_q$.

\begin{algorithm*}
  \caption{\alg{Lercier-Sirvent}}
  \label{alg:le-si}
  \begin{algorithmic}[1]
    \REQUIRE $\ell>1$, $E,E'$ $\ell$-isogenous defined over $\F_q$.
    \ENSURE An isogeny $\I:E\ra E'$ of degree $\ell$.

    \STATE \label{alg:le-si:lift1}Take any lift
    $\bar{E}\;:\;y^2=x^3+\bar{a}x+\bar{b}$ of $E$ in $\Q_q$;
    
    \STATE \label{alg:le-si:modpol}Compute a root $\bar{j}'$ of
    $\Modpol_\ell(X,j_{\bar{E}})$ in $\Q_q$ by lifting the solution
    $j_{E'}$;
    
    \STATE \label{alg:le-si:elkies} Compute an $\ell$-normalized model
    $\bar{E}'':y^2=x^3+\bar{a}'x+\bar{b}'$ for $\bar{j}'$;
    
    \STATE \label{alg:le-si:bmss}Apply \titleref{alg:bmss} to $\bar{E}$ and
    $\bar{E}''$ to obtain $\bar{\I}:\bar{E}\ra\bar{E}''$;
    
    \STATE \label{alg:le-si:reduce}Reduce $\bar{E}''$ and $\bar{\I}$
    to $E''$ and $\I$ modulo $p$;
    
    \STATE \label{alg:le-si:isom}Apply an isomorphism $E'\isom E''$ to
    recover $\I:E\ra E'$.
  \end{algorithmic}
\end{algorithm*}

\pdfmcthree{A little detail on the model returned by Elkies' formulas}
Step~\ref{alg:le-si:elkies} uses Elkies' formulas~\cite{elkies98} to
find the $\ell$-normalized model of $\bar{j}'$; these formulas allow
to compute a normalized model of the form $y^2=x^3+ax+b$ from the
knowledge of $\partial\Modpol_\ell/\partial X$ and
$\partial\Modpol_\ell/\partial Y$, and the sum of the abscissas of the
kernel from the knowledge of $\partial^2\Modpol_\ell/\partial X^2$,
$\partial^2\Modpol_\ell/\partial X\partial Y$ and
$\partial^2\Modpol_\ell/\partial Y^2$, using $O(\ell^2)$ operations in
the base field ($\Q_q$, in this case). Analogous formulas exist for
other types of modular polynomials, we address the interested reader
to~\cite{schoof95,morain95,elkies98,lercier-algorithmique}. Notice
that this step fails when $(j_E,j_{E'})$ is a singular point of the
curve $X_0(\ell)$; this condition is very rare for ordinary curves of
large discriminant, as pointed out in~\cite[$\S7$]{schoof95}.

Computations in $\Q_q$ must be approximated to a certain
precision. Lercier and Sirvent show the following fundamental
property.

\begin{proposition}
  \label{th:ls-diffeq}
  If $p\ge5$, on inputs $\ell$, $E$, $E'$, the previous algorithm
  computes the correct answer using at most
  $\left\lceil\log^2\ell/\log p\right\rceil$ $p$-adic digits.
\end{proposition}

Building on this, we now analyze the complexity of the algorithm, for
simplicity we count operations in $\F_q$ rather than in $\F_p$.

\begin{proposition}
  \label{th:lercier-sirvent}
  Algorithm \titleref{alg:le-si} computes an $\ell$-degree isogeny in
  $\tildO(\ell^2)$ operations in $\F_q$, plus $\tildO(\ell^3)$ binary
  operations.
\end{proposition}
\begin{proof}
  We do not take into account the complexity of building the field
  $\Q_q$. Lifting $E$ in $\Q_q$ can be done for free by taking a
  trivial lift. Since the computations in $\Q_q$ are truncated to a
  fixed precision, we only need the coefficients of $\Modpol_\ell\bmod
  p^{\left\lceil\log^2\ell/\log p\right\rceil}$; this can be computed
  using $O(\ell^3\log^3\ell\loglog\ell)$ binary operations and
  $O(\ell^2\log^2\ell)$ bits of space by
  \cite[Algorithm~6.1]{sutherland10:modpol}. The output is a
  polynomial in $\Z_p[X,Y]$ having $O(\ell^2)$ coefficients, this
  requires an equivalent storage of $\tildO(\ell^2)$ elements of
  $\F_p$.

  Then, step~\ref{alg:le-si:modpol} can be done in $\tildO(\ell)$
  operations using Hensel lifting. Steps~\ref{alg:le-si:elkies}
  and~\ref{alg:le-si:bmss} take $O(\ell^2)$ and
  $O(\Mult(\ell)\log\ell)$ operations in $\Q_q$ respectively (slightly
  less if the sum of the abscissas of the kernel of the isogeny is
  computed by Elkies' formulas), this is equivalent to
  $\tildO(\ell^2)$ operations in $\F_q$. The rest of the computation
  is negligible. Thus, the dominating step is~\ref{alg:le-si:elkies}.
\end{proof}

\begin{remark}
  \pdfmcone{More details on how we changed the complexity analysis.}
  Our presentation of the algorithm slightly deviates from the
  original paper~\cite{lercier+sirvent08}. Since we want to compare it
  to Couveignes' algorithm, we assume that the elliptic curve $E'$
  isogenous to $E$ is provided as an input, while
  in~\cite{lercier+sirvent08} it is assumed that only $E$ is known.

  The consequence is that in the original version, before
  step~\ref{alg:le-si:modpol} one has to factor the univariate
  polynomial $\Modpol(X,j_E)$ in $\F_q$ to find an isogenous
  $j$-invariant. Lercier and Sirvent,
  citing~\cite{lidl+niederreiter:2}, estimate this cost to be
  $O(\Mult(\ell)\log q + \Mult(\ell)\log\ell)$ operations in
  $\F_q$. This contribution must be added to the complexity announced
  in Proposition~\ref{th:lercier-sirvent} if one wants to work in the
  original setting.

  Another difference is that we rely on an algorithm to compute the
  modular polynomial $\Modpol_\ell$ in the ring $\Z/m\Z[X,Y]$,
  recently appeared in~\cite{sutherland10:modpol}. This algorithm does
  not improve the binary complexity, however it permits to compute
  $\Modpol_\ell$ in $\Z_p[X,Y]$ truncated to the required precision
  using only $O(\ell^2\log^2\ell)$ bits of memory, instead of
  $\tildO(\ell^3)$, thus improving the algebraic complexity.
\end{remark}


\begin{nota}
  \pdfmcthree{Changed this note again, moved to the bottom of the
    section.}  In the cases $p=2,3$, Elkies' formulas yield a curve
  over $\Q_q$ that reduces badly in $\F_q$. As a consequence, each
  iteration of algorithm~\ref{alg:le-si-diff} introduces some
  additional divisions by $p$, and Proposition~\ref{th:ls-diffeq}
  fails to hold. While it is still possible to apply
  \titleref{alg:le-si} in this case, its complexity gets much worse
  because of the higher $p$-adic precision needed.

  In~\cite{lercier+sirvent08}, Lercier and Sirvent say:
  \begin{quote}
    ``For $p = 2$ (or $p = 3$), Weierstrass models of the form $y^2 + xy
    = x^3 + a_2 x^2 + a_6$ (or $y^2 = x^3 + a_2 x^2 + a_6$) must be
    considered. This yields completely different equations\dots{} [The
    algorithm] can be easily extended to these fields but for the
    sake of simplicity we prefer to omit the details here.''
  \end{quote}
  
  It is true that in the case $p=3$ it is possible to obtain, via
  isomorphism, normalized models for $\bar{E}$ and $\bar{E}''$ of the
  form $y^2 = x^3 + a_2 x^2 + a_6$ that reduce well in $\F_q$. Hence,
  algorithm \ref{alg:le-si-diff} can still be applied to solve the
  differential equation, and the isogeny can be computed using the
  same $p$-adic precision as in Proposition~\ref{th:ls-diffeq}.

  On the other hand, when $p=2$, while it is still possible to obtain
  models of the form $y^2 + xy = x^3 + a_2 x^2 + a_6$ that reduce well
  in $\F_q$, isogenies in such models do not verify an equation as
  simple as Eq.~\eqref{eq:166}. We think that in this case the
  techniques known to solve differential equations are not enough to
  find a solution to this problem.
\end{nota}


% Local Variables:
% mode:flyspell
% ispell-local-dictionary:"american"
% mode:TeX-PDF
% mode:reftex
% TeX-master: "../these"
% End:
%

\section{The algorithm C2}
\label{sec:C2}

The algorithm we refer to as C2 was originally proposed in
\cite{Cou96}. It takes as input two elliptic curves $E, E'$ and an
integer $\ell$ prime to $p$ and it returns, if it exists, an
$\F_q$-rational isogeny of degree $\ell$ between $E$ and $E'$. It only
works in odd characteristic.

\subsection{The original algorithm}
Suppose there exists an $\F_q$-rational isogeny
$\I:E\rightarrow E'$ of degree $\ell$. Since $\ell$ is prime to $p$
one has $\I(E[p^k]) = E'[p^k]$ for any $k$.

Recall that $E[p^k]$ and $E'[p^k]$ are cyclic groups. C2 iteratively
computes generators $P_k,P_k'$ of $E[p^k]$ and $E'[p^k]$
respectively. Now C2 makes the guess $\I(P_k) = P_k'$; then, if $\I$
is given by rational fractions as in \eqref{eq:isog},
\begin{equation}
  \label{eq:C2:I}
  \frac{g\bigl(x([i]P_k)\bigr)}{h\bigl(x([i]P_k)\bigr)} = x([i]P_k')
  \quad\text{for $i\in\Z/p^k\Z$} 
\end{equation}
and by \eqref{eq:velu-deg} $\deg g = \deg h + 1 = \ell$.

Using \eqref{eq:C2:I} one can compute the rational fraction
$\frac{g(X)}{h(X)}$ through Cauchy interpolation over the points of
$E[p^k]$ for $k$ large enough. C2 takes $p^k > 4\ell - 2$,
interpolates the rational fraction and then checks that it corresponds
to the restriction of an isogeny to the $x$-axis. If this is the case,
the whole isogeny is computed through Vélu formulae and the algorithm
terminates. Otherwise the guess $\I(P_k) = P_k'$ was wrong, then C2
computes a new generator for $E'[p^k]$ and starts over again.

We now go through the details of the algorithm.

\paragraph{The $p$-torsion}
The computation of the $p$-torsion points follows from the work of
Gunji \cite{Gun76}. Here we suppose $p\ne2$.

\begin{definition}
  \label{def:hasse}
  Let $E$ have equation $y^2 = f(x)$. The \emph{Hasse invariant} of
  $E$, noted $H_E$, is the coefficient of $X^{p-1}$ in
  $f(X)^{\frac{p-1}{2}}$.
\end{definition}

Gunji shows the following proposition and gives formulae to compute
the $p$-torsion points.

\begin{proposition}
  \label{th:gunji}
  Let $c=\sqrt[p-1]{H_E}$; then, the $p$-torsion points of $E$ are
  defined in $\F_q[c]$ and their abscissae are defined in $\F_q[c^2]$.
\end{proposition}


\paragraph{The $p^k$-torsion}
$p^k$-torsion points are iteratively computed via $p$-descent. The
basic idea is to split the multiplication map as $[p] = \frobisog\circ
V$ and invert each of the components. The purely inseparable isogeny
$\frobisog$ is just a frobenius map and the separable isogeny $V$ can
be computed by Vélu formulae once the $p$-torsion points are
known. Although this is reasonably efficient, pulling $V$ back may
involve factoring polynomials of degree $p$ in some extension field.

A finer way to do the $p$-descent, as suggested in the original paper
\cite{Cou96}, is to use the work of Voloch \cite{Vol90}. Suppose
$p\ne2$, let $E$ and $\widetilde{E}$ have equations respectively
\begin{align*}
  y^2&=f(x)=x^3+a_2x^2+a_4x+a_6 \;\text{,}\\
  \tilde{y}^2&=\tilde{f}(\tilde{x}) = \tilde{x}^3 +
  \sqrt[p]{a_2}\tilde{x}^2 + \sqrt[p]{a_4}\tilde{x} + \sqrt[p]{a_6}
  \;\text{,}
\end{align*}
set
 \begin{equation}
  \label{eq:voloch:cover}
  \tilde{f}(X)^{\frac{p-1}{2}} = \alpha(X) + H_{\widetilde{E}}X^{p-1} + X^p\beta(X)
\end{equation}
with $\deg \alpha < p-1$ and $H_{\widetilde{E}}$ the Hasse invariant
of $\widetilde{E}$. Voloch shows the following proposition.

\begin{proposition}
  \label{th:voloch}
  Let $\tilde{c} = \sqrt[p-1]{H_{\widetilde{E}}}$, the cover of
  $\widetilde{E}$ defined by
  \begin{equation}
    \label{th:voloch:cover}
    C:\; \tilde{z}^p - \tilde{z} = \frac{\tilde{y}\beta(\tilde{x})}{\tilde{c}^p}
  \end{equation}
  is an étale cover of degree $p$ and is isomorphic to $E$ over
  $\F_q[\tilde{c}]$; the isomorphism is given by
  \begin{equation}
    \label{th:voloch:isom}
    \left\{
      \begin{aligned}
        (\tilde{x}, \tilde{y}) &= V(x, y)\\
        \tilde{z} &= -\frac{y}{\tilde{c}^p}\sum_{i=1}^{p-1}\frac{1}{x - x([i]P_1)}
      \end{aligned}
    \right.
  \end{equation}
  where $P_1$ is a primitive $p$-torsion point of $E$.
\end{proposition}

The descent is then performed as follows: starting from a point $P$ on
$E$, first pull it back along $\frobisog$, then take one of its
pre-images in $C$ by solving equation \eqref{th:voloch:cover}, finally
use equation \eqref{th:voloch:isom} to land on a point $P'$ in $E$.
The proposition guarantees that $[p]P' = P$. The descent is pictured
in figure \ref{fig:voloch}.

\begin{figure}
  \centering
  \[
  \xymatrix{\widetilde{E}\ar@/^/[r]^{\frobisog} & E\ar@/^/[l]^{V}}
  %%
  \qquad
  %%
  \xymatrix{
    \widetilde{E}\ar@/^/[r]^{\frobisog} & E\ar@/^/@{-->}[l]^{V}\ar[d]_{\simeq}\\
    & C\ar@/^/[ul]
  }
  \]
  
  \caption{Two ways of doing the $p$-descent: standard on the left and via a degree $p$ cover on the right}
  \label{fig:voloch}
\end{figure}


The reason why this is more efficient than a standard descent is the
shape of equation \eqref{th:voloch:cover}: it is an Artin-Schreier
equation and it can be solved by many techniques, the simplest being
linear algebra (as was suggested in \cite{Cou96}). Once a solution
$\tilde{z}$ to \eqref{th:voloch:cover} is known, solving in $x$ and $y$ the
bivariate polynomial system \eqref{th:voloch:isom} takes just a GCD
computation (explicit formulae were given by Lercier in
\cite[$\S$6.2]{Ler97}, we give some slightly improved ones in Section
\ref{sec:implementation}). Compare this with a generic factoring
algorithm needed by standard descent.

Solving Artin-Schreier equations is the most delicate task of the
descent and we will further discuss it.


\paragraph{Cauchy interpolation}
Interpolation reconstructs a polynomial from the values it takes on
some points; Cauchy interpolation reconstructs a rational
fraction. The Cauchy interpolation algorithm is divided in two phases:
first find the polynomial $P$ interpolating the evaluation points,
then use rational fraction reconstruction to find a rational fraction
congruent to $P$ modulo the polynomial vanishing on the points. The
first phase is carried out through any classical interpolation
algorithm, while the second is similar to an XGCD computation. See
\cite[$\S$5.8]{vzGG} for details.

Cauchy interpolation needs $n+2$ points to reconstruct a degree
$(k,n-k)$ rational fraction. This, together with \eqref{eq:velu-deg},
justifies the choice of $k$ such that $p^k > 4\ell - 2$. Some of our
variants of C2 will interpolate only on the primitive $p^k$-torsion
points, thus requiring the slightly larger bound $\euler(p^k) \ge
4\ell - 2$. This is not very important to our asymptotical analysis
since in both cases $p^k \in O(\ell)$.

\paragraph{Recognising the isogeny}
Once the rational fraction $\frac{g(X)}{h(X)}$ has been computed, one
has to verify that it is indeed an isogeny. The first test is to check
that the degrees of $g$ and $h$ match equation \eqref{eq:velu-deg}, if
they don't, the equation can be discarded right away and the algorithm
can go on with the next trial. Next, one can check that $h$ is indeed
the square of a polynomial (or, if $\ell$ is even, the product of one
factor of the $2$-division polynomial and a square polynomial). This
two tests are usually enough to detect an isogeny, but, should they
lie, one can still check that the resulting rational function is
indeed a group morphism by trying some random points on $E$.


\subsection{The case $p=2$}
\label{sec:p=2}
The algorithm as we have presented it only works when $p\ne2$, it is
however an easy matter to generalise it. The only phase that doesn't
work is the computation of the $p^k$-torsion points. For curves in the
form \eqref{eq:weierstrass=2} the only $2$-torsion point is
$(0,\sqrt{b})$.

Voloch formulae are hard to adapt, nevertheless a $2$-descent on the
Kummer surface of $E$ can easily be performed since the doubling
formula reads
\begin{equation}
  x([2]P) = \frac{b}{x(P)^2} + x(P)^2 =
  \frobisog\left(\frac{\sqrt{b} + x(P)^2}{x(P)} \right) = \frobisog\circ V
  \;\text{.}
\end{equation}
Given point $x_P$ on $K_E$, a pull-back along $\frobisog$ gives a
point $\tilde{x}_P$ on $K_{\widetilde{E}}$. Then pulling $V$ back
amounts to solve
\begin{equation}
  \label{eq:2-descent}
  x^2 + \tilde{x}_Px = \sqrt{b}
\end{equation}
and this can be turned in an Artin-Schreier equation through the
change of variables $x \rightarrow x'\tilde{x}_P$.

From the descent on the Kummer surfaces one could deduce a full
$2$-descent on the curves by solving a quadratic equation at each step
in order recover the $y$ coordinate, but this would be too
expensive. Fortunately, the $y$ coordinates are not needed by the
subsequent steps of the algorithm, thus one may simply ignore
them. Observe in fact that even if $K_E$ does not have a group law,
the restriction of scalar multiplication is well defined and can be
computed through Montgomery formulae \cite{Mon87}. This is enough to
compute all the abscissae of the points in $E[p^k]$ once a generator
is known.


\subsection{Complexity analysis}
\label{sec:C2:complexity}
Analysing the complexity of C2 is a delicate matter since the
algorithm relies on some black-box computer algebra algorithms in
order to deal with finite extensions of $\F_q$. The choice of the
actual algorithms may strongly influence the overall complexity of C2.
In this section we will only give some lower bounds on the complexity
of C2, since a much more accurate complexity analysis will be carried
out in Section \ref{sec:C2-AS}.

\paragraph{$p$-torsion}
Applying Gunji formulae first requires to find $c$ and $c'$, $p-1$-th
roots of $H_E$ and $H_{E'}$, and build the field extension $\F_q[c] =
\F_q[c']$. Independently of the actual algorithm used, observe that in
the worst case $\F_q[c]$ is a degree $p-1$ extension of $\F_q$, thus
simply representing one of its elements requires $\Theta(pd)$ elements
of $\F_p$.

Subsequently, the main cost in Gunji's formulae is the computation of
the determinant of a $\frac{p-1}{2}\times\frac{p-1}{2}$
quadri-diagonal matrix (see \cite{Gun76}). This takes $\Theta(p^2)$
operations in $\F_q[c]$ by Gauss elimination, that is no less than
$\Omega(p^3d)$ operations in $\F_p$.

\paragraph{$p^k$-torsion}
During the $p$-descent, factoring of equations \eqref{th:voloch:cover}
or \eqref{eq:2-descent} may introduce some field extensions over
$\F_q[c]$. Observe that an Artin-Schreier polynomial is either
irreducible or totally split, so at each step of the $p$-descent we
either stay in the same field or we take a degree $p$ extension. This
shows that in the worst case, we have to take an extension of degree
$p^{k-1}$ over $F_q[c]$. The following proposition, which is a
generalisation of \cite[Prop. 26]{Ler97}, states precisely how likely
this case is.

\begin{proposition}
  \label{th:tower}
  Let $E$ be an elliptic curve over $\F_q$, we note $\U_i$ the
  smallest field extension of $\F_q$ such that $E[p^i]\subset
  E(\U_i)$. For any $i\ge1$, either $[\U_{i+1}:\U_i] = p$ or
  $\U_{i+1}=\U_i=\cdots=\U_1$.
\end{proposition}
\begin{proof}
  Observe that the action of the Frobenius $\frobisog$ on $E[p]$ is
  just multiplication by the trace $t$, in fact the equation
  \[\frobisog^2 - [t \bmod p]\circ\frobisog + [q \bmod p] = 0\]
  has two solutions, namely $[t \bmod p]$ and $[0 \bmod p]$, but the
  second can be discarded since it would imply that $\frobisog$ has
  non-trivial kernel.  By lifting this solution, one sees that the
  action of $\frobisog$ on the Tate module $\mathcal{T}_p(E)$ is equal
  to multiplication by some $\tau\in\Z_p$.

  Note $G$ the absolute Galois group of $\F_q$, there is a well known
  action of $G$ on $\mathcal{T}_p(E)$. Since $G$ is generated by the
  Frobenius automorphism of $\F_q$, the restriction of this action to
  $E[p^k]$ is equal to the action (via multiplication) of the subgroup
  of $(\Z/p^k\Z)^\ast$ generated by $\tau_k = \tau \bmod p^k$. Hence
  $[\U_k:\F_q] = \ord(\tau_k)$.

  Then, for any $k>1$, \cite[Corollary 4]{Ler97} applied to
  $\tau_{k+1}=\tau\bmod p^{k+1}$ shows that
  $\ord(\tau_{k+1})=\ord(\tau_k)$ implies
  $\ord(\tau_k)=\ord(\tau_{k-1})$ and this concludes the proof.
\end{proof}

Thus for any elliptic curve there is an $i_0$ such that $[\U_i:\U_1] =
p^{i-i_0}$ for any $i \ge i_0$. This shows that the worst and the
average case coincide since for any fixed curve $[\U_k:\U_1] \in
\Theta(p^k)$ asymptotically. In this situation, one needs
$\Theta(p^kd)$ elements of $\F_p$ to store an element of $\U_k$.

Now the last iteration of the $p$-descent needs to solve an
Artin-Schreier equation in $\U_k$. To do this C2 precomputes the
matrix of the $\F_q$-linear application $(x^q-x):\U_k\rightarrow\U_k$
and its inverse, plus the matrix of the $\F_p$-linear application
$(x^p-x):\F_q\rightarrow\F_q$ and its inverse. The former is the most
expensive one and takes $\Theta(p^{\omega k})$ operations in $\F_q$,
that is $\Omega(p^{\omega k}d) = \Omega(\ell^\omega d)$ operations in
$\F_p$, plus a storage of $\Theta(\ell^2d)$ elements of
$\F_p$. Observe that this precomputation may be used to compute any
other isogeny with domain $E$.

After the precomputation has been done, C2 successively applies the
two inverse matrices; details can be found in
\cite[$\S$2.4]{Cou96}. This costs at least $\Omega(\ell^2d)$.


\paragraph{Interpolation}
The most expensive part of Cauchy interpolation is the polynomial
interpolation phase. In fact, simply representing a polynomial of
degree $p^k-1$ in $\U_k[X]$ takes $\Theta(p^{2k}d)$ elements, thus at
least $\Omega(\ell^2d)$ operations are needed to interpolate unless
special care is taken. This contribution due to arithmetics in $\U_k$
had been underestimated in the complexity analysis of \cite{Cou96},
which gave an estimate of $\Omega(\ell d\log\ell)$ operations for this
phase. We will give more details on interpolation in Section
\ref{sec:C2-AS-FI}.


\paragraph{Recognising the isogeny}
The cost of testing for squareness of the denominator and other tests
is negligible compared to the rest of the algorithm. Nevertheless it
is important to realize that on average half of the $\euler(p^k)$
mappings from $E[p^k]$ to $E'[p^k]$ must be tried before finding the
isogeny, for only one of these mappings corresponds to it. This
implies that the Cauchy interpolation step must be repeated an average
of $\Theta(p^k)$ times, thus contributing a $\Omega(\ell^3d)$ to the
total complexity.

Summing up all the contributions one ends up with the following lower
bound
\begin{equation}
  \label{eq:C2:complexity}
  \Omega(\ell^3d + p^3d)
\end{equation}
plus a precomputation step whose cost is negligible compared to this
one and a space requirement of $\Theta(\ell^2d)$ elements. In the next
sections we will see how to make all these costs drop.


% \subsection{The case $(p,\ell)\ne1$}
% \label{sec:C2:non-prime}
% If we are interested in finding a separable isogeny whose degree is
% not prime to $p$, the best way is to compute the curve $\widetilde{E}$
% such that $E = \widetilde{E}^{(p)}$, then compute an isogeny of degree
% $\ell/p$ between $\widetilde{E}$ and $E'$ and finally compose it with
% the separable $p$-isogeny $V$ from $E$ to $\widetilde{E}$.

% Observe however that C2 can be easily adapted to directly compute such
% an isogeny. In fact let $v=v_p(\ell)$, then $\I(E[p^k]) =
% E'[p^{k-v}]$. All one needs to do in this case is to modify the Cauchy
% interpolation so that it interpolates the rational function that sends
% a generator of $E[p^k]$ over a generator of $E'[p^{k-v}]$ and the
% other points accordingly. The maximum number of trials to do before
% finding the isogeny is $\euler(p^{k-v})$, thus the overall complexity
% is
% \begin{equation}
%   \label{eq:C2:complexity-non-prime}
%   \Omega\left(\frac{\ell^3}{p^v}d + p^3d\right)
%   \;\text{.}
% \end{equation}

% Although this method is less efficient than the first one, it will
% come handy in Section \ref{sec:bounded}.



% Local Variables:
% mode:flyspell
% ispell-local-dictionary:"british"
% mode:TeX-PDF
% TeX-master: "ec-isogeny"
% End:
%
% LocalWords:  Schreier Artin pseudotrace frobenius bivariate Joux Sirvent FFT
% LocalWords:  Couveignes isogenies Schoof isogeny cryptosystems Lercier
% LocalWords:  precomputation arithmetics polylogarithmic Karatsuba precomputes
% LocalWords:  endomorphisms asymptotical

%% these.tex
%% Copyright 2010 Luca De Feo
%% All rights reserved


\section{The algorithm \alg{C2-AS}}
\label{sec:C2-AS}

One of the most expensive steps of \ctwo{} is the resolution of an
Artin-Schreier equation in an extension field $\U_i$. We call \ctwoas{}
the variant of Couveignes' algorithm that uses the fast Artin-Schreier
towers of Chapter~\ref{cha:artin-schr-towers}; in this section we
analyze the complexity of \ctwoas{}

\subsection{Complexity analysis}
\label{sec:C2-AS:complexity}
We borrow the complexity notations $\Lift(i)$ (Theorem~\ref{theo:L})
and $\Ptr(i)$ (Theorem~\ref{th:b-pseudo}) from
Chapter~\ref{cha:artin-schr-towers}.

\paragraph{$p$-torsion}
The construction of $\F_q[c]$ may be done in many ways. The only
requirements of Theorem~\ref{th:cantor}
\begin{enumerate}
\item that its elements have a representation as elements of
  $F_p[X]/Q_1(X)$ for some irreducible polynomial $Q_1$,
\item that either $(d,p)=1$ or $\deg Q_1' + 2 = \deg Q_1$.
\end{enumerate}
\pdfmctwo{Yes, there are deterministic algorithms to generate
  irreducible polynomials, and it would be easier to compute with Q1,
  if it were sparse. But if I want to have a good chance of meeting
  condition 2 when p divides d, I must take a random polynomial and
  certainly not a sparse one! This was already said in the remark.}
Selecting a random polynomial $Q_1$ and testing for irreducibility is
usually enough to meet these conditions, as we saw in
Remark~\ref{rk:comp-minim-polyn}.  This costs $O\bigl(pd\Mult(pd)\log
(pd)\log(p^2d)\bigr)$ according to~\cite[Th.  14.42]{vzGG}.

Now we need to compute the embedding $\F_q\subset\F_q[c]$. Supposing
$\F_q$ is represented as $\F_p[X]/Q_0(X)$, we factor $Q_0$ in
$\F_q[c]$, which costs $O\bigl(pd\Mult(pd^2)\log d\log p\bigr)$
using~\cite[Coro. 14.16]{vzGG}. Then the most naive technique to
express the embedding is linear algebra. This requires the computation
of $pd$ elements of $\F_q[c]$ at the expense of
$\Theta\bigl(pd\Mult(pd)\bigr)$ operations in $\F_p$, then the
inversion of the matrix holding such elements, at a cost of
$\Theta\bigl((pd)^\omega\bigr)$ operations. This is certainly not
optimal, yet this phase will have negligible cost compared to the rest
of the algorithm.

Now we can compute $c$ and $c'$ by factoring the polynomials
$Y^{p-1}-H_E$ and $Y^{p-1}-H_{E'}$ in $\F_p[X]/Q_1(X)$. This costs
\[O\bigl((p\ModComp(pd) + \ModComp(p)\Mult(pd) + \Mult(p)\Mult(pd)\log
p)(\log^2 p+\log d)\bigr)\] using~\cite[Section 3]{kaltofen+shoup97}.

Finally, computing the determinants needed by Gunji's formulas takes
$\Theta(p^2)$ multiplications in $\F_q[c]$, that is
$\Theta\bigl(p^2\Mult(pd)\bigr)$.

Letting out logarithmic factors, the overall cost of this phase is
\begin{equation}
  \label{eq:gunji-complexity}
  \tildO\bigl(p^2d^3 + p\ModComp(pd) + \ModComp(p)pd + (pd)^\omega \bigr)
\end{equation}


\paragraph{$p^k$-torsion}
Application of Voloch formulas requires at each of the levels
$\U_2,\ldots,\U_k$
\begin{enumerate}
\item to solve equation \eqref{th:voloch:cover} by factoring an
  Artin-Schreier polynomial,
\item to solve the system \eqref{th:voloch:isom}.
\end{enumerate}
If we assume the worst case $[\U_2:\U_1] = p$, according to
Theorem~\ref{theo:main}, at each level $i$ the first step costs
\begin{equation*}
  O\bigl((pd)^\omega i + {\Ptr}(i-1) + \Mult(p^{i+1}d)\log p\bigr)
\end{equation*}
while the second takes the GCD of two degree $p$ polynomials in
$\U_i[X]$ for each $i$ (see Section \ref{sec:implementation}), at a
cost of $O\bigl(\Mult(p^{i+1}d)\log p\bigr)$ operations using a
\hyperref[sec:eucl-algor-rati]{fast Euclidean algorithm}.

Summing up over $i$, the total cost of this phase up to logarithmic
factors is
\begin{equation}
  \label{eq:C2-AS:complexity:p^k}
  \tildO_{p,d,\log\ell}\left((pd)^\omega \log_p^2\ell + p^2\ell d\log_p^4\ell +
  \frac{\ell}{p}\ModComp(pd)\right)
  \;\text{.}  
\end{equation}
Also notice that there is no need to store a $p^{k-1}d\times p^{k-1}d$
matrix to solve the Artin-Schreier equation, thus the space
requirements are not anymore quadratic in $\ell$.


\paragraph{Interpolation}
The interpolation phase does not change in a significant way: one
needs first to interpolate a degree $p^k-1$ polynomial with
coefficients in $\U_k$, then use
\titleref{alg:push-down} to obtain the corresponding
polynomial in $\F_q[X]$ and finally do a rational fraction
reconstruction.

The first step costs $O\bigl(\Mult(p^{2k}d)\log p^k\bigr)$ using fast
techniques as in Section~\ref{sec:chin-rema-algor}, then converting to
$\F_q[c][X]$ takes $O\bigl(p^k\Lift(k-1)\bigr)$ and further
converting to $\F_q[X]$ takes $\Theta\bigl((pd)^2\bigr)$ by linear
algebra. The \hyperref[sec:eucl-algor-rati]{rational function
  reconstruction} then takes $O\bigl(\Mult(p^kd)\log p^k\bigr)$.

The overall complexity of one interpolation is then
\begin{equation}
  \label{eq:C2-AS:complexity:interp}
  O\bigl(\Mult(\ell^2d)\log_p\ell + \ell\Lift(k-1) + (pd)^2\bigr)
  \;\text{.}
\end{equation}
Remember that this step has to be repeated an average number of
$\euler(p^k)/4$ times, thus the dependency of \ctwoas{} in $\ell$ is still cubic.



% Local Variables:
% mode:flyspell
% ispell-local-dictionary:"american"
% mode:TeX-PDF
% TeX-master: "../these"
% mode:reftex
% End:
%
% LocalWords:  Schreier Artin pseudotrace Frobenius bivariate Joux Sirvent FFT
% LocalWords:  Couveignes isogenies Schoof isogeny cryptosystems Lercier
% LocalWords:  precomputation arithmetics polylogarithmic Karatsuba
% LocalWords:  irreducibility

\section{The algortihm C2-AS-FI}
\label{sec:C2-AS-FI}

The most expensive step of C2-AS is the polynomial interpolation step
which is part of the Cauchy interpolation. If we use a standard
interpolation algorithm, its input consists in a list of $\Theta(p^k)$
pairs $\bigl(P, \I(P)\bigr)$, with $P$ having coordinates in $\U_k$,
thus a lower bound for any such algorithm is $\Omega(p^{2k}d)$. Notice
however that the output is a polynomial of degree $\Theta(p^k)$ in
$\F_q[X]$, hence, if supplied with a shorter input, an \emph{ad hoc}
algorithm could reach the bound $\Omega(p^kd)$.

In this section we give an algorithm that reaches this bound up to
some logarithmic factors. It realizes the polynomial interpolation on
the primitive points of $E[p^k]$, thus its output is a degree
$\euler(p^k)/2-1$ polynomial in $\F_q[X]$. Using the Chinese remainder
theorem it is straightforward to generalise this to an algorithm,
having the same asymptotic complexity, that realizes the polynomial
interpolation on all the points of $E[p^k]$. We call C2-AS-FI the
variant of C2-AS resulting from applying this new algorithm.


\subsection{The algorithm}
Let $P\in E[p^k]$ and $P'\in E'[p^k]$ be primitive $p^k$ torsion
points. We want to compute the smallest degree polynomial
$A\in\F_q[X]$ such that
\begin{equation}
  A\bigl(x\bigl([n]P\bigr)\bigr) = x\bigl([n]P'\bigr)
  \quad\text{for any $n\in\left(\Z/p^k\Z\right)^\ast$.}
\end{equation}
To be more precise, we want to compute the canonical representative of
$A$ in $\F_q[X]/T(X)$, where
\begin{equation}
  T(X) = \prod_{n\in\left(\Z/p^k\Z\right)^\ast} \bigl(X - x([n]P)\bigr)
  \text{.}
\end{equation}
 
There are two equivalent ways to look at this problem. The first one
is as the interpolation problem we just stated. The second one is as
an isomorphism of finite fields problem. Both viewpoints will be
important.

For notational convenience, we set $\U_0=\F_q$.  Let

\begin{equation}
  \label{eq:T}
  T = \prod T^{(j)}
\end{equation}
be the factorisation of $T$ over $\U_0$, and set
\begin{equation}
  \label{eq:A}
  A^{(j)} = A \bmod T^{(j)}
  \;\text{.}
\end{equation}

It was already pointed out in \cite[$\S$2.3]{Cou96} that, knowing the
factorisation of $T$ over $\U_0$ and all the $A^{(j)}$'s, we can
recover $A$ using the Chinese remainder theorem. Thus we will focus on
computing, say, $A^{(0)}$.

$T^{(0)}$ is irreducible over $\F_q$. Chose any root of $T^{(0)}$,
without loss of generality we can take $x(P)$ to be such a root.  Fix
the $\F_q$-linear embedding of finite fields
\begin{equation}
  \label{eq:embed}
  \xymatrix{
    ^{\F_q[X]}/_{T^{(0)}(X)} \ar@{^{(}->}[r]^-\iota & \U_k
  }
\end{equation}
given by $\iota(X) = x(P)$. It is evident that
\begin{equation}
  \iota\bigl(A^{(0)}(X)\bigr) = A^{(0)}\left(\iota(X)\right)=x\bigl(P'\bigr)
  \text{,}
\end{equation}
thus in order to compute $A^{(0)}$ one just needs to compute
$\iota^{-1}\bigl(x(P')\bigr)$.

This is a classic problem in computer algebra: given an algebraic
extension $\LK/\K$ and elements $x\in\LK$ and $y\in\K[x]$, find the
minimal polynomial $Q$ of $x$ over $\K$, identify $\K[X]/Q(X)$ to
$\K[x]$ and find the canonical image of $y$ in $\K[X]/Q(X)$. The
fastest techniques available are \cite{shoup99,PS06}, which are largely
used in \cite{DFS09}. However, they both require to solve a
\emph{power projection} problem, that is, given a $\K$-linear form
$\ell$, compute
\begin{equation}
  \ell(1), \ell(x), \ell(x^2), \dots, \ell(x^n)
\end{equation}
up to some bound $n$. As explained in \cite{shoup99}, by the
\emph{transposition principle}, solving the power projection has the
same complexity as computing $g(x)$ given $g\in\K[X]$.

But, in our specific case, the extension we work with is $\U_k/\F_q$.
Without previous knowledge of the minimal polynomial of $x\in\U_k$, we
have no method to evaluate a polynomial $g\in\F_q[X]$ on $x$, faster
than lift $g$ in $\U_k[X]$ and evaluate by Horner rule. Unfortunately,
this is too expensive, thus we will study an alternative approach that
amounts to decompose $\iota$ as a chain of morphisms and invert them
one-by-one going down in the tower $(\U_0,\U_1,\ldots,\U_k)$. This is
similar to the way \cite{Cou00} solves an Artin-Schreier equation by
moving it down from $\U_k$ to $\U_1$.


\paragraph{Interpolation in towers of extensions}
We switch back to the interpolation viewpoint. The algorithm we give
here can be applied in any tower of cyclic extensions, provided the
action of the Galois groups can be computed. However we will present
it only for our specific tower $(\U_0,\dots,\U_k)$, to avoid adding
unnecessary notation.

Consider the following problem: given elements $x,y\in\U_k$ such that
$x$ generates $\U_k$ over $\F_q$, find a polynomial $A\in\F_q[X]$ such
that
\begin{equation}
  \label{eq:affine-minimal}
  A(x) = y
  \text{.}
\end{equation}
Let $A$ be such a polynomial and let $T$ be the minimal polynomial of
$x$ over $\F_q$, then it is evident that $A+T$ satisfies
\eqref{eq:affine-minimal}. Hence, we can look for a representative of
minimal degree of the class of $A$ in $\F_q[X]/T(X)$. If one such
class exists, then it is unique, because otherwise $x$ would be root
of a polynomial in $\F_q[X]$ of degree smaller than $T$. 

Let $A$ be a polynomial satisfying \eqref{eq:affine-minimal} it is
clear that $A(\sigma(x)) = \sigma(y)$ for any
$\sigma\in\Gal(\U_k/\F_q)$. Conversely, the polynomial interpolating
$\sigma(x)$ over $\sigma(y)$ for any $\sigma$ is invariant under
$\Gal(\U_k/\F_q)$, thus it has coefficients in $\F_q$. Hence we can
construct $A$ by interpolation.

A fast interpolation algorithm as in \cite[10.1-2]{vzGG} would compute
$T$ via a binary subproduct tree, and then interpolate $A$ recursively
applying the Chinese remainder theorem along the branches of the
tree. However this is too expensive. We can do better by using a
non-binary subproduct tree on which the tower of Galois groups
associated to $(\U_0,\ldots,\U_k)$ acts.

First we need to compute $T$. Let $T_i$ be the minimal polynomial of
$x$ over $\U_i$, it is computed recursively as
\begin{align}
  T_k &= (X - x)\text{,}\\
  \label{eq:minprod}
  T_i &= \prod_{\sigma\in\Gal(\U_{i+1}/\U_i)}T_{i+1}^\sigma\text{.}
\end{align}
Then $T=T_0$. Observe that, rather than computing a whole subproduct
tree of $T$, we have only computed one branching as shown in figure
\ref{fig:tree}.

\begin{figure}[tb]
  \centering
  
  \begin{tikzpicture}
    \begin{scope}
      [level distance=1cm]
      \node{$\U_0$}[grow'=up]
      child {node {$\U_1$}
        child {node {$\U_2$}
          child {node {$\U_3$}}
        }
      };
    \end{scope}    
  \end{tikzpicture}
  % 
  \hfill
  %
  \begin{tikzpicture}
    \begin{scope}
      [level distance=1cm,
      level/.style={sibling distance=6cm/#1},
      nc/.style={gray}]
      \node{$T$}[grow'=up]
      child {node {$T_1$}
        child {node {$T_2$}
          child {node {$T_3$}}
          child {node {$T_3^{\sigma^4}$}}
        }
        child {node {$T_2^{\sigma^2}$}
          child[nc] foreach \l in {2,6} {node {$T_3^{\sigma^{\l}}$} edge from parent[dashed]}
        }
      }
      child {node {$T_1^\sigma$}
        child[nc] {node {$T_2^\sigma$} edge from parent[dashed]
          child[nc] foreach \l in {,5} {node {$T_3^{\sigma^{\l}}$} edge from parent[dashed]}
        }
        child[nc] {node {$T_2^{\sigma^3}$} edge from parent[dashed]
          child[nc] foreach \l in {3,7} {node {$T_3^{\sigma^{\l}}$} edge from parent[dashed]}
        }
      };
    \end{scope}
  \end{tikzpicture}
  
  \caption{The subproduct tree of $T$, in the case of a tower of
    quadratic extensions. Any generator of $\Gal(\U_3/\U_0)$ can be
    taken as $\sigma$. We gray out the nodes that the algorithm does
    not compute.}
  \label{fig:tree}
\end{figure}



Now we compute recursively the polynomials in $A_i\in\U_i[X]$ such
that $A_i(x)=y$. We start from $A_k=y$. Suppose $A_{i+1}$ is known,
then we apply the Chinese remainder algorithm of \cite[$\S$10.3]{vzGG}
to compute the polynomial $P\in\U_{i+1}[X]/T_i(X)$ such that
\begin{equation}
  \label{eq:crt}
  P \equiv A_{i+1}^\sigma \bmod T_{i+1}^\sigma
  \qquad\text{for any $\sigma\in\Gal(\U_{i+1}/\U_i)$.}
\end{equation}
It is clear that $P$ is invariant under $\Gal(\U_{i+1}/\U_i)$, hence
$P\in\U_i[X]/T_i(X)$ and by \eqref{eq:crt} it is evident that
$P(x)=A_{i+1}(x)=y$, thus $P=A_i$.

We have thus succeeded in interpolating $A=A_0$, without having to
build the whole subproduct tree. A similar algorithm was already given
in \cite{EnMo03}.


\paragraph{Back to our problem}
It is easy to realise that, on inputs $x(P)$ and $x(P')$, the
algorithm we just gave computes $A^{(0)}$. In fact, $T^{(0)}$ is
the minimal polynomial of $x(P)$ over $\F_q$ and $A^{(0)}$ is the
unique polynomial in $\F_q[X]/T^{(0)}(X)$ that satisfies
\eqref{eq:affine-minimal}.

This can be viewed as decomposing $\iota$ as the chain of
$\F_q$-linear isomorphisms
\begin{equation}
  \xymatrix{
    ^{\U_0[X_0]}/_{T_0(X_0)} \ar@{^{(}->}[r]^-{\iota_0} &
    \;\cdots\; \ar@{^{(}->}[r]^-{\iota_{k-1}} &
    ^{\U_k[X_k]}/_{T_k(X_k)} \ar@{^{(}->}[r]^-{\iota_k} &
    \U_k
  }
\end{equation}
defined by $\iota_k\circ\cdots\circ\iota_i(X_i) = x(P)$ for any $i$,
and then finding the preimage of $x(P')$ by inverting them one by
one.

Then, the Chinese remainder step we applied in \eqref{eq:crt} amounts
to invert $\iota_i$ by descending the lower path in the diagram below
\begin{equation}
  \xymatrix{
    ^{\U_i[X_i]}/_{T_i(X_i)} \ar@{^{(}->}[r]^-{\iota_i} \ar@{^{(}->}[d]^{\varepsilon} &
    ^{\U_{i+1}[X_{i+1}]}/_{T_{i+1}(X_{i+1})} \\
    ^{\U_{i+1}[Y]}/_{T_i(Y)} \ar@{^{(}->>}[r]^-{\gamma} &
    \bigoplus_\sigma {}^{\U_{i+1}[Y_{j}]}/_{\left(T_{i+1}\right)^\sigma(Y_{j})} \ar@{->>}[u]_{\pi}
  }
\end{equation}
where $\varepsilon$ is the canonical injection extending
$\U_i\subset\U_{i+1}$, $\gamma$ is the Chinese remainder isomorphism
and $\pi$ is projection onto the first coordinate.

Some care must be taken when $x(P)$ does not generate $\U_k$, but only
a subfield of index $2$. This happens when $c\not\in\F_q[c^2]$, and in
this case $\iota_0$ is not a field isomorphism. It is not to
difficult, however, to handle this case, as one only needs to take a
subgroup of index $2$ of $\Gal(\U_1/\U_0)$, instead of the whole
group, in the interpolation algorithm given above.

Observe that we could have used a different approach: after $T^{(0)}$
has been computed by \eqref{eq:minprod}, a polynomial $g\in\F_q[X]$
can efficiently be evaluated at $x(P)$ by successively lifting in
$\U_i$ and reducing modulo $T_i$ for $i=1,\ldots,k$. Thus, as noted
before, by the transposition principle we also have an efficient
algorithm to compute the power projection on $x(P)$; hence we can
apply \cite{PS06} to efficiently find $A^{(0)}$. However, this
approach cannot improve the overall complexity as it will be clear
in the next section.


\subsection{Complexity analysis}
\label{sec:C2-AS-FI:complexity}

The two algorithms for computing $T^{(0)}$ and $A^{(0)}$ are very
similar and run in parallel. We can merge them in one unique
algorithm.

We set some notation. Let $i_0$ be the largest index such that
$\U_{i_0} = \U_1$ and let $\frac{p-1}{2r} = [\F_q[c^2]:\F_q]$.  Remark
that all the $T^{(j)}$'s have degree $\frac{\euler(p^{k-i_0+1})}{2r}$.
At each level $i\ge i_0$, it does the following

\begin{enumerate}
\item for $\sigma \in \Gal(\U_{i+1}/\U_i)$, compute
  \begin{enumerate}
  \item\label{alg:T:gal} $\left(T_{i+1}\right)^\sigma$ and
  \item\label{alg:A:gal} $\left(A_{i+1}\right)^\sigma$ using
    \cite[\alg{IterFrobenius}]{DFS09},
  \end{enumerate}
\item\label{alg:T:prod} compute $T_i$ by \eqref{eq:minprod}
  through a subproduct tree as in \cite[Algo. 10.3]{vzGG},
\item\label{alg:A:CRA} compute $A_i$ by \eqref{eq:crt} through
  Chinese Reminder Algorithm \cite[Algo. 10.16]{vzGG},
\item\label{alg:T:push} convert $T_i$ and $A_i$ into
  elements of $\U_i[X]$ using \cite[\alg{Push-down}]{DFS09}.
\end{enumerate}

Steps \ref{alg:T:gal} and \ref{alg:A:gal} are identical. Both are
repeated $p$ times, each iteration taking $O\bigl(p^{k-i}{\sf
  L}(i-i_0)\bigr) \subset O\bigl({\sf L}(k-i_0)\bigr)$ by
\cite[Th. 17]{DFS09}.

Step \ref{alg:T:prod} takes $O\bigl(\Mult(p^{k-i_0+1}d/r)\log p\bigr)$
by \cite[Lemma 10.4]{vzGG} and step \ref{alg:A:CRA} has the same
complexity by \cite[Coro. 10.17]{vzGG}.

Step \ref{alg:T:push} takes $O\bigl(p^{k-i+1}{\sf L}(i-i_0)\bigr)
\subset O\bigl(p{\sf L}(k-i_0)\bigr)$.

When $i=0$ and $\U_1\ne\F_q$ the algorithm is identical but steps
\ref{alg:T:gal} and \ref{alg:A:gal} must be computed through a generic
Frobenius algorithm (using~\cite[Algorithm 5.2]{vzGS92}, for example)
and step \ref{alg:T:push} must use the implementation of $F_q[c]$ to
make the conversion (for example, linear algebra). In this case steps
\ref{alg:T:gal} and \ref{alg:A:gal} cost
$\Theta\bigl(\frac{p^{k-i_0}}{r}\ModComp(pd)\log d \bigr)$
by~\cite[Lemma 5.3]{vzGS92} and step \ref{alg:T:push} costs
$\Theta\bigl(p^{k-i_0}(pd)^2\bigr)$.

The total cost of the algorithm is then
\begin{equation*}
  \label{eq:T:complexity}
  O\left(\bigl(k-i_0\bigr)\bigl(p{\sf L}(k-i_0) + \Mult(p^{k-i_0+1}d/r)\log p\bigr) +
    \frac{p^{k-i_0}}{r}\bigl(\ModComp(pd)\log d + r(pd)^2\bigr) \right)
  \;\text{.}
\end{equation*}


\paragraph{The complete interpolation}
We compute all the $A^{(j)}$'s using this algorithm; there's
$p^{i_0-1}r$ of them. We then recombine them through a Chinese
remainder algorithm at a cost of $O\bigl(\Mult(p^kd)\log
p^{i_0-1}r\bigr)$. The total cost of the whole interpolation phase is
then
\begin{equation*}
  O\left(\bigl(k-i_0\bigr) \bigl(p{\sf L}(k) + \Mult(p^kd)\log p\bigr) +
    p^{k-1}\ModComp(pd)\log d + p^{k-1}r(pd)^2 + i_0\Mult(p^kd)\log p
  \right)
  \;\text{,}
\end{equation*}
that is
\begin{equation}
  \label{eq:interp}
  O\left(p{\sf L}(k)\log\left(\frac{\ell}{p^{i_0}}\right) + 
    \Mult(\ell d)\log\ell\log p +
    \frac{\ell}{p}\ModComp(pd)\log d +
    \ell (pd)^2
  \right)
  \;\text{.}
\end{equation}

Alternatively, once $A^{(0)}$ is known, one could compute the other
$A^{(j)}$'s using modular composition with the multiplication maps
of $E$ and $E'$ as suggested in \cite{Cou96}. However this approach
doesn't give a better asymptotic complexity because in the worst case
$A^{(0)}=A$. From a practical point of view, though, Brent's and
Kung's algorithm for modular composition \cite{BrKu78}, despite having
a worse asymptotic complexity, could perform faster for some set of
parameters. We will discuss this matter in Section
\ref{sec:C2-AS-FI-MC}.

If more than $\euler(p^k)/2$ points are needed, but less than
$\frac{p-1}{2}$, one can use the previous algorithm to interpolate
over the primitive $p^i$-torsion points for each $i=1,\ldots,k$. The
interpolating polynomials can then be recombined through a Chinese
remainder algorithm at a cost of $O\bigl(\Mult(p^kd)\log p^k\bigr)$,
which doesn't change the overall complexity of C2-AS-FI.


Putting together the complexity estimates of C2-AS and C2-AS-FI, we
have the following theorem.

\begin{theorem}
  \label{th:complexity}
  Assuming $\Mult(n) = n\log n\log\log n$, the algorithm C2-AS-FI has
  worst case complexity
  \begin{equation*}
    \tildO_{p,d,\log\ell}\left(
      p^2d^3 +
      \ModComp(p)pd +
      (pd)^\omega\log^2\ell +
      p^3\ell^2 d\log^3\ell + 
      p^2\ell^2 d^2+
      \left(\frac{\ell^2}{p} + p\right)\ModComp(pd)
    \right)
    \;\text{.}
  \end{equation*}
\end{theorem}



% Local Variables:
% mode:flyspell
% ispell-local-dictionary:"british"
% mode:TeX-PDF
% TeX-master: "ec-isogeny"
% End:
%
% LocalWords:  Schreier Artin pseudotrace Frobenius bivariate Joux Sirvent FFT
% LocalWords:  Couveignes isogenies Schoof isogeny cryptosystems Lercier moduli
% LocalWords:  precomputation arithmetics polylogarithmic Karatsuba embeddings
% LocalWords:  irreducibility

\section{The algorithm C2-AS-FI-MC}
\label{sec:C2-AS-FI-MC}

However asymptotically fast, the polynomial interpolation step is
quite expensive for reasonably sized data. Instead of repeating it
$\frac{\euler(p^k)}{2}$ times, one can use composition with the
Frobenius endomorphism $\frobisog_E$ in order to reduce the number of
interpolations in the final loop.

\subsection{The algorithm}
Suppose we have computed, by the algorithm of the previous Section,
the polynomial $T$ vanishing on the abscissae of $E[p^k]$ and an
interpolating polynomial $A_0\in\F_q[X]$ such that
\begin{equation*}
  A_0\bigl(x\bigl([n]P\bigr)\bigr) = x\bigl([n]P'\bigr)
  \quad\text{for any $n$.}
\end{equation*}
The group $\Gal(\U_k/\F_q) = \langle\frob\rangle$ acts on $E'[p^k]$
permuting its points and preserving the group structure. Thus, the
map (where polynomials act by evaluation) 
\begin{equation*}
  A_1 = A_0\circ\frob = \frob\circ A_0
\end{equation*}
is such that
\begin{equation*}
  A_1\bigl(x\bigl([n]P\bigr)\bigr) = x\bigl([n]\frobisog_{E'}(P')\bigr)
  \quad\text{for any $n$,}
\end{equation*}
where $\frobisog_{E'}$ is the Frobenius endomorphism of $E'$.  Since
$\frobisog_{E'}(P')$ is a generator of $E'[p^k]$, $A_1$ is one of the
polynomials that the algorithm C2 tries to identify to an isogeny. By
iterating this construction we obtain $[\U_k:\F_q]/2$ different
polynomials $A_i$ for the algorithm C2 with only one interpolation.

To compute the $A_i$'s, we first compute $F\in\F_q[X]$
\begin{equation}
  \label{eq:frob}
  F(X) = X^q \bmod T(X)
  \text{,}
\end{equation}
then for any $1\le i<[\U_k:\F_q]/2$
\begin{equation}
  \label{eq:modcomp}
  A_i(X) = A_{i-1}(X)\circ F(X) \bmod T(X)\text{.}
\end{equation}

If $\frac{\euler(p^k)}{[\U_k:\F_q]} = p^{i_0-1}r$, we must compute
$p^{i_0-1}r$ polynomial interpolations and apply this algorithm to
each of them in order to deduce all the polynomials needed by C2.


\subsection{Complexity analysis}
We compute \eqref{eq:frob} via square-and-multiply, this costs
$\Theta(d\Mult(p^kd)\log p)$ operations. Each application of
\eqref{eq:modcomp} is done via a \emph{modular composition}, the cost
is thus $O(\ModComp(p^k))$ operations in $\F_q$, that is
$O(\ModComp(p^k)\Mult(d))$ operations in $\F_p$. Using the algorithm
of~\cite{KeUm08} for modular composition, the complexity of
C2-AS-FI-MC wouldn't be essentially different from the one of
C2-AS-FI; however, in practice the fastest algorithm for modular
composition is~\cite{BrKu78}, and in particular the variant
in~\cite[Lemma 3]{KS98}, which has a worse asymptotic complexity, but
performs better on the instances we treat in
Section~\ref{sec:benchmarks}.

Notice that a similar approach could be used inside the polynomial
interpolation step (see Section \ref{sec:C2-AS-FI}) to deduce
$A_k^{(0)}$ from $A_0^{(0)}$ using modular composition with the
multiplication maps of $E$ and $E'$ as described in
\cite[$\S$2.3]{Cou96}. This variant, though, has an even worse
complexity because of the cost of computing multiplication maps.




% Local Variables:
% mode:flyspell
% ispell-local-dictionary:"british"
% mode:TeX-PDF
% TeX-master: "ec-isogeny"
% End:
%
% LocalWords:  Schreier Artin pseudotrace Frobenius bivariate Joux Sirvent FFT
% LocalWords:  Couveignes isogenies Schoof isogeny cryptosystems Lercier moduli
% LocalWords:  precomputation arithmetics polylogarithmic Karatsuba embeddings
% LocalWords:  irreducibility

\section{Smallest degree isogeny}
\label{sec:bounded}

We now present an extension to Couveignes algorithm that could be
useful in cryptographic application. It is well known that two curves
having the same number of points over a finite field are isogenous,
however this doesn't say anything on the degree of the isogeny
connecting them. Given two elliptic curves $E$ and $E'$ defined over
$\F_q$ and having the same number of points, we want to find the
smallest degree isogeny between them.

The simplest solution is to take any algorithm computing a fixed
degree isogeny and try all the degrees until an isogeny is found. If
$\ell$ is the degree of the smallest isogeny, this of course adds a
factor $\ell$ to the complexity of any polynomial time algorithm.

Couveignes' algorithm can be easily adapted to solve this problem at
no additional cost. We call this algorithm C2SD and we will only
discuss its most efficient variant C2SD-AS-FI.

Observe that, apart for the choice of $k$, the computation of $E[p^k]$
and the polynomial interpolation step do not depend at all on
$\ell$. The degree of the isogeny only comes into play in the last
part of the Cauchy interpolation, that is the rational function
reconstruction. We study more in detail this last step.


\paragraph{Rational Function Reconstruction}
Rational function reconstruction takes as input a degree $n$
polynomial $T$, a polynomial $A$ of degree less than $n$ and a target
degree $m\le n$ and outputs the unique rational function such that
\begin{equation*}
  A \equiv \frac{R}{V} \bmod T
\end{equation*}
and $\deg R < m$, $\deg V \le n-m$. This is done by computing a Bezout
relation $AV + TU = R$ with the expected degrees via an XGCD
algorithm. If a classical XGCD algorithm is used, one simply computes
all the lines
\begin{equation}
  \label{eq:XGCD}
  \begin{aligned}
    R_0 &= T, & U_0 &= 1, & V_0 &= 0,\\
    R_1 &= A, & U_1 &= 0, & V_1 &= 1,\\
    R_{i-1} &= Q_iR_i + R_{i+1}, & U_{i+1} &= U_{i-1}-Q_iU_i, & V_{i+1} &= V_{i-1}-Q_iV_i
  \end{aligned}
\end{equation}
and stops as soon as a remainder $R_{i+1}$ with $\deg R_{i+1}<m$ is
found. If a fast XGCD algorithm as \cite[Algo. 11.4]{vzGG} is used,
one directly aims at the two lines
\begin{equation}
  \label{eq:FastGCD}
  \begin{aligned}
    R_{h-2} &= Q_{h-1}R_{j-1} + R_h\\
    R_{h-1} &= Q_hR_h + R_{h+1}
  \end{aligned}
\end{equation}
such that $\deg R_{h+1} < m \le \deg R_h$ without computing the
intermediate lines.

When looking for an $\ell$-isogeny, one simply sets
$m=\ell+1$. Observe that if the algorithm doesn't return a rational
fraction $\frac{R}{V}$ with $\deg R = \ell$ and $\deg V = \ell -1 $,
then no such fraction congruent to $A$ modulo $T$ exists.

If $\ell$ is not \emph{a priori} known, we can still use the fact that
a separable isogeny with cyclic kernel must have $\deg R = \deg V +
1$. In fact if we suppose $R = R_i$ and $V = V_i$, then
\begin{align*}
  \deg T &= \deg V_{i+1} + \deg R_i,\\
  \deg R_i - \deg V_i &= \deg R_{i-1} - \deg V_{i+1}
\end{align*}
implies
\begin{equation*}
  \deg T + 1 = \deg R_{i-1} + \deg R_i  
  \;\text{.}
\end{equation*}
Hence, if $A$ is congruent to an $\ell$-isogeny with $\ell =
\left\lfloor\frac{\deg T}{2}\right\rfloor - t$ for some $t\ge0$, then
\begin{equation}
  \label{eq:degseq}
  \deg R_{i-1} =
  \left\lceil\frac{\deg T}{2}\right\rceil + t + 1 >
  \left\lfloor\frac{\deg T}{2}\right\rfloor - t = \deg R_i
  \;\text{.}
\end{equation}
Thus we can recover any isogeny having degree less than
$\left\lfloor\frac{\deg T}{2}\right\rfloor$ using either a classical
or a fast XGCD algorithm setting $m = \left\lceil\frac{\deg
    T}{2}\right\rceil + 1$.


\paragraph{Recognising an isogeny}
Once we have a rational fraction with the required degree, we have to
test if it really is an isogeny. In order to understand how often we
have to make this test, we introduce some more terminology. Let $n_i =
\deg R_i$, we call $(n_0,\ldots,n_r)$ the \emph{degree sequence} of
$A$ and $T$; a degree sequence is said \emph{normal} if $n_i = n_{i+1}
+ 1$ for any $i$.

\begin{proposition}
  \label{th:normseq}
  Let $f,g\in\F_q[X]$ be uniformly chosen random polynomials of
  respective degrees $n_0>n_1>0$ and let $(n_0, n_1, \ldots, n_r)$ be
  their degree sequence. For $0\le i < n_1$ define the binary random
  variables $X_i = 1 \Leftrightarrow i\in(n_0,n_1,\ldots,n_r)$, then
  the $X_i$ are independent random variables and $\mathrm{Prob}(X_i=0) =
  \frac{1}{q}$.
\end{proposition}
\begin{proof}
  Pairs of polynomials $f,g$ are in bijection with the GCD-sequence
  $(R_r, Q_r, \ldots, Q_1)$ constituted by their GCD and the quotients
  of the GCD algorithm. To each such sequence is associated a degree
  sequence
  \begin{equation*}
    (n_0,n_1,\ldots,n_r) =
    \left(\deg R_r + \sum_{i=1}^r\deg Q_i, \ldots, \deg R_r + \sum_{i=1}^1\deg Q_i, \deg R_r\right)
    \;\text{,} 
  \end{equation*}
  thus for any given degree sequence there are
  \begin{equation*}
    (q-1)q^{n_0-n_1}\cdot(q-1)q^{n_1-n_2}\cdot\cdots\cdot(q-1)q^{n_r} =
    (q-1)^{r+1}q^{n_0}
  \end{equation*}
  GCD-sequences.
 
  Let $I$ and $O$ be two disjoints subsets of $\{X_i\}$, the number of
  GCD-sequences such that $X\in I \Rightarrow X=1$ and $X\in O
  \Rightarrow X=0$,
  \begin{equation*}
     \sum_{s=0}^{n_1-\card{I}-\card{O}}\binom{n_1-\card{I}-\card{O}}{s}(q-1)^{s+2+\card{I}}q^{n_0} =
    (q-1)^{2+\card{I}}q^{n_0}q^{n_1-\card{I}-\card{O}}
    \;\text{.}
  \end{equation*}
  There are $(q-1)^2q^{n_0}q^{n_1}$ pairs of polynomials of degrees
  $n_0,n_1$, thus
  \begin{equation}
   \label{th:normseq:prob}
    \mathrm{Prob}\bigl(\{X = 1 \mid X\in I\},
    \{X=0\mid X\in O\}\bigr) = \left(\frac{q-1}{q}\right)^{\card{I}}\left(\frac{1}{q}\right)^{\card{O}}
    \;\text{.}
  \end{equation}
  The claim follows.
\end{proof}

Degree sequences associated to isogenies are in general not normal, in
fact if $\ell\le\left\lfloor\frac{\deg T}{2}\right\rfloor-t$, equation
\eqref{eq:degseq} shows that there must be at least a gap of degree
$2c$ in the degree sequence. Heuristically, we can expect that if the
polynomial $A$ doesn't correspond to an isogeny, then $A$ and $T$ act
like random polynomials, thus, by the proposition above, the
probability that $A$ looks like an isogeny of degree
$\ell\le\left\lfloor\frac{\deg T}{2}\right\rfloor-t$ is less than $\frac{1}{q^{2t}}$.

Therefore, by choosing an appropriate $t\in O(\log_q p^k)$, C2SD can
find any isogeny of degree less than $\frac{p^k-1}{4}-t$ at the same
cost of one run of C2. Also notice that C2SD is not restricted to
isogenies of degree prime to $p$ as it was already mentioned in
Section \ref{sec:C2:non-prime}.

No other method for computing isogenies is known to have a similar
generalisation, this makes C2SD-AS-FI interesting for practical
applications.




% Local Variables:
% mode:flyspell
% ispell-local-dictionary:"british"
% mode:TeX-PDF
% TeX-master: "ec-isogeny"
% End:
%
% LocalWords:  Schreier Artin pseudotrace frobenius bivariate Joux Sirvent FFT
% LocalWords:  Couveignes isogenies Schoof isogeny cryptosystems Lercier
% LocalWords:  precomputation arithmetics polylogarithmic Karatsuba precomputes
% LocalWords:  endomorphisms  isogenous


\chapter{Experimental results}
\label{cha:experimental-results}
In this chapter we describe the implementations we made of the
algorithms of the previous chapter and some experimental results.

\section{Implementation of Couveignes' algorithm}
\label{sec:implementation}

We implemented \ctwoasfimc{} as \texttt{C++} programs using the
libraries \texttt{NTL}~\cite{shoup2003ntl} for finite field
arithmetics, \texttt{gf2x}~\cite{gf2x} for fast arithmetics in
characteristic $2$ and \texttt{FAAST} (see
Section~\ref{sec:artin-benchmarks}) for fast arithmetics in
Artin-Schreier towers.  We also have a Magma~\cite{MAGMA} prototype of
the same algorithm, not making use of the fast algorithms of
Chapter~\ref{cha:artin-schr-towers}.  This section mainly deals with
some tricks we implemented in order to speed up the computation.

\subsection{Building \texorpdfstring{$E[p^k]$}{E[pk]} and \texorpdfstring{$E'[p^k]$}{E[pk]}}
\label{sec:impl:torsion}

\paragraph{$p$-Torsion}
For $p\ne2$, \ctwo{} and its variants require to build the extension
$\F_q[c]$ where $c$ is a $(p-1)$-th root of $H_E$. In order to deal with
the lowest possible extension degree, it is a good idea to modify the
curve so that $[\F_q[c]:\F_q]$ is the smallest possible.

$[\F_q[c]:\F_q]$ is invariant under isomorphism, but taking a twist
can save us a quadratic extension. Let $u=c^{-2}$, the curve
\begin{equation*}
  \bar{E} : y^2 = x^3 + a_2ux^2 + a_4u^2x + a_6u^3
\end{equation*}
is defined over $\F_q[c^2]$ and is isomorphic to $E$ over $\F_q[c]$
via $(x,y)\mapsto(\sqrt{u}^2x,\sqrt{u}^3y)$. Its Hasse invariant is
$H_{\bar{E}} = (u)^{\frac{p-1}{2}}H_E = 1$, thus its $p$-torsion
points are defined over $\F_q[c^2]$.

In order to compute the $p^k$-torsion points of $E$ we build
$\F_q[c^2]$, we compute $\bar{P}$ a $p^k$-torsion points of $\bar{E}$
using $p$-descent, then we invert the isomorphism to compute the
abscissa of $P\in E[p^k]$. Since the Cauchy interpolation only needs
the abscissas of $E[p^k]$, this is enough to complete the
algorithm. Scalar multiples of $P$ can be computed without knowledge
of $y(P)$ using \hyperref[rk:montgomery]{Montgomery formulas}.

\pdfmcone{"Kummer surface" -> "Kummer variety".}  Remark that
for $p=2$ we use the same construction in an implicit way since we do
a $p$-descent on the Kummer variety.


\paragraph{$p^k$-Torsion points}
For $p\ne2$ we use Voloch's $p$-descent to compute the $p^k$-torsion
points iteratively as described in Section \ref{sec:C2}. To factor the
Artin-Schreier polynomial \eqref{th:voloch:cover}, we use the
algorithms from Section~\ref{sec:couveignes-algorithm} implemented in
\texttt{FAAST}.

To solve system \eqref{th:voloch:isom} we first compute
\begin{equation*}
  V(x,y) = \left(\frac{g(x)}{h^2(x)}, 
    sy\left(\frac{g(x)}{h^2(x)}\right)'\right)
\end{equation*}
through \titleref{sec:velu-formulas}.\footnote{Vélu formulas compute
  this isogeny up to an indeterminacy on the sign of the ordinate, the
  actual value of $s$ must be determined by composing $V$ with
  $\frobisog$ and verifying that it corresponds to $[p]$ by trying
  some random points.} Recall that we work on a curve having Hasse
invariant $1$, system \eqref{th:voloch:isom} can then be rewritten
\begin{equation*}
  \left\{
    \begin{aligned}
      \tilde{x}(P) &= \frac{g(x)}{h^2(x)}\\
      \tilde{y}(P) &= sy\left(\frac{g(x)}{h^2(x)}\right)'\\
      \tilde{z}(P) &= -2y\frac{h'(x)}{h(x)}
    \end{aligned}
  \right.
\end{equation*}
where $P$ is the point on the cover $C$ that we want to pull back
($\tilde{x}(P)$, $\tilde{y}(P)$ and $\tilde{z}(P)$ are just its
coordinates). After some substitutions this is equivalent to
\begin{equation*}
  \left\{
    \begin{aligned}
      \tilde{x}(P)h^2(x) - g(x) &= 0\\
      \left(\tilde{x}(P)h^2(x) - g(x) - \frac{\tilde{y}(P)}{s\tilde{x}(P)}h^2(x)\right)' &= 0
    \end{aligned}
  \right.
\end{equation*}
Then a solution in $x$ to this system is given by the GCD of the two
equations. Remark that proposition \ref{th:voloch} ensures there is
one unique solution. These formulas are slightly more efficient than
the ones in \cite[$\S$6.2]{lercier-algorithmique}.

For $p=2$ we use the library \texttt{FAAST} (for solving
Artin-Schreier equations) on top of \texttt{gf2x} (for better
performance). There is nothing special to remark about the
$2$-descent.


\subsection{Cauchy interpolation and loop}
\label{sec:impl:cauchy}
The polynomial interpolation step is done as described in Section
\ref{sec:C2-AS-FI}. As a result of this implementation, the polynomial
interpolation algorithm was added to the library \texttt{FAAST}.

The rational fraction reconstruction is implemented using a fast XGCD
algorithm on top of \texttt{NTL} and \texttt{gf2x}. This algorithm was
added to \texttt{FAAST} too.

The loop uses modular composition as in Section~\ref{sec:C2-AS-FI-MC}
in order to minimise the number of interpolations. The timings in the
next section clearly show that this non-asymptotically-optimal variant
performs much faster in practice.

To check that the rational fractions are isogenies we test their
degrees, that their denominator is a square and that they act as group
morphisms on a fixed number of random points. All these checks take a
negligible amount of time compared to the rest of the algorithm.


\subsection{Parallelisation of the loop}
\label{parallel}

The most expensive step of \ctwoasfimc{}, in theory as well as in
practice, is the final loop over the points of $E'[p^k]$. Fortunately,
this phase is very easy to parallelise with very little overhead.

Let $n$ be the number of processors we wish to parallelise on, suppose
that $[\U_k:\F_q]$ is maximal, then we make only one interpolation
followed by $\euler(p^k)/2$ modular compositions.\footnote{If
  $[\U_k:\F_q]$ is not maximal, the parallelisation is
  straightforward: we simply send one interpolation to each processor
  in turn.} We set $m=\left\lfloor\frac{\euler(p^k)}{2n}\right\rfloor$
and we compute the action of $\frobisog^{m}$ on $E[p^k]$ as in
Section~\ref{sec:C2-AS-FI-MC}:
\begin{equation*}
  F^{(m)}(X) = F(X) \circ \cdots \circ F(X) \bmod T(X)
  \;\text{,}
\end{equation*}
this can be done with $\Theta(\log m)$ modular compositions via a
binary square-and-multiply approach as in
Section~\ref{sec:modular-composition}.

Then we compute the $n$ polynomials
\begin{equation*}
  A_{mi}(X) = A_{m(i-1)}(X) \circ F^{(m)}(X) \bmod T(X)
\end{equation*}
and distribute them to the $n$ processors so that they each work on a
separate slice of the $A_i$'s. The only overhead is $\Theta(\log
(\ell/n))$ modular compositions with coefficients in $\F_q$, this is
acceptable in most cases.

\section{Implementation of \ctwoud{}}
\label{sec:implementation-c2-ud}
We modified our \texttt{C++} implementation of \ctwoasfimc{} to obtain
two variants of \ctwoud{}.

The first one takes an integer $k$ and looks for all isogenies of
degree $p^c\ell$ with $\ell$ prime to $p$, $\ell<\euler(p^k)/4$
and $c$ arbitrary. This is done by slightly modifying the modular
composition step of Section~\ref{sec:C2-AS-FI-MC}. Suppose we know an
interpolating polynomial $A_0$, that we view as a morphisms $E[p^k]\ra
E'[p^k]$ such that
\begin{equation}
  \label{eq:184}
  A_0\circ[n](P) = [n](P')
  \quad\text{for any $n$.}
\end{equation}
Then we compose with the Frobenius isogeny $\frobisog$
\begin{equation}
  \label{eq:185}
  A_1 \eqdef \frobisog\circ A_0:E[p^k]\ra {E'}^{(p)}[p^k]
  \text{,}
\end{equation}
where ${E'}^{(p)}$ is the curve
\begin{equation}
  \label{eq:186}
  E^{(p)}\;:\; y^2 = x^3+{a'}^px + {b}'^p
  \text{.}
\end{equation}
So $A_1$ is one of the polynomials that Couveignes algorithm computes
when looking for an isogeny between $E$ and ${E'}^{(p)}$. If $q=p^d$,
iterating $d$ times this construction, we fall back on $E'$, as we
would have if we had directly applied the
\hyperref[sec:curves-over-finite]{Frobenius automorphism} as in
Section~\ref{sec:C2-AS-FI-MC}. Thus, paying an additional factor of
$\log_p q$, we can compute any isogeny of degree $p^c\ell$ with
arbitrary $c$ and $\ell$ bounded as before.

When $\log_pq$ is large, the previous variant becomes unpractical. We
implemented a second variant using the algorithm described in
Section~\ref{sec:C2:non-prime}, this allows to compute any isogeny of
degree $\ell\le\euler(p^k)/4$, even if $p$ divides $\ell$. The
asymptotic cost of this variant is the same as one run of \ctwoasfimc{},
because the search for $\ell$ prime to $p$ dominates.



\section{Implementation of Lercier-Sirvent}
We implemented a Magma prototype of \titleref{alg:bmss}
and \titleref{alg:le-si}. In both cases we did
not take Remark~\ref{rk:bmss} into account, and only implemented the
variant using rational fraction reconstruction. We used Magma native
support for $p$-adics to construct the field $\Q_q$.

Both implementations differ from the description we gave in
Sections~\ref{sec:bmss} and~\ref{sec:lercier-sirvent} in that they
only take the degree $\ell$ and the equation of $E$ as input, and
compute an $\F_q$-isogenous curve, if it exists, by factoring the
modular polynomial in $\F_q[X]$.

\pdfmctwo{Atkin's MODULAR polynomial. Not "canonical".}
Instead of the classical modular polynomials $\Modpol_\ell$ we used
Atkin's modular polynomials $\Modpol^\ast_\ell$ since they have
smaller coefficients and degree; this does not change the other steps
of the algorithm.

The modular polynomials were not computed on the fly as suggested in
Section~\ref{sec:lercier-sirvent}, instead they were taken from the
tables precomputed in Magma. This implies that the implementation has
an asymptotic complexity cubic in $\ell$, however we will see that
even this implementation behaves well in practice.


% Local Variables:
% mode:flyspell
% ispell-local-dictionary:"american"
% mode:TeX-PDF
% TeX-master: "../these"
% mode:reftex
% End:
%
% LocalWords:  Schreier Artin pseudotrace Frobenius bivariate Joux Sirvent FFT
% LocalWords:  Couveignes isogenies Schoof isogeny cryptosystems Lercier Hasse
% LocalWords:  precomputation arithmetics polylogarithmic Karatsuba precomputes
% LocalWords:  endomorphisms 

\section{Benchmarks}
\label{sec:benchmarks}
We ran various experiments to compare the different variants of the
algorithm \ctwo{} between themselves and to \titleref{alg:le-si}. All
the experiments were run on four dual-core Intel Xeon E5520 (2.26GHz),
using the parallelized version of the algorithm in some cases. Magma
2.16 was used to run Magma experiments.

\paragraph{Magma vs. \texttt{FAAST}}
\label{sec:magma-vs.-textttf}
The first set of experiments was run to evaluate the benefits of using
the algorithms of Chapter~\ref{cha:artin-schr-towers}. We selected
pairs of isogenous curves over $\F_{2^{101}}$ such that the height of
the tower is maximal (observe that this is always the case for
cryptographic curves).  We compared the Magma prototype to the
\texttt{FAAST}-based implementation of \ctwoasfimc{} using the
\texttt{zz\_p} and \texttt{GF2} data types (see
Section~\ref{sec:artin-benchmarks}).

\begin{figure}
  \centering
  \includegraphics[width=0.9\textwidth]{isogeny/p2}
  \caption{Comparative timings for different implementations of \ctwoasfimc{} with curves defined over $\F_{2^{101}}$. Plot in logarithmic scale.}
  \label{fig:2-101}
\end{figure}

The results are in figure~\ref{fig:2-101}: we plot a line for the
average running time of the algorithm and bars around it for minimum
and maximum execution times of the final loop. Besides the dramatic
speedup obtained by using the ad-hoc type \texttt{GF2}, the
algorithmic improvements of \texttt{FAAST} over Magma are evident as
even \texttt{zz\_p} is one order of magnitude faster.

\begin{table}
  \centering
  \begin{tabular}{r r r r r r r r}
    \hline
    $\ell$ & $E[p^k]$ & $E'[p^k]$ & FI & RFR & MC & Avg tries & Avg loop time\\
    \hline
    31  &  0.3529 &  0.3529 & 0.3569 & 0.00125 & 0.00055 &  32 &   0.058\\
    61  &  0.9848 &  0.9848 & 0.8268 & 0.00343 & 0.00228 &  64 &   0.365\\
    127 &  2.6636 &  2.6626 & 1.8927 & 0.01090 & 0.00872 & 128 &   2.511\\
    251 &  6.9809 &  6.9779 & 4.2833 & 0.03092 & 0.03494 & 256 &  16.860\\
    397 & 18.1052 & 18.0952 & 9.7385 & 0.07325 & 0.14117 & 512 & 109.783\\
  \end{tabular}
  \caption{Comparative timings for the phases of \ctwoasfimc{} for curves over $\F_{2^{101}}$.}
  \label{tab:C2}
\end{table}

Table~\ref{tab:C2} shows detailed timings for each phase of
\ctwoasfimc{}. The column FI reports the time for one interpolation, the
column MC the time for one modular composition; comparing these two
columns the gain from passing from \ctwoasfi{} to \ctwoasfimc{} is
evident. Columns RFR (rational fraction reconstruction) and MC
constitute the Cauchy interpolation step that is repeated in the final
loop. The last column reports the average time spent in the loop: it
is by far the most expensive phase and this justifies the attention we
paid to FI and MC; only on some huge examples we approached the
crosspoint between these two algorithms.


\paragraph{\ctwoud{}}
\label{sec:c2-ud}
Next we ran experiments on \ctwoud{}. The first observation was that the
heuristic argument --on the probability that a degree sequence not
associated to an isogeny is not normal-- is well verified in practice:
except for a degree $2$ symmetry verified in characteristic $2$,
polynomials not associated to an isogeny very rarely gave a degree
sequence with a gap around the middle.

\pdfmcone{Added some blabla on Teske's cryptosystem.}
Looking for isogenies of unknown degree may be of some cryptographic
significance. For example, Teske's trapdoor cryptosystem selects a
binary field of composite degree ($\F_{2^{7\cdot 23}}$, in the
proposal) and chooses an elliptic curve $E$ vulnerable to the GHS
attack~\cite{gaudry+hess+smart02}. Then hides $E$ by taking a random
path of isogenies of small degrees landing on a curve $E'$ not
vulnerable to GHS, and uses $E'$ as public key. The security of the
cryptosystem comes from the assumption that it is infeasible to find a
GHS-vulnerable curve isogenous to $E'$, without the knowledge of the
isogeny path. 

The \emph{trapdoor} of the cryptosystem is the curve $E$: it is given
to a trusted authority so that --using an isogeny path from $E'$ to
$E$ and a GHS attack-- it has the power of deciphering messages at a
relatively high computational cost. This feature rests on the
assumption that it is feasible, but relatively hard, to compute any
isogeny path from $E$ to $E'$.

In this context, it may be interesting to verify that $E$ and $E'$ are
not related by an isogeny of too low degree.
From~\cite[Appendix~A]{teske06}, we took the two curves defined over
$F_{2^{161}}=\F_2[z]/(z^{161}+z^{18}+1)$ of $j$ invariants:

$1/j = z^{152} + z^{143} + z^{139} + z^{136} + z^{135} + z^{133} +
z^{130} + z^{125} + z^{124} + z^{122} + z^{120} + z^{119} + z^{118} +
z^{117} + z^{116} + z^{114} + z^{113} + z^{112} + z^{110} + z^{109} +
z^{106} + z^{105} + z^{103} + z^{102} + z^{101} + z^{99} + z^{97} +
z^{96} + z^{92} + z^{91} + z^{88} + z^{87} + z^{86} + z^{85} + z^{81}
+ z^{78} + z^{77} + z^{76} + z^{75} + z^{73} + z^{71} + z^{69} +
z^{68} + z^{67} + z^{66} + z^{63} + z^{59} + z^{58} + z^{53} + z^{51}
+ z^{50} + z^{49} + z^{48} + z^{46} + z^{45} + z^{44} + z^{42} +
z^{38} + z^{34} + z^{3} + z^{32} + z^{31} + z^{29} + z^{27} + z^{26} +
z^{24} + z^{23} + z^{22} + z^{21} + z^{20} + z^{19} + z^{18} + z^{17}
+ z^{16} + z^{15} + z^{14} + z^{13} + z^{12} + z^{10} + z^{7} + z^{6}
+ z^{4} + z^{3} + z^{2}$,

$1/j'=z^{160} + z^{156} + z^{155} + z^{153} +z^{152} +z^{151} +z^{150}
+z^{149} +z^{148} +z^{147} +z^{146} +z^{145} +z^{143} +z^{142}
+z^{141} +z^{130} +z^{129} + z^{127} + z^{126} + z^{125} + z^{124} +
z^{123} + z^{120} + z^{118} + z^{112} + z^{109} + z^{104} + z^{103} +
z^{102} + z^{101} + z^{99} + z^{98} +z^{97} +z^{96} +z^{93} +z^{92}
+z^{91} +z^{90} +z^{88} +z^{85} +z^{83} +z^{77} +z^{74} +z^{70}
+z^{68} +z^{65} +z^{64} +z^{63} + z^{62} + z^{61} + z^{60} + z^{58} +
z^{57} + z^{55} + z^{50} + z^{48} + z^{45} + z^{41} + z^{38} + z^{37}
+ z^{36} + z^{33} + z^{31} + z^{30} + z^{27} +z^{26} +z^{24} +z^{23}
+z^{22} +z^{21} +z^{20} +z^{19} +z^{17} +z^{16} +z^{14} +z^{13}
+z^{10} +z^{8} +z^{7} +z^{4} +z^{3} +z$.

We ran our two variants of \ctwoud{} on the two curves to certify the
conjectured property that no unexpected isogeny of low degree exists
between the two curves.

In 258 cpu-hours we were able to prove that no isogeny of degree
$p^c\ell$ for $\ell<2^{11}$ and $c$ arbitrary exists between the two
curves; in 694 cpu-hours we were able to prove that no isogeny of
degree less than $2^{13}$ exists either. We stress the fact that,
albeit of little interest, this computation would have been impossible
without the (surprising) discovery of \ctwoud{}.


\paragraph{Couveignes vs. Lercier-Sirvent}
Finally, we ran experiments on
\titleref{alg:le-si}. 
Table~\ref{tab:ls} shows
timings for the different phases of the algorithm for some isogeny
degrees. The first column is the time spent to find a root of
$\Modpol_\ell(X,j_E)$ in $\F_q$, the second column summarizes the time
spent to lift this root in $\Q_q$ and apply Elkies'
formulas. DiffSolve is the time spent solving the differential
equation, it is clearly the most expensive phase, although not the
most important asymptotically. RFR is the time for rational fraction
reconstruction, its rapid growth is justified by the fact that we
implemented it on top of a quadratic XGCD algorithm.

\begin{table}
  \pdfmcthree{This table has changed.}
  \centering
  \begin{tabular}{r r r r}
    \hline
    $\ell$ & Lift & DiffSolve & RFR\\
    \hline
    31  &   0.570 &   14.830 & 0.010\\
    103 &   5.160 &  274.550 & 0.250\\
    149 &  12.510 &  815.320 & 0.590\\
    239 &  21.420 & 1470.240 & 1.950\\
    331 & 113.500 & 4204.610 & 4.890\\
    389 & 147.340 & 5166.730 & 7.360\\
  \end{tabular}
  \caption{Comparative timings for the phases of \titleref{alg:le-si} 
    for curves over $\F_{3^{64}}$.}
  \label{tab:ls}
\end{table}


\begin{figure}
  \centering
  \includegraphics[height=0.45\textwidth]{isogeny/C2-LS}
  \includegraphics[height=0.45\textwidth]{isogeny/C2-LS2}
  \caption{Comparative timings for \ctwoasfimc{} (C2) and
    \titleref{alg:le-si} (LS) over different curves. Plot in
    logarithmic scale.}
  \label{fig:comp}
\end{figure}

We also compared the running times of \ctwoasfimc{} and
\titleref{alg:le-si} over curves of half the cryptographic size in
figure~\ref{fig:comp} (left) and five times the cryptographic size in
figure~\ref{fig:comp} (right). We only plot average times for \ctwo{},
in characteristic $2$ we only plot the timings for \texttt{GF2}. From
the plot it is clear that \ctwoasfimc{} only performs better than
\titleref{alg:le-si} for $p=2$, but in this case Lercier's
algorithm~\cite{lercier96} is much faster.  Contradicting theory, the
asymptotic behavior of \titleref{alg:le-si} looks worse than the one
of \ctwoasfimc{}; however comparing a Magma prototype to our highly
optimized implementation of \ctwoasfimc{} is somewhat unfair.

Furthermore, it is unlikely that \ctwoasfimc{} could be practical for
any $p>3$ because of its high dependence on $p$, while
\titleref{alg:le-si} scales pretty well with the characteristic as
shown in figure~\ref{fig:LSp}.

Considering that the asymptotic dependency of Couveignes' algorithm in
$\log q$ and in $p$ is worse than the one of \titleref{alg:le-si}
(compare Eq.~\eqref{eq:interp} to
Proposition~\ref{th:lercier-sirvent}), there are very few regions
where Couveignes' algorithm stays of practical or theoretical
interest.

\pdfmcone{People don't like pessimism.}  Ironically, the
techniques presented in this document were developed in view of an
efficient implementation of Couveignes' algorithm, but, for the
moment, their only practical application seems to be \ctwoud{}. Our hope
is that other interesting applications may be found in the future.

\begin{figure}
  \centering
  \includegraphics[width=0.9\textwidth]{isogeny/LSp}
  \caption{Timings for \titleref{alg:le-si} for different fields. We
    increase $p$ while keeping constant $d$ and the isogeny degree.}
  \label{fig:LSp}
\end{figure}


% Local Variables:
% mode:flyspell
% ispell-local-dictionary:"american"
% TeX-master: "../these"
% mode: TeX-PDF
% mode:reftex
% End:
%
% LocalWords:  Schreier Artin pseudotrace Frobenius bivariate Joux Sirvent FFT
% LocalWords:  Couveignes isogenies Schoof isogeny cryptosystems Lercier
% LocalWords:  precomputation arithmetics polylogarithmic Karatsuba precomputes
% LocalWords:  endomorphisms  isogenous



% Local Variables:
% mode:flyspell
% ispell-local-dictionary:"american"
% mode:TeX-PDF
% mode:reftex
% TeX-master: "../these"
% End:
%



%\backmatter
\appendix
\pdfbookmark[-1]{Appendices}{part:appendices}
\part*{Appendices}

\subsection{Coevaluation}
When dealing with a construction in category theory, it is natural to
simultaneously study its dual, that is the construction obtained by
\emph{reversing all the arrows}. If in definition \ref{def:eval} we
substitute the product $\prod^nR$ by its dual $\coprod^nR$, called
\emph{coproduct}, we obtain a new way of evaluating an arithmetic
circuit that we will call \emph{coevaluation}. We study here the
properties of coevaluation, its interest will be clear in the next
sections.

In this context, we will make an abuse by using the same notation
$R^n$ we used for the product to signify the coproduct $\coprod^nR$ in
the category. Whether $R^n$ is product or coproduct will always be
clear from the context.

\begin{definition}[Arithmetic co-operator, cobasis]
  Let $R$ be a ring. An arithmetic co-operator over $R$ is a function
  $f:R^i\ra R^o$ for some $i,o\in\N$; here $R^n$ is coproduct.

  An arithmetic $R$-cobasis is a set of arithmetic co-operators over
  $R$.
\end{definition}

In particular when the category is $\mathsf{Set}$ the coproduct is the
disjoint union of sets, thus the bases $\Sbasis$ and $\Tbasis$ make no
sense in this context.

The definitions of node and circuit naturally extend to cobases, but
we need to define a new evaluation for arithmetic circuits defined
over them.

\begin{definition}[coevaluation of an arithmetic circuit]
  \label{def:coeval}
  Let $C$ be an arithmetic circuit with $i$ inputs and $o$ outputs
  over a cobasis $\mathcal{B}$. Its coevaluation is a function
  $\lave_C:R^i\ra R^o$.

  We use the same notation as in definition \ref{def:eval}. As we did
  there, we simultaneously define $\lave_v$ for each $v\in V$ and
  $\lave_e$ for each $e\in E$.
  \begin{itemize}
  \item Let $v\in V$ have in-degree $m$, let its coevaluation be
    $\lave_v:R^m\ra R^o$ and let $\iota_1,\ldots,\iota_n$ be the
    canonical injections from $R$ to $R^m$. Let $i_1<_v\cdots<_vi_m$
    be the input ports of $v$ and let
    $e_j=\bigl(i_j,E^{-1}(i_j)\bigr)$ be the corresponding edges
    incident to $v$, then $\lave_{e_j} = \lave_v\circ\iota_j$ for any
    $j$.
  \item Let $y_1<_V\cdots<_Vy_n$ be the output nodes and let
    $\iota_1,\ldots,\iota_o$ be the canonical injections from $R$ to
    $R^o$, then $\lave_{y_j}=\iota_j$ for any $j$.
  \item For every evaluation node $v$ with out-degree $n$, let
    $o_1<_v\cdots<_vo_n$ be the output ports of $v$ and let
    $e_j=\bigl(E(o_j),o_j\bigr)$ be the corresponding edges
    stemming from $v$, then
    \begin{equation}
      \label{eq:lave_v}
      \lave_v = (\lave_{e_1} \oplus \cdots \oplus \lave_{e_n}) \circ \beta(v) 
      \text{.}
    \end{equation}
  \item For every input node $x$, let $e\in E$ be the only edge
    stemming from $x$, then $\lave_x=\lave_e$.
  \end{itemize}

  We can finally define $\lave_C:R^i\ra R^o$. Let $x_1<_V\cdots<_Vx_i$
  be the input nodes, then
  \begin{equation}
    \label{eq:lave}
    \lave_C = \lave_{x_1} \oplus \cdots \oplus \lave_{x_i}
    \text{.}
  \end{equation}
\end{definition}

As before, the sums of equations \eqref{eq:lave_v} and \eqref{eq:lave}
are formally defined via the universal property of the coproduct.

The coevaluation in general does not attach the same semantics to a
circuit as the evaluation. For example in the case of $\mathsf{Set}$
the coevaluation is a function from the disjoint union of $i$ copies
of $R$ to the disjoint union of $o$ copies of $R$. We can regard
circuits over cobases in $\mathsf{Set}$ as objects that are fed one
single element of $R$ on one out of their $n$ inputs and then take
decisions depending on which input was fed. An example is given in
figure \ref{fig:coffee}.

\begin{figure}[!ht]
  \centering
  
  \begin{tikzpicture}
    \tikzstyle{node}=[circle,thick,draw=black,minimum size=4mm]
    \tikzstyle{arg}=[rectangle,thin,draw=black,minimum size=4mm]

    \begin{scope}
      \node[arg](in){$x$};

      \node[node,below of=in](s10){$\ge_{10}$};

      \node[node,right of=s10,xshift=2mm](s20){$\ge_{20}$};
      \node[arg,below of=s10](o10){$10$c};

      \node[node,right of=s20,xshift=2mm](s50){$\ge_{50}$};
      \node[arg,below of=s20](o20){$20$c};

      \node[node,right of=s50,xshift=2mm](s1){$\ge_{100}$};
      \node[arg,below of=s50](o50){$50$c};

      \node[arg,below of=s1](o1){$1$\euro};
      \node[arg,right of=o1](o2){$2$\euro};

      \path[->]
      (in) edge (s10)
      (s10) edge (o10)
      (s10) edge (s20)
      (s20) edge (o20)
      (s20) edge (s50)
      (s50) edge (o50)
      (s50) edge (s1)
      (s1) edge (o1)
      (s1) edge (o2);
    \end{scope}
  \end{tikzpicture}  
  
  \caption{The coffee machine circuit. On input $r\in R$, the operator
    $\ge_x:R\ra R\uplus R$ gives $r$ on its first output if $r\ge x$, on its
    second output otherwise. The circuit is an euro coin separator.}
  \label{fig:coffee}
\end{figure}

In some cases, howevever, evaluation and coevaluation coincide. The
following lemma shows one important case when this happens.

\begin{lemma}
  \label{th:coeval}
  Let $C$ be a circuit over $(R,\Tbasis)$. In the category $RMod{R}$
  $\eval_C\simeq\lave_C$.
\end{lemma}
\begin{proof}
  For finite dimensional modules, the product and the direct sum are
  the same object. More formally, when working in $RMod{R}$ there is
  a natural isomorphism $\prod^n R\simeq\coprod^n R$ for any $n$ (this
  is true for any additive category, see \cite[VIII.2]{McLane}). Thus
  $\Tbasis$ is both a basis and a cobasis, up to isomorphism, and both
  evaluation and coevaluation of circuits over it are meaningful.

  The rest of the proof is just induction on the size of the
  circuit. First, it is obvious that for elementary circuits with an
  unique evaluation node $v$ we have $\eval_C \simeq \beta(v) \simeq
  \lave_C$. Then it suffices to show that the property is maintained
  upon composition of circuits.
\end{proof}

The equivalence of evaluations and coevalutions suggests that there is
some unexploited symmetry in circuits over $\Tbasis$. The next
section explores it.




% Local Variables:
% mode:flyspell
% ispell-local-dictionary:"american"
% mode:TeX-PDF
% mode:reftex
% TeX-master: "../these"
% End:
%

\chapter{Linearity inference of karatsuba multiplication}
\label{cha:line-infer-karats}
\lstset{language=haskell}

\begin{lstlisting}
data L = L Integer
data S = S Integer

class Ring r where
  zero :: r
  (<+>) :: r -> r -> r
  (<*>) :: r -> S -> r
  neg :: r -> r
  
class Ring r => Module m r | m -> r where
  zeroM :: m
  (<<*) :: m -> S -> m
  (>>>) :: m -> Integer -> r
  (<<<) :: r -> Integer -> m
  (<++>) :: m -> m -> m
  add :: m -> m -> Integer -> m
  add a b n = foldl (<++>) zeroM 
              [((a>>>i) <+> (b>>>i))<<<i | i <- [1..n]]
  
instance Ring L where
  zero = L 0
  (L x) <+> (L y) = L (x+y)
  (L x) <*> (S y) = L (x*y)
  neg (L x) = L (-x)
  
instance Ring S where
  zero = S 0
  (S x) <+> (S y) = S (x+y)  
  (S x) <*> (S y) = S (x*y)
  neg (S x) = S (-x)

one = S 1

instance Ring r => Module [r] r where
  zeroM = [zero]
  [] <<* x = []
  (x:xs) <<* y = (x <*> y):(xs <<* y)
  [] >>> i = zero
  (x:xs) >>> i =
    if i < 1 then zero else if i == 1 then x else xs >>> (i-1)
  x <<< i = if i <= 1 then [x] else zero:(x <<< (i-1))
  [] <++> [] = []
  (x:xs) <++> [] = x:(xs <++> [])
  [] <++> (y:ys) = y:([] <++> ys)
  (x:xs) <++> (y:ys) = (x <+> y):(xs <++> ys)
  add [] [] n = []
  add [] (y:ys) n =
    if n > 0 then y:(add [] ys (n-1)) else []
  add (x:xs) [] n =
    if n > 0 then x:(add xs [] (n-1)) else []
  add (x:xs) (y:ys) n =
    if n > 0 then (x<+>y):(add xs ys (n-1)) else []


-- Karatsuba multiplication : the system will infer
-- shift :: Ring r => [r] -> Integer -> [r]
-- split :: Ring r => [r] -> Integer -> ([r], [r])
-- kara :: Ring r => [r] -> [S] -> Integer -> [r]

shift x n = if n <= 0 then x else shift (zero:x) (n-1)

split [] n = ([], [])
split (x:xs) n =
  if n <= 0
  then ([], x:xs)
  else let (a, b) = split xs (n-1) in (x:a, b)

kara [] y n = []
kara x [] n = []
kara x y n =
  if n <= 0
  then []
  else if n == 1 
       then [(x!!0) <*> (y!!0)]
       else 
         let h = n `div` 2 in
         let (a0, a1) = split x h in
         let (b0, b1) = split y h in
         let x0 = kara a0 b0 h in
         let x2 = kara a1 b1 (n-h) in
         let xx1 = kara (a1 <++> a0) (b1 <++> b0) (n-h) in
         let x1 = xx1 <++> ((x0 <++> x2) <<* (neg one)) in
         (shift x2 n) <++> (shift x1 h) <++> x0
\end{lstlisting}                    
                  

% Local Variables:
% mode:flyspell
% ispell-local-dictionary:"american"
% mode:TeX-PDF
% mode:reftex
% TeX-master: "../these"
% End:
%

\chapter{Proof of Vélu's formulas}
\label{cha:proof-velus-formulas}

\pdfmcone{Changed person.}  We always had admiration for our
colleagues who can develop by hand two pages full of calculations
without making mistakes. When it comes to us, we usually make a sign
mistake at the third term. Tired of having to check for sign errors in
other people's papers any time we had to use Vélu formulas, we decided
to make an automatic proof of it.

The following Magma code proves the passage from Eq.~\eqref{eq:155} to
Eq.~\eqref{eq:161} and from there to~\eqref{eq:157}.

\begin{xcomment}{lstlisting}
\chapter{Proof of Vélu's formulas}
\label{cha:proof-velus-formulas}

\pdfmcone{Changed person.}  We always had admiration for our
colleagues who can develop by hand two pages full of calculations
without making mistakes. When it comes to us, we usually make a sign
mistake at the third term. Tired of having to check for sign errors in
other people's papers any time we had to use Vélu formulas, we decided
to make an automatic proof of it.

The following Magma code proves the passage from Eq.~\eqref{eq:155} to
Eq.~\eqref{eq:161} and from there to~\eqref{eq:157}.

\begin{xcomment}{lstlisting}
\chapter{Proof of Vélu's formulas}
\label{cha:proof-velus-formulas}

\pdfmcone{Changed person.}  We always had admiration for our
colleagues who can develop by hand two pages full of calculations
without making mistakes. When it comes to us, we usually make a sign
mistake at the third term. Tired of having to check for sign errors in
other people's papers any time we had to use Vélu formulas, we decided
to make an automatic proof of it.

The following Magma code proves the passage from Eq.~\eqref{eq:155} to
Eq.~\eqref{eq:161} and from there to~\eqref{eq:157}.

\begin{xcomment}{lstlisting}
\input{isogeny/veluproof.mgm}
\end{xcomment}

The first and second line of output are the differences between each
term of the sums in Eqs.~\eqref{eq:155} and~\eqref{eq:161}. In both
lines, to conclude one must observe that all the terms in the
difference contain an odd power of $y(Q)$, thus they sum up to $0$
over $G^\ast$.

The third line is the difference between each term of the sums in
Eqs.~\eqref{eq:161} and~\eqref{eq:157}. The result is
self-explanatory.


% Local Variables:
% mode:flyspell
% ispell-local-dictionary:"american"
% mode:TeX-PDF
% mode:reftex
% TeX-master: "../these"
% End:

\end{xcomment}

The first and second line of output are the differences between each
term of the sums in Eqs.~\eqref{eq:155} and~\eqref{eq:161}. In both
lines, to conclude one must observe that all the terms in the
difference contain an odd power of $y(Q)$, thus they sum up to $0$
over $G^\ast$.

The third line is the difference between each term of the sums in
Eqs.~\eqref{eq:161} and~\eqref{eq:157}. The result is
self-explanatory.


% Local Variables:
% mode:flyspell
% ispell-local-dictionary:"american"
% mode:TeX-PDF
% mode:reftex
% TeX-master: "../these"
% End:

\end{xcomment}

The first and second line of output are the differences between each
term of the sums in Eqs.~\eqref{eq:155} and~\eqref{eq:161}. In both
lines, to conclude one must observe that all the terms in the
difference contain an odd power of $y(Q)$, thus they sum up to $0$
over $G^\ast$.

The third line is the difference between each term of the sums in
Eqs.~\eqref{eq:161} and~\eqref{eq:157}. The result is
self-explanatory.


% Local Variables:
% mode:flyspell
% ispell-local-dictionary:"american"
% mode:TeX-PDF
% mode:reftex
% TeX-master: "../these"
% End:


\renewcommand{\nomname}{List of Symbols}
\printnomenclature
\clearpage
\phantomsection
\addcontentsline{toc}{chapter}{\indexname}
\printindex
\bibliographystyle{amsalpha}
\bibliography{defeo}

\end{document}


% Local Variables:
% mode:TeX-PDF
% mode:reftex
% End:
%
