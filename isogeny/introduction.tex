\section{Introduction}

The problem of computing an explicit degree $\ell$ isogeny between two
given elliptic curves over $\F_q$ was originally motivated by point
counting methods based on Schoof's algorithm \cite{Atk91},
\cite{Elk91}, \cite{Sch95}. A review of the most efficient algorithms
to solve this problem is given in \cite{BoMoSaSc08} together with a
new quasi-optimal algorithm; however, all the algorithms presented in
\cite{BoMoSaSc08} are limited to the case $\ell\ll p$ where $p$ is the
characteristic of $\F_q$. This is satisfactory for cryptographic
applications where one takes $p=q$ or $p=2$; indeed in the former case
Schoof's algorithm needs $\ell\in O(\log p)$, while in the latter case
there's no need to compute explicit isogenies since $p$-adic methods
based on \cite{Sat00} are preferred to Schoof's algorithm.

Nevertheless, the problem of computing explicit isogenies in the case
where $p$ is small compared to $\ell$ remains of theoretical interest
and can find practical applications in newer cryptosystems such as
\cite{Tes06}, \cite{RoSt06}. The first algorithm to solve this problem
was given by Couveignes and made use of formal groups \cite{Cou94}; it
takes $\tildO(\ell^3\log q)$ operations in $\F_p$ assuming $p$ is
constant, however it has an exponential complexity in $\log
p$. Another algorithm by Lercier specific to $p=2$ uses some linear
properties of the problem to build a linear system from whose solution
the isogeny can be deduced \cite{Ler96}; its complexity is conjectured
to be $\tildO(\ell^3\log q)$ operations in $\F_p$, but it has a much
better constant factor than \cite{Cou94}. At the moment we write, the
latter algorithm is by many orders of magnitude the fastest algorithm
to solve practical instances of the problem when $p=2$, thus being the
\emph{de facto} standard for cryptographic use.

$p$-adic methods were used by Joux and Lercier \cite{JL06} and Lercier
and Sirvent \cite{LeSi09} to solve the isogeny problem. The former
method has complexity $\tildO(\ell^2(1 + \ell/p)\log q)$ operations in
$\F_p$, which makes it well adapted to the case where $p\sim\log
q$. The latter has complexity $\tildO(\ell^3 + \ell\log q^2)$
operations in $\F_p$, making it the best algorithm to our knowledge
for the case where $p$ is not constant.

\paragraph{The algorithm C2 and its variants}
Finally, the algorithm having the best asymptotic complexity in $\ell$
was proposed again by Couveignes in \cite{Cou96}; we will refer to
this original version as ``C2''\footnote{As opposed to the algorithm
  presented in \cite{Cou94}, an algorithm ``C2'' shares many
  similarities with.}. Its complexity --supposing $p$ is fixed-- was
estimated in \cite{Cou96} as being $\tildO(\ell^2\log q)$ operations
in $\F_p$, but with a precomputation step requiring $\tildO(\ell^3\log
q)$ operations and large memory requirements. However, some more work
is needed to effectively reach these bounds, while a straightforward
implementation of C2 has an overall asymptotic complexity of
$\tildO(\ell^3\log q)$ operations, as we will argue in Section
\ref{sec:C2}.

Subsequent work by Couveignes \cite{Cou00}, and more recently
\cite{DFS09}, use Artin-Schreier theory to avoid the precomputation
step of C2 and drop the memory requirements to $\tildO(\ell\log q +
\log^2 q)$ elements of $\F_p$. However, this is still not enough to
reduce the overall complexity of the algorithm, as we will argue in
Section \ref{sec:C2-AS}. We refer to this variant as ``C2-AS''.

In the present paper we give a complete review of Couveignes'
algorithm, we present new variants that reach the foreseen quadratic
bound in $\ell^2$ and prove an accurate complexity estimate which
doesn't suppose $p$ to be fixed. We also run experiments to compare
the performances of C2 with other algorithms.

\paragraph{Notation and plan}
In the rest of the paper $p$ is a prime, $d$ a positive integer,
$q=p^d$ and $\F_q$ is the field with $q$ elements. For an elliptic
curve $E$ and a field $\K$ embedded in an algebraic closure
$\clot{\K}$, we note by $E(\K)$ the set of $\K$-rational points and by
$E[m]$ the $m$-torsion subgroup of $E(\clot{\K})$. The group law on
the elliptic curve is noted additively, its zero is the point at
infinity, noted $\0$. For an affine point $P$ we note by $x(P)$ its
abscissa and by $y(P)$ its ordinate. We will restrict ourselves to the
case of ordinary elliptic curves, thus $E[p^k]\isom\Z/p^k\Z$.

Unless otherwise stated, all time complexities will be measured in
number of operations in $\F_p$ and all space complexities in number of
elements of $\F_p$; we do not assume $p$ to be constant. We use the
$O$, $\Theta$ and $\Omega$ notations to state respectively upper
bounds, tight bounds and lower bounds for asymptotic complexities. We
also use the notation $\tildO_x$ that forgets polylogarithmic factors
in the variable $x$, thus $O(xy\log x \log y)\subset\tildO_x(xy\log
y)\subset\tildO_{x,y}(xy)$. We simply note $\tildO$ when the variables
are clear from the context.

We let $2<\omega\le3$ be the exponent of linear algebra, that is an
integer such that $n\times n$ matrices can be multiplied in $n^\omega$
operations. We let $\Mult:\N\rightarrow\N$ be a \emph{multiplication
  function}, such that polynomials of degree at most $n$ with
coefficients in $\F_p$ can be multiplied in $\Mult(n)$ operations,
under the conditions of \cite[Ch. 8.3]{vzGG}. Typical orders of
magnitude are $O(n^{\log_23})$ for Karatsuba multiplication or
$O(n\log n\log\log n)$ for FFT multiplication. Similarly, we let
$\ModComp:\N\rightarrow\N$ be the complexity of \emph{modular
  composition}, that is a function such that $\ModComp(n)$ is the
number of field operations needed to compute $f\circ g\bmod h$ for
$f,g,h\in\K[X]$ of degree at most $n$ with coefficients in an
arbitrary field $\K$. The best known algorithm is~\cite{BrKu78}, this
implies $\ModComp(n)\in O\left(n^{\frac{\omega+1}{2}}\right)$. Note
that in a boolean RAM model, the algorithm of~\cite{KeUm08} takes
quasi-linear time.

\paragraph{Organisation of the paper}
In Section~\ref{sec:preliminaries} we give preliminaries on elliptic
curves and isogenies. In Sections~\ref{sec:C2}
through~\ref{sec:C2-AS-FI-MC} we develop the algorithm C2 and we
incrementally improve it by giving a new faster variant in each
Section. Section~\ref{sec:implementation} gives technical details on
our implementations of the algorithms of this paper and
of~\cite{LeSi09}. Finally in Section~\ref{sec:benchmarks} we comment
the results of the experiments we ran on our implementations.



% Local Variables:
% mode:flyspell
% ispell-local-dictionary:"british"
% mode:TeX-PDF
% TeX-master: "ec-isogeny"
% End:
%
% LocalWords:  Schreier Artin pseudotrace frobenius bivariate Joux Sirvent FFT
% LocalWords:  Couveignes isogenies Schoof isogeny cryptosystems Lercier
% LocalWords:  precomputation arithmetics polylogarithmic Karatsuba
