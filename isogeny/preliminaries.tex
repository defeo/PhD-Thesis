\section{Preliminaries on Isogenies}
\label{sec:preliminaries}

Let $E$ be an ordinary elliptic curve over the field $\F_q$. We
suppose it is given to us as the locus of zeroes of an affine
Weierstrass equation
\[y^2 + a_1xy + a_3y = x^3 + a_2x^2 + a_4x + a_6
\qquad a_1,\ldots,a_6\in\F_q\text{.}\]

\paragraph{Simplified forms} If $p>3$ it is well known that the curve
$E$ is isomorphic to a curve in the form
\begin{equation}
  \label{eq:weierstrass>3}
  y^2 = x^3 + ax + b
\end{equation}
and its $j$-invariant is $j(E) = \frac{1728(4a)^3}{16(4a^3 + 27b^2)}$.

When $p=3$, since $E$ is ordinary, it is isomorphic to a curve
\begin{equation}
  \label{eq:weierstrass=3}
  y^2 = x^3 + ax^2 + b
\end{equation}
and its $j$-invariant is $j(E) = -\frac{a^3}{b}$.

Finally, when $p=2$, since $E$ is ordinary, it is isomorphic to a curve
\begin{equation}
  \label{eq:weierstrass=2}
  y^2 + xy = x^3 + ax^2 + b
\end{equation}
and its $j$-invariant is $j(E) = \frac{1}{b}$.

These isomorphism are easy to compute and we will always assume that
the elliptic curves given to our algorithms are in such simplified
forms.

\paragraph{Isogenies}
Elliptic curves are endowed with the classic group structure through
the chord-tangent law. A group morphism having finite kernel is called
an \emph{isogeny}. Isogenies are regular maps, as such they can be
represented by rational functions. An isogeny is said to be
$\K$-rational if it is $\K$-rational as regular map; its degree is the
degree of the regular map.

One important property about isogenies is that they factor the
multiplication-by-$m$ map.

\begin{definition}[Dual isogeny]
  Let $\I : E \rightarrow E'$ be a degree $m$ isogeny. There exists an
  unique isogeny $\hat{\I} : E' \rightarrow E$, called the \emph{dual
    isogeny} such that
  \[\I\circ\hat{\I} = [m]_E \qquad\text{and}\qquad \hat{\I}\circ\I =
  [m]_{E'}\]
\end{definition}

As regular maps, isogenies can be separable, inseparable or purely
inseparable. In the case of finite fields, purely inseparable
isogenies are easily understood as powers of the frobenius map. Let
\[E^{(p)} : y^2 + a_1^pxy + a_3^py = x^3 + a_2^px^2 + a_4^px + a_6^p\]
then the map
\begin{align*}
  \frobisog : E &\rightarrow E^{(p)}\\
          (x,y) &\mapsto (x^p,y^p)
\end{align*}
is a degree $p$ purely inseparable isogeny. Any purely inseparable
isogeny is a composition of such frobenius isogenies.

Let $E$ and $E'$ be two elliptic curves defined over $\F_q$, by
finding an \emph{explicit isogeny} we mean to find an
($\F_q$-rational) rational function from $E(\clot{\F}_q)$ to
$E'(\clot{\F}_q)$ such that the map it defines is an isogeny.

\begin{figure}
  \centering
  \[\xymatrix{
    E \ar[r]^{[m]}\ar@/_1pc/[rrr]_{\I'} & E \ar[r]^\I & E' \ar[r]^{\frobisog^n} & E'^{(p^n)}\\
  }\]
  \label{fig:fact}
  \caption{Factorization of an isogeny. $\I'$ has kernel $E[m]\oplus\ker\I$.}
\end{figure}

Since an isogeny can be uniquely factored in the product of a
separable and a purely inseparable isogeny, we focus ourselves on the
problem of computing explicit separable isogenies. Furthermore one can
factor out multiplication-by-$m$ maps, thus reducing the problem to
compute explicit separable isogenies with cyclic kernel (see figure
\ref{fig:fact}).

In the rest of this paper, unless otherwise stated, by $\ell$-isogeny
we mean a separable isogeny with kernel isomorphic to $\Z/\ell\Z$.

\paragraph{Vélu formulae}
For any finite subgroup $G \subset E(\clot{\K})$, Vélu formulae
\cite{Vel71} give in a canonical way an elliptic curve $\bar{E}$ and
an explicit isogeny $\I:E\rightarrow \bar{E}$ such that
$\ker\I=G$. The isogeny is $\K$-rational if and only if the polynomial
vanishing on the abscissae of $G$ belongs to $\K[X]$.

In practice, if $E$ is defined over $\F_q$ and if
\[h(X) = \prod_{\substack{P\in G\\P\ne\0}}(X - x(P)) \in \F_q[X]\]
is known, Vélu formulae compute a rational function
\begin{equation}
  \label{eq:isog}
  \bar{\I}(x,y) = \left(\frac{g(x)}{h(x)}, \frac{k(x,y)}{l(x)}\right)  
\end{equation}
and a curve $\bar{E}$ such that $\bar{\I} : E\rightarrow\bar{E}$ is an
$\F_q$-rational isogeny of kernel $G$. A consequence of Vélu formulae
is
\begin{equation}
  \label{eq:velu-deg}
  \deg g = \deg h + 1 = \card{G}
  \text{.}
\end{equation}

Given two curves $E$ and $E'$, Vélu formulae reduce the problem of
finding an explicit isogeny between $E$ and $E'$ to that of finding
the kernel of an isogeny between them. Once the polynomial $h(X)$
vanishing on $\ker\I$ is found, the explicit isogeny is computed
composing Vélu formulae with the isomorphism between $\bar{E}$ and
$E'$ as in figure \ref{fig:velu}.

\begin{figure}
  \centering
  \[\xymatrix{
    E \ar[r]^{\bar{\I}} \ar[rd]^\I & \bar{E} \ar[d]^{\simeq}\\
    & E'
  }\]
  \caption{Using Vélu formulae to compute an explicit isogeny.}
  \label{fig:velu}
\end{figure}




% Local Variables:
% mode:flyspell
% ispell-local-dictionary:"british"
% mode:TeX-PDF
% TeX-master: "ec-isogeny"
% End:
%
% LocalWords:  Schreier Artin pseudotrace frobenius bivariate Joux Sirvent FFT
% LocalWords:  Couveignes isogenies Schoof isogeny cryptosystems Lercier
% LocalWords:  precomputation arithmetics polylogarithmic Karatsuba
% LocalWords:  endomorphisms
