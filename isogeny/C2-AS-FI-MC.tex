%% these.tex
%% Copyright 2010 Luca De Feo
%% All rights reserved


\section{The algorithm \alg{C2-AS-FI-MC}}
\label{sec:C2-AS-FI-MC}

However asymptotically fast, the polynomial interpolation step is
quite expensive for reasonably sized data. Instead of repeating it
$\frac{\euler(p^k)}{2}$ times, one can use composition with the
Frobenius endomorphism $\frobisog_E$ in order to reduce the number of
interpolations in the final loop. We call this variant \ctwoasfimc{} (MC
for Modular Composition).

\subsection{The algorithm}
Suppose we have computed, by the algorithm of the previous Section,
the polynomial $T$ vanishing on the abscissas of $E[p^k]$ and an
interpolating polynomial $A_0\in\F_q[X]$ such that
\begin{equation*}
  A_0\bigl(x\bigl([n]P\bigr)\bigr) = x\bigl([n]P'\bigr)
  \quad\text{for any $n$.}
\end{equation*}
We view $A_0$ as a morphism of varieties (not necessarily an isogeny!)
from $E[p^k]$ to $E'[p^k]$.

The \hyperref[sec:curves-over-finite]{Frobenius automorphism}
$\frobisog_E$ acts on $E[p^k]$ permuting its points. We define
\begin{equation}
  \label{eq:183}
  A_1\eqdef A_0\circ\frobisog_E = \frobisog_{E'}\circ A_0
  \text{,}
\end{equation}
where the equality comes from the fact that $A_0$ is
$\F_q$-rational. Hence
\begin{equation*}
  A_1\circ[n](P) = [n]\circ\frobisog_{E'}(P')
  \quad\text{for any $n$.}
\end{equation*}
$A_1$ has no poles at $E[p^k]$, thus it is a polynomial map; and since
$\frobisog_{E'}(P')$ is a generator of $E'[p^k]$, $A_1$ is one of the
polynomials that the algorithm \ctwo{} tries to identify to an isogeny. By
iterating this construction we obtain $[\U_k:\F_q]/2$ different
polynomials $A_i$ for the algorithm \ctwo{} with only one interpolation.

To compute the $A_i$'s, we first compute $F\in\F_q[X]$
\begin{equation}
  \label{eq:frob}
  F(X) = X^q \bmod T(X)
  \text{,}
\end{equation}
then for any $1\le i<[\U_k:\F_q]/2$
\begin{equation}
  \label{eq:modcomp}
  A_i(X) = A_{i-1}(X)\circ F(X) \bmod T(X)\text{.}
\end{equation}

If $\frac{\euler(p^k)}{[\U_k:\F_q]} = p^{i_0-1}r$, we must compute
$p^{i_0-1}r$ polynomial interpolations and apply this algorithm to
each of them in order to deduce all the polynomials needed by \ctwo{}.


\subsection{Complexity analysis}
We compute \eqref{eq:frob} via square-and-multiply, this costs
$\Theta(d\Mult(p^kd)\log p)$ operations. Each application of
\eqref{eq:modcomp} is done via a
\hyperref[sec:modular-composition]{modular composition}, the cost is
thus $O(\ModComp(p^k))$ operations in $\F_q$, that is
$O(\ModComp(p^k)\Mult(d))$ operations in $\F_p$. Using the algorithm
of~\cite{kedlaya+umans08} for modular composition, the complexity of
\ctwoasfimc{} would not be essentially different from the one of
\ctwoasfi{}. However, in practice the fastest algorithm for modular
composition is~\cite{brent+kung}, and in particular the variant
in~\cite[Lemma~3]{kaltofen+shoup98}: these have worse asymptotic
complexity, but perform better on the instances we treat in
Section~\ref{sec:benchmarks}.

Notice that a similar approach could be used inside the polynomial
interpolation step (see Section \ref{sec:C2-AS-FI}) to deduce
$A_k^{(0)}$ from $A_0^{(0)}$ using modular composition with the
multiplication maps of $E$ and $E'$ as described
in~\cite[$\S$2.3]{couveignes96}. This variant, though, has an even
worse complexity because of the cost of computing multiplication maps.




% Local Variables:
% mode:flyspell
% ispell-local-dictionary:"american"
% mode:TeX-PDF
% mode:reftex
% TeX-master: "../these"
% End:
%
% LocalWords:  Schreier Artin pseudotrace Frobenius bivariate Joux Sirvent FFT
% LocalWords:  Couveignes isogenies Schoof isogeny cryptosystems Lercier moduli
% LocalWords:  precomputation arithmetics polylogarithmic Karatsuba embeddings
% LocalWords:  irreducibility
