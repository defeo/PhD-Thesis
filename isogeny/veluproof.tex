\chapter{Proof of Vélu's formulas}
\label{cha:proof-velus-formulas}

\pdfmcone{Changed person.}  We always had admiration for our
colleagues who can develop by hand two pages full of calculations
without making mistakes. When it comes to us, we usually make a sign
mistake at the third term. Tired of having to check for sign errors in
other people's papers any time we had to use Vélu formulas, we decided
to make an automatic proof of it.

The following Magma code proves the passage from Eq.~\eqref{eq:155} to
Eq.~\eqref{eq:161} and from there to~\eqref{eq:157}.

\begin{xcomment}{lstlisting}
\chapter{Proof of Vélu's formulas}
\label{cha:proof-velus-formulas}

\pdfmcone{Changed person.}  We always had admiration for our
colleagues who can develop by hand two pages full of calculations
without making mistakes. When it comes to us, we usually make a sign
mistake at the third term. Tired of having to check for sign errors in
other people's papers any time we had to use Vélu formulas, we decided
to make an automatic proof of it.

The following Magma code proves the passage from Eq.~\eqref{eq:155} to
Eq.~\eqref{eq:161} and from there to~\eqref{eq:157}.

\begin{xcomment}{lstlisting}
\chapter{Proof of Vélu's formulas}
\label{cha:proof-velus-formulas}

\pdfmcone{Changed person.}  We always had admiration for our
colleagues who can develop by hand two pages full of calculations
without making mistakes. When it comes to us, we usually make a sign
mistake at the third term. Tired of having to check for sign errors in
other people's papers any time we had to use Vélu formulas, we decided
to make an automatic proof of it.

The following Magma code proves the passage from Eq.~\eqref{eq:155} to
Eq.~\eqref{eq:161} and from there to~\eqref{eq:157}.

\begin{xcomment}{lstlisting}
\chapter{Proof of Vélu's formulas}
\label{cha:proof-velus-formulas}

\pdfmcone{Changed person.}  We always had admiration for our
colleagues who can develop by hand two pages full of calculations
without making mistakes. When it comes to us, we usually make a sign
mistake at the third term. Tired of having to check for sign errors in
other people's papers any time we had to use Vélu formulas, we decided
to make an automatic proof of it.

The following Magma code proves the passage from Eq.~\eqref{eq:155} to
Eq.~\eqref{eq:161} and from there to~\eqref{eq:157}.

\begin{xcomment}{lstlisting}
\input{isogeny/veluproof.mgm}
\end{xcomment}

The first and second line of output are the differences between each
term of the sums in Eqs.~\eqref{eq:155} and~\eqref{eq:161}. In both
lines, to conclude one must observe that all the terms in the
difference contain an odd power of $y(Q)$, thus they sum up to $0$
over $G^\ast$.

The third line is the difference between each term of the sums in
Eqs.~\eqref{eq:161} and~\eqref{eq:157}. The result is
self-explanatory.


% Local Variables:
% mode:flyspell
% ispell-local-dictionary:"american"
% mode:TeX-PDF
% mode:reftex
% TeX-master: "../these"
% End:

\end{xcomment}

The first and second line of output are the differences between each
term of the sums in Eqs.~\eqref{eq:155} and~\eqref{eq:161}. In both
lines, to conclude one must observe that all the terms in the
difference contain an odd power of $y(Q)$, thus they sum up to $0$
over $G^\ast$.

The third line is the difference between each term of the sums in
Eqs.~\eqref{eq:161} and~\eqref{eq:157}. The result is
self-explanatory.


% Local Variables:
% mode:flyspell
% ispell-local-dictionary:"american"
% mode:TeX-PDF
% mode:reftex
% TeX-master: "../these"
% End:

\end{xcomment}

The first and second line of output are the differences between each
term of the sums in Eqs.~\eqref{eq:155} and~\eqref{eq:161}. In both
lines, to conclude one must observe that all the terms in the
difference contain an odd power of $y(Q)$, thus they sum up to $0$
over $G^\ast$.

The third line is the difference between each term of the sums in
Eqs.~\eqref{eq:161} and~\eqref{eq:157}. The result is
self-explanatory.


% Local Variables:
% mode:flyspell
% ispell-local-dictionary:"american"
% mode:TeX-PDF
% mode:reftex
% TeX-master: "../these"
% End:

\end{xcomment}

The first and second line of output are the differences between each
term of the sums in Eqs.~\eqref{eq:155} and~\eqref{eq:161}. In both
lines, to conclude one must observe that all the terms in the
difference contain an odd power of $y(Q)$, thus they sum up to $0$
over $G^\ast$.

The third line is the difference between each term of the sums in
Eqs.~\eqref{eq:161} and~\eqref{eq:157}. The result is
self-explanatory.


% Local Variables:
% mode:flyspell
% ispell-local-dictionary:"american"
% mode:TeX-PDF
% mode:reftex
% TeX-master: "../these"
% End:
