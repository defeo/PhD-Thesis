Here we recall the basic concepts from abstract algebra that
constitute the background for all the chapters that follow. To the
reader interested in reading more about these topics, we recommend
\cite{lang} for general algebra, \cite{lidl+niederreiter:2} for finite
fields and \cite{silverman:elliptic,silverman:advanced} for elliptic
curves.

\section{Groups, Ring, Fields}
\label{sec:ring-fields}

\subsection{Objects}
\label{sec:ring-fields:objects}

A \index{group}\textbf{group} is a pair $(G,\cdot)$ such that $G$ is a
set and $\cdot:G\times G\ra G$ is an \emph{internal composition law}
satisfying:
\begin{itemize}
\item \index{associativity}\textbf{Associativity}: $(a\cdot b)\cdot c
  = a \cdot (b\cdot c)$ for any $a,b\in G$;
\item There is an element $e\in G$, called the
  \index{identity~element}\textbf{identity}, such that $a\cdot e =
  e\cdot a = a$ for any $a\in G$;
\item For any $a\in G$ there is an
  \index{inverse~element}\textbf{inverse element} $a^{-1}$ such that
  $a\cdot a^{-1} = a^{-1}\cdot a = e$.
\end{itemize}
If $\cdot$ also satisfies $a\cdot b=b\cdot a$
(\index{commutativity}\textbf{commutativity}), the group is said to be
\index{group!abelian}\textbf{abelian}. The group composed of one
single element with the obvious law is called the
\index{group!trivial}\textbf{trivial group}.

A \index{subgroup}\textbf{subgroup} of a group $(G,\cdot)$ is a group
$(H,\circ)$ such that $H\subset G$ and $\circ$ is the restriction of
$\cdot$ to $H$. Any group has two trivial subgroups: the trivial group
and itself. The \index{opposite!group}
\index{group!opposite~group}opposite group of a group $(G,\cdot)$,
denoted by $G^\op$ is the group $(G,\circ)$ where $a\circ b=b\cdot a$
for any $a,b\in G$.

Let $(G,\cdot)$ be a group, $H$ a subgroup and $g\in G$. The subset
$gH = \{g\cdot h | h \in H \}$ of $G$ is called a
\index{coset!left~coset}\textbf{left coset}, or simply
\index{coset}coset. A \index{coset!right~coset}\textbf{right coset},
denoted by $Hg$ is a left coset for the opposite group; when $G$ is
commutative the two notions coincide. A subgroup $H$ is called
\index{subgroup!normal}normal if $gH=Hg$ for any $g\in G$; note that
if $G$ is abelian, any subgroup is normal.

%todo

A \index{ring}\textbf{ring} is a tuple $(R,+,\cdot)$ such that $(R,+)$
is an abelian group and $\cdot:R\times R\ra R$ is an internal
composition law satisfying associativity, existence of the identity
and \index{distributivity}\textbf{distributivity} over $+$
\[a \cdot (b + c) = (a\cdot b) + (a\cdot c) \quad\text{for any
  $a,b,c\in R$.}\] When $\cdot$ satisfies commutativity, the ring is
said to be \index{ring!commutative}\textbf{commutative}.  The law $+$
is called \emph{addition}, $\cdot$ is called \emph{multiplication},
the identity for $+$ is denoted by $0$ and the identity for $\cdot$ by
$1$.  A commutative ring such that $1\ne 0$ and where $\cdot$ also
satisfies the existence of the inverse, is called a
\index{field}\textbf{field}.

A \index{subring}\textbf{subring} of a ring $(R,+,\cdot)$ is a ring
$(S,\ast,\circ)$ such that $(S,\ast)$ is a subgroup of $(R,+)$ and
$\circ$ is the restriction of $\cdot$ to $S$.  The
\index{opposite!ring}\index{ring!opposite~ring}\textbf{opposite ring}
of a ring $(R,+,\cdot)$, denoted by $R^\op$ is the ring $(R,+,\circ)$
where $a\circ b=b\cdot a$ for any $a,b\in R$.

The groups $R^+=(R,+)$ and $R^\ast=(R\diffset\{0\},\cdot)$ are called
respectively the \index{subgroup!additive}\textbf{additive} and
\index{multiplicative!subgroup}\textbf{multiplicative} group of
$(R,+,\cdot)$.

The simplest example of ring is $\Z$, the set of integers; the
rational numbers $\Q$ are an example of field, it is the
\emph{smallest} field containing $\Z$ as a subring. The
\index{ring!trivial}\textbf{trivial ring} is the ring composed of one
unique element $r=0=1$ with the evident laws; note that by definition
this is not a field.

Given a ring $(R,+,\cdot)$ a \index{module!left~module}\textbf{left
  module}, or simply \textbf{module}, over $R$ is a tuple $(M, \ast,
\circ)$ such that $(M,\ast)$ is an abelian group and $\circ:R\times
M\ra M$ is an \emph{external law} such that for any $r,r'\in R$ and
$m,m'\in R$
\begin{itemize}
\item $(r + r')\circ m = (r \circ m) \ast (r'\circ m)$,
\item $r\circ(m\ast m') = (r\circ m) + (r'\circ m)$,
\item $r'\circ(r\circ m ) = (r'\cdot r)\circ m$,
\item $1\circ m = m$.
\end{itemize}
The law $+$ is called \emph{addition}, its identity is denoted by $0$;
the law $\circ$ is called
\index{multiplication!scalar}\index{scalar~multiplication}\textbf{scalar}
or
\index{mutliplication!external}\index{external~multiplication}\textbf{external}
multiplication.

A \index{module!right~module}\textbf{right module} is a left module
for the opposite ring $R^\op$, a
\index{module!two-sided}\textbf{two-sided module} also called
\index{bimodule}\textbf{bimodule} is an object that is both a left and
a right module. When $R$ is commutative, the three notions coincide
and we simply speak of a \index{module}\textbf{module}.  $R$-module is
another way of saying ``module over $R$''. When $\K$ is a field, a
$\K$-module is called a \index{vector~space}\textbf{$\K$-vector
  space}.

A \index{module!submodule}\index{submodule!left~submodule}left
(\index{submoudle!right~submodule}right,
\index{submodule!two-sided}two-sided) \textbf{submodule} of a left
(right, two-sided) $R$-module $(M,+,\cdot)$ is a left (right, two-sided)
$R$-module $(N,\ast,\circ)$ such that $(N,\ast)$ is a subgroup of
$(M,+)$ and $\circ$ is the restriction of $\cdot$ to $R\times N$.

The module containing one unique element with the evident laws is
called the \index{module!zero~module}\textbf{zero module}; any
$R$-module contains a submodule that is isomorphic to the zero
module. Any group $(G,+)$ can be given a $\Z$-module structure by the
law
\[n\cdot g = \underbrace{g + \cdots + g}_{n\text{ times}} \text{.}\]
Any ring $R$ is trivially a two-sided module over itself; a
\index{ideal!left~ideal} (\index{ideal!right~ideal}right,
\index{ideal!two-sided}two-sided) \textbf{ideal} of a ring $R$ is a
submodule of the left (right, two-sided) $R$-module $R$.  When $R$ is
commutative one simply speaks of an \index{ideal}ideal.  

Any ring contains at least two submodules: the zero module and itself;
these are called the \index{ideal!trivial}\textbf{trivial ideals}. The
only non-trivial ideals of $\Z$ are the $n\Z$ for any $n\ne0,1$. A
field has no non-trivial ideals.

The \index{direct~sum}\textbf{direct sum} $M\oplus N$ of two
$R$-modules $(M,+_M,\cdot_M)$ and $(N,+_N,\cdot_N)$ is the module
$(M\times N,+,\cdot)$ where the laws $+$ and $\cdot$ are defined
component-wise. This generalizes to sums of an arbitrary number of
modules: let $(M_i)_{i\in I}$ be a sequence of $R$-modules, the direct
sum $\bigoplus_{i\in I}M_i$ is the $R$-module whose elements are the
sequences $(m_1,m_2,\ldots)$ where $m_i\in M_i$ and $m_i=0$ for all
but a finite number of them; the laws are defined component-wise.
Although less commonly used, there also exists a notion of
\index{direct~product}\textbf{direct product}: given $(M_i)_{i\in I}$,
the direct product $\prod_{i\in I}M_i$ is the $R$-module whose
elements are the sequences $(m_1,m_2,\ldots)$ where $m_i\in M_i$ with
the laws defined component-wise. Clearly, the two definition coincide
when $I$ is a finite set.

When $R$ is seen as an $R$-module over itself, we denote by $R^n$ the
direct sum $\bigoplus_{0<i\le n}R$ and by $R^\infty$ the direct sum
$\bigoplus_{i>0}R$. An $R$-module that is isomorphic to the direct sum
$\bigoplus_{I}R$ for some $I$ is called a
\index{module!free}\textbf{free module}. A \index{basis}\textbf{basis}
of a module $M$ is a family $(m_i)_{i\in I}$ of elements of $M$ such
that any $m\in M$ can be written as 
\begin{equation}
  \label{eq:module-basis}
  m = \sum_{i\in I} r_i\circ m_i
  \quad\text{with $r_i\in R$}
\end{equation}
in an unique way. Clearly, if we note by $e_i$ the element of
$\bigoplus_IR$ that has $1$ in the $i$-th position and $0$ elsewhere,
the family $(e_i)_{i\in I}$ forms a basis; hence, any free module has
a basis and, conversely, any module that has a basis is free. One
important statement about bases of modules is the following.

\begin{proposition}
  Any two bases for a free module $M$ over a commutative ring $R$ have
  the same cardinality.
\end{proposition}

For this reason, when $M$ is a free module over a commutative ring $R$
we call \index{dimension} \index{module!free!dimension~of}
\index{vector~space!dimension~of}\textbf{dimension} the cardinality of
any of its bases. It is a well known result in linear algebra that any
vector space has a basis, hence any $\K$-vector space is free as a
$\K$-module.


\subsection{Arrows}
\label{sec:ring-fields:arrows}

Morphisms, kernels, isomorphism theorems, 

\section{Linear algebra}
\label{sec:linear-algebra}

duality
matrices, trace, determinant, resultant

\section{Basic Galois theory}
\label{sec:basic-galois-theory}
Field extension, splitting field, Galois extensions, algebraic closure

roots of unity, cyclotomic polynomials

trace, norm

Artin-Schreier

finite fields


\section{Elliptic curves}
\label{sec:elliptic-curves}



%%% Local Variables: 
%%% mode:flyspell
%%% ispell-local-dictionary:"american"
%%% mode: TeX-PDF
%%% mode: reftex
%%% TeX-master: "../these"
%%% End: 


