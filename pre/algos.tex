\section{Asymptotic complexity}
\label{sec:asympt-compl}
% Many algorithms below rely on fast multiplication; thus, we let $\Mult
% : \N \rightarrow \N$ be a {\em multiplication function}, such that
% polynomials in $\F_p[X]$ of degree less than $n$ can be multiplied in
% $\Mult(n)$ operations, under the conditions of~\cite[Ch.~8.3]{vzGG}.
% Typical orders of magnitude for $\Mult(n)$ are $O(n^{\log_2(3)})$ for
% Karatsuba multiplication or $O(n\log (n) \log\log (n))$ for FFT
% multiplication. Using fast multiplication, fast algorithms are
% available for Euclidean division or extended GCD~\cite[Chapter~9 \&
% 11]{vzGG}.

% The cost of {\em modular composition}, that is, of computing $F(G)
% \bmod H$, for $F,G,H\in\F_p[X]$ of degrees at most $n$, will be
% written $\ModComp(n)$. We refer to~\cite[Chapter~12]{vzGG} for a
% presentation of known results in an algebraic computational model: the
% best known algorithms have subquadratic (but superlinear) cost in
% $n$. Note that in a boolean RAM model, the algorithm of~\cite{KeUm08}
% takes quasi-linear time.

\section{Fundamental algorithms}
\label{sec:fund-algor}
In this section we review some fundamental algorithms that we will
repeatedly use in the rest of the document. Most of the algorithms we
present are taken from~\cite{vzGG}; another source of inspiration
is~\cite{poly-formel}.

\subsection{Polynomial multiplication}
\label{sec:polyn-mult}
Multiplication of polynomials with coefficients in a ring is a
fundamental brick to which most of the algorithms in computer algebra
reduce.

In the previous section we introduced the notation $\Mult(n)$ to
denote the number of operations in $R$ required to multiply two
polynomials of degree at most $n$ in $R[X]$.  Using the school-book
algorithm, we have $\Mult(n) = O(n^2)$. The first major step forward
in the complexity of multiplication was done by \index{Karatsuba
  multiplication}Karatsuba~\cite{karatsuba}. He observed that using
the formula
\begin{gather*}
  f = f_1X^n + f_2\text{,}\qquad g = g_1X^n + g2\text{,}\\
  fg = f_1g_1X^{2n} + \bigl((f_1+f_2)(g_1+g_2)-f_1g_1-f_2g_2\bigr)X^n + f_2g_2
  \text{,}
\end{gather*}
multiplication can be computed recursively using only $3$ recursive
calls.  It follows that $\Mult(n)=O(n^{\log_23})$.

When the base ring $R$ is a field containing a primitive $n$-th root
of unit $\omega$, polynomials can be multiplied by evaluating at the
powers of $\omega$, multiplying each evaluation, and interpolating
back. The map that sends a polynomial of degree $n$ over its
evaluations at the $n$-th roots of unit is called
\index{discrete~Fourier~transform}\textbf{discrete Fourier transform},
there are many algorithms of complexity $O(n\log n)$ to compute it,
they all go under the generic name of
\index{FFT}\index{fast~Fourier~transform}\textbf{fast Fourier transform}
(FFT).

Thus, multiplication in certain fields can be carried out in time
$O(n\log n)$. In the famous paper~\cite{schonage+strassen}, Schönage
and Strassen showed how performing an FFT in an extension ring of $R$
containing the required roots of unit yields an algorithm of
complexity $O(n\log n\log\log n)$ to multiply polynomials in $R[X]$.

\subsection{Formal power series}
\label{sec:formal-power-series}
We denote by $R[[X]]$ the ring of
\index{formal~power~series}\textbf{formal power series} on $R$. Its
elements are the sequences $(f_i)_{i>0}$ of elements of $R$, they are
denoted by
\begin{equation}
  \label{eq:197}
  f(X) = \sum_{i>0}f_iX^i
  \text{.}
\end{equation}
Multiplication and evaluation are defined in the obvious way. An
element $f\in R[[X]]$ is invertible if and only if $f(0)$ is an unit
of $R$.

Since formal power series are infinite objects, to be used in a
discrete algorithm they must be approximated. We denote by $f\bmod
X^n$ the polynomial
\begin{equation}
  \label{eq:198}
  f\bmod X^n = \sum_{0\le i < n}f_iX^i
  \text{.}
\end{equation}
We write $f = g + O(X^n)$, where $g$ is a polynomial or a power
series, whenever
\[f\bmod X^n=g\bmod X^n\text{,}\] and we say that $g$ approximates $f$
to the precision $n$.

Using polynomial multiplication, the product of two series known up to
precision $n$ can be computed in $O(\Mult(n))$ operations.

\paragraph{Derivative, integral}
\label{sec:derivative-integral}
If $R$ has characteristic $0$, we define the
\index{formal~power~series!derivative}derivative and the
\index{formal~power~series!integral}integral of a power series as
\begin{align}
  \label{eq:200}
  f'(X) &= \sum_{i\ge0}(i+1)f_{i+1}X^i\text{,}\\
  \int f(X) &= \sum_{i\ge0}\frac{f_i}{i+1}X^{i+1}\text{.}
\end{align}
Derivatives and integrals up to precision $n$ can be computed in
$O(n)$ operations by their definition.

\paragraph{Logarithm, exponential}
\label{sec:logarithm}
The \index{formal~power~series!logarithm}logarithm of a power series
$f$ such that $f(0)=1$ is defined as
\begin{equation}
  \label{eq:194}
  \log f = \int\frac{f'}{f}
  \text{.}
\end{equation}
The \index{formal~power~series!exponential}exponential of a power
series $f$ such that $f(0)=0$, is defined as
\begin{equation}
  \label{eq:195}
  \exp(f) = 1 + f/1! + f^2/2! + \cdots
\end{equation}
All the usual identities involving multiplication, derivatives,
integrals, logarithms and exponentials are verified on power
series. In the next subsection we shall see that integrals, logarithms
and powers up to precision $n$ can all be computed in time
$O(\Mult(n))$.

In characteristic different from $0$, these definition do not make
sense anymore, however we can still do computations on truncated power
series in positive characteristic, the theoretical justification being
a lift in characteristic $0$.


\subsection{Newton's iteration}
\label{sec:newtons-iteration}
Let $\Phi:\R\ra\R$ be a $C^1$ function, the
\index{Newton's~iteration}Newton's iteration is a classical method to
approximate a root $x$ of $\Phi$. Start from an approximation $x_0$, and
\emph{linearize} $\Phi$ to compute
\begin{equation}
  \label{eq:192}
  x_1 = x_0 - \frac{\Phi(x_0)}{\Phi'(x_0)}
  \text{,}
\end{equation}
then iterate this step until the desired precision is obtained. When
$x_0$ is taken close enough to a root, and when the derivative at this
root is non-zero, Newton's iteration converges \emph{quadratically} to
the solution, meaning that at each iteration the distance to the
solution is squared.

In computer algebra, Newton's iteration is applied to operators
$\Phi:R[[X]]\ra R[[X]]$ on formal power series; in this context,
\emph{quadratic} convergence means that the number of correct terms is
doubled at each iteration. Many fast algorithms for some fundamental
operations on power series and polynomials are obtained by this
method, here we summarize the most important ones.

\paragraph{Inversion}
If $f\in R[[X]]$ is invertible, the operator $\Phi(y) = 1/y - f$
applied to $y_0=1$ converges quadratically to the inverse of
$f$. Since the iteration associated to $\Phi$ is
\begin{equation}
  \label{eq:193}
  y_{i+1} = y_i(2 - y_if)
  \text{,}
\end{equation}
the cost of inverting a power series is $O(\Mult(n))$. From
Eq.~\eqref{eq:194} we deduce that computing the logarithm of a power
series has the same cost.

Another important consequence of this algorithm is that the Euclidean
division of polynomials of degree at most $n$ can also be performed in
$O(\Mult(n))$ operations.

\paragraph{Exponential}
If $f$ is such that $f(0)=0$, we compute its exponential using the
operator $\Phi(y)=f-\log y$, which gives the iteration
\begin{equation}
  \label{eq:196}
  y_{i+1} = y_i(1 + f - \log y)
  \text{.}
\end{equation}
Thus, the cost of computing an exponential is $O(\Mult(n))$ too. Using
the formula
\begin{equation}
  \label{eq:201}
  f^\alpha = \exp(\alpha\log f)
  \text{,}
\end{equation}
we deduce that, in characteristic $0$, computing arbitrary rational
powers of power series costs $O(\Mult(n))$ too.


\subsection{Modular composition}
\label{sec:modular-composition}


also frobenius composition(vzGS92
Algorithm~5.2),



\subsection{Differential equations}
\label{sec:diff-equat}


\subsection{Euclidean algorithm and rational fraction reconstruction}
\label{sec:eucl-algor-rati}
\cite[$\S$11.1]{vzGG},
\cite[$\S$5.8]{vzGG} 

\subsection{Chinese remainder algorithm and interpolation}
\label{sec:chin-rema-algor}
\cite[$\S$10]{vzGG}

\subsection{Multivariate polynomials}
\label{sec:mult-polyn}
Kronecker substitution (vzGS92)\\
maybe bivariate operations (some Pascal-Schost and some Li-Moreno-Schost) ?

\subsection{Transposed algorithms}
\label{sec:transp-algor}
transposed mul, transposed mod,


%%% Local Variables: 
%%% mode:flyspell
%%% ispell-local-dictionary:"american"
%%% mode: TeX-PDF
%%% mode: reftex
%%% TeX-master: "../these"
%%% End: 
